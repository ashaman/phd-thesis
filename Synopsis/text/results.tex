\begin{enumerate}
  \item Предложен комплекс методов для моделей с чёткими отношениями и нечёткими параметрами, позволяющий применять классические методы решения задач и достигать требуемых качественных свойств решения~--- устойчивости, сохранения естественных математических соотношений и т.\,п..
  \item На основе результатов анализа существующих моделей представления нечёткой числовой информации разработана параметрическая модель представления нечёткого числа, позволяющая максимально сохранять исходную экспертную информацию, а также метод двухточечных вычислений, приводящий к эффективной численной реализации решения задач, основанной на подходящих алгебраических структурах.
  \item В рамках метода двухточечных вычислений рассмотрена проблема устойчивости решения задачи линейного программирования с нечёткими параметрами, обосновано введение свёртки критериев для управления устойчивостью и сформулирован алгоритм получения устойчивого решения задачи.
  \item Предложенные методы решения задач с нечёткими параметрами апробированы на задаче сетевого планирования с нечёткими временными оценками. Полученное в результате решение соответствует решениям, найденным с помощью других методов, хорошо зарекомендовавших себя в мировой практике.
  \item Разработан программный комплекс, позволяющий решать задачу оценки сроков и рисков при разработке программного обеспечения как задачу сетевого планирования с нечёткими временными оценками.
\end{enumerate}