\begin{enumerate}
  \item Разработана и исследована модель представления нечётких числовых параметров математического описания объектов в классе треугольных LR-чисел, обеспечивающая возможность построения алгебраической структуры нечётких чисел, сохраняющей требуемые свойства решения задач выбора: ограничение роста неопределённости, сохранение истинности модельных отношений и возможность интерпретации полученного результата.
  \item Разработан метод приближённого численного решения задач выбора с нечёткими параметрами, инвариантный к форме математического описания задачи, позволяющий строить нечёткое решение задач как линейную комбинацию чётких решений, полученных на границах интервального представления параметров, снизить вычислительную сложность процесса получения решения и применять стандартные программные продукты для нечётких вычислений.
  \item Предложенные методы решения задач выбора с нечёткими параметрами апробированы на задаче сетевого планирования с нечёткими временными оценками. В процессе апробации рассмотрена проблема устойчивости критического пути, обосновано введение свёртки критериев для управления устойчивостью и сформулирован алгоритм, обеспечивающий получение устойчивого решения задачи. Достоверность полученного решения подтверждается его сравнением с решениями, найденным с помощью других методов, хорошо зарекомендовавших себя в мировой практике.
  \item Разработан программный комплекс, позволяющий решать задачу оценки сроков при разработке программного обеспечения как задачу сетевого планирования с нечёткими временными оценками и обеспечивающий учёт возможных рисков, возникающих при разработке программного обеспечения. Практическая ценность коплекса подтверждается актом о внедрении.   
\end{enumerate}