\textbf{Актуальность темы.} Развитие теории нечётких множеств позволило расширить возможности учёта различных видов неопределённости, для описания которых в~течение долгого времени в моделях использовались методы теории вероятностей и математической статистики. Исторически сложилось два основных направления нечёткой математики~--- нечёткий логический вывод, использующий модели с нечёткими отношениями, и мягкие вычисления, применяемые в моделях с чёткими отношениями и нечёткими параметрами. Удобство классических моделей состоит в том, что они достаточно хорошо проработаны и испытаны временем, а для подкласса линейных моделей возможно получать решение в аналитическом виде, не применяя различные методологии имитационного моделирования.

Фаззификация известных ранее классических задач и создание новых нечётких моделей привела к~необходимости разработки новых методов решения, позволяющих применять экспертные оценки на различных этапах моделирования. В работах известных зарубежных (D. Dubois, R. Fuller, A. Prade, R. Yager, L. Zadeh, H. Zimmermann и др.) и отечественных (В.~Г. Балашов, А.~H. Борисов, В.~В. Борисов, В.~В. Круглов, С.~Л. Блюмин, А.~А. Усков и др.) учёных и исследователей рассмотрено и проанализировано множество применений результов теории нечётких множеств и мягких вычислений к решению задач выбора, управления и принятия решений. Обратной стороной использования моделей с нечёткостью стало возникновение противоречий между решениями, полученными с применением новых методов, и результатами классических теорий, потеря устойчивости решений, нарушение естественных отношений в моделях, в которых нечёткими являются только параметры, неоправданное расширение степени нечёткости результата, повышение вычислительной сложности задач.

Актуальность темы исследования определяется необходимостью разработки математических моделей, численных методов и программ, инвариантных к широкому кругу различных задач с чёткими отношениями и нечёткой неопределённостью параметров и позволяющих решать их как совокупность нескольких чётких, используя при этом классические методы моделирования и стандартное ПО и обеспечивая требуемые в конкретной задаче качественные свойства решения~--- устойчивость, сохранение чётких математических соотношений и т.\,п.

Диссертационная работа выполнена в рамках одного из основных научных направлений Воронежского государственного университета <<Разработка моделей, методов и алгоритмов обработки информации для создания информационных технологий и систем нового поколения>> (№ гос. регистрации 01201263910).

\textbf{Целью} диссертационной работы является построение и исследование моделей учёта нечёткой неопределённости, обеспечивающих требуемые свойства решения различных прикладных задач~--- устойчивость, сохранение чётких математических соотношений, ограничение расширения неопределённости, а также разработка методов численного решения на основе вводимых моделей. Для достижения поставленной цели в работе решались следующие \textbf{задачи}:
\begin{enumerate}
  \item анализ существующих методик нечётких вычислений с~точки зрения сохранения свойств решения задач;
  \item разработка модели представления нечётких чисел, позволяющей максимально сохранять исходную экспертную информацию и обеспечить требуемые качественные свойства решений (устойчивость, сохранение чётких математических соотношений и т.\,п.);
  \item разработка методики эффективной численной реализации решения задач с нечёткими параметрами, основанной на подходящих алгебраических структурах и её тестирование на примере задачи сетевого планирования с нечёткими параметрами;
  \item разработка и верификация программного обеспечения, реализущего предложенную модель представления нечётких параметров и методики численного решения задач с нечёткими параметрами.
\end{enumerate}

\textbf{Методы исследования.} В~диссертационной работе использованы основные положения и~методы теории нечётких множеств, мягких вычислений, интервального анализа, теории алгебраических структур, теории графов, численных методов. При~создании программного обеспечения использовались технологии модульного и~объектно-ориентированного программирования.

\textbf{Тематика работы.} Содержание диссертации соответствует п. 1 <<Разработка новых математических методов моделирования объектов и явлений>>, п. 3 <<Разработка, обоснование и тестирование эффективных вычислительных методов с применением современных компьютерных технологий>>, п. 8 <<Разработка систем компьютерного и имитационного моделирования>> паспорта специальности 05.13.18~--- <<Математическое моделирование, численные методы и комплексы программ>>.

\textbf{Научная новизна.} В диссертационной работе получены следующие результаты, характеризующиеся научной новизной:
\begin{enumerate}
  \item предложена и исследована модель представления расплывчатых числовых оценок в классе треугольных LR-чисел, отличающаяся модификацией нечёткого LR-числа, основанной на применении предложенного L-преобразования LR-чисел в соответствующие LL/RR-числа;
  \item разработаны вычислительные методы приближённого решения задач выбора с нечёткими параметрами, отличающиеся построением и использованием изоморфной алгебраической структуры (поле модифицированных нечётких чисел), инвариантные к форме математического описания задачи и позволяющие параметрически управлять устойчивостью решения;
  \item разработан программный комплекс для решения задач с нечёткими параметрами, реализующий предложенные в работе вычислительные методы, отличающийся использованием стандартных вычислительных операций над действительными переменными (в отличие от специализированных программных пакетов, работающих с нечеткими числами).
\end{enumerate}

\textbf{Достоверность научных результатов.} Научные положения, теоретические выводы и практические рекомендации обоснованы корректным использованием выбранного математического аппарата и подтверждены результатами вычислительного эксперимента.

\textbf{Практическая значимость исследования} заключается в расширении сферы применимости методов моделирования с использованием чётких отношений и нечётких параметров. Подходы к нечётким вычислениям, предложенные в диссертации, позволяют существенно упростить процедуру расчётов без значительных потерь экспертной информации, а также использовать существующее стандартное программное обеспечение для решения различных производственных задач.

\textbf{Реализация и внедрение результатов работы.} Разработанный программный комплекс <<CSBusinessGraph>> используется в практической деятельности по первоначальной оценке проектов ООО <<Философт>> (DataArt). Результаты диссертации в форме моделей, алгоритмов и программ используются в производственном процессе ООО <<Философт>>, что подтверждается актом о внедрении. Признана целесообразность использования предложенной в диссертации методики для оптимизации процедур первоначальной оценки проектов по разработке программного обеспечения.

\textbf{Апробация работы.} Основные результаты работы докладывались на ежегодных научных сессиях Воронежского государственного университа и следующих конференциях различного уровня: международная конференция <<Современные проблемы прикладной математики, теории управления и математического моделирования>> (Воронеж, 2012 г.); международная конференция <<Информатика: проблемы, методология, технологии>> (Воронеж, 2013--2014 гг.); международный научно-технический семинар <<Современные технологии в задачах управления, автоматики и обработки информации>> (Алушта, 2013--2014 гг.); научно--техническая конференция студентов и аспирантов <<Радиоэлектроника, электротехника и энергетика>> (Москва, 2014).

\textbf{Публикации.} По теме диссертационного исследования опубликовано 11 научных работ \cite{PMTYMM}--\cite{Kanischeva}, в~том~числе 4 \cite{Kanischeva, Vorontsov_Compare, Vorontsov_PI, Vorontsov_VSTU}~--- в~изданиях, рекомендованных ВАК РФ. В~работах, выполненных в~соавторстве: в~\cite{Vorontsov_PI} предложено преобразование~L и~алгебра модифицированных нечётких чисел; в~\cite{Vorontsov_Compare} выполнен анализ существующих методов сравнения нечётких чисел и предложен метод сравнения для модифицированных LL/RR-чисел; а в~\cite{Vorontsov_VSTU}~--- предложено определение устойчивости задачи нечёткого линейного программирования и разработан алгоритм решения задачи календарно-сетевого планирования и управления с нечёткими параметрами, позволяющий получать устойчивое решение.

\textbf{Объем и структура работы.} Диссертация состоит из~введения, четырех глав, заключения и~списка литературы. Полный объем диссертации \textbf{158}~страниц текста с~\textbf{16}~рисунками и~\textbf{3}~таблицами. Список литературы содержит \textbf{131}~наименование, включая работы автора.