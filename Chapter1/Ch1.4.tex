Проведённый анализ алгебраических методов решения нечётких задач показал, что они не позволяют применять хорошо изученные чёткие методы и модели в задачах и не обеспечивают требуемые свойства решения~--- устойчивость, непротиворечивость естественным математическим отношениям, ограничение расширения неопределенности. Успешное решение таких задач возможно при создании и применении методики нечётких вычислений, которая, с одной стороны, обеспечивала бы требуемые свойства, а с другой~---простоту решения, позволяющую использовать классические методы.

%Проведённый выше анализ алгебраических методов решения нечётких задач показал, что они не позволяют применять хорошо изученные чёткие методы и модели в задачах с нечёткими параметрами и не рассматривают проблему устойчивости решения, которая естественно возникает в задачах с нечеткими числами. Успешное решение таких задач возможно при создании и применении методики нечётких вычислений, которая, с одной стороны, обеспечивала бы простоту и универсальность вычислений, подобно методам, использующим дискретизацию и дефаззификацию нечётких параметров, а с другой, имела бы под собой строгую теоретическую базу, как существующие алгебры нечётких чисел.


%Целью диссертации является разработка моделей и методов решения задач выбора в условиях нечёткой параметрической неопределённости, представленной треугольными числами, в частности, задачи линейного программирования с нечёткими параметрами, с учётом сохранения устойчивости конечного результата.
Целью диссертационной работы является построение и исследование моделей учёта нечёткой неопределённости, обеспечивающих требуемые свойства решения различных производственных задач.

Для достижения поставленной цели в работе решались следующие задачи.
\begin{enumerate}
  \item Анализ существующих методик нечётких вычислений с~точки зрения сохранения свойств решения задач.
  \item Разработка модели представления нечётких чисел, позволяющей максимально сохранять исходную экспертную информацию и обеспечить требуемые качественные свойства решений (устойчивость, сохранение математических соотношений и т.\,п.).
  \item Разработка методики эффективной численной реализации решения задач с нечёткими параметрами, основанной на подходящих алгебраических структурах и её тестирование на примере задачи сетевого планирования с нечёткими параметрами.
  \item Разработка и верификация программного обеспечения, реализущего предложенную модель представления нечётких параметров и методики численного решения задач с нечёткими параметрами.
\end{enumerate}