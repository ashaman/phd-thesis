Проведённый анализ нечётких методов моделирования и классических моделей представления нечёткости показал, что они не позволяют применять хорошо изученные чёткие методы и модели в задачах с нечёткими параметрами и не~гарантируют обеспечение требуемых свойств решения~--- устойчивости, непротиворечивости естественным математическим отношениям, ограничения расширения неопределенности. Успешное решение задач моделирования с нечёткими параметрами и чёткими отношениями возможно при создании и применении методики нечётких вычислений, которая, с одной стороны, обеспечивала бы требуемые свойства, а с другой~---простоту решения, позволяющую использовать классические методы.

Целью диссертационной работы является построение и исследование моделей учёта нечёткой неопределённости, обеспечивающих требуемые свойства решения различных производственных задач.

Для достижения поставленной цели в работе решались следующие задачи.
\begin{enumerate}
  \item Анализ существующих методик нечётких вычислений с~точки зрения сохранения свойств решения задач.
  \item Разработка модели представления нечётких чисел, позволяющей максимально сохранять исходную экспертную информацию и обеспечить требуемые качественные свойства решений (устойчивость, сохранение математических соотношений и т.\,п.).
  \item Разработка методики эффективной численной реализации решения задач с нечёткими параметрами, основанной на подходящих алгебраических структурах и её тестирование на примере задачи сетевого планирования с нечёткими параметрами.
  \item Разработка и верификация программного обеспечения, реализущего предложенную модель представления нечётких параметров и методики численного решения задач с нечёткими параметрами.
\end{enumerate}