\subsection{Цель и задачи исследования}

Проведённый выше анализ алгебраических методов решения нечётких задач показал, что они не позволяют применять хорошо изученные чёткие методы и модели в задачах с нечёткими параметрами и не рассматривают проблему устойчивости решения, которая естественно возникает в задачах с нечеткими числами. Успешное решение таких задач возможно при создании и применении методики нечётких вычислений, которая, с одной стороны, обеспечивала бы простоту и универсальность вычислений, подобно методам, использующим дискретизацию и дефаззификацию нечётких параметров, а с другой, имела бы под собой строгую теоретическую базу, как существующие алгебры нечётких чисел.

Целью диссертации является разработка моделей и методов решения задач выбора в условиях нечёткой параметрической неопределённости, представленной треугольными числами, в частности, задачи линейного программирования с нечёткими параметрами, с учётом сохранения устойчивости конечного результата.

Для достижения поставленной цели в работе решались следующие задачи.

\begin{enumerate}
	\item Анализ существующих методик нечётких вычислений с точки зрения абстрактных алгебр и вычислительной эффективности.
	\item Разработка модели представления треугольных нечётких чисел, позволяющей извлекать необходимую для выбора или принятия решения информацию и сохранять большую часть параметров числа неизменными. 
	\item Разработка методов обработки нечётких величин – алгебры модифицированных нечётких чисел и эквивалентной ей эффективной численной методики решения нечётких задач как совокупности чётких.
	\item Разработка устойчивого метода решения задачи линейного программирования (о критическом пути) с нечёткими параметрами
	\item Разработка и верификация программного обеспечения, реализущего предложенную модель представления нечётких параметров и подходы к решению задачи линейного программирования с нечёткими параметрами.
\end{enumerate}

\subsection{Выводы по главе 1}