\begin{enumerate}
  \item введены основные понятия теории нечётких множеств. Описаны актуальные модели представления нечёткой информации~--- нечёткие множества и нечёткие числа, определены основные операции над ними; 
  \item произведена классификация нечётких моделей в~зависимости от этапа применения нечёткой математики и от применяемого в них языка описания выбора. Кратко описаны модели, использующие нечёткий логический вывод, перечислены их достоинства и недостатки. В качестве исследуемых выбраны модели, использующие чёткие отношения и нечёткие параметры;
  \item рассмотрены основные типы абстрактных алгебр и их проекция на нечёткие вычисления и показаны недостатки классических подходов к нечётким вычислениям;
  \item сформулированы требования к алгебраической системе, которая необходима для решения задач второго типа с нечёткими параметрами и чёткими отношениями~--- устойчивость решения, непротиворечивость естественным математическим отношениям, ограничение расширения неопределенности;
  \item определены цель и задачи исследования.
\end{enumerate}
