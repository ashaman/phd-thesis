\begin{enumerate}
  \item Приведены основные понятия теории нечётких множеств и описаны актуальные модели представления нечёткой информации~--- нечёткие множества и нечёткие числа,~--- используемые в дальнейшем при описании исследования.
  \item Нечёткие модели обычно классифицируют в~зависимости от этапа применения нечёткой математики~--- при описании системы, при задании параметров, при задании входов, выходов и состояний (модели первого, второго и третьего типа). Предлагается классифицирировать их на основе применяемого в них языка описания выбора и объединить две классификации в одну. В качестве исследуемых выбраны модели, использующие чёткие отношения и нечёткие параметры.
  \item Хорошо разработанный нечёткий логический вывод не может применяться в моделях второго типа, поскольку рассчитан на нечёткость отношений, отсутствие формализованных адекватных математических моделей либо способов решения с помощью классической теории. Другие подходы к нечётким вычислениям далеко не всегда применимы в рассматриваемых моделях, поскольку лежащие в их основании алгебраические структуры (в основном решётки) и отсутствие отношения линейного порядка приводят к нарушениям чётких математических отношений и неоправданному расширению неопределённости результата.
  \item Сформулированы основные требования к алгебраической системе, которая необходима для решения задач второго типа с нечёткими параметрами и чёткими отношениями~--- устойчивость решения, непротиворечивость естественным математическим отношениям, ограничение расширения неопределенности. Также введены требования вычислительной эффективности и возможности применения стандартных программных комплексов, предназначенных для чётких вычислений.
  \item Определены цель и задачи исследования.
\end{enumerate}
