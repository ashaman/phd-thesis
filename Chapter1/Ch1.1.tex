\subsection{Основные понятия теории нечётких множеств}
Теория нечётких множеств появилась в~1965 году с~выходом статьи Лотфи Заде <<Fuzzy Sets>>~\cite{Zadeh_Classic}. Понятие нечёткого множества~--- попытка математической формализации нечёткой информации с~целью её~использования при~построении математических моделей сложных систем~\cite{Orlovskiy}. В~основе этого понятия лежит представление о том, что элементы некоторого множества обладают каким-то общим свойством в разной степени, и, следовательно, принадлежат этому множеству в различной степени. Ключевая идея, изложенная в статье Заде, расширяет классическое понятие множества, допуская, что~функция принадлежности $\mu_{\tilde A}\left( x \right)$ некоторого элемента $x$ множеству может принимать любые значения из интервала $\left[ 0;1 \right]$, а не только 0 или 1. Само~множество в~этом~случае представляется в~виде совокупности пар
\begin{equation}
	\tilde{A}=\left\{ \left( x, \mu_{\tilde A}\left( x \right) \right)\left| x\in X \right. \right\} 
\end{equation}
где уже упомянутая функция принадлежности $\mu_{\tilde A} \left( x \right)$ характеризует степень, с которой элемент $x$ можно отнести к~нечётком множеству $\tilde{A}$. При помощи нечётких множеств можно выразить неточные понятия вроде <<низкий дом>>, <<пожилой человек>>, <<много денег>>, однако это~требует задания чёткого множества $X$, которое обычно называется областью рассуждений, либо универсальным множеством.

В~\cite{Rutkovskaya, Borisov_Krumberg_Riga} для конечных нечётких множеств применяется следующий символьный способ записи нечётких множеств. Если~$X$ – чёткое множество с~конечным числом элементов, т.е. $X=\left\{x_1, \dots, x_N \right\}$, то~нечёткое множество $\tilde{A}\subseteq X$  записывается в~виде суммы дробей, в~числителе которых стоит степень принадлежности элемента множеству, а~в~знаменателе~--- его~значение,~т.е.
\begin{equation}
\label{eq:discrete-fuzzy-set}
	\tilde{A}=\sum\limits_{i=1}^{N} \frac{\mu_{\tilde A} \left( x_i \right)}{x_i}
\end{equation}

В~формуле~\eqref{eq:discrete-fuzzy-set} дробь не~несёт в~себе семантики деления, а~всего~лишь является другой формой записи пары $\left( {{x}_{i}},{{\mu }_{{\tilde{A}}}}\left( {{x}_{i}} \right) \right)$. Аналогично, для~бесконечного множества $X$ нечёткое подмножество $\tilde{A}\subseteq X$ записывается в форме
\begin{equation}
\label{eq:infinite-fuzzy-set}
	\tilde{A}=\int\limits_{x}{\frac{{{\mu }_{{\tilde{A}}}}\left( x \right)}{x}dx}
\end{equation}

Ещё одним вариантом представления нечётких множеств является т.\,н. горизонтальная форма~\cite{Pegat}, т.\,е. их~выражение в~виде совокупности чётких подмножеств множества Х, каждое из~которых называется $\alpha$--сечением. 
\begin{mydef} 
$\alpha$--сечением (срезом, разрезом) нечёткого множества $\tilde{A}$ называется чёткое множество $A_\alpha$, определяемое в~\cite{Rutkovskaya, Pospelov, Borisov_Alexeev_Msk} следующим образом 
\begin{equation}
\label{eq:alpha-cut}
	A_{\alpha}= \left\{ x\in X \left| \mu_{\tilde A}\left( x \right)\geqslant \alpha \right. \right\}
\end{equation}
где $\chi \left( {{A}_{\alpha }} \right)$~--- характеристическая функция, определяемая выражением~\eqref{eq:alpha-char-function}:
\begin{equation}
\label{eq:alpha-char-function}
	\chi \left( {{A}_{\alpha }} \right)=\left\{
		\begin{aligned}
			1, & \mu_{\tilde A}\left( x \right) \geqslant \alpha; \\
			0, & \mu_{\tilde A}\left( x \right) < \alpha.
		\end{aligned}		 
	\right.
\end{equation}
\end{mydef}

Для~$\alpha$-сечений нечёткого множества справедлива теорема о~декомпозиции, которая позволяет не только выполнять разложение нечёткого множества на совокупность чётких, но и синтезировать исходное нечёткое множество из совокупности чётких $\alpha$-интервалов~\cite{Kaufmann}. 
\begin{theorem}
Любое нечёткое множество $\tilde{A}$ можно представить в~виде объединения его~$\alpha$--сечений, т.е.
\begin{equation}
\label{eq:alpha-cut-theorem}
	\tilde{A}=\bigcup\limits_{\alpha \in \left[ 0;1 \right]}{A_\alpha}
\end{equation}
\end{theorem}

Например, для множества $\displaystyle \tilde{A}=\frac{0,1}{1}+\frac{0.4}{2}+\frac{0.6}{3}+\frac{1}{4}+\frac{0.9}{5}+\frac{0.5}{6}$, определённого в~пространстве $\displaystyle X=\left\{ 1 \cdots 6 \right\}$, декомпозиция по~$\alpha$-уровням выглядит следующим образом:
\begin{gather*}
	  \tilde{A}=\left( \frac{0,1}{1}+\frac{0,1}{2}+\frac{0,1}{3}+\frac{0,1}{4}+\frac{0,1}{5}+\frac{0,1}{6} \right)\bigcup \left( \frac{0,4}{2}+\frac{0,4}{3}+\frac{0,4}{4}+\right.\\
	  \left. +\frac{0,4}{5}+\frac{0,4}{6} \right) \bigcup\left( \frac{0,5}{3}+\frac{0,5}{4}+\frac{0,5}{5}+\frac{0,5}{6} \right) \bigcup \left( \frac{0,6}{3}+\frac{0,6}{4}+\frac{0,6}{5} \right) \bigcup {} \\
	  {} \bigcup \left( \frac{0,9}{4}+\frac{0,9}{5} \right) \bigcup \left( \frac{1}{4} \right) 
\end{gather*}

Также для~нечёткого множества в~\cite{Pospelov, Rutkovskaya, Yakhyaeva} вводятся понятия носителя и высоты.
\begin{mydef}
Носителем нечёткого множества $\tilde{A}$ называется чёткое множество $supp\left( {\tilde{A}} \right)$, определяемое~как
\begin{equation}
\label{eq:support}
	supp\left( \tilde A \right)=\left\{ x\in X \left| \mu_{\tilde A}\left( x \right)>0 \right. \right\}.
\end{equation}
\end{mydef}
Иными словами, носитель является множеством строгого уровня $A_\alpha= \left\{ x\in X \left| \mu_{\tilde A}\left( x \right)> \alpha \right. \right\}$ при $\alpha = 0$.

\begin{mydef}
Высотой нечёткого множества $\tilde A$ называется величина
\begin{equation}
\label{eq:number-height}
	h \left( \tilde A \right)= \underset{x\in R}{\mathop {\sup}} {}\, \left( \mu_{\tilde A} \left( x \right) \right).
\end{equation}
\end{mydef}
Если высота $h\left( \tilde A \right)$ нечёткого множества равна 1, то оно является нормальным; если же $h\left( \tilde A \right) < 1$, то множество называется субнормальным.

\subsection{Операции над нечёткими множествами. Принцип обобщения Заде}
Операции над нечёткими множествами можно определить по-разному. При~этом нужно учитывать, что нечёткие множества охватывают и множества в обычном смысле, поэтому вводимые операции не должны противоречить уже известным теоретико-множественным операциям. Рассмотрим классические максиминные формулировки объединения, пересечения и дополнения нечётких множеств, приведённые в~\cite{Rutkovskaya, Orlovskiy, Borisov_Alexeev_Msk, Borisov_Krumberg_Riga, Kaufmann}.
\begin{mydef}
Пересечением нечётких множеств $\tilde{A}$ и $\tilde{B}$ называется нечёткое множество $\displaystyle \tilde{C}=\tilde{A}\bigcap \tilde{B}$, функция принадлежности которого
\begin{equation}
\label{eq:fuzzy-cross}
	{{\mu }_{{\tilde{C}}}}\left( x \right)=\min \left[ {{\mu }_{{\tilde{A}}}}\left( x \right);{{\mu }_{{\tilde{B}}}}\left( x \right) \right]
\end{equation}
для любых $x\in X$.
\end{mydef}

\begin{mydef}
Объединением нечётких множеств $\tilde{A}$ и $\tilde{B}$ называется нечёткое множество $\displaystyle \tilde{C}=\tilde{A}\bigcup \tilde{B}$, функция принадлежности которого равна
\begin{equation}
\label{eq:fuzzy-union}
	{{\mu }_{{\tilde{C}}}}\left( x \right)=\max \left[ {{\mu }_{{\tilde{A}}}}\left( x \right);{{\mu }_{{\tilde{B}}}}\left( x \right) \right]
\end{equation}
для любых $x\in X$.
\end{mydef}

\begin{mydef}
Дополнением нечёткого множества $\tilde{A}\subseteq X$ называется нечёткое множество $\displaystyle \tilde{C}=\bar{\tilde{A}}$, функция принадлежности которого равна
\begin{equation}
\label{eq:fuzzy-minus}
	{{\mu }_{{\tilde{C}}}}\left( x \right)=1-{{\mu }_{{\tilde{A}}}}\left( x \right)
\end{equation}
для любых $x\in X$.
\end{mydef}

В тех~же источниках~\cite{Rutkovskaya, Borisov_Alexeev_Msk, Borisov_Krumberg_Riga, Kaufmann} предлагаются и~альтернативные определения~операций пересечения и~объединения нечётких множеств, называемые алгебраическим произведением и~алгебраической~суммой.
\begin{mydef}
Алгебраическим произведением нечётких множеств $\tilde{A}$ и $\tilde{B}$ называется нечёткое множество $\displaystyle \tilde{C}=\tilde{A}\ \widehat{\cdot}\ \tilde{B}$, функция принадлежности которого
\begin{equation}
\label{eq:fuzzy-cross-mult}
	\mu_{\tilde C} \left( x \right)=\mu_{\tilde A}\left( x \right) \cdot \mu_{\tilde B}\left( x \right)
\end{equation}
для любых $x\in X$.
\end{mydef}

\begin{mydef}
Алгебраической суммой нечётких множеств $\tilde{A}$ и $\tilde{B}$ называется нечёткое множество $\displaystyle \tilde{C}=\tilde{A}\, \widehat{+}\, \tilde{B}$, функция принадлежности которого равна
\begin{equation}
\label{eq:fuzzy-union-summ}
		\mu_{\tilde C} \left( x \right)=\mu_{\tilde A}\left( x \right) + \mu_{\tilde B}\left( x \right) - \mu_{\tilde A}\left( x \right) \cdot \mu_{\tilde B}\left( x \right)
\end{equation}
для любых $x\in X$.
\end{mydef} 

Как отмечается в~\cite{Axioms_Fuzzy_Algebra, Kaufmann, Lipetsk}, максиминные и~алгебраические операции над нечёткими множествами обладают свойствами коммутативности, ассоциативности и~отвечают законам де~Моргана, идемпотентности и некоторым другим правилам, справедливым для чётких множеств. Если обозначить операции объединения и алгебраической суммы за $\oplus$, a пересечения и алгебраического произведения за $\otimes$, то свойства и законы будут выглядеть следующим образом:
\begin{gather*}
	\tilde A \oplus \tilde B = \tilde B \oplus \tilde A;\ \tilde A \otimes \tilde B = \tilde B \otimes \tilde A; \allowbreak \\
	\tilde A \oplus \left(\tilde B \oplus \tilde C \right) = \left( \tilde A \oplus \tilde B \right) \oplus \tilde C;\ \tilde A \otimes \left(\tilde B \otimes C \right) = \left( \tilde A \otimes \tilde B \right) \otimes \tilde C; \allowbreak \\
	\overline{\tilde A \oplus \tilde B}=\overline{\tilde A} \otimes \overline{\tilde B};\  \overline{\tilde A \otimes \tilde B}=\overline{\tilde A} \oplus \overline{\tilde B}; \allowbreak \\
	\tilde A \otimes \tilde A = \tilde A;\ \tilde A \oplus \tilde A = \tilde A;
\end{gather*}
\begin{gather*}
	\tilde A \otimes \varnothing = \varnothing;\ \tilde A \oplus \varnothing = \tilde A; \allowbreak \\
	\tilde A \otimes X = \tilde A;\ \tilde A \oplus X = X; \\
	\overline{\overline{\tilde A}} = \tilde A.
\end{gather*}
Однако для всех операций не~выполняются условие дополнения,~т.е.
\begin{gather*}
	\tilde{A} \oplus \overline{\tilde{A}}\ne X; \\
	\tilde{A} \otimes \overline{\tilde{A}}\ne \varnothing,
\end{gather*}
а для алгебраических~--- ещё и свойство дистрибутивности, которое выполняется для операций пересечения и объединения:
\begin{gather*}
	\tilde{A}\, \widehat{+}\, \left( \tilde{B}\ \widehat{\cdot}\ \tilde C \right) \neq \left( \tilde A\, \widehat{+}\, \tilde B \right)\ \widehat{\cdot}\ \left(\tilde A\, \widehat{+}\, \tilde C \right); \\
		\tilde{A}\ \widehat{\cdot}\ \left( \tilde{B}\, \widehat{+}\, \tilde C \right) \neq \left( \tilde A\ \widehat{\cdot}\ \tilde B \right)\, \widehat{+}\, \left(\tilde A\ \widehat{\cdot}\ \tilde C \right); \\
	\tilde A \bigcup \left( \tilde B \bigcap \tilde C \right) = \left(\tilde A \bigcup \tilde B \right) \bigcap \left(\tilde A \bigcup \tilde C \right); \\
	\tilde A \bigcap \left( \tilde B \bigcup \tilde C \right) = \left(\tilde A \bigcap \tilde B \right) \bigcup \left(\tilde A \bigcap \tilde C \right).
\end{gather*}

Например, пусть нечёткое множество $\tilde A$ задано на носителе $\displaystyle X=\left[ 3;8 \right]\subset N$ совокупностью пар $\displaystyle \tilde{A}=\frac{0,7}{3}+\frac{0,9}{5}+\frac{1}{6}+\frac{0,2}{8}$. 
В соответствие с определением дополнения нечёткого множества~\eqref{eq:fuzzy-minus}
\begin{equation*}
  \bar{\tilde{A}}=\frac{0,3}{3}+\frac{1}{4}+\frac{0,1}{5}+\frac{0}{6}+\frac{1}{7}+\frac{0,8}{8}
\end{equation*}
Пересечение множеств $\tilde A$ и $\overline{\tilde A}$ является непустым, т.\,к.
\begin{gather*}
  \tilde{A}\bigcap \bar{\tilde A}=\frac{\min \left( 0,7;0,3 \right)}{3}+\frac{\min \left( 1;0 \right)}{4}+\frac{\min \left( 0,9;0,1 \right)}{5}+\frac{\min (1;0)}{6}+\\
  +\frac{\min \left( 0;1 \right)}{7}+\frac{\min \left( 0,2;0,8 \right)}{8}=\frac{0,3}{3}+\frac{0,1}{5}+\frac{0,2}{8} 
\end{gather*}
Аналогично, их~объединение не~даёт множества-носителя $X$:
\begin{gather*}
  \tilde{A}\bigcup \bar{\tilde{A}}=\frac{\max \left( 0,7;0,3 \right)}{3}+\frac{\max \left( 0;1 \right)}{4}+\frac{\max \left( 0,9;0,1 \right)}{5}+\frac{\max \left( 1;0 \right)}{6}+{}\\
  {}+\frac{\max \left( 0;1 \right)}{7}+\frac{\max \left( 0,2;0,8 \right)}{8}=\frac{0,7}{3}+\frac{1}{4}+\frac{0,9}{5}+\frac{1}{6}+\frac{1}{7}+\frac{0,8}{8} 
\end{gather*}

Ещё одна фундаментальная операция, которая переносится с чётких множеств на нечёткие~--- прямое (декартово) произведение. Классическое определение этой операции дано в~\cite{Zadeh_LingVar}, а в~\cite{Shtovba, Lipetsk} с~помощью прямого произведения также вводится понятие нечёткого отношения.
\begin{mydef}
  Пусть даны нечёткие множества $\tilde A_i \subseteq X_i,\ i=\overline{1, N}$. Декартово произведение нечётких подмножеств $\tilde A_i$ определяется как нечёткое подмножество множества $X_1 \times X_2 \times \cdots \times X_N$ с~функцией принадлежности
  \begin{equation}
    \label{eq:fuzzy-decartes}
    \mu_{\tilde A_1 \times \cdots \times \tilde A_N} \left( x_1, \cdots, x_N \right) = \mathop{ \min} \left[ \mu_{\tilde A_1}\left( x_1\right), \cdots, \mu_{\tilde A_N}\left(x_N \right) \right].
  \end{equation}
\end{mydef}
\begin{mydef}
  Нечётким отношением $\tilde R$ на множествах $X_i,\ i=\overline{1,N}$ называется нечёткое подмножество декартова произведения $X_1 \times X_2 \times \ldots \times X_N$ с функцией принадлежности $\mu_{\tilde R}\left(x_1, x_2, \ldots, x_N \right)$, которая показывается степень выполнения отношения между элементами $x_1, x_2, \ldots, x_N$.
\end{mydef}

Задание N-арного нечёткого отношения состоит в~указании значений функции принадлежности для всех кортежей вида $\left(x_1, x_2, \ldots, x_N \right)$. Поскольку нечёткие отношения являются нечёткими множествами, над ними определены все вышеупомянутые операции~\eqref{eq:fuzzy-cross}--\eqref{eq:fuzzy-union-summ}, дополненные также операцией максиминной композиции отношений. Также для нечётких бинарных отношений справедливы те~же~свойства (рефлексивность, антирефлексивность, симметричность, антисимметричность, транзитивность), что~и для~обычных~\cite{Shtovba, Orlovskiy}.

Принцип обобщения Заде является основополагающим в~теории нечётких множеств и~позволяет перенести различные математические операции с~чётких на~нечёткие множества. Его суть состоит в следующем. Пусть дано биективное отображение $f:X\to Y$ из чёткого множества $X$ в чёткое множество $Y$. Пусть $\tilde{A}\subseteq X$ – нечёткое подмножество, имеющее вид
\begin{equation}
\label{eq:a-set-discrete-zadeh}
	\tilde A=\sum\limits_{n} \frac{\mu_{\tilde A}\left( x_i \right)}{x_i}
\end{equation}
В этом случае генерируемое отображением $f$ нечёткое множество $\tilde{B}\subseteq Y$ имеет вид
\begin{equation}
\label{eq:b-set-discrete-zadeh}
	\tilde{B}=f\left( {\tilde{A}} \right)=\sum\limits_{n}{\frac{{{\mu }_{{\tilde{A}}}}\left( {{x}_{i}} \right)}{f\left( {{x}_{i}} \right)}}
\end{equation}

Если отображение $f$ не является взаимно однозначным, то~степень принадлежности элемента к нечёткому множеству $\tilde B$ равна максимальной степени принадлежности среди всех элементов исходного множества $X$, которые отображаются в один и тот же $y\in Y$. В этому случае выражение~\eqref{eq:b-set-discrete-zadeh} принимает вид
\begin{equation}
\label{eq:b-set-zadeh}
	\tilde B=f\left( \tilde A \right)=\sum\limits_{n}{\frac{\underset{j=\overline{1,n}:\ f\left( {{x}_{i}} \right)=f\left( x_j \right)}{\mathop{\sup }}\,\left( {{\mu }_{{\tilde{A}}}}\left( x_j \right) \right)}{f\left( x_i \right)}}
\end{equation}

Однако наиболее универсальной и широко употребляемой является следующая формулировка принципа обобщения, данная в~\cite{Rutkovskaya, Yakhyaeva}. 
\begin{mydef}
Пусть $f:X\to Y$~--- чёткое отображение, в котором $X=X_1 \times X_2 \times \cdots \times X_N$, а $\tilde A_i \subseteq X_i;\ i=\overline{1,N}$~--- нечёткие множества. В этом случае нечёткое множество $\tilde{B}=f\left( {{{\tilde{A}}}_{1}},\cdots,{{{\tilde{A}}}_{N}} \right)$ имеет вид
\begin{equation}
\label{eq:zadeh-classic-setb}
	\tilde{B}=\left\{ \left( y; \mu_{\tilde B} \left( y \right) \right)\left| y=f\left( x_1, x_2, \dots, x_N \right),\ x_i \in X\ \forall i=\overline{1,N} \right. \right\}
\end{equation}
а~его~функция принадлежности равна
\begin{equation}
\label{eq:zadeh-classic-mfunction}
	\mu_{\tilde B} \left( y \right)=\underset{
		\begin{smallmatrix} 
			 y=f\left( x_1, \cdots, x_N \right) \\ 
			 {{x}_{i}}\in supp\left( {{A}_{i}} \right) 
		\end{smallmatrix}}
	{\mathop{\sup }} {} \underset{i=1\cdots N}{\mathop{\min }} {}\, 
	\left\{ \mu_{\tilde {A_i}} \left( x_i \right ) \right \}
\end{equation}
\end{mydef}
Используя принцип обобщения, вводимый формулами~\eqref{eq:zadeh-classic-setb}--\eqref{eq:zadeh-classic-mfunction}, можно переносить действие известных арифметических операций на~нечёткие множества, а также определять операцию композиции нечётких отношений. Особый интерес представляет перенос арифметических операций на~нечёткие подмножества множества действительных чисел $\mathbb{R}$, иначе называемые нечёткими величинами.

\subsection{Нечёткие числа}

Рассмотрим важные определения и свойства, касающиеся нечётких чисел, которые вводятся в~работах~\cite{Rutkovskaya, Pegat, Borisov_Alexeev_Msk, Pospelov} и~широко используются в дальнейших параграфах.
\begin{mydef}
Нечёткое число~--- разновидность нечёткой величины, функция принадлежности которой $\mu_{\tilde A}\left( x \right):R\to \left[ 0;1 \right]$ обладает следующими свойствами:
\begin{itemize}
	\item кусочная непрерывность;
	\item выпуклость
	\begin{gather}
			 \forall x_1, x_2 \in \mathbb{R}; \forall \gamma \in \left[ 0;1 \right] \notag \\
	 		\label{eq:convex-set}
			 \mu_{\tilde A}\left( \gamma x_1+\left( 1-\gamma  \right)x_2 \right)\geqslant \min \left\{ \mu_{\tilde A}\left( x_1 \right),\mu_{\tilde A}\left( x_2 \right) \right\};
	\end{gather}
	\item нормальность
	\begin{equation}
		\label{eq:normal-set}
		\underset{x\in \mathbb{R}}{\mathop {\sup}}{}\, \left( \mu_{\tilde A} \left( x \right) \right)=1.
	\end{equation}
\end{itemize}
\end{mydef}

Нечёткие числа представляют огромный интерес с~точки зрения практической применимости именно ввиду непрерывности функций принадлежности. В~\cite{Zimmermann} упоминаются также нечёткие величины с дискретной функцией принадлежности, при выполнении операций над которыми возникают проблемы, показанные в~\cite{Rutkovskaya}: результатом арифметических операций, выполненных над произвольными нечёткими величинами согласно принципу обобщения Заде, далеко не всегда будет являться нечёткое число. 

Например, пусть даны две нечёткие величины $\displaystyle \tilde{A}=\frac{0,5}{2}+\frac{1}{3}+\frac{0,3}{4}$ и $\displaystyle \tilde{B}=\frac{0,7}{3}+\frac{0,9}{5}+\frac{0,6}{6}$. Найдём их произведение согласно принципу обобщения~\eqref{eq:zadeh-classic-setb}:
\begin{gather*}
	\tilde{A}\cdot \tilde{B}=\frac{\min \left( 0,5;0,7 \right)}{6}+\frac{\min \left( 0,5;0,9 \right)}{10}+\frac{\max \left\{\min \left( 0,5;0,6 \right),\min \left( 0,3;0,7 \right) \right\}}{12}+{} \\
	{} +\frac{\min \left( 1;0,7 \right)}{9}+\frac{\min \left( 1;0,9 \right)}{15}+\frac{\min \left( 1;0,6 \right)}{18}+\frac{\min \left( 0,3;0,9 \right)}{20}+\frac{\min \left( 0,3;0,6 \right)}{24}={} \\
	{}=\frac{0,5}{6}+\frac{0,7}{9}+\frac{0,5}{10}+\frac{0,5}{12}+\frac{0,9}{15}+\frac{0,6}{18}+\frac{0,3}{20}+\frac{0,3}{24}.
\end{gather*}

Данный пример иллюстрирует, что в~результате умножения нечётких чисел с~дискретной функцией~принадлежности может получиться субнормальное нечёткое множество. Ввиду этого предпочтительнее использовать нечёткие числа, поскольку принадлежность результата арифметических действий классу нечётких чисел гарантируется описанной в~\cite{Rutkovskaya} теоремой Дюбуа и~Прейда
\begin{theorem}
\label{th:dubois-prade}
(Дюбуа и~Прейда). Если~два нечётких числа имеют непрерывные функции принадлежности, то~результатом арифметических операций над ними будут нечёткие числа.
\end{theorem}

Арифметические операции над нечёткими числами в~общем случае требуют проведения достаточно сложных вычислений. Дюбуа и~Прейд в своей работе~\cite{Dubois_Prade} предложили частную форму представления нечётких чисел с помощью двух функций с определёнными свойствами, которая позволяет существенно упростить нечёткие арифметики. 
\begin{mydef}
Нечёткие числа $LR$-типа~--– разновидность нечётких чисел, функция принадлежности которых задаётся с~помощью двух функций $L(x):\mathbb{R} \to \mathbb{R}$, $R(x):\mathbb{R} \to \mathbb{R}$ таких, что 
\begin{gather*}
	L\left( -x \right)=L\left( x \right); \\
	R\left( -x \right)=R\left( x \right); \\
	L\left( 0 \right)=R\left(  0 \right)=1.
\end{gather*}
Кроме того, $L$ и~$R$ являются невозрастающими на~интервале $\left[ 0;+\infty  \right)$~\cite{Rutkovskaya}.
\end{mydef}
Функция принадлежности нечёткого $LR$-числа выглядит следующим образом
\begin{equation}
\label{eq:membership-lr-general}
	 \mu_{\tilde A}\left( x \right)=\left\{ 
		 \begin{aligned}
			 & L\left( \frac{m-x}{a} \right);\ x \leqslant m \\ 
			 & R\left( \frac{x-m}{b} \right);\ x>m \\ 
		 \end{aligned} 
	 \right.
\end{equation}
При известной форме функции принадлежности, $LR$-числа гораздо удобнее записывать как кортеж из~трёх параметров $\tilde A = \left(m; a; b \right)$, называемых модой и левым и правым коэффициентами нечёткости соответственно.

Частным случаем нечётких чисел $LR$-типа являются треугольные числа, которые широко распространены во всевозможных математических задачах.
\begin{mydef}
Треугольным (триангулярным) нечётким числом называется $LR$-число $\tilde{A}$, задаваемое тройкой $\left\langle a,m,b \right\rangle $, с функцией принадлежности
\begin{equation}
\label{eq:membership-abm-form}
	\mu_{\tilde A}\left( x \right)=
	\left\{ 
		\begin{aligned}
			& \frac{x-m+a}{a};\ x\in \left[ m-a;m \right] \\ 
			& \frac{m+b-x}{b};\ x\in \left( m;m+b \right] \\ 
			& 0;\ \text{в остальных случаях} 
	 	\end{aligned} 
	 \right.
\end{equation}
Если правый (левый) коэффициент нечёткости треугольного числа равен нулю, то такое число, согласно~\cite{Vorontsov_PI}, будем называть числом $LL$ $\left( RR \right)$-типа.
\end{mydef}

Обозначим точки пересечения левой и правой ветвей функции принадлежности с осью $Ox$ как $x_{{\tilde{A}}}^{L}$ и $x_{{\tilde{A}}}^{R}$ соответственно. В~\cite{Borisov_Krumberg_Riga} эти~точки называются границами функции принадлежности.~Тогда
\begin{equation}
\label{eq:xlxr-to-abm}
	\left[ 
		\begin{aligned}
			 & x_{\tilde A}^L=m-a \\ 
			 & x_{\tilde A}^R=m+b 
		 \end{aligned} 
 	\right.
\end{equation}
и~функция принадлежности~\eqref{eq:membership-abm-form} с~учётом~\eqref{eq:xlxr-to-abm} будет выглядеть следующим образом
\begin{equation}
\label{eq:membership-xlxr-form}
	\mu_{\tilde A}\left( x \right)=\left\{ 
	\begin{aligned}
		& \frac{x-x_{{\tilde{A}}}^{L}}{m-x_{{\tilde{A}}}^{L}};\ x\in \left[ x_{{\tilde{A}}}^{L};m \right] \\ 
		& \frac{x-x_{{\tilde{A}}}^{R}}{m-x_{{\tilde{A}}}^{R}};\ x\in \left( m;x_{{\tilde{A}}}^{R} \right] \\ 
		& 0;\ \text{в остальных случаях} 
	\end{aligned} 
	\right.
\end{equation}
а само число можно записать в виде тройки $\left\langle x_{{\tilde{A}}}^{L},m,x_{{\tilde{A}}}^{R} \right\rangle$.

Согласно теореме о~декомпозиции~\eqref{eq:alpha-cut-theorem}, для~нечёткого треугольного числа $\tilde{A}$ также можно использовать представление в~виде совокупности чётких $\alpha$-интервалов $X_\alpha$, границы которых определяются как~функции параметра $\alpha \in \left[ 0;1 \right]$:
\begin{equation}
\label{eq:membership-alphacut-form}
	\left[ 
		\begin{aligned}
			& x^L(\alpha )=m-a+a\alpha  \\ 
			& x^R(\alpha )=m+b-b\alpha
		 \end{aligned}
	\right.
\end{equation}
Представление числа $\tilde{A}$ в~виде объединения интервалов $\left[ {{x}^{L}}\left( \alpha  \right);{{x}^{R}}\left( \alpha  \right) \right]$, концы которых определяются согласно формулам~\eqref{eq:membership-alphacut-form}, позволяет сохранить неопределённость в~интервальной форме.

Используя введённый ранее принцип обобщения Заде, можно расширить четыре арифметические действия на множество нечётких чисел. Пусть~$\tilde{A}$ и~$\tilde{B}$~--- нечёткие числа с функциями принадлежности $\mu_{\tilde A}\left( x \right)$ и $\mu_{\tilde B}\left( x \right)$ соответственно, а $g:\mathbb{R}\times \mathbb{R}\to \mathbb{R}$~--- некоторая функция двух действительных переменных. Согласно принципу обобщения Заде, результат $\tilde{C}=g\left( \tilde{A},\tilde{B} \right)$ будет определяться следующей функцией принадлежности~\cite{Borisov_Alexeev_Msk, Pospelov, Rutkovskaya, Yakhyaeva}:
\begin{equation}
\label{eq:zadeh-algebra}
	\begin{matrix}
		\mu_{\tilde C}\left( x \right)=\underset{g\left( a,b \right)=x}{\mathop{\sup }}{}\,\min \left\{ {{\mu }_{{\tilde{A}}}}\left( a \right);{{\mu }_{{\tilde{B}}}}\left( b \right) \right\} \\ 
		  a\in supp\left( {\tilde{A}} \right),\ b\in supp\left( {\tilde{B}} \right) 
	\end{matrix}
\end{equation}

Если в качестве $g$ берётся одна из арифметических операций, то~\eqref{eq:zadeh-algebra} определяет результат арифметической операции над~нечёткими~числами:
\begin{gather*}
		\tilde{A}+\tilde{B} \leftrightarrow \underset{x+y}{\mathop{\sup }}\,\min \left( {{\mu }_{{\tilde{A}}}}\left( x \right),{{\mu }_{{\tilde{B}}}}\left( y \right) \right) \\
		\tilde{A}-\tilde{B} \leftrightarrow \underset{x-y}{\mathop{\sup }}\,\min \left( {{\mu }_{{\tilde{A}}}}\left( x \right),{{\mu }_{{\tilde{B}}}}\left( y \right) \right) \\
		\tilde{A}\cdot \tilde{B} \leftrightarrow \underset{x\cdot y}{\mathop{\sup }}\,\min \left( {{\mu }_{{\tilde{A}}}}\left( x \right),{{\mu }_{{\tilde{B}}}}\left( y \right) \right) \\
		\tilde{A}/\tilde{B} \leftrightarrow \underset{x/y}{\mathop{\sup }}\,\min \left( \mu_{\tilde A}\left( x \right), \mu_{\tilde B} \left( y \right) \right) 
\end{gather*}

Как отмечается в~\cite{Pospelov, Borisov_Alexeev_Msk, Yakhyaeva}, [146], операции сложения и умножения, вводимые с~помощью~\eqref{eq:zadeh-algebra}, обладают следующими свойствами:
\begin{enumerate}
	\item коммутативность $\tilde{A}+\tilde{B}=\tilde{B}+\tilde{A}$, $\tilde{A}\cdot \tilde{B}=\tilde{B}\cdot \tilde{A}$;
	\item ассоциативность $\tilde{A}+\left( \tilde{B}+\tilde{C} \right)=\left( \tilde{A}+\tilde{B} \right)+\tilde{C}$; $\tilde{A}\cdot \left( \tilde{B}\cdot \tilde{C} \right)=\left( \tilde{A}\cdot \tilde{B} \right)\cdot \tilde{C}$;
	\item дистрибутивность умножения относительно сложения при совпадении знаков $\tilde{B}$ и $\tilde{C}$: $\tilde{A}\cdot \left( \tilde{B}+\tilde{C} \right)=\tilde{A}\cdot \tilde{B}+\tilde{A}\cdot \tilde{C}$.
\end{enumerate}

Сравнение нечётких чисел производится как сравнение двух нечётких подмножеств множества $\mathbb{R}$. Очевидно, числа $\tilde A$ и $\tilde B$ считаются равными тогда и только тогда, когда их функции принадлежности совпадают~\cite{Ibragimov}:
\begin{equation}
\label{eq:fuzzy-membership-equality}
  \tilde A = \tilde B \Leftrightarrow \mu_{\tilde A} \left( x \right) = \mu_{\tilde B} \left( x \right).
\end{equation}

Также тривиален случай, когда носители чисел не пересекаются~--- то нечёткое число, носитель которого расположен правее по оси действительных чисел, будет больше~\cite{Ibragimov}:
\begin{equation}
\label{eq:fuzzy-inequality}
  supp \left( \tilde A \right) \bigcap supp \left( \tilde B \right) = \varnothing,\ \forall x \in \mathbb{R}\ \mu_{\tilde A} \left( x \right) \leqslant \mu_{\tilde B} \left( x \right) \Rightarrow \tilde A < \tilde B.
\end{equation}

В остальных случаях, не описываемых формулами~\eqref{eq:fuzzy-membership-equality} и~\eqref{eq:fuzzy-inequality}, числа либо считаются несравнимыми, и в случае близости функций принадлежности друг~к~другу вычисляется степень равенства множеств~\cite{Rutkovskaya, Siler_Buckley}
\begin{gather*}
  E\left(\tilde A = \tilde B \right)=1 - \underset{x \in T} {\mathop{\max}} \left| \mu_{\tilde A} \left( x \right) - \mu_{\tilde B} \left( x \right) \right|; \\
  T = \left \{ x \in X:\ \mu_{\tilde A} \left( x \right) \neq \mu_{\tilde B} \left( x \right) \right \},
\end{gather*}
либо производится их сравнение на основании методов одного из семейств, описанных в~\cite{Vorontsov_Compare, Cheng_Comparison}, суть которых в основном сводится к вычислению некоторой оценочной функции и упорядочиванию чисел на основании её значений.

Классическим вариантом такой оценочной функции является индекс ранжирования. Для нечётких чисел, удовлетворяющих условию $supp \left( \tilde A \right) \bigcap supp \left( \tilde B \right) \neq \varnothing$, вычисляется индекс ражирования $H\left(\tilde A, \tilde B \right)$, конкретный вид которого зависит от вида сравниваемых чисел~\cite{Borisov_Alexeev_Msk, Skorokhod}. Значение индекса позволяет вычислить степень, с~которой одно из чисел больше/меньше другого, а результатом сравнения является нечёткое подмножество множества \textit{ \{да, нет\}}. Таким образом, результат сравнения нечётких чисел также может быть нечётким~\cite{Siler_Buckley}.