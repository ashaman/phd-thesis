%Модели с нечеткими отношениями, нечеткими функциями, нечеткими параметрами, нечеткими исходными переменными. Что представляет собой решение в этих случаях.
%Два направления решения задач: нечеткий логический вывод с использованием логических операций и математические модели задач с нечеткими параметрами и необходимостью арифметических операций и операций сравнения нечетких чисел. Преимущества второго направления. Экспертный характер задания нечеткой информации.

Модели статических и динамических систем, построение, использование и анализ которых базируется на положениях теории нечётких множеств, называют нечёткими моделями~\cite{Borisov_Fedulov}. Многие исследователи отмечают тот факт, что нечёткие модели могут рассматриваться как обобщение интервальных, которые, в свою очередь, обобщают известные чёткие модели. К примеру, рассмотрим некоторую функцию $y=f \left(x \right)$, которую с точки зрения дискретной математики можно представить как отношение на декартовом произведении $X \times Y$. Вне зависимости от типа модели, вычисление выходного значения $y$ для заданного значения входного параметра $x$ происходит в три этапа~\cite{Borisov_Fedulov}:
\begin{itemize}
	\item задание значения входной переменной $x \in X$;
	\item нахождение пересечения $x$ с отношением $f$;
	\item проецирование пересечения $x$ и $f$ на $Y$.
\end{itemize}

Однако результаты во всех случаях различны по своему роду. На рис. 1 приведены результаты вычислений для чёткой, интервальной и нечёткой функций при различных видах аргументов. 

На основании этого примера можно выделить взаимосвязи между описаниями и переменными чётких и нечётких моделей, а также математические методы, которые применимы для описания моделей. Выделенные взаимосвязи изображены в таблице~\ref{t:fuzzy-modeling-approaches}.
\begin{table}[H]
\caption{Взаимосвязи между описаниями и переменными чётких и нечётких моделей}
\label{t:fuzzy-modeling-approaches}
\begin{center}
\begin{tabularx}{\textwidth}{|p{0.15\linewidth}|p{0.15\linewidth}|p{0.15\linewidth}|X|}
	\hline
		\centering \textit{Описание модели} & \centering \textit{Входные данные} & \centering \textit{Выходные данные} & \centering \textit{Математические методы} \tabularnewline	\hline
	\hline
		Чёткое & Чёткие & Чёткие & Функциональный анализ, линейная алгебра и т.д. \tabularnewline
	\hline
		Чёткое & Нечёткие & Нечёткие & Принцип обобщения Заде \tabularnewline
	\hline
		Чёткое & Нечёткие & Чёткие & Нечёткие модели и вычисления \tabularnewline
	\hline
		Нечёткое & Чёткие/\allowbreak Нечёткие & Нечёткое & Нечёткие модели и вычисления \tabularnewline
  \hline
\end{tabularx}
\end{center}
\end{table}

Очевидно, что нечёткое моделирование не подменяет собой другие методологии моделирования сложных систем, в которых существенные зависимости выражены настолько хорошо, что они могут быть выражены в числах или символах, получающих в итоге численные оценки~\cite{Borisov_Fedulov}. Нечёткие модели скорее представляют необходимый инструмент для исследования как отдельных аспектов, так и системы в целом на различных этапах её анализа в случае доминирования качественных элементов над количественными. Об этом же говорится и в~\cite{Kaufmann, Borisov_Alexeev_Msk} ~--- теория нечётких множеств не призвана конкурировать с теорией вероятности и статистическими методами, она заполняет пробел в области структуризованной неопределённости там, где нельзя корректно применять статистику и вероятности ввиду неизвестности распределения величин или малого размера статистической выборки.

%Конечно, за удобство применения нечётких моделей приходится платить, поскольку предлагаемые теорией решения, основанные на нечёткой информации, несут на себе печать нечёткости. Они могут рассматриваться лишь как рекомендации по выбору для лица, принимающего решения, требуя от него выбора одного из предлагаемых вариантов. Тем не менее, даже этот факт можно рассматривать как достоинство теории~--- он показывает, как увеличение информированности ЛПР сказывается на достоверности и правильности принимаемых решений.

В~\cite{Borisov_Fedulov} предложена оригинальная классификация подходов к созданию нечётких моделей в зависимости от того, в какой момент моделирования используется теория нечётких множеств, а также соответствующие ей сферы применения нечётких моделей. Рассмотрим данную классификацию подробнее: нечёткость может применяться
\begin{enumerate}
	\item при описании системы~--- речь идёт об информационной неопределённости~\cite{Pospelov, Borisov_Alexeev_Msk}. Система описывается моделями нечёткой логики: продукционными/реляционными/функциональными. Обычно такой подход применяется, когда имеются неполные или неопределённые знания об исследуемом объекте, а их дополнение является либо невозможным, либо нецелесообразным, либо значительная часть информации об объекте является качественной и не~выражается с~помощью известных математических зависимостей, но может быть описана системой предпочтений на~естественном языке в форме правил <<если-то>>;
	\item при задании параметров системы~--- в традиционной, чёткой модели системы используются нечёткие параметры (например, нечёткие коэффициенты обычных алгебраических или дифференциальных уравнений). Данный подход оправдывает себя в ситуации полной определённости модели, когда необходимо учесть присущую параметрам неопределённость, а традиционный вероятностный подход неприменим ввиду того, что неоднозначность параметров не является физической согласно классификации, используемой в~\cite{Borisov_Alexeev_Msk}. В таких ситуациях приходится прибегать к~услугам экспертов, которые выражают своё мнение в виде качественных оценок, а принадлежность объектов задаётся с помощью лингвистических операторов («много», «мало», «около» и т.п.);
	\item нечёткость при~задании входов, выходов и состояний системы~--- в традиционной модели системы с чёткими или нечёткими параметрами могут применяться нечёткие переменные. Этот подход в основном применяется при идентификации динамических или нелинейных систем на основе их входных и выходных параметров и позволяет при наличии обучающей выборки аппроксимировать искомые функции или измеренные данные с наперёд заданной точностью;
	\item комбинированные модели~--- создаются на основе совмещения двух или более подходов.
\end{enumerate}

Если рассматривать описанную выше классификацию подходов к синтезу нечётких моделей через призму выбора, который являются неотъемлемой частью моделирования как целенаправленного процесса, и языков его описания, то можно заметить, как модель и используемый в ней язык выбора проецируются на~два основных раздела современной нечёткой математики~--- нечёткий логический вывод и алгебры нечётких множеств и чисел. К настоящему моменту сложилось три основных языка описания выбора~--- язык функций выбора, язык бинарных отношений и критериальный язык~\cite{Choice_Languages}, которые позволяют говорить об одном и том же объекте или явлении с разной степенью общности. Два последних языка~--- язык бинарных отношений и критериальный язык~--- достаточно хорошо изучены и отражены в рамках теории нечётких множеств.

Язык бинарных отношений является более общим и основывается на том факте, что в~реальности дать объективную оценку той или~иной альтернативе затруднительно или невозможно, однако, при~рассмотрении альтернатив в~паре, можно указать более или~менее предпочтительную. Основные предположения этого~языка выбора сводятся к~следующим:
\begin{itemize}
	\item отдельная альтернатива не~оценивается;
	\item для~каждой пары альтернатив некоторым образом можно установить, что~одна из~них предпочтительнее другой, либо они равноценны или несравнимы;
	\item отношение предпочтения внутри любой пары альтернатив не~зависит от~остальных альтернатив, предъявленных к~выбору.
\end{itemize}

Нечёткие модели первого типа, в которых нечёткость присутствует на этапе описания системы~\cite{Choice_Languages}, как раз используют язык бинарных отношений. В~рамках нечёткой математики, этот язык проецируется на нечёткий логический вывод и основанное на нём нечёткое управление. Основополагающими для логического вывода являются понятия нечёткого отношения, лингвистической переменной и нечёткой импликации, на которой основаны правила логического вывода.

\begin{mydef}
Лингвистической переменной называется переменная, значения которой представляют слова или суждения на естественном языке. С точки зрения нечёткой математики, это кортеж $\left\lbrace \beta, T, X, G, M \right\rbrace$, где $\beta$~--- название нечёткой переменной; $T$~--- базовое терм-множество лингвистической переменной, каждый элемент которого (терм) представляется как нечеткое множество на универсальном множестве $X$; $G$~--- cинтаксические правила, часто задаваемые в виде грамматики, для порождения названий термов; $М$~--- семантические правила, задающие функции принадлежности нечетких термов, порожденных синтаксическими правилами $G$.
\end{mydef}

Нечёткая импликация является нечётким отношением $\tilde R \subseteq X \times Y$ с функцией принадлежности, задаваемой одним из правил

В нечётком логическом выводе и управлении используются 

 (сюда относятся известные алгоритмы Такаги--Сугено и Мамдани), однако этот раздел нечёткой математики лежит за пределами данной диссертации. У него есть свои преимущества и недостатки (описать)
 
Достоинства и недостатки, а также рекомендации по применению нечёткого логического вывода и нечёткого управления хорошо описаны в~\cite{Bauer_Winkler}. Согласно~\cite{Bauer_Winkler}, использование моделей первого типа рекомендуется для моделирования очень сложных процессов, когда не существует простой математической модели, для нелинейных процессов высоких порядков и для~обработки лингвистически сформулированных экспертных знаний. Эти же модели не рекомендуется применять, если приемлемый результат может быть получен с~помощью общей теории управления, либо существует формализованная и адекватная математическая модель, либо проблема не разрешима методами современной математики. Нечёткому логическому выводу и~управлению посвящено множество книг и~публикаций, однако подробное рассмотрение моделей первого вида выходит за~рамки данной диссертации.

Второй язык, более простой и~узкий, однако и~более изученный~--- критериальный. Он~основывается на~предположении, что~каждую отдельно взятую альтернативу возможно оценить конкретным числом, называемым значением критерия. Сравнение альтернатив в~таком случае сводится к~сравнению соответствующих~им числовых значений. Пусть $x\in X$ - некоторая альтернатива из~множества альтернатив $X$. Критерием будем называть функцию $q\left( x \right);\ x\in X$, обладающую тем~свойством, что~если альтернатива ${x_1}$ предпочтительнее ${x_2}$, то $q\left( x_1 \right)>q\left( x_2 \right)$ и~наоборот. Естественно считать, что наилучшей альтернативой ${{x}^{*}}$ считается~та, значение критерия которой максимально:
\begin{equation*}
  x^{*}=\arg \underset{i}{\mathop{\max }}\,\left\{ q\left( x_i \right) \right\}
\end{equation*}
Задача отыскания $x^{*}$, достаточно простая по~постановке, часто оказывается весьма сложной в решении, поскольку зависит от характера множества $X$ и критерия $q\left( x \right)$. 

%Частным случаем моделей с лингвистической неопределённостью являются модели, использующие нечёткие числа в качестве параметров. Они удобны тем, что это хорошо проработанные и испытанные временем модели. Основная проблема – создание алгебраических систем для нечётких чисел либо численных методов решения таких моделей.

Нечёткие модели второго и третьего типа, в которых нечёткими являются либо их параметры, либо состояния и входные и выходные данные, описываются с помощью критериального языка. Этот язык в~нечёткой~математике соответствует т.\,н. <<мягким вычислениям>>~--- нечётким множествам, нечётким числам и определённым на них алгебрам. Как уже упоминалось ранее, описание с помощью нечётких множеств имеет существенные преимущества перед языком теории вероятностей в том случае, когда имеется лингвистическая неоднозначность в смысле полисемии~\cite{Borisov_Alexeev_Msk}, и~оценки получаются c помощью опроса экспертов. Известно, что люди в большинстве своём неправильно оценивают вероятности (особенно большие и малые), поэтому требовать от экспертов, коими обычно являются специалисты в конкретных предметных областях, а не математики, оценок в форме распределения вероятностей зачастую невозможно~\cite{Gubko}. Кроме того, описание в форме нечётких множеств гораздо менее требовательно к квалификации экспертов и зачастую гораздо точнее отражает суть исследуемого объекта или явления.

Данная диссертация посвящена исследованию способов представления нечёткости в моделях второго типа, в которых отношения и функции чёткие, а параметры заданы нечёткими числами.