\subsection{Классификация нечётких моделей}


Модели с нечеткими отношениями, нечеткими функциями, нечеткими параметрами, нечеткими исходными переменными. Что представляет собой решение в этих случаях.

Модели статических и динамических систем, построение, использование и анализ которых базируется на положениях теории нечётких множеств, называют нечёткими моделями \cite{Borisov_Fedulov}. Многие исследователи отмечают тот факт, что нечёткие модели могут рассматриваться как обобщение интервальных, которые, в свою очередь, обобщают известные чёткие модели. К примеру, рассмотрим некоторую функцию $y=f \left(x \right)$, которую с точки зрения дискретной математики можно представить как отношение на декартовом произведении $X \times Y$. Вне зависимости от типа модели, вычисление выходного значения $y$ для заданного значения входного параметра $x$ происходит в три этапа \cite{Borisov_Fedulov}:
\begin{itemize}
	\item задание значения входной переменной $x \in X$;
	\item нахождение пересечения $x$ с отношением $f$;
	\item проецирование пересечения $x$ и $f$ на $Y$.
\end{itemize}

Однако результаты во всех случаях различны по своему роду. На рис. 1 приведены результаты вычислений для чёткой, интервальной и нечёткой функций при различных видах аргументов. 

На основании этого примера можно выделить взаимосвязи между описаниями и переменными чётких и нечётких моделей, а также математические методы, которые применимы для описания моделей. Всё это описывается в таблице 1.

Нечёткое моделирование не подменяет собой другие методологии моделирования сложных систем, в которых существенные зависимости выражены настолько хорошо, что они могут быть выражены в числах или символах, получающих в итоге численные оценки. Нечёткие модели скорее представляют необходимый инструмент для исследования как отдельных аспектов, так и системы в целом на различных этапах её анализа в случае доминирования качественных элементов над количественными. Об этом же говорится и в~\cite{Kaufmann}~--- теория нечётких множеств не призвана конкурировать с теорией вероятности и статистическими методами, она заполняет пробел в области структуртизованной неопределённости там, где нельзя корректно применять статистику и вероятности.

В \cite{Borisov_Fedulov} предложена следующая классификация подходов к созданию нечётких моделей в зависимости от того, в какой момент моделирования используется теория нечётких множеств:
\begin{enumerate}
	\item при описании системы~--- система описывается моделями нечёткой логики: продукционными/реляционными/функциональными. Наиболее общими здесь являются продукционные модели, которые обычно ассоциируют с системами нечёткого логического вывода;
	\item при задании параметров системы~--- в традиционной, чёткой модели системы могут использоваться нечёткие параметры (например, нечёткие коэффициенты обычных алгебраических или дифференциальных уравнений);
	\item при задании входов, выходов и состояний системы~--- в традиционной модели системы с чёткими параметрами могут применяться нечёткие переменные
	\item комбинированные модели~--- создаются на основе совмещения двух или более подходов.
\end{enumerate}

Для каждого из этих подходов существует своя сфера применения

Два направления решения задач: нечеткий логический вывод с использованием логических операций и математические модели задач с нечеткими параметрами и необходимостью арифметических операций и операций сравнения нечетких чисел. Преимущества второго направления. Экспертный характер задания нечеткой информации.


Рассмотрим классификацию моделей с точки зрения выбора. Выбор является неотъемлемой частью моделирования как целенаправленного процесса.

%Выбор является действием, которое придаёт всей деятельности целенаправленность. С математической точки зрения, проблема выбора нетривиальна и допускает существенно различающиеся постановки задач. К настоящему моменту сложилось три основных языка описания выбора~--- язык функций выбора, язык бинарных отношений и критериальный язык \cite{Choice_Languages}, два последних из~которых достаточно хорошо изучены. Рассмотрим, как~теория нечётких множеств соотносится с~существующими на~сегодняшний день языками выбора.

С точки зрения современных языков выбора, описываемых в \cite{Choice_Languages}, используемые в них методы относятся к языку бинарных отношений. Его общность основана на~том факте, что в~реальности дать объективную оценку той или~иной альтернативе затруднительно или невозможно, однако, при~рассмотрении альтернатив в~паре, можно указать более или~менее предпочтительную. Основные предположения этого~языка выбора сводятся к~следующим:

\begin{itemize}
	\item отдельная альтернатива не~оценивается;
	\item для~каждой пары альтернатив некоторым образом можно установить, что~одна из~них предпочтительнее другой, либо они равноценны или несравнимы;
	\item отношение предпочтения внутри любой пары альтернатив не~зависит от~остальных альтернатив, предъявленных к~выбору.
\end{itemize}

Модели первого типа как раз описываются с помощью языка бинарных отношений. В~рамках теории нечётких множеств, этот язык проецируется на нечёткие отношения и нечёткий логический вывод. Основополагающим понятием здесь является понятие лингвистической переменной


 (сюда относятся известные алгоритмы Такаги--Сугено и Мамдани), однако этот раздел нечёткой математики лежит за пределами данной диссертации. У него есть свои преимущества и недостатки (описать)

Более простой и~узкий, однако и~более изученный язык~--- критериальный. Он~основывается на~предположении, что~каждую отдельно взятую альтернативу возможно оценить конкретным числом, называемым значением критерия. Сравнение альтернатив в~таком случае сводится к~сравнению соответствующих~им числовых значений.

Пусть $x\in X$ - некоторая альтернатива из~множества альтернатив $X$. Критерием будем называть функцию $q\left( x \right);\ x\in X$, обладающую тем~свойством, что~если альтернатива ${{x}_{1}}$ предпочтительнее ${{x}_{2}}$, то $q\left( {{x}_{1}} \right)>q\left( {{x}_{2}} \right)$ и~наоборот. Естественно считать, что наилучшей альтернативой ${{x}^{*}}$ считается~та, значение критерия которой максимально:
\[
	{{x}^{*}}=\arg \underset{i}{\mathop{\max }}\,\left\{ q\left( {{x}_{i}} \right) \right\}
\]

Задача отыскания ${{x}^{*}}$, достаточно простая по~постановке, часто оказывается весьма сложной в решении, поскольку зависит от характера множества $X$ и критерия $q\left( x \right)$. Если~рассматривать теорию нечётких множеств в~применении к критериальному языку выбора, то на~него проецируются нечёткие числа и правила работы с ними. Язык нечётких множеств имеет существенные преимущества перед языком теории вероятностей в том случае, когда оценки получаются из опроса экспертов. Известно, что люди в большинстве своём неправильно оценивают вероятности (особенно большие и малые). Поэтому требовать от экспертов, коими являются специалисты в конкретных предметных областях, а не математики, оценок в форме распределения вероятностей зачастую невозможно. Использование же полученных результатов для принятия решений можно квалифицировать как самонадеянное. Оаисани в форме нечётких множеств гораздо менее требовательно к квалификации экспертов и зачастую гораздо точнее отражает суть дела и имеющуюся у ЛПР информацию \cite{Gubko}. 



В данной диссертации основной интерес представляют модели второго рода, в которых отношения и функции чёткие, а параметры заданы нечёткими числами.

Для уточнения применяемых при моделировании и решении методов, рассмотрим взаимосвязи между описаниями модели, используемыми в них переменными и применяемыми математическими методами.


% Способы решения задач


\subsection{Требования к методам численной реализации решения задач с нечеткими параметрами и четкими отношениями}

Конечно, за удобство применения нечётких моделей приходится платить, поскольку предлагаемые теорией решения, основанные на нечёткой информации, несут на себе печать нечёткости. Они могут рассматриваться лишь как рекомендации по выбору для лица, принимающего решения, требуя от него выбора одного из предлагаемых вариантов. Тем не менее, даже этот факт можно рассматривать как достоинство теории~--- он показывает, как увеличение информированности ЛПР сказывается на достоверности и правильности принимаемых решений.

% Губко Лекции по нечёткой логике
Как мы увидели из предыдущего раздела, понятие образа нечёткого множества при нечётком отображении позволяет ЛПР вычислять нечёткую реакцию системы на нечёткие же управляющие воздействия. Тем не менее, для теории принятия решений горазо важнее обратная задача~--- найти действия, которые приводят к желаемому результату. Решить данную задачу позволяет понятие прообраза нечёткого множества, описанию которого посвящён настоящий раздел.

\begin{itemize}
	\item Ограничение роста неопределенности результатов обработки нечеткой информации.
	\item Наличие нечетких параметров не должно исключать четкие отношения модельных уравнений.
	\item Необходимость возможности представления полного порядка.
	\item Возможность использования стандартных программных средств реализации численных методов решений.
	\item Проблема устойчивости решения.
\end{itemize}