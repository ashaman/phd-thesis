Понятие алгебраической системы как совокупности алгебры и модели.

Однако в тех же источниках показывается, что не существует истинно противоположного и обратного элементов, т.\,е.~справедливы выражения
\begin{gather}
	\label{eq:zadeh-zero-nonequality}
	\tilde{A}+\left( -\tilde{A} \right)\ne 0; \\
	\label{eq:zadeh-reverse-nonequality}
	\tilde{A}\cdot {{\tilde{A}}^{-1}}\ne 1.
\end{gather}

Проблемы сравнения нечётких чисел (очень кратко)

 
Необходимым условием четкого равенства должно быть наличие групповых свойств операций над нечеткими числами. Т.е. надо подобрать такую модель представления нечеткой числовой информации, которая при сохранении основных исходных свойств экспертных оценок обеспечивает возможность построения алгебры нечетких чисел эквивалентной  полю действительных чисел. Это обеспечит ограничение роста неопределенности.

Полный порядок должен обеспечиваться выбором подходящей модели в рамках соответствующей алгебраической системы.

Надо выбрать такие условия устойчивости, которые можно было бы использовать непосредственно в алгоритме решения. Более того, модель представления нечеткой числовой информации должна позволять управлять устойчивостью

Ввиду~\eqref{eq:zadeh-zero-nonequality} и~\eqref{eq:zadeh-reverse-nonequality}, алгебра нечётких чисел, основанная на операциях, которые определены через принцип обобщения, не является ни полем, ни кольцом. Авторы~\cite{Rutkovskaya} отмечают, что отсутствие нуля и единицы не позволяет решать уравнения и системы уравнений с нечёткими коэффициентами с помощью метода исключения. Это означает, что нарушается одно из важных


\subsection{Требования к методам численной реализации решения задач с нечеткими параметрами и четкими отношениями}

% Губко Лекции по нечёткой логике
Как мы увидели из предыдущего раздела, понятие образа нечёткого множества при нечётком отображении позволяет ЛПР вычислять нечёткую реакцию системы на нечёткие же управляющие воздействия. Тем не менее, для теории принятия решений горазо важнее обратная задача~--- найти действия, которые приводят к желаемому результату. Решить данную задачу позволяет понятие прообраза нечёткого множества, описанию которого посвящён настоящий раздел.

\begin{itemize}
	\item Ограничение роста неопределенности результатов обработки нечеткой информации.
	\item Наличие нечетких параметров не должно исключать четкие отношения модельных уравнений.
	\item Необходимость возможности представления полного порядка.
	\item Возможность использования стандартных программных средств реализации численных методов решений.
	\item Проблема устойчивости решения.
\end{itemize}

\subsection{Общие алгебраические структуры}

Для~решения задач с использованием нечётких параметров необходимо уметь выполнять различные операции над нечёткими числами. В различных работах вводятся различные алгебры нечётких чисел. Однако для начала рассмотрим, как вводится понятие алгебры и как оно проецируется на множество нечётких чисел.

Предметом рассмотрения абстрактной алгебры являются произвольные множества с~заданными на~них операциями, при этом природа этих~множеств и~операций может существенно отличаться от~привычных числовых множеств и~известных операций над~числами~\cite{Bauman_DM}. Проблема работы со~сложно организованными числовыми и~нечисловыми структурами возникла в~связи с~развитием современных информационных технологий и~переходом от~обычных вычислений к~обработке сложных структур данных.

\begin{mydef}
Пусть $A$ – произвольное непустое множество, $n\in \mathbb{N}$. Любое отображение 
\begin{equation*}
	f:A^N \to A
\end{equation*}
называют $n$-арной операцией на множестве $A$.
\end{mydef}

В алгебрах наиболее важными и исследуемыми являются бинарные $\left( n=2 \right)$ операции. Если $*$~--- некая абстрактная бинарная операция, то, согласно~\cite{Bauman_DM}, она~является
\begin{itemize}
	\item коммутативной, если $\forall x,y\in A\ x*y=y*x$;
	\item ассоциативной, если $\forall x,y,z\in A\ x*\left( y*z \right)=\left( x*y \right)*z$;
	\item идемпотентной, если $\forall x\in A\ x*x=x$.
\end{itemize}

Элемент $0$ множества $A$ называют нулём относительно операции $*$, если $\forall x\in A\ 0*x=0,\ x*0=0$. Нуль в множестве $A$ единственен. В~самом деле, если~предположить существование другого нулевого элемента ${0}'$ относительно операции $*$, то,~согласно определению нуля
\begin{equation*}
	0*{0}'=0,\ {0}'*0={0}',
\end{equation*}
откуда следует равенство $0={0}'$.

Элемент 1 множества $A$ называют нейтральным относительно операции $*$, если $\forall x\in A\ 1*x=x,\ x*1=1$.
Нейтральный элемент в множестве $A$ также единственен, доказательство этого факта аналогично доказательству единственности нулевого элемента.

В~\cite{Bauman_DM, Adelson_Velskiy} даётся следующее определение алгебры.
\begin{mydef}
Алгебра считается заданной, если задано некоторое множество $D$, называемое носителем алгебры, и некоторое множество операций $\Omega $ на $D$, называемое сигнатурой данной алгебры. Алгебру можно записать как упорядоченную пару множеств $\left( D,\Omega  \right)$.
\end{mydef}

Стоит отметить, что~операции, включенные в~сигнатуру, задаются как~некоторые специальные отображения. При~этом не~оговариваются свойства, которыми операции обладают на~носителе~--– они обычно указываются дополнительно.
Кратко рассмотрим основные виды алгебр, описанные в~\cite{Bauman_DM, Adelson_Velskiy, Voevodin}. Вначале дадим определения для алгебр, сигнатура которых состоит из одной абстрактной бинарной операции $*$.

Группоидом называют любую алгебру $\left( D,* \right)$, сигнатура которой состоит из~одной бинарной операции, на~которую не~наложено никаких ограничений. Если~же операция $*$ ассоциативна, то~группоид является полугруппой. Отдельно выделяют коммутативные полугруппы~--– полугруппы, в~которых операция $*$ обладает свойством коммутативности.

Моноидом называется такая полугруппа, относительно операции которой существует нейтральный элемент. Такой элемент называется единицей моноида $\left( D,* \right)$ и~обозначается как 1. Для моноида справедливы следующие свойства:
\begin{itemize}
	\item $\forall x,y,z\in D\quad x*\left( y*z \right)=\left( x*y \right)*z$;
	\item $\forall x\in D\quad x*1=1*x=x$.
\end{itemize}

Если в моноиде $\forall x\in D\ \exists \,{{x}^{-1}}\in D$, называемый обратным, такой, что~$x*{{x}^{-1}}={{x}^{-1}}*x=1$, то~моноид является группой. В~\cite{Bauman_DM, Adelson_Velskiy, Voevodin} доказывается теорема о~единственности обратного элемента для~каждого~$x\in D$. Если~же операция~$*$ коммутативна, то~группа называется абелевой. Для~абелевой~группы свойства моноида дополняются ещё двумя~свойствами:
\begin{itemize}
	\item $\forall x,y\in D\quad x*y=y*x$;
	\item $\forall x\in D\quad x*{{x}^{-1}}=1$.
\end{itemize}

Для~наглядности записи, в~сигнатуру алгебры допускается включать нейтральные относительно операции элементы, поскольку, как указано в~\cite{Bauman_DM}, они являются нульарной операцией. В этом случае моноид можно записать как совокупность $\left( D,*,1 \right)$.

Перейдём к рассмотрению алгебр с сигнатурой, состоящей из двух бинарных операций. Кольцом называют алгебру $\left( D,+,\cdot ,1,0 \right)$, такую, что алгебра $\left( D,+,0 \right)$ является абелевой группой, алгебра $\left( D,\cdot ,1 \right)$ является моноидом, а операция умножения кольца $\left( \cdot  \right)$ дистрибутивна относительно операции сложения кольца $\left( + \right)$, т.е. справедливо равенство:
\begin{equation*}
	\forall x,y,z\in D:\quad x\cdot \left( y+z \right)=x\cdot y+x\cdot z.
\end{equation*}
Элементы 0~и~1 называют нулём и~единицей кольца соответственно. Если операция~умножения коммутативна, то~кольцо является коммутативным.

Если~же в~кольце алгебра всех ненулевых по~умножению элементов кольца образует группу, то~кольцо называется телом. Коммутативное же~тело является полем. Другими словами, поле есть алгебра $\left( D,+,\cdot ,0,1 \right)$, сигнатура которой состоит из~двух бинарных и~двух нульарных операций, для~которых должны выполняться следующие тождества~\cite{Adelson_Velskiy, Bauman_DM, Yakhyaeva}:
\begin{enumerate}
	\item ассоциативность по сложению: $\forall x,y,z\in D:\ x+\left( y+z \right)=\left( x+y \right)+z$;
	\item коммутативность по сложению: $\forall x,y\in D:\ x+y=y+x$;
	\item наличие нуля (нейтрального по сложению элемента): $\exists \,0\in D:\ \forall x\in D\ \ x+0=x$;
	\item существование противоположного элемента:$\forall x\in D\ \exists \,-x\in D:\ x+\left( -x \right)=0$;
	\item ассоциативность по умножению: $\forall x,y,z\in D:\ x\cdot \left( y\cdot z \right)=\left( x\cdot y \right)\cdot z$;
	\item коммутативность по умножению: $\forall x,y\in D:\ x\cdot y=y\cdot x$;
	\item наличие единицы (нейтрального по умножению элемента): $\exists \,1:\ \forall x\in D:\ x\cdot 1=x$;
	\item существование обратного элемента для ненулевых элементов: $\forall x\in D\backslash \left\{ 0 \right\}\ \ \exists \ {{x}^{-1}}:\ \ x\cdot {{x}^{-1}}=1$;
	\item дистрибутивность умножения относительно сложения: $\forall x,y,z\in D:x\cdot y+x\cdot z=x\cdot \left( y+z \right)$.
\end{enumerate}

