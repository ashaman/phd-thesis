Для~создания моделей с~использованием нечётких параметров необходимо уметь выполнять различные операции над нечёткими числами, а также сравнивать их между собой. Основной проблемой является подбор адекватной структуры и подходящей алгебры для~множества нечётких чисел либо численных методов решения таких моделей. В~отечественной и~зарубежной литературе предложено множество различных алгебр и алгебраических систем на множестве нечётких чисел, которые различаются с~точки зрения свойств их~операций. Кратко рассмотрим основные понятия дискретной математики, касающиеся алгебр и алгебраических систем для того, чтобы в дальнейшем применять их к анализу существующих алгебр нечётких чисел и сформировать критерии оценки <<идеальной>> алгебраической системы с~точки зрения моделирования.

Предметом рассмотрения абстрактной алгебры являются произвольные множества с~заданными на~них операциями, при этом природа этих~множеств и~операций может существенно отличаться от~привычных числовых множеств и~известных операций над~числами~\cite{Bauman_DM}.

\begin{mydef}
Пусть $A$ – произвольное непустое множество, $n\in \mathbb{N}$. Любое отображение 
\begin{equation*}
	f:A^N \to A
\end{equation*}
называют $n$-арной операцией на множестве $A$.
\end{mydef}

В алгебрах наиболее важными и исследуемыми являются бинарные $\left( n=2 \right)$ операции. Если $*$~--- некая абстрактная бинарная операция, то, согласно~\cite{Bauman_DM}, она~является
\begin{itemize}
	\item коммутативной, если $\forall x,y\in A\ x*y=y*x$;
	\item ассоциативной, если $\forall x,y,z\in A\ x*\left( y*z \right)=\left( x*y \right)*z$;
	\item идемпотентной, если $\forall x\in A\ x*x=x$.
\end{itemize}

Элемент $0$ множества $A$ называют нулём относительно операции $*$, если $\forall x\in A\ 0*x=0,\ x*0=0$. Нуль в множестве $A$ единственен. В~самом деле, если~предположить существование другого нулевого элемента ${0}'$ относительно операции $*$, то,~согласно определению нуля
\begin{equation*}
	0*{0}'=0,\ {0}'*0={0}',
\end{equation*}
откуда следует равенство $0={0}'$.

Элемент 1 множества $A$ называют нейтральным относительно операции $*$, если $\forall x\in A\ 1*x=x,\ x*1=1$.
Нейтральный элемент в множестве $A$ также единственен, доказательство этого факта аналогично доказательству единственности нулевого элемента.

В~\cite{Bauman_DM, Adelson_Velskiy} даётся следующее определение алгебры и алгебраической системы.
\begin{mydef}
Алгебра считается заданной, если задано некоторое множество $D$, называемое носителем алгебры, и некоторое множество операций $\Omega $ на $D$, называемое сигнатурой данной алгебры. Алгебру можно записать как упорядоченную пару множеств $\left( D,\Omega  \right)$. Алгебра $\left(D, \Omega \right)$, дополненная множеством отношений $\Pi$ на $D$, называется алгебраической системы и обозначается $\left(D, \Omega, \Pi \right)$.
\end{mydef}
Понятие алгебры является частным случаем алгебраической системы с пустым множеством отношений. Другим предельным случаем алгебраической системы является модель~--- множество, на котором заданы только отношения.

Стоит отметить, что~операции, включенные в~сигнатуру, задаются как~некоторые специальные отображения. При~этом не~оговариваются свойства, которыми операции обладают на~носителе~--– они обычно указываются дополнительно.
Кратко рассмотрим основные виды алгебр, описанные в~\cite{Bauman_DM, Adelson_Velskiy, Voevodin}. Вначале дадим определения для алгебр, сигнатура которых состоит из одной абстрактной бинарной операции $*$.

Группоидом называют любую алгебру $\left( D,* \right)$, сигнатура которой состоит из~одной бинарной операции, на~которую не~наложено никаких ограничений. Если~же операция $*$ ассоциативна, то~группоид является полугруппой. Отдельно выделяют коммутативные полугруппы~--– полугруппы, в~которых операция $*$ обладает свойством коммутативности, а также полурешётки~--- коммутативные полугруппы, операция которых идемпотентна: $\forall x \in D \ x*x=x$.

Моноидом называется такая полугруппа, относительно операции которой существует нейтральный элемент. Такой элемент называется единицей моноида $\left( D,* \right)$ и~обозначается как 1. Для моноида справедливы следующие свойства:
\begin{itemize}
	\item $\forall x,y,z\in D\quad x*\left( y*z \right)=\left( x*y \right)*z$;
	\item $\forall x\in D\quad x*1=1*x=x$.
\end{itemize}

Если в моноиде $\forall x\in D\ \exists \,x^{-1}\in D$, называемый обратным, такой, что~$x*{{x}^{-1}}={{x}^{-1}}*x=1$, то~моноид является группой. В~\cite{Bauman_DM, Adelson_Velskiy, Voevodin} доказывается теорема о~единственности обратного элемента для~каждого~$x\in D$. Если~же операция~$*$ коммутативна, то~группа называется абелевой. Для~абелевой~группы свойства моноида дополняются ещё двумя~свойствами:
\begin{itemize}
	\item $\forall x,y\in D\quad x*y=y*x$;
	\item $\forall x\in D\quad x*{{x}^{-1}}=1$.
\end{itemize}

Для~наглядности записи, в~сигнатуру алгебры допускается включать нейтральные относительно операции элементы, поскольку, как указано в~\cite{Bauman_DM}, они являются нульарной операцией. В этом случае моноид можно записать как совокупность $\left( D,*,1 \right)$.

Перейдём к рассмотрению алгебр с сигнатурой, состоящей из двух бинарных операций. Решёткой называют такую алгебру $\left( D, \bigvee, \bigwedge \right)$, что каждая из алгебр $\left( D, \bigvee\right)$ и $\left( D, \bigwedge \right)$ является полурешёткой и справедливы тождества поглощения~\cite{Hybrid_Systems}:
\begin{equation*}
  \forall x,y\in D:\ x \bigvee \left( x \bigwedge y \right) = x,\ x \bigwedge \left(x \bigvee y \right) = x.
\end{equation*}
Операции $\bigvee$ и $\bigwedge$ называют решёточным объединением и пересечением соответственно. Если для элементов решётки выполняется свойство дистрибутивности, то она является дистрибутивной, а если на дистрибутивной решётке введена операция дополнения и для неё выполняются законы де Моргана и~свойство $\forall x\in D\ \overline{\bar x}=x$, то алгебра является дистрибутивной решёткой с дополнениями~\cite{Lipetsk}. В~\cite{Kaufmann, Pospelov} отмечается, что теоретико-множественные операции над нечёткими множествами образуют дистрибутивную решётку.

Кольцом называют алгебру $\left( D,+,\cdot ,1,0 \right)$, такую, что алгебра $\left( D,+,0 \right)$ является абелевой группой, алгебра $\left( D,\cdot ,1 \right)$ является моноидом, а операция умножения кольца $\left( \cdot  \right)$ дистрибутивна относительно операции сложения кольца $\left( + \right)$, т.е. справедливо равенство:
\begin{equation*}
	\forall x,y,z\in D:\quad x\cdot \left( y+z \right)=x\cdot y+x\cdot z.
\end{equation*}
Элементы 0~и~1 называют нулём и~единицей кольца соответственно. Если операция~умножения коммутативна, то~кольцо является коммутативным.

Если~же в~кольце алгебра всех ненулевых по~умножению элементов кольца образует группу, то~кольцо называется телом. Коммутативное же~тело является полем. Другими словами, поле есть алгебра $\left( D,+,\cdot ,0,1 \right)$, сигнатура которой состоит из~двух бинарных и~двух нульарных операций, для~которых должны выполняться следующие тождества~\cite{Adelson_Velskiy, Bauman_DM, Yakhyaeva}:
\begin{enumerate}
	\item ассоциативность по сложению: $\forall x,y,z\in D:\ x+\left( y+z \right)=\left( x+y \right)+z$;
	\item коммутативность по сложению: $\forall x,y\in D:\ x+y=y+x$;
	\item наличие нуля (нейтрального по сложению элемента): $\exists \,0\in D:\ \forall x\in D\ \ x+0=x$;
	\item существование противоположного элемента:$\forall x\in D\ \exists \,-x\in D:\ x+\left( -x \right)=0$;
	\item ассоциативность по умножению: $\forall x,y,z\in D:\ x\cdot \left( y\cdot z \right)=\left( x\cdot y \right)\cdot z$;
	\item коммутативность по умножению: $\forall x,y\in D:\ x\cdot y=y\cdot x$;
	\item наличие единицы (нейтрального по умножению элемента): $\exists \,1:\ \forall x\in D:\ x\cdot 1=x$;
	\item существование обратного элемента для ненулевых элементов: $\forall x\in D\backslash \left\{ 0 \right\}\ \ \exists \ {{x}^{-1}}:\ \ x\cdot {{x}^{-1}}=1$;
	\item дистрибутивность умножения относительно сложения: $\forall x,y,z\in D:x\cdot y+x\cdot z=x\cdot \left( y+z \right)$.
\end{enumerate}

В~чётких моделях в~качестве параметров используются элементы множества действительных чисел $\mathbb{R}$, на~котором определена алгебра действительных чисел, являющаяся по~своей структуре полем. Для~того, чтобы нечёткие числа можно было применять в~качестве параметров чётких задач, алгебра, применяемая к~ним, также должна быть полем. Например, при решении алгебраических или дифференциальных уравнений необходимым условием четкого равенства является наличие групповых свойств операций над нечеткими числами~\cite{Serbia_Algebras}. Однако в~моделировании, помимо выполнения операций над численными величинами, требуется уметь сравнивать эти величины между собой. Поскольку на множестве действительных чисел определено отношение линейного порядка <<меньше или равно>>, логичным является требование установления линейного порядка и на множестве нечётких чисел.

Очевидно, что алгебра, основанная на арифметических операциях, введённых в~п.~\ref{chapter1_3}, не~является ни~полем, ни~кольцом, а~псевдорешёткой~\cite{Kaufmann, Fuzzy_Lattices} c~различными вариантами аксиоматических систем~\cite{Axioms_Fuzzy_Algebra, Philosophy_Fuzzy_Structures}. Помимо доказательства невыполнения свойства дистрибутивности, в~\cite{Pospelov, Borisov_Alexeev_Msk, Yakhyaeva, Yager_Arithmetics} показывается, что не~существует истинно противоположного и~обратного элементов, т.\,е.~справедливы выражения
\begin{gather}
	\label{eq:zadeh-zero-nonequality}
	\tilde{A}+\left( -\tilde{A} \right)\ne 0; \\
	\label{eq:zadeh-reverse-nonequality}
	\tilde{A}\cdot {{\tilde{A}}^{-1}}\ne 1.
\end{gather}
Авторы~\cite{Rutkovskaya} отмечают, что ввиду~\eqref{eq:zadeh-zero-nonequality} и~\eqref{eq:zadeh-reverse-nonequality}, уравнения и системы уравнений с нечёткими коэффициентами неразрешимы с помощью метода исключения, поскольку нарушается четкая тождественность уравнения после подстановки нечеткого решения. Схожие выводы сделаны в~\cite{Sokolov, Bocharnkinov_Ukraine}~--- любые вычислительные операции над нечеткими числами могут приводить к~нарушению естественных отношений (например, операция вычитания с равными нечеткими операндами не приводит к нулю, уже упоминавшееся отсутствие баланса в уравнениях и т.п.). В~связи с~этим возникает проблема выбора модели представления нечеткой числовой информации~\cite{Koroteev_Fuzzy_Arithmetics}, которая при сохранении основных исходных свойств экспертных оценок обеспечивает возможность построения алгебры нечетких чисел, эквивалентной полю действительных чисел. 

Ещё одним недостатком классической нечёткой арифметики является тот факт, что функция принадлежности результата определяется 
на~максимально широком носителе, что при обеспечении математической строгости завышает степень неопределенности~\cite{Evdokimov, Kreinovich_100plus1, Hanss_Engineering}. В статье~\cite{Hanss_Strict_Arithmetic} это объясняется тем фактом, что в~классической нечёткой арифметике эффект <<переучёта>> неопределённости возникает из-за того, что параметры нечёткого числа рассматриваются как независимые переменные, что далеко не всегда соответствует реальности. В~\cite{Yakhyaeva} также отмечается, что степень неопределённости зависит не только от длины носителя числа, но и от его положения на~числовой~оси, а в~\cite{Kreinovich_100plus1} при попытке разрешения этой проблемы возникают проблемы с нарушением групповых свойств операций сложения и умножения. Поэтому к~модели представления нечётких чисел и к определяемой на них алгебре выдвигается требование ограничения роста неопределенности и~независимости степени размытости результата операций от расположения операндов на числовой оси.

Для установления линейного порядка на множестве нечётких чисел требуется определить отношение $Q$ на множестве нечётких чисел $F$, которое обладает свойствами рефлексивности $\forall \tilde X \in F:\ \tilde X Q \tilde X$, антисимметричности $\forall \tilde X, \tilde Y \in F:\allowbreak\ \tilde X Q \tilde Y = \tilde Y Q \tilde X \Rightarrow \tilde X = \tilde Y$ и транзитивности $\forall \tilde X, \tilde Y, \tilde Z \in F:\allowbreak \tilde X Q \tilde Y \bigwedge \tilde Y Q \tilde Z \Rightarrow \tilde X Q \tilde Z$, и при этом любые элементы $\tilde X, \tilde Y \in F$ сравнимы~\cite{Vorontsov_Compare}. Очевидно, что~способ сравнения с~помощью индексов ранжирования, описанный в п.~\ref{chapter1_3}, даёт нечёткий результат сравнения и непригоден для определения отношения~$Q$. Исследования различных способов установления линейного порядка, проведённые в~\cite{Vorontsov_Compare}, показали, что большинство методов сравнения нечётких чисел либо не решают проблему существования несравнимых чисел (теоретико-множественные и $\alpha$-уровневые методы~\cite{Cheng_Comparison, Zak}), либо решают её искусственным образом (интегральные и~метрические методы~\cite{Skorokhod, Ledeneva_Nguen}), вводя дополнительные оценочные функции (метод $\alpha$-взвешенного сравнения~\cite{Detyniecki_Yager, Ukhobotov_Comparison, Ledeneva_VSTU_Comparison}) и~при~этом приводя к~противоречивым результатам (центроидный метод~\cite{Centroid} и~метод построения максимизирующей и~минимизирующей точек~\cite{Max_Min_Points}). Многообразие методов, сложности вычислений и неочевидные моменты в~сравнении нечётких величин наводят на~мысль о~том, что линейный~порядок должен обеспечиваться выбором подходящей модели представления нечётких чисел в рамках соответствующей алгебраической системы~\cite{Kosinski}.

Наконец, при нечётком моделировании возникают ещё две проблемы, не упомянутые ранее. Во-первых, это проблема создания специфического программного обеспечения для каждой решаемой задачи~\cite{Uskov_Complex}, поскольку при моделировании могут использоваться различные способы представления нечётких величин и их сравнения. Каждый новый метод вычислений и новое представление нечётких чисел обычно приводит к необходимости написания новых модулей ПО~\cite{Gallyamov}, при том, что сама постановка задачи и алгоритм решения не меняются~\cite{Koroteev_Fuzzy_Arithmetics}. Во-вторых, это проблема устойчивости результата~\cite{Fuller}, возникающая, например, в задачах линейного программирования с нечёткими коэффициентами.  Необходимо выбрать такие условия устойчивости, которые можно было бы использовать непосредственно в алгоритме решения задачи. Более того, в идеальном случае, модель представления нечеткой числовой информации должна позволять параметрически управлять устойчивостью решения.

Таким образом, к~алгебраической системе, которую можно использовать в моделях с чёткими отношениями и нечёткими параметрами, выдвигаются следующие требования:
\begin{itemize}
	\item ограничение роста неопределенности результатов обработки нечеткой информации;
	\item сохранение чётких отношений в модельных уравнениях при подстановке данных;
	\item возможность представления линейного порядка на множестве нечётких чисел;
	\item возможность использования стандартных программных средств реализации численных методов решений;
	\item возможность управления устойчивостью решения решаемой задачи.
\end{itemize}
