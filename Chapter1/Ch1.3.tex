Понятие алгебраической системы как совокупности алгебры и модели.
 
Необходимым условием четкого равенства должно быть наличие групповых свойств операций над нечеткими числами. Т.е. надо подобрать такую модель представления нечеткой числовой информации, которая при сохранении основных исходных свойств экспертных оценок обеспечивает возможность построения алгебры нечетких чисел эквивалентной  полю действительных чисел. Это обеспечит ограничение роста неопределенности.

Полный порядок должен обеспечиваться выбором подходящей модели в рамках соответствующей алгебраической системы.

Надо выбрать такие условия устойчивости, которые можно было бы использовать непосредственно в алгоритме решения. Более того, модель представления нечеткой числовой информации должна позволять управлять устойчивостью

\subsection{Общие алгебраические структуры}

Для~решения задач с использованием нечётких параметров необходимо уметь выполнять различные операции над нечёткими числами. В различных работах вводятся различные алгебры нечётких чисел. Однако для начала рассмотрим, как вводится понятие алгебры и как оно проецируется на множество нечётких чисел.

Предметом рассмотрения абстрактной алгебры являются произвольные множества с~заданными на~них операциями, при этом природа этих~множеств и~операций может существенно отличаться от~привычных числовых множеств и~известных операций над~числами \cite{Bauman_DM}. Проблема работы со~сложно организованными числовыми и~нечисловыми структурами возникла в~связи с~развитием современных информационных технологий и~переходом от~обычных вычислений к~обработке сложных структур данных.

\begin{mydef}
Пусть $A$ – произвольное непустое множество, $n\in \mathbb{N}$. Любое отображение 
\[
	f:A^N \to A
\] 
называют $n$-арной операцией на множестве $A$.
\end{mydef}

В алгебрах наиболее важными и исследуемыми являются бинарные $\left( n=2 \right)$ операции. Если $*$~--- некая абстрактная бинарная операция, то, согласно \cite{Bauman_DM}, она~является
\begin{itemize}
	\item коммутативной, если $\forall x,y\in A\ x*y=y*x$;
	\item ассоциативной, если $\forall x,y,z\in A\ x*\left( y*z \right)=\left( x*y \right)*z$;
	\item идемпотентной, если $\forall x\in A\ x*x=x$.
\end{itemize}

Элемент $0$ множества $A$ называют нулём относительно операции $*$, если $\forall x\in A\ 0*x=0,\ x*0=0$. Нуль в множестве $A$ единственен. В~самом деле, если~предположить существование другого нулевого элемента ${0}'$ относительно операции $*$, то,~согласно определению нуля
\[
	0*{0}'=0,\ {0}'*0={0}'
\],
откуда следует равенство $0={0}'$.

Элемент 1 множества $A$ называют нейтральным относительно операции $*$, если $\forall x\in A\ 1*x=x,\ x*1=1$.
Нейтральный элемент в множестве $A$ также единственен, доказательство этого факта аналогично доказательству единственности нулевого элемента.

В~\cite{Bauman_DM, Adelson_Velskiy} даётся следующее определение алгебры.
\begin{mydef}
Алгебра считается заданной, если задано некоторое множество $D$, называемое носителем алгебры, и некоторое множество операций $\Omega $ на $D$, называемое сигнатурой данной алгебры. Алгебру можно записать как упорядоченную пару множеств $\left( D,\Omega  \right)$.
\end{mydef}

Стоит отметить, что~операции, включенные в~сигнатуру, задаются как~некоторые специальные отображения. При~этом не~оговариваются свойства, которыми операции обладают на~носителе~--– они обычно указываются дополнительно.
Кратко рассмотрим основные виды алгебр, описанные в~\cite{Bauman_DM, Adelson_Velskiy, Voevodin},. Вначале дадим определения для алгебр, сигнатура которых состоит из одной абстрактной бинарной операции $*$.

Группоидом называют любую алгебру $\left( D,* \right)$, сигнатура которой состоит из~одной бинарной операции, на~которую не~наложено никаких ограничений. Если~же операция $*$ ассоциативна, то~группоид является полугруппой. Отдельно выделяют коммутативные полугруппы~--– полугруппы, в~которых операция $*$ обладает свойством коммутативности.

Моноидом называется такая полугруппа, относительно операции которой существует нейтральный элемент. Такой элемент называется единицей моноида $\left( D,* \right)$ и~обозначается как 1. Для моноида справедливы следующие свойства:
\begin{itemize}
	\item $\forall x,y,z\in D\quad x*\left( y*z \right)=\left( x*y \right)*z$;
	\item $\forall x\in D\quad x*1=1*x=x$.
\end{itemize}

Если в моноиде $\forall x\in D\ \exists \,{{x}^{-1}}\in D$, называемый обратным, такой, что~$x*{{x}^{-1}}={{x}^{-1}}*x=1$, то~моноид является группой. В~\cite{Bauman_DM, Adelson_Velskiy, Voevodin} доказывается теорема о~единственности обратного элемента для~каждого~$x\in D$. Если~же операция~$*$ коммутативна, то~группа называется абелевой. Для~абелевой~группы свойства моноида дополняются ещё двумя~свойствами:
\begin{itemize}
	\item $\forall x,y\in D\quad x*y=y*x$;
	\item $\forall x\in D\quad x*{{x}^{-1}}=1$.
\end{itemize}

Для~наглядности записи, в~сигнатуру алгебры допускается включать нейтральные относительно операции элементы, поскольку, как указано в [21-томник], они являются нульарной операцией. В этом случае моноид можно записать как совокупность $\left( D,*,1 \right)$.

Перейдём к рассмотрению алгебр с сигнатурой, состоящей из двух бинарных операций. Кольцом называют алгебру $\left( D,+,\cdot ,1,0 \right)$, такую, что алгебра $\left( D,+,0 \right)$ является абелевой группой, алгебра $\left( D,\cdot ,1 \right)$ является моноидом, а операция умножения кольца $\left( \cdot  \right)$ дистрибутивна относительно операции сложения кольца $\left( + \right)$, т.е. справедливо равенство:
	\[\forall x,y,z\in D:\quad x\cdot \left( y+z \right)=x\cdot y+x\cdot z\].
Элементы 0~и~1 называют нулём и~единицей кольца соответственно. Если операция~умножения коммутативна, то~кольцо является коммутативным.

Если~же в~кольце алгебра всех ненулевых по~умножению элементов кольца образует группу, то~кольцо называется телом. Коммутативное же~тело является полем. Другими словами, поле есть алгебра $\left( D,+,\cdot ,0,1 \right)$, сигнатура которой состоит из~двух бинарных и~двух нульарных операций, для~которых должны выполняться следующие тождества \cite{Adelson_Velskiy, Bauman_DM, Yakhyaeva}:
\begin{enumerate}
	\item ассоциативность по сложению: $\forall x,y,z\in D:\ x+\left( y+z \right)=\left( x+y \right)+z$;
	\item коммутативность по сложению: $\forall x,y\in D:\ x+y=y+x$;
	\item наличие нуля (нейтрального по сложению элемента): $\exists \,0\in D:\ \forall x\in D\ \ x+0=x$;
	\item существование противоположного элемента:$\forall x\in D\ \exists \,-x\in D:\ x+\left( -x \right)=0$;
	\item ассоциативность по умножению: $\forall x,y,z\in D:\ x\cdot \left( y\cdot z \right)=\left( x\cdot y \right)\cdot z$;
	\item коммутативность по умножению: $\forall x,y\in D:\ x\cdot y=y\cdot x$;
	\item наличие единицы (нейтрального по умножению элемента): $\exists \,1:\ \forall x\in D:\ x\cdot 1=x$;
	\item существование обратного элемента для ненулевых элементов: $\forall x\in D\backslash \left\{ 0 \right\}\ \ \exists \ {{x}^{-1}}:\ \ x\cdot {{x}^{-1}}=1$;
	\item дистрибутивность умножения относительно сложения: $\forall x,y,z\in D:x\cdot y+x\cdot z=x\cdot \left( y+z \right)$.
\end{enumerate}

\subsection{Анализ существующих алгебр нечётких чисел}

В~чётких задачах в~качестве параметров используются элементы множества действительных чисел $\mathbb{R}$, на~котором определена алгебра действительных чисел, являющаяся по~своей структуре полем. Для~того, чтобы нечёткие числа можно было применять в~качестве параметров чётких задач, алгебра, применяемая к~ним, также должна быть полем.

%В~отечественной и~зарубежной литературе вводится множество различных алгебр нечётких чисел, однако концептуально они различаются только по тому, используется ли в них принцип обобщения Заде (обычный или~$\alpha$-уровневый) или~нет. Также существующие алгебры нечётких чисел можно классифицировать с точки зрения свойств их операций (моноиды, группоиды, кольца и т.д.).

\subsubsection*{Вычисления с использованием интервальной нечёткости}
Как указано в~\cite{Rotshtein, Borisov_Krumberg_Riga}, алгебры, основанные на~принципе~обобщения Заде, обладают существенным недостатком~--- неэффективными промежуточными вычислениями, необходимыми для получения функции принадлежности результата. Поскольку для~нечётких~множеств справедлива теорема о декомпозиции, то для решения практических задач, обычная нечёткость может быть сведена к интервальной, а операции над числами – к соответствующим операциям над их $\alpha$-сечениями. В~\cite{Borisov_Krumberg_Riga} предлагается выполнить дискретизацию непрерывного нечёткого числа по конечному числу $\alpha$-уровней $\alpha_i;i=\overline{1,k}$, причём $\alpha_1=0$, $\alpha_k=1$. Нечёткое число будет представлено совокупностью интервалов $\displaystyle \bigcup\limits_{i=1}^{k}{{{\left[ {{x}_{i1}};{{x}_{i2}} \right]}_{{{\alpha }_{i}}}}}$, над которыми можно совершать некоторые операции.

В~статье~\cite{Levin} описывается алгебра для~интервалов вида $\tilde{x}=\left[ {{x}_{1}};{{x}_{2}} \right];\,{{x}_{1}}\le {{x}_{2}};\ {{x}_{1}},{{x}_{2}}\in \mathbb{R};\ \ \tilde{x}\in X$. Произвольная интервальная функция $F\left( \tilde{x},\tilde{y},...,\tilde{z} \right)$ определяется с помощью сопоставленной ей чёткой функции $f\left( x,y,...,z \right)$, $x\in \tilde{x}$, $y\in \tilde{y}$, …$z\in \tilde{z}$ следующим образом:
\begin{equation}
\label{eq:interval-function}
	F\left( \tilde{x},\tilde{y},...,\tilde{z} \right)=\left\{ f\left( x,y,...,z \right)\left| x\in \tilde{x},y\in \tilde{y},...,z\in \tilde{z} \right. \right\}
\end{equation}

В~соответствие с~определением~\eqref{eq:interval-function}, автор вводит конкретные алгебраические операции для интервалов:
\begin{itemize}
	\item сложение $\tilde{v}=\tilde{x}+\tilde{y}=\left\{ \left( x+y \right)\left| x\in \tilde{x},y\in \tilde{y} \right. \right\}=\left[ {{x}_{1}}+{{y}_{1}};{{x}_{2}}+{{y}_{2}} \right]$;
	\item вычитание $\tilde{v}=\tilde{x}-\tilde{y}=\left\{ \left( x-y \right)\left| x\in \tilde{x},y\in \tilde{y} \right. \right\}=\left[ {{x}_{1}}-{{y}_{2}};{{x}_{2}}-{{y}_{1}} \right]$;
	\item умножение на число: $\tilde{v}=k\tilde{x}=\left\{ kx\left| x\in \tilde{x},k\in \mathbb{R} \right. \right\}=\left[\underset{i}{\mathop {\min}} \left( kx_i \right); \underset{i}{\mathop {\max}} \left( kx_i \right) \right]$;
	
%	\left\{ \begin{aligned}
% & \left[ k{{x}_{1}};k{{x}_{2}} \right];k\geqslant 0 \\ 
%& \left[ k{{x}_{2}};k{{x}_{1}} \right];k<0 \\ 
%\end{aligned} \right.$;
	\item умножение $\tilde{v}=\tilde{x}\cdot \tilde{y}=\left\{ \left( x\cdot y \right)\left| x\in \tilde{x},y\in \tilde{y} \right. \right\}=\left[ \underset{i,j}{\mathop{\min }}\,\left( {{x}_{i}}{{y}_{j}} \right);\underset{i,j}{\mathop{\max }}\,\left( {{x}_{i}}{{y}_{j}} \right) \right]$;
	\item деление $\displaystyle \tilde{v}=\frac{{\tilde{x}}}{{\tilde{y}}}=\left\{ \left. \left( \frac{x}{y} \right) \right|x\in \tilde{x},y\in \tilde{y} \right\}=\left[ {{x}_{1}};{{x}_{2}} \right]\cdot \left[ \frac{1}{{{y}_{2}}};\frac{1}{{{y}_{1}}} \right]$.
\end{itemize}

Также определяются интервальная единица $\tilde{1}=\left[ 1;1 \right]$ и интервальный ноль $\tilde{0}=\left[ 0;0 \right]$ как нейтральные элементы по умножению и сложению соответственно. Во вводимой алгебре справедливы следующие законы:
\begin{enumerate}
	\item коммутативность по сложению $\tilde{x}+\tilde{y}=\tilde{y}+\tilde{x};\ \ \tilde{x},\tilde{y}\in X$;
	\item ассоциативность по сложению $\tilde{x}+\left( \tilde{y}+\tilde{z} \right)=\left( \tilde{x}+\tilde{y} \right)+\tilde{z};\ \tilde{x},\tilde{y},\tilde{z}\in X$;
	\item коммутативность по умножению $\tilde{x}\cdot \tilde{y}=\tilde{y}\cdot \tilde{x};\ \ \tilde{x},\tilde{y}\in X$;
	\item ассоциативность по умножению $\tilde{x}\cdot \left( \tilde{y}\cdot \tilde{z} \right)=\left( \tilde{x}\cdot \tilde{y} \right)\cdot \tilde{z};\ \ \tilde{x},\tilde{y},\tilde{z}\in X$;
	\item существование единственного нейтрального по сложению элемента (нуля) $\tilde{x}+\tilde{0}=\tilde{0}+\tilde{x}=\tilde{x};\ \tilde{x}\in X$;
	\item существование единственного нейтрального по умножению элемента (единицы) $\tilde{x}\cdot \tilde{1}=\tilde{1}\cdot \tilde{x}=\tilde{x};\ \tilde{x}\in X$;
	\item дистрибутивность умножения на действительную константу относительно сложения $k\left( \tilde{x}+\tilde{y} \right)=k\tilde{x}+k\tilde{y};\ \ \tilde{x},\tilde{y}\in X,\ k\in \mathbb{R}$.
\end{enumerate}

Как~отмечается в~\cite{Pospelov, Sokolov}, интервальная алгебра определяет операции вычитания и~деления как самостоятельные операции, при~этом, в~общем~случае, интервальные числа не обладают свойством дистрибутивности умножения относительно сложения; отсутствуют понятия противоположного элемента (по сложению) и обратного элемента (по умножению); вычитание из~интервального числа равного ему в~общем~случае не~приводит к~нуль-интервалу, как и деление интервального числа на равное ему не дает вырожденного единичного интервала. Эти недостатки подтверждаются в~\cite{Levin} в~виде следующих законов:
\begin{enumerate}
	\item $\tilde{x}-\tilde{x}=\tilde{0}\Leftrightarrow \tilde{x}=\left[ {{x}_{1}};{{x}_{1}} \right]$, т.е. разность интервального числа самого с собой равна нулю тогда и только тогда, когда это число вырожденное;
	\item $\displaystyle \frac{{\tilde{x}}}{{\tilde{x}}}=\tilde{1}\Leftrightarrow \tilde{x}=\left[ {{x}_{1}};{{x}_{1}} \right]$, т.е. результат деления интервального числа на само себя равен единице в том и только в том случае, когда это число вырожденное;
	\item противоположные по сложению элементы существуют только для вырожденных интервалов: $\tilde{x}+\tilde{y}=\tilde{0}\Leftrightarrow \tilde{x}=\left[ {{x}_{1}};{{x}_{1}} \right];\tilde{y}=\left[ -{{{\tilde{x}}}_{1}};-{{{\tilde{x}}}_{1}} \right]$;
	\item обратные по умножению элементы существую только для вырожденных интервалов: $\displaystyle \tilde{x}\cdot \tilde{y}=\tilde{1}\Leftrightarrow \tilde{x}=\left[ {{x}_{1}};{{x}_{1}} \right];\tilde{y}=\left[ \frac{1}{{{x}_{1}}};\frac{1}{x_1} \right]$;
	\item закон субдистрибутивности умножения на интервальное число относительно сложения $\tilde{x}\cdot \left( \tilde{y}+\tilde{z} \right)\subseteq \tilde{x}\cdot \tilde{y}+\tilde{x}\cdot \tilde{z}$. Только в частном случае, когда $\tilde y,\tilde z>\tilde 0$, имеет место истинная дистрибутивность.
\end{enumerate}

Как указано в~\cite{Levin}, невыполнение некоторых аксиом поля, а также перечисленные выше особенности интервальных чисел приводят к значительной специфике их использования на практике (в основном, для оценки множества решений некоторой задачи при интервально заданных параметрах). Стоит также упомянуть описанную в~\cite{Spesivtsev, Yakhyaeva} проблему размытости произведения, т.\,е. необоснованного увеличения нечёткости. Размытость зависит не только от длины $d={{x}_{2}}-{{x}_{1}}$ интервала, но и от его положения на~числовой~оси. К~примеру, интервалы одинаковой длины $\left[ 1;3 \right]$ и $\left[ 99;101 \right]$, будучи умноженными на один и тот же интервал$\left[ 2;4 \right]$, дадут результаты с сильно различающейся степенью нечёткости:
\[
	\begin{matrix}
		\left[ 2;4 \right]\cdot \left[ 1;3 \right]=\left[ 2;12 \right];\quad d=12-2=10 \\ 
		\left[ 2;4 \right]\cdot \left[ 99;101 \right]=\left[ 198;404 \right];\quad d=404-198=206 
	\end{matrix}
\]

\subsubsection*{Алгебры нечётких LR-чисел}

% (Поспелов + [28], Борисов + [54] [107] [25])
На практике для нечётких вычислений часто используются удобные для понимания и записи нечёткие числа $LR$-типа. Такое число, согласно определению, представимо в виде тройки $\tilde{A}=\left( m;a;b \right)$. Арифметические операции над~числами $LR$-типа в~\cite{Pospelov, Yakhyaeva} вводятся следующим образом:
\begin{itemize}
	\item сложение ${{\tilde{A}}_{1}}+{{\tilde{A}}_{2}}$ : $\left( {{m}_{1}},{{a}_{1}},{{b}_{1}} \right)+\left( {{m}_{2}},{{a}_{2}},{{b}_{2}} \right)=\left( {{m}_{1}}+{{m}_{2}},{{a}_{1}}+{{a}_{2}},{{b}_{1}}+{{b}_{2}} \right)$;
	\item противоположное число $-{{\tilde{A}}_{1}}$: $-\left( m,a,b \right)=\left( -m,b,a \right)$;
	\item вычитание вводится как сумма числа ${{\tilde{A}}_{1}}$ и~числа, противоположного числу ${{\tilde{A}}_{2}}$: $\left( {{m}_{1}},{{a}_{1}},{{b}_{1}} \right)-\left( {{m}_{2}},{{a}_{2}},{{b}_{2}} \right)=\left( {{m}_{1}}-{{m}_{2}},{{a}_{1}}+{{b}_{2}},{{b}_{1}}+{{a}_{2}} \right)$;
	\item умножение положительных чисел: $\left( {{m}_{1}},{{a}_{1}},{{b}_{1}} \right)\times \left( {{m}_{2}},{{a}_{2}},{{b}_{2}} \right)=\left( {{m}_{1}}{{m}_{2}},{{m}_{2}}{{a}_{1}}+{{m}_{1}}{{a}_{2}},{{m}_{2}}{{b}_{1}}+{{m}_{1}}{{b}_{2}} \right)$;
	\item умножение разнознаковых чисел $\left( {{m}_{1}},{{a}_{1}},{{b}_{1}} \right)\times \left( {{m}_{2}},{{a}_{2}},{{b}_{2}} \right)=\left( {{m}_{1}}{{m}_{2}},{{m}_{2}}{{a}_{1}}-{{m}_{1}}{{b}_{2}},{{m}_{2}}{{b}_{1}}-{{m}_{1}}{{a}_{2}} \right)$;
	\item умножение отрицательных чисел $\left( {{m}_{1}},{{a}_{1}},{{b}_{1}} \right)\times \left( {{m}_{2}},{{a}_{2}},{{b}_{2}} \right)=\left( {{m}_{1}}{{m}_{2}},-{{m}_{2}}{{b}_{1}}-{{m}_{1}}{{b}_{2}},-{{m}_{2}}{{a}_{1}}-{{m}_{1}}{{a}_{2}} \right)$;
	\item обратное число $\displaystyle {\tilde A}^{-1}$: ${{\left( m,a,b \right)}^{-1}}=\left( \frac{1}{m};\frac{b}{{{m}^{2}}};\frac{a}{{{m}^{2}}} \right)$;
	\item деление нечётких чисел $\displaystyle \frac{{{{\tilde{A}}}_{1}}}{{{{\tilde{A}}}_{2}}}$ рассматривается~как умножение числа ${{\tilde{A}}_{1}}$ на~число, обратное числу ${{\tilde{A}}_{2}}$: $\displaystyle \left( \frac{{{m}_{1}}}{{{m}_{2}}},\frac{{{m}_{1}}{{b}_{2}}+{{m}_{2}}{{a}_{1}}}{m_{2}^{2}},\frac{{{m}_{1}}{{a}_{2}}+{{m}_{2}}{{b}_{1}}}{m_{2}^{2}} \right)$.
\end{itemize}

Для~вводимых~операций справедливы свойства коммутативности и~ассоциативности операций сложения и~вычитания. Свойство дистрибутивности, как и~в~случае вычислений, основанных~на принципе~Заде, выполняется не~всегда. Кроме того, и~для~этой алгебры нечётких чисел характерны недостатки, описанные в~\cite{Spesivtsev, Yakhyaeva}~--– носитель результата может необоснованно расшириться ввиду зависимости результата от~степени нечёткости операндов и~их~местоположения на~числовой~оси. Также при построении данной алгебры не вводятся нейтральные по сложению и умножению элементы (ноль и единица). В~статье~\cite{Uskov_PPS} отмечается, что~если в~качестве нулевого~элемента использовать тройку, соответствующую нулю во множестве действительных чисел
\begin{equation}
\label{eq:zero-element}
	\tilde{0}=\left( 0;0;0 \right)
\end{equation}
а в~качестве единицы~--– тройку
\begin{equation}
\label{eq:one-element}
	\tilde{1}=\left( 1;0;0 \right)
\end{equation}
то в~этом~случае вводимые ранее операции не~позволяют получить поле из-за~невыполнения тождеств $\tilde{A}+\left( -\tilde{A} \right)=\tilde{0}$ и $\tilde{A}\cdot {{\tilde{A}}^{-1}}=\tilde{1}$.

Для~получения алгебры типа~поле с~использованием нейтральных элементов \eqref{eq:zero-element}~и~\eqref{eq:one-element}, авторы~\cite{Uskov_PPS} предлагают ввести групповые операции над нечёткими числами. Для этого вводятся несколько иные противоположный и~обратный элементы:
\begin{gather}
	\label{eq:reverse-minus-element}
	-\left( m;a;b \right)=\left( -m;-a;-b \right); \\ 
	\label{eq:reverse-div-element}
	\left( m;a;b \right)^{-1}=\left( \frac{1}{m};-\frac{a}{{{m}^{2}}};-\frac{b}{{{m}^{2}}} \right). 
\end{gather}

При~этом признаётся тот~факт, что~вводимые с~помощью~  элементы имеют отрицательные коэффициенты нечёткости, а, следовательно, лишены физического смысла и не являются элементами множества нечётких $LR$-чисел. Однако, согласно определениям, которые вводились при рассмотрении абстрактных алгебр, противоположный и обратный элемент также должны являться элементами несущего множества алгебры, что в (1.35) не выполняется. Таким образом, алгебру групповых нечётких чисел нельзя называть полем в строгом смысле этого термина. Более того, авторы статьи ограничивают применимость вводимой ими алгебры только тем случаем, когда в рассматриваемой задаче есть только один независимый нечёткий параметр/сигнал, поскольку в этом случае предотвращается необоснованное увеличение степени нечёткости результата. Во всех остальных случаях предлагается использовать классический подход к нечётким вычислениям, который был описан выше, со всеми свойственными ему недостатками.
В~статье~\cite{Piter_SCM} на~основании идей, изложенных в~\cite{Borisov_Krumberg_Riga, Alexeev_Riga}, для~решения проблем чрезмерной размытости результата и~нетождественности выражений вида $\tilde A+\tilde B-\tilde B=\tilde A$, вводятся дополнительные арифметические операции и~новое представление нечёткого числа, нечувствительное к~его~знаку. Дополнительной арифметической операцией $\underset{d}{\mathop{*}}\,$ для~операции $*$ является такая, что~для~нечётких чисел $\tilde A. \tilde B, \tilde C$
\begin{equation}
\label{eq:complementary-operations}
	\tilde{C}=\tilde{A}\underset{d}{\mathop{*}}\,\tilde{B}\to \tilde{C}*\tilde{B}=\tilde{A}
\end{equation}

Для определения нового представления нечёткого числа $\left( m,d\left( {\tilde{A}} \right),\Delta \left( {\tilde{A}} \right) \right)$, называемого симметризованным параметрическим, авторы~\cite{Piter_SCM} вводят понятия показателя нечёткости числа $\displaystyle d\left( {\tilde{A}} \right)=\frac{a+b}{2}$ и коэффициента асимметрии $\displaystyle \Delta \left( {\tilde{A}} \right)=\frac{b-a}{2}$. Дополнительные арифметические операции над нечёткими числами в новом представлении выглядят следующим образом:
\begin{gather*}
		\tilde{A}\underset{d}{\mathop{+}}\,\tilde{B}=\left( {{m}_{{\tilde{A}}}}+{{m}_{{\tilde{B}}}};\left| d\left( {\tilde{A}} \right)-d\left( {\tilde{B}} \right) \right|;\Delta \left( {\tilde{A}} \right)+\Delta \left( {\tilde{B}} \right) \right); \\ 
		\tilde{A}\underset{d}{\mathop{-}}\,\tilde{B}=\left( {{m}_{{\tilde{A}}}}-{{m}_{{\tilde{B}}}};\left| d\left( {\tilde{A}} \right)-d\left( {\tilde{B}} \right) \right|;\Delta \left( {\tilde{A}} \right)-\Delta \left( {\tilde{B}} \right) \right); \\ 
		\tilde{A}\underset{d}{\mathop{\cdot }}\,\tilde{B}=\left( {{m}_{{\tilde{A}}}}{{m}_{{\tilde{B}}}};\left| \left| {{m}_{{\tilde{B}}}} \right|d\left( {\tilde{A}} \right)-\left| {{m}_{{\tilde{A}}}} \right|d\left( {\tilde{B}} \right) \right|;\left| {{m}_{{\tilde{A}}}} \right|\Delta \left( {\tilde{B}} \right)+\left| {{m}_{{\tilde{B}}}} \right|\Delta \left( {\tilde{A}} \right) \right); \\ 
		\tilde{A}\underset{d}{\mathop{:}}\,\tilde{B}=\left( \frac{{{m}_{{\tilde{A}}}}}{{{m}_{{\tilde{B}}}}};\frac{\left| \left| {{m}_{{\tilde{B}}}} \right|d\left( {\tilde{A}} \right)-\left| {{m}_{{\tilde{A}}}} \right|d\left( {\tilde{B}} \right) \right|}{m_{{\tilde{B}}}^{2}};\frac{\left| {{m}_{{\tilde{B}}}} \right|\Delta \left( {\tilde{A}} \right)-\left| {{m}_{{\tilde{A}}}} \right|\Delta \left( {\tilde{B}} \right)}{m_{{\tilde{B}}}^{2}} \right).
\end{gather*}

Для~дополнительных арифметических операций не~выполняются свойства ассоциативности по~сложению и~умножению, т.\,е.
\begin{gather*}
		\tilde{A}\underset{d}{\mathop{+}}\,\left( \tilde{B}\underset{d}{\mathop{+}}\,\tilde{C} \right)\ne \left( \tilde{A}\underset{d}{\mathop{+}}\,\tilde{B} \right)\underset{d}{\mathop{+}}\,\tilde{C}; \\ 
		\tilde{A}\underset{d}{\mathop{\cdot }}\,\left( \tilde{B}\underset{d}{\mathop{\cdot }}\,\tilde{C} \right)\ne \left( \tilde{A}\underset{d}{\mathop{\cdot }}\,\tilde{B} \right)\underset{d}{\mathop{\cdot }}\,\tilde{C},
\end{gather*}
однако при этом они обеспечивают выполнение следующих соотношений (нечёткие ноль и единица вводятся согласно \eqref{eq:zero-element} и~\eqref{eq:one-element}):
\begin{equation}
\label{eq:complementary-arithmetics-1}
\begin{matrix}
  \tilde{A}\underset{d}{\mathop{-}}\,\tilde{0}=\tilde{A} \\ 
  \tilde{A}\underset{d}{\mathop{-}}\,\tilde{A}=\tilde{0} \\ 
  \tilde{A}\underset{d}{\mathop{:}}\,\tilde{1}=\tilde{A} \\ 
  \tilde{A}\underset{d}{\mathop{:}}\,\tilde{A}=1 
\end{matrix}
\end{equation}
\begin{equation}
\label{eq:complementary-arithmetics-2}
\begin{matrix}
  \left( \tilde{A}+\tilde{B} \right)\underset{d}{\mathop{-}}\,\tilde{A}=\tilde{B} \\ 
  \left( \tilde{A}\cdot \tilde{B} \right)\underset{d}{\mathop{:}}\,\tilde{A}=\tilde{B} \\ 
  \left( \tilde{A}\underset{d}{\mathop{\cdot }}\,\tilde{B} \right)\underset{d}{\mathop{:}}\,\tilde{A}=-\tilde{B} \\ 
  \left( \tilde{A}\underset{d}{\mathop{+}}\,\tilde{B} \right)\underset{d}{\mathop{-}}\,\tilde{A}=-\tilde{B} 
\end{matrix}
\end{equation}

Свойства~\eqref{eq:complementary-arithmetics-1}~и~\eqref{eq:complementary-arithmetics-2} показывают, что~вводимые в~\cite{Piter_SCM} дополнительные алгебраические операции позволяют сохранять первоначальную нечёткость чисел, что~немаловажно для~возможности получения корректных решений нечётких уравнений. К~недостаткам подхода можно отнести его~громоздкость, неочевидность и~необходимость использования восьми алгебраических операций вместо двух, дополненных понятиями противоположного и~обратного элементов.

В~\cite{Borisov_Krumberg_Riga} класс чисел, над которыми осуществляются арифметические операции, сужается до треугольных. В целях удобства, треугольное нечёткое число $\tilde{A}=\left( {{m}_{{\tilde{A}}}};{{a}_{{\tilde{A}}}};{{b}_{{\tilde{A}}}} \right)$ представляется в виде совокупности двух ветвей функции принадлежности, аналогично представлению~\eqref{eq:infinite_fuzzy_set} для нечёткого множества:
\begin{equation}
\label{eq:integral-operation}
	\tilde{A}=\int\limits_{{{m}_{{\tilde{A}}}}-{{a}_{{\tilde{A}}}}}^{{{m}_{{\tilde{A}}}}}{\frac{{{\mu }_{{\tilde{A}}}}\left( x \right)}{x}}+\int\limits_{{{m}_{{\tilde{A}}}}}^{{{m}_{{\tilde{A}}}}+{{b}_{{\tilde{A}}}}}{\frac{{{\mu }_{{\tilde{A}}}}\left( x \right)}{x}}=\int\limits_{x_{{\tilde{A}}}^{L}}^{{{m}_{{\tilde{A}}}}}{\frac{{{\mu }_{{\tilde{A}}}}\left( x \right)}{x}}+\int\limits_{{{m}_{{\tilde{A}}}}}^{x_{{\tilde{A}}}^{R}}{\frac{{{\mu }_{{\tilde{A}}}}\left( x \right)}{x}}.
\end{equation}

Для обобщённой бинарной операции $*$, выполняемой над нечёткими треугольными числами $\tilde{A}=\left( x_{{\tilde{A}}}^{L};{{m}_{{\tilde{A}}}};x_{{\tilde{A}}}^{R} \right)$ и $\tilde{B}=\left( x_{{\tilde{B}}}^{L};{{m}_{{\tilde{B}}}};x_{{\tilde{B}}}^{R} \right)$ в~форме~\eqref{eq:integral-operation}, справедливо следующее выражение:
\[
	\tilde{C}=\tilde{A}*\tilde{B}=\left( \int\limits_{x_{{\tilde{A}}}^{L}}^{{{m}_{{\tilde{A}}}}}{\frac{{{\mu }_{{\tilde{A}}}}\left( x \right)}{x}}+\int\limits_{{{m}_{{\tilde{A}}}}}^{x_{{\tilde{A}}}^{R}}{\frac{{{\mu }_{{\tilde{A}}}}\left( x \right)}{x}} \right)*\left( \int\limits_{x_{{\tilde{B}}}^{L}}^{{{m}_{{\tilde{B}}}}}{\frac{{{\mu }_{{\tilde{B}}}}\left( x \right)}{x}}+\int\limits_{{{m}_{{\tilde{B}}}}}^{x_{{\tilde{B}}}^{R}}{\frac{{{\mu }_{{\tilde{B}}}}\left( x \right)}{x}} \right)={}
\]
\[
	=\int\limits_{x_{{\tilde{C}}}^{L}}^{{{m}_{{\tilde{C}}}}}{\frac{{{\mu }_{{\tilde{C}}}}\left( x \right)}{x}}+\int\limits_{{{m}_{{\tilde{C}}}}}^{x_{{\tilde{C}}}^{R}}{\frac{{{\mu }_{{\tilde{C}}}}\left( x \right)}{x}} ,
\]
где $m_{\tilde C}=m_{\tilde A}*m_{\tilde B}$, а границы результата $x_{\tilde C}^{L}$ и $x_{\tilde C}^{R}$ зависят от операции:
\begin{itemize}
	\item для сложения $x_{{\tilde{C}}}^{L}=x_{{\tilde{A}}}^{L}+x_{{\tilde{B}}}^{L}$; $x_{{\tilde{C}}}^{R}=x_{{\tilde{A}}}^{R}+x_{{\tilde{B}}}^{R}$;
	\item для вычитания $x_{{\tilde{C}}}^{L}=x_{{\tilde{A}}}^{L}-x_{{\tilde{B}}}^{R}$; $x_{{\tilde{C}}}^{R}=x_{{\tilde{A}}}^{R}-x_{{\tilde{B}}}^{L}$;
	\item для умножения $x_{{\tilde{C}}}^{L}=x_{{\tilde{A}}}^{L}\cdot x_{{\tilde{B}}}^{L}$; $x_{{\tilde{C}}}^{R}=x_{{\tilde{A}}}^{R}\cdot x_{{\tilde{B}}}^{R}$;
	\item для деления $\displaystyle x_{{\tilde{C}}}^{L}=\frac{x_{{\tilde{A}}}^{L}}{x_{{\tilde{B}}}^{R}}$; $\displaystyle x_{{\tilde{C}}}^{R}=\frac{x_{{\tilde{A}}}^{R}}{x_{{\tilde{B}}}^{L}}$.
\end{itemize}

Функция принадлежности результата ${{\mu }_{{\tilde{C}}}}\left( x \right)$ ищется в виде линейной функции ${{\mu }_{{\tilde{C}}}}\left( x \right)=kx+b$ для операций сложения и вычитания, а также в виде ${{\mu }_{{\tilde{C}}}}\left( x \right)=k\sqrt{x}+b$ для умножения и ${{\mu }_{{\tilde{C}}}}\left( x \right)=\frac{k}{x}+b$ для деления.

Исходя из~определения границ результата, можно сразу отметить нерешённость проблемы чрезмерного размытия результата арифметических операций и~его выхода из~класса треугольных чисел. В~\cite{Borisov_Krumberg_Riga} приводится пример операций над~нечёткими числами $\tilde{6}=\left( 5;6;7 \right)$ с длиной носителя ${{d}_{{\tilde{6}}}}=8-6=2$ и $\tilde{8}=\left( 6;8;10 \right)$ с длиной носителя ${{d}_{{\tilde{8}}}}=10-6=4$. Их умножение приводит к границам нечёткого результата $x_{{\tilde{C}}}^{L}=30$, $x_{{\tilde{C}}}^{R}=70$ и моде ${{m}_{{\tilde{C}}}}=48$. Длина носителя ${{d}_{{\tilde{c}}}}=70-30=40\gg {{d}_{{\tilde{6}}}}\cdot {{d}_{{\tilde{8}}}}$.

\subsubsection*{Создание алгебры, изоморфной существующим алгебрам}
В~статье~\cite{Uskov_Complex} предлагаются теоремы, которые позволяют сводить арифметические операции над нечёткими числами к арифметическим операциям над комплексными числами и над матрицами размера $2\times 2$. Для~определения связи между алгеброй симметричных нечётких чисел $LR$-типа $\tilde{A}=\left( x;y;y \right)$ и~комплексных чисел $x=y+iz$ доказывается следующая теорема.

\begin{theorem}
Пусть биективное преобразование $f:X\to \mathbb{C}$ ставит каждому элементу $\tilde{A}=\left( x;y;y \right)\in X$ множества нечётких симметричных $LR$-чисел элемент $x=y+iz\in \mathbb{C};\ i=\sqrt{-1}$ множества комплексных чисел. Пусть даны симметричные нечёткие $LR$-числа $\tilde{A}=\left( {{m}_{1}};a;a \right)$ и $\tilde{B}=\left( {{m}_{2}};b;b \right)$, которым с помощью $f$ взаимно однозначно сопоставлены комплексные числа ${{x}_{A}}={{m}_{1}}+ai$ и ${{x}_{B}}={{m}_{2}}+bi$. Тогда при ${{m}_{1}}{{m}_{2}}\gg ab$ и $m\gg a>0$ арифметические операции над $\tilde A$ и $\tilde B$ соответствуют операциям над комплексными числами ${{x}_{A}}$ и ${{x}_{B}}$:
\begin{gather*}
	\tilde{A}+\tilde{B}\leftrightarrow {{x}_{A}}+{{x}_{B}} \\ 
	\tilde{A}\cdot \tilde{B}\leftrightarrow {{x}_{A}}\cdot {{x}_{B}} \\ 
	-\tilde{A}\leftrightarrow -{{{\bar{x}}}_{A}} \\ 
  	{{{\tilde{A}}}^{-1}}\leftrightarrow \bar{x}_{A}^{-1} 
\end{gather*}
где ${{\bar{x}}_{A}}={{m}_{1}}-ai$~--- комплексно-сопряженное по отношению к $x_A$.
\end{theorem}

Связь же операций над симметричными нечёткими $LR$-числами и матрицами $2\times 2$ вида $\left[ \begin{matrix}
   m & -a  \\
   a & m 
\end{matrix} \right]$ доказывается с помощью следующей теоремы.
\begin{theorem}
Пусть биективное отображение $g:X\to M$ ставит каждому элементу $\tilde{A}=\left( x;y;y \right)\in X$ множества нечётких симметричных $LR$-чисел элемент ${{M}_{A}}=\left[ \begin{matrix}
   x & -y  \\
   y & x  \\
\end{matrix} \right]$ множества $M$ матриц размерности $2\times 2$. Пусть даны симметричные нечёткие $LR$-числа $\tilde{A}=\left( {{m}_{1}};a;a \right)$ и $\tilde{B}=\left( {{m}_{2}};b;b \right)$, которым с~помощью $g$ взаимно однозначно сопоставлены матрицы ${{M}_{A}}=\left[ \begin{matrix}
   {{m}_{1}} & -a  \\
   a & {{m}_{1}}  \\
\end{matrix} \right]$ и ${{M}_{B}}=\left[ \begin{matrix}
   {{m}_{2}} & -b  \\
   b & {{m}_{2}}  \\
\end{matrix} \right]$. Тогда~при ${{m}_{1}}{{m}_{2}}\gg ab$ и $m\gg a>0$ арифметические операции над~$\tilde A$ и~$\tilde B$ соответствуют операциям над~матрицами~${{M}_{A}}$ и~${{M}_{B}}$: 
\begin{gather*}
  \tilde{A}+\tilde{B}\leftrightarrow {{M}_{A}}+{{M}_{B}} \\ 
  \tilde{A}\cdot \tilde{B}\leftrightarrow {{M}_{A}}\cdot {{M}_{B}} \\ 
  -\tilde{A}\leftrightarrow -M_{A}^{T} \\ 
  {{{\tilde{A}}}^{-1}}\leftrightarrow {{\left[ M_{A}^{T} \right]}^{-1}}
\end{gather*}
где~${{M}^{T}}$~--- транспонированная матрица, а~${{M}^{-1}}$~--- матрица, обратная матрице $M$.
\end{theorem}

В~качестве преимуществ данного подхода авторы заявляют стандартизацию нечётких вычислений с~участием~чисел $LR$-типа, поскольку в~этом~случае возможно использование современных систем компьютерной математики (Maple, Mathematica, MATLAB, Mathcad), а~также возможность наглядного изображения операций и~их~результатов на комплексной плоскости. Однако в~рамках описанного подхода явно не~решаются проблемы нейтральных по~сложению и~умножению элементов. Кроме~того, приводимый авторами в статье численный пример демонстрирует нерешённость проблемы необоснованного расширения носителя при вычислениях. Наконец, применимость таких алгебр ограничена лишь~нечёткими симметричными числами.

Те~же авторы, пытаясь решить последнюю проблему (т.\,е. ограниченность области применения алгебр), в~статье~\cite{Uskov_Quaternion} распространяют полученные ранее результаты на~случай с~асимметричными нечёткими $LR$-числами. В~качестве второго множества для~построения изоморфизма алгебр выступает множество кватернионов~$H$. Далее доказывается следующее утверждение.
\begin{theorem}
Пусть биективное преобразование $f:X\to H$ ставит каждому элементу $\tilde{A}=\left( x;y;z \right)\in X$ множества нечётких $LR$-чисел элемент $a=x+yi+zj+\xi k\in H;\ {{i}^{2}}={{j}^{2}}={{k}^{2}}=ijk=-1$ множества кватернионов, где $\xi$~--– произвольный параметр, удовлетворяющий условиям $\xi >0$, $\xi \ll y$, $\xi \ll z$. Пусть даны нечёткие $LR$-числа $\tilde{A}=\left( {{m}_{1}};{{a}_{1}};{{b}_{1}} \right)$ и $\tilde{B}=\left( {{m}_{2}};{{a}_{2}};{{b}_{2}} \right)$, которым с помощью $f$ взаимно однозначно сопоставлены кватернионы ${{x}_{A}}={{m}_{1}}+{{a}_{1}}i+{{b}_{1}}j+\xi k$ и ${{x}_{B}}={{m}_{2}}+{{a}_{2}}i+{{b}_{2}}j+\xi k$. Тогда при~${{m}_{1}}\gg {{a}_{1}},{{a}_{2}},{{b}_{1}},{{b}_{2}}$ и ${{m}_{2}}\gg {{a}_{1}},{{a}_{2}},{{b}_{1}},{{b}_{2}}$ арифметические операции над~$\tilde A$ и $\tilde B$ соответствуют операциям над кватернионами ${{x}_{A}}$ и ${{x}_{B}}$:
\begin{gather*}
  \tilde{A}+\tilde{B}\leftrightarrow {{x}_{A}}+{{x}_{B}} \\ 
  \tilde{A}\cdot \tilde{B}\leftrightarrow {{x}_{A}}\cdot {{x}_{B}} \\ 
  -\tilde{A}\leftrightarrow -\bar{x}_{A}^{R} \\ 
  {{{\tilde{A}}}^{-1}}\leftrightarrow {{\left[ \bar{x}_{A}^{-1} \right]}^{R}}
\end{gather*}
где~$\bar a=x-yi-zj-\xi k$~--- кватернион, сопряженный по~отношению к~кватерниону $a$, а~${{a}^{R}}$~–-- операция рокировки кватерниона, выполняемая~как перемена~мест $i$-го и~$j$-го компонентов кватерниона $a^R=x+zi+yj+\xi k$.
\end{theorem}

Далее, по~аналогии со~статьёй [Комплексный и матричный методы…], доказывается изоморфность алгебры нечётких $LR$-чисел $\tilde{A}=\left( m;a;b \right)$ и матриц 4х4 вида
	\[\left[ \begin{matrix}
   m & a & b & -\xi   \\
   -a & m & -\xi  & -b  \\
   -b & \xi  & m & a  \\
   \xi  & b & -a & m  \\
\end{matrix} \right]\]
где $\xi $ – произвольный параметр, удовлетворяющий условиям $\xi>0$, $\xi \ll a,b$.

Данный подход решает проблему асимметричных нечётких чисел, однако при~этом сводится к~громоздким матричным вычислениям и~по-прежнему не~позволяет устранить проблему необоснованного расширения носителя результата арифметической операции. Приводя в~качестве примера анализ системы управления с~нечёткими параметрами, коэффициенты нечёткости которых не~превышают~$3$, в~результате авторы получают решение с~коэффициентами нечёткости порядка $1000$.
