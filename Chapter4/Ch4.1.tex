Перспективность данного направления исследований заключается в преимуществах теории нечётких множеств при обработке нечётких данных, которыми изобилует реальная практика разработки ПО в сфере аутсорсинга. Действительно, характерными чертами процесса проектирования и разработки ПО являются:
\begin{itemize}
  \item относительная уникальность разрабатываемого проекта~--- у некоторых участников может быть опыт участия в схожих проектах, однако он нивелируется тем, что остальная часть команды 
  \item частое отсутствие полных требований к программному продукту либо необходимых для разработки ресурсов (дизайны, переводы и т.д.)
  \item разный уровень подготовки членов команды разработчиков;
  \item возможность изменения списка работ на поздних стадиях разработки;
  \item бюрократическая составляющая~--- затягивание сроков разработки из-за проблем в согласовании изменений или списка работ.
\end{itemize}

В связи с этим в настоящее время применяемые при планировании разработки ПО методы базируются на использовании мнений нескольких экспертов. Эксперты, производящие первоначальную оценку проекта, обычно знают о вышеперечисленных потенциальных рисках и потому стараются учесть все риски в самих временных оценках. Обычно применяется вариант интервальной неопределённости, т.\,н. <<вилка>>~--- оценка наиболее пессимистичного и наиболее оптимистичного вариантов разработки. Однако у этого варианта есть несколько существенных недостатков:
\begin{itemize}
  \item с психологической точки зрения, заказчик <<цепляется>> за нижнюю границу и задаёт вопрос <<А почему так долго и дорого?>> относительно верхней оценки, вынуждая руководителя проекта надавливать на экспертов с целью получения более оптимистичной верхней оценки;
  \item существует негласное правило <<умножения на три>>, которое применяется руководителем проекта при анализе оценок, полученных от экспертов: для каждого этапа проекта из полученных оценок выбирается самая пессимистичная и утраивается~--- в результате неопределённость учитывается дважды; 
  \item эксперт, коим обычно является инженер-программист, исходит из своего опыта и оценки собственной производительности, поэтому оценка может оказаться чрезмерно персонализированной~--- другой исполнитель задания просто не успеет выполнить запланированные работы в срок.
\end{itemize}

В связи с этим логичным выглядит получение оценок в виде нечётких треугольных чисел, поскольку эксперт на основании своего опыта может оценить, сколько времени обычно занимает некоторый этап разработки, и сколько времени он занимал в наилучшем и наихудшем случаях. Учитывая общую склонность программистов к <<перестраховке>>, оценки обычно получаются асимметричными.

Рассмотрим следующий случай. В фирму по разработке программного обеспечения обращается заказчик с просьбой оценить срок и примерную стоимость реализации комплексного решения~--- веб-портала и двух мобильных приложений под наиболее популярные платформы~--- приуроченного к празднованию круглой даты одного из государственных праздников.

Для оценки продолжительности проекта собирается команда экспертов, состоящая из руководителя проекта, графического дизайнера, разработчика серверных приложений и двух разработчиков мобильных приложений.