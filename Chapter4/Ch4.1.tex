Перспективность данного направления исследований заключается в преимуществах теории нечётких множеств при обработке нечётких данных, которыми изобилует реальная практика разработки ПО в сфере аутсорсинга. Действительно, характерными чертами процесса проектирования и разработки программного обеспечения являются:
\begin{itemize}
  \item относительная уникальность разрабатываемого проекта~--- у некоторых участников может быть опыт участия в схожих проектах, однако он нивелируется тем, что остальная часть команды обычно впервые сталкивается с конкретной задачей автоматизации в некоторой предметной области;
  \item частое отсутствие полных требований к программному продукту либо необходимых для разработки ресурсов (согласованные и утверждённые дизайны, переводы и т.д.);
  \item разный уровень подготовки членов команды разработчиков;
  \item возможность изменения списка работ на поздних стадиях разработки~--- либо увеличение списка требований к продукту, либо его сокращение с целью ускорить разработку и приурочить выпуск готового продукта к некоторому событию;
  \item бюрократическая составляющая~--- затягивание сроков разработки из-за проблем в согласовании изменений или списка работ;
\end{itemize}

В связи с~этим в настоящее время применяемые при планировании разработки ПО методы в основном являются неформальными и базируются на использовании мнений нескольких экспертов. Эксперты, производящие первоначальную оценку проекта, обычно знают о вышеперечисленных потенциальных рисках и~потому стараются учесть все риски в самих временных оценках. Чаще всего применяется вариант интервальной неопределённости, т.\,н. <<вилка>>~--- оценка наиболее пессимистичного и наиболее оптимистичного вариантов разработки. Однако у этого варианта есть несколько существенных недостатков:
\begin{itemize}
  \item обычно заказчики ориентируются на нижнюю границу оценки и не уточняют список потенциальных факторов риска, которые приводят к достижению верхней границы интервала, при этом вынуждая руководителя проекта надавливать на экспертов с целью получения более оптимистичной и точной оценки;
  \item существует негласное правило <<умножения на три>>, которое применяется руководителем проекта при анализе оценок, полученных от экспертов: для каждого этапа проекта из полученных оценок выбирается самая пессимистичная и утраивается~--- в результате неопределённость учитывается дважды; 
  \item эксперт, коим обычно является инженер-программист, исходит из своего опыта и оценки собственной производительности, поэтому оценка может оказаться чрезмерно персонализированной~--- другой исполнитель задания просто не успеет выполнить запланированные работы в срок.
\end{itemize}

В связи с этим логичным выглядит получение оценок в виде нечётких треугольных чисел, поскольку эксперт на основании своего опыта может оценить, сколько времени обычно занимает некоторый этап разработки, и сколько времени он занимал в наилучшем и наихудшем случаях. Учитывая общую склонность инженеров-программистов к пессимистическому оцениванию времени разработки, оценки обычно получаются асимметричными. Применение к оценкам преобразования L позволяет получить модифицированные значения оценок, носители которых могут быть в дальнейшем интерпретированы как нечёткие интервалы, а тип (LL или RR)~--- как степень пессимизма или оптимизма. 

Рассмотрим следующий случай. В фирму по разработке программного обеспечения обращается заказчик с просьбой оценить срок и примерную стоимость реализации комплексного решения~--- веб-портала и двух мобильных приложений под наиболее популярные платформы~--- приуроченного к празднованию круглой даты одного из государственных праздников. Для оценки продолжительности проекта собирается команда экспертов, состоящая из руководителя проекта, графического дизайнера, разработчика серверных приложений и двух разработчиков мобильных приложений. Требуется предоставить заказчику <<вилку>> по времени и указать возможные факторы риска незавершения проекта в срок.

В~cтандартный процесс вносятся некоторые изменения. Как и ранее, эксперты-разработчики формулируют интервальные оценки для всех операций проекта, но при этом их дополнительно просят 


Ещё одной проблемой является необходимость создания алгоритма преобразования вершинного графа в стрелочный...