\textit{Средства разработки}. В качестве средства разработки применяется интегрированная среда Microsoft Visual Studio 2010. Эта среда разработки поддерживает компонентную технологию, позволяет легко создавать собственные компоненты и интегрироваться с внешними модулями ПО и офисными пакетами.

\textit{Требования к аппаратному и программному обеспечению}. Для нормального функционирования программы необходимы следующие программные и технические средства:
\begin{itemize}
  \item Intel-совместимый процессор с тактовой частотой не менее 1 ГГц;
  \item объём оперативной памяти~--- 1 ГБ и более;
  \item свободное место на жёстком диске~--- 100 МБ и более;
  \item операционная система~--- Windows 7 и выше;
  \item предустановленная среда выполнения .NET Framework Client Profile версии не ниже 4.0;
  \item предустановленный пакет визуализации Graphviz версии не ниже 2.28;
  \item предустановленный офисный пакет Microsoft Office версии не ниже 14 (Office 2010).
\end{itemize}

\textit{Условия применения программы.} Разработанная программа решает научно-исследовательскую и прикладную задачу. Программа распространяется свободно и без каких-либо ограничений и гарантий.

\subsection{Программный модуль CSBusinessGraph}

\subsubsection*{Назначение программы}

Разработанная программа предназначена для проведения вычислительного эксперимента, цель которого заключалась в выполнении сетевого анализа проекта по разработке программного обеспечения

\subsubsection*{Интерфейс пользователя}
Программа имеет интуитивно понятный интерфейс, представленный на рис.

Окна для вывода данных представлены на рис.

При нажатии на форме кнопки....

Главное окно состоит из элементов, представленных на рис....

Скриншоты из диплома

\subsubsection*{Функциональные возможности программы}
Программа имеет следующие функциональные возможности:
\begin{itemize}
  \item создание модели проекта в виде вершинного графа в ручном режиме или импорт существующей модели из XML-файла;
  \item поддержка модели проекта в согласованном состоянии~--- проверка отсутствия циклов в графе и наличие только одной компоненты связности;
  \item редактирование временных оценок выполнения операций, выраженных в~виде треугольных чисел, c~помощью изменения параметров нечёткго числа;
  \item графическое представление функций принадлежности для~временных оценок;
  \item автоматическое преобразование вершинного графа в~стрелочный;
  \item выбор пользователем используемых при~расчётах значений параметров $\lambda$ преобразования L, соответствующих середине $\alpha$-сечения, проекции центра тяжести или оптимальному в смысле сохранения экспертной информации значению;
  \item реализация механизма расчёта критического пути на~основе $\alpha$-уровневых и~двухточечных вычислений;
  \item экспорт отчёта о~решении задачи в~формат Microsoft Excel с~формированием графиков для модифицированных нечётких оценок, общего времени выполнения проекта и~построением преобразованного стрелочного графа c~выделением критических операций.
\end{itemize}

\subsubsection*{Реализация}

\textit{Структура данных}

Входными данными для механизма построения вял

Тут про импорт из файла, reflection и конвертацию всего во вся

\textit{Логическая структура программы}

Программа состоит из нескольких модулей. Схема взаимодействия между модулями программы показана на рис.

В программе были использованы третьесторонние компоненты, позволившие сократить общее время разработки и упростить тестирование~--- библиотека для работы с графами QuickGraph и~компонент для визуального представления и редактирования графа NetworkView.

\subsubsection*{Тестирование приложения}


Таблица с тестами и анализом результатов