%%% Макет страницы %%%
%\geometry{a4paper,top=2cm,bottom=2cm,left=2.5cm,right=1cm}
\setmarginsrb{3cm}{2cm}{1cm}{2cm}{0pt}{0mm}{0pt}{10mm} % поля и колонтитулы

%%% Кодировки и шрифты %%%
\renewcommand{\rmdefault}{ftm} % Включаем Times New Roman

%%% Выравнивание и переносы %%%
\parindent=1.25cm		% Отступ параграфа (красная строка)
\sloppy					% Избавляемся от переполнений
\clubpenalty=10000		% Запрещаем разрыв страницы после первой строки абзаца
\widowpenalty=10000		% Запрещаем разрыв страницы после последней строки абзаца

% Абзацные настройки
\renewcommand{\baselinestretch}{1.5}

%%% Библиография %%%
\makeatletter
%\bibliographystyle{bst/utf8gost705s}	% Оформляем библиографию в соответствии с ГОСТ 7.0.5
\bibliographystyle{ugost2008ls}
\renewcommand{\@biblabel}[1]{#1.}	% Заменяем библиографию с квадратных скобок на точку:
\makeatother

%%% Изображения %%%
\graphicspath{{images/}} % Пути к изображениям

%%% Цвета гиперссылок %%%
\definecolor{linkcolor}{rgb}{0.9,0,0}
\definecolor{citecolor}{rgb}{0,0.6,0}
\definecolor{urlcolor}{rgb}{0,0,1}
\hypersetup{
    colorlinks, linkcolor={linkcolor},
    citecolor={citecolor}, urlcolor={urlcolor}
}

%%% Оглавление %%%
\setcounter{tocdepth}{1}

\renewcommand{\cftchapdotsep}{\cftdotsep}
% \renewcommand{\cftsecleader}{\bfseries\cftdotfill{\cftdotsep}}
% \renewcommand{\cftchappresnum}{Глава }
\renewcommand{\cftchapaftersnum}{.}
\renewcommand{\cftsecpresnum}{}
\renewcommand{\cftsecaftersnum}{.}
\renewcommand{\cftsubsecaftersnum}{.}

\renewcommand{\thesection}{\arabic{chapter}.\arabic{section}}
\renewcommand{\thesubsection}{\arabic{chapter}.\arabic{section}.\arabic{subsection}}

%%% Формат заголовков глав %%%
\titleformat{\chapter}[block]{\bfseries\Large\center}{Глава \arabic{chapter}.}{1ex}{}
\titleformat{\section}[block]{\bfseries\large}{\arabic{chapter}.\arabic{section}.}{1ex}{}
\titleformat{\subsection}[block]{\bfseries\normalsize}{\arabic{chapter}.\arabic{section}.\arabic{subsection}.}{1ex}{}
\titleformat{\subsubsection}[block]{\bfseries\small}{\arabic{chapter}.\arabic{section}.\arabic{subsection}.\arabic{subsubsection}.}{1ex}{}

\titlespacing*{\chapter}		 {0pt}{1ex plus 1ex minus .2ex}{2.0ex plus .2ex}
\titlespacing*{\section}      {0pt}{3.50ex plus 1ex minus .2ex}{1.5ex plus .2ex}
\titlespacing*{\subsection}   {0pt}{3.25ex plus 1ex minus .2ex}{1.0ex plus .2ex}
\titlespacing*{\subsubsection}{0pt}{3.25ex plus 1ex minus .2ex}{1.0ex plus .2ex}

%%% Математика %%%
\newtheorem{mydef}{Определение}[chapter]
\newtheorem{prop}{Свойство}[chapter]
\newtheorem{theorem}{Теорема}[chapter]
\newtheorem{cor}{Следствие}[chapter]

%%% Нумерация и названия таблиц и рисунков %%%

\def\capfigure{figure}
%\def\captable{table}
\long\def\@makecaption#1#2
{%
  \vskip\abovecaptionskip
  \ifx\@captype\capfigure
      \centering #1~--~#2 \par
  \else
      #1~--~#2 \par
  \fi
  \vskip\belowcaptionskip
}

% Russian-styled figure and table captions
\captionsetup[figure]{font=small}
\captionsetup[table]{font=normal, justification=raggedleft}
