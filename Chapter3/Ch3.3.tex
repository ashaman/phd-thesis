\begin{enumerate}
  \item рассматривается классическая задача сетевого планирования и её нечёткий случай, когда оценки продолжительности операций получены от экспертов и выражаются нечёткими треугольными числами. Приводятся определения, используемые в дальнейшем. Формулируется задача поиска критического пути в нечётких условиях как задача линейного программирования;
  \item анализируются существующие семейства способов решения рассматриваемой задачи. На примере нескольких методов, описанных в отечественной и зарубежной литературе, демонстрируются недостатки способов решения, использующих принцип обобщения Заде и арифметики нечётких чисел. В качестве наиболее универсального способа выделяется $\alpha$-уровневое или основанное на функциях ранжирования решение, позволяющее перейти к совокупности чётких задач линейного программирования;
  \item на основании полученных в главе~\ref{chapter2} результатов, предлагается алгоритм решения задачи сетевого планирования с нечёткими временными оценками как двух чётких задач линейного программирования, позволяющий получать устойчивое в смысле определений, введённых в главе~\ref{chapter2}, решение. Показывается необходимость применения аддитивной свёртки критериев в целевой функции возмущённой задачи линейного программирования как средства управления устойчивостью и~предотвращения потерь экспертной информации;
  \item в конце главы рассматривается пример применения предложенного алгоритма. Полученное модифицированное решение сравнивается с решениями, найденными другими описанными ранее способами.
\end{enumerate}