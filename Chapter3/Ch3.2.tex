Подробное описание примера применения разработанных методов к задаче сетевого планирования: поиск критического пути, распределение 
ресурсов и решение проблемы устойчивости.

Для решения задачи воспользуемся вводимой в~п.\ref{chapter2_3} алгеброй модифицированных нечётких чисел. Для начала сформулируем задачу для произвольного $\alpha$-уровня. Преобразование $L$, применяемое в данной задаче, несколько отличается от вводимого в главе~\ref{chapter2} формулой~\eqref{} (1.45), поскольку параметры $\lambda$ необходимо изменять, управляя, таким образом, устойчивостью задачи линейного программирования
\begin{equation}
\label{eq:times-l-transform}
  L(\tau_i(\alpha ))=\bar{\tau}_i\left(\alpha, \lambda_{\tau_i}\right)=\lambda_{\tau_i}\tau_{i}^{L}\left(\alpha \right)+(1-\lambda_{\tau_i})\tau_{i}^{R}\left(\alpha \right).
\end{equation}

В формуле \eqref{eq:times-l-transform} границы числа $\tilde \tau_i$ определяются согласно~\eqref{}:
\begin{equation*}
  \left[ \begin{aligned}
    & \tau_{i}^{L}\left(\alpha \right)=m_{\tilde \tau_i}-a_{\tilde \tau_i}+a_{\tilde \tau_i}\alpha; \\ 
    & \tau_{i}^{R}\left(\alpha \right)={{m}_{{{{\tilde{\tau }}}_{i}}}}+{{b}_{{{{\tilde{\tau }}}_{i}}}}-{{b}_{{{{\tilde{\tau }}}_{i}}}}\alpha  \\ 
  \end{aligned} \right.
\end{equation*}

В результате, на~каждом из~$\alpha$-уровней задача линейного программирования будет выглядеть следующим образом:
\begin{equation}
\label{eq:modified-fcmp-lp}
  \left\{ \begin{aligned}
    & T(\alpha )=t_n-t_1\to \min  \\ 
    & t_{j_s}-t_{i_s}\geqslant \bar{\tau}_s\left(\alpha,\lambda_s \right),\ \forall s=\overline{1,m}.
  \end{aligned} \right.
\end{equation}

Результатом решения задачи~\eqref{eq:modified-fcmp-lp} является вектор времён $t\left( \alpha \right)=\left\{ t_0\left(\alpha, \lambda_0\right),\ldots,t_n\left(\alpha, \lambda_n\right) \right\}$, который является календарным планом $\alpha$-уровня, а~также множество критических операций $S_1\left( \alpha \right)$~\cite{Vorontsov_VSTU}. 

Нечеткость оценок $\tilde{\tau}_i$ обуславливает проблему устойчивости решений задачи~\eqref{eq:modified-fcmp-lp} в~смысле~\eqref{eq:fuzzy-solution-stability}. Для неустойчивой задачи на различных $\alpha $-уровнях решения соответствуют различным критическим путям и возникает проблема объединения разнородных $\alpha$-уровневых решений $S_1\left(\alpha \right)$. Очевидно, что задача будет устойчива, если критический путь не меняется при переходе с~одного $\alpha$-уровня на~другой. Поэтому в данной ситуации целесообразно исследовать задачу~\eqref{eq:modified-fcmp-lp} на~устойчивость и выяснить, при каких значениях параметра $\lambda$ критический путь одинаков на~всех $\alpha$-уровнях~\cite{Vorontsov_VSTU}. 

При решении задачи~\eqref{eq:modified-fcmp-lp}, будем использовать параметр $\displaystyle \lambda_{s}^{*}=\frac{a_s}{a_s+b_s}$, оптимальный в смысле сохранения максимального количества характеристик нечёткого числа.

Согласно данному ранее определению~\eqref{eq:fuzzy-solution-stability}, задачу~\eqref{eq:modified-fcmp-lp} поиска критического пути на $\alpha$-уровне будем считать устойчивой по~решению, если она устойчива и если $S_1$ не~зависит от $\alpha$, т.\,е. на всех $\alpha$-уровнях критический путь не изменяется и~проходит по~одним и~тем~же рёбрам. Условие устойчивости выполняется автоматически, поскольку в~сетевом графике всегда существует хотя бы один путь от истока к стоку, следовательно, независимо от величин весов рёбер, всегда существует путь максимальной длины~\cite{Kormen}. Что~касается устойчивости по~решению, то, поскольку результатом решения задачи о критическом пути является множество номеров критических операций, целесообразно выбрать в качестве метрики сходства решений мощность симметрической разности двух множеств. Очевидно, что если для $\alpha_1, \alpha_2 \in \left[ 0;1 \right];\ \alpha_1\ne \alpha_2$ $S_1\left( \alpha_1 \right)=S_1\left(\alpha_2 \right)$, то мощность симметрической разности этих множеств $S_1\left( \alpha_1 \right)\Delta S_2\left(\alpha_2 \right)$ равна нулю, т.\,е. критические пути на разных $\alpha$-уровнях совпадают.

Таким образом, задачу поиска критического пути на $\alpha $-уровне будем считать устойчивой по решению, если она устойчива и если 
\begin{equation*}
  \forall \alpha_1, \alpha_2\in \left[ 0;1 \right];\ \alpha_1\ne \alpha_2\ S_1\left(\alpha_1 \right)\Delta S_2\left(\alpha_2 \right)=\varnothing,
\end{equation*}
т.\,е. на всех $\alpha $-уровнях критический путь не~изменяется и~проходит по~одним и~тем~же дугам. 

В~п.\ref{chapter2_3} было показано, что для получения функции принадлежности модифицированного нечёткого числа, достаточно знать его значения в двух точках – при $\alpha=0$ и $\alpha=1$. На этом основывается упрощённый способ вычислений

причём будем вести расчёты только на двух $\alpha$-уровнях – при $\alpha=0$ и $\alpha=1$.

В~качестве невозмущённой задачи возьмём исходную задачу~\eqref{eq:modified-fcmp-lp} при $\alpha=1$. Возмущения возникают при изменении $\alpha$ и~$\lambda$, поскольку в~этом случае меняется чёткое значение числа на~каждом $\alpha$-срезе. На устойчивость задачи можно влиять, изменяя параметры $\lambda$ для каждого из чисел. Поскольку преобразование L является линейным, то для дальнейшего анализа задачи~\eqref{eq:modified-fcmp-lp} на~устойчивость достаточно решить её при~$\alpha=1$, а~затем, зафиксировав полученный критический путь с~помощью изменения части ограничений на строгие равенства для~соответствующих~операций, проверить существование решения при $\alpha=0$ и степень его~совпадения с~решением при $\alpha=1$.

АЛГОРИТМ РЕШЕНИЯ ЗДЕСЬ!