Подробное описание примера применения разработанных методов к задаче сетевого планирования: поиск критического пути, распределение 
ресурсов и решение проблемы устойчивости.

Для решения задачи воспользуемся вводимой в~п.~\ref{chapter2_3} алгеброй модифицированных нечётких чисел. Для начала сформулируем задачу для произвольного $\alpha$-уровня. Преобразование $L$, применяемое в данной задаче, несколько отличается от вводимого в главе 2 формулой (1.45), поскольку параметры $\lambda$ необходимо изменять, управляя, таким образом, устойчивостью задачи линейного программирования
	\[L({{\tau }_{i}}(\alpha ))={{\bar{\tau }}_{i}}(\alpha ,{{\lambda }_{{{\tau }_{i}}}})={{\lambda }_{{{\tau }_{i}}}}\tau _{i}^{L}(\alpha )+(1-{{\lambda }_{{{\tau }_{i}}}})\tau _{i}^{R}(\alpha )\] 	(1.132)

В формуле (1.132)
	\[\left[ \begin{aligned}
  & \tau _{i}^{L}(\alpha )={{m}_{{{{\tilde{\tau }}}_{i}}}}-{{a}_{{{{\tilde{\tau }}}_{i}}}}+{{a}_{{{{\tilde{\tau }}}_{i}}}}\alpha  \\ 
 & \tau _{i}^{R}(\alpha )={{m}_{{{{\tilde{\tau }}}_{i}}}}+{{b}_{{{{\tilde{\tau }}}_{i}}}}-{{b}_{{{{\tilde{\tau }}}_{i}}}}\alpha  \\ 
\end{aligned} \right.\] 	(1.133)

В результате, на каждом из $\alpha$-уровней задача линейного программирования будет выглядеть следующим образом:
	\[\left\{ \begin{aligned}
  & T(\alpha )={{t}_{n}}-{{t}_{1}}\to \min  \\ 
 & {{t}_{{{j}_{s}}}}-{{t}_{{{i}_{s}}}}\ge {{{\bar{\tau }}}_{s}}(\alpha ,{{\lambda }_{s}}),\ \forall s=\overline{1,m} \\ 
\end{aligned} \right.\] 	(1.134)

Результатом решения задачи (1.134) является вектор времён $t\left( \alpha  \right)=\left\{ {{t}_{0}}(\alpha ,{{\lambda }_{0}}),...,{{t}_{n}}(\alpha ,{{\lambda }_{n}}) \right\}$, который является календарным планом $\alpha$-уровня, а также множество критических операций ${{S}_{1}}\left( \alpha  \right)$. 

Нечеткость оценок ${{\tilde{\tau }}_{i}}$ обуславливает проблему устойчивости решений задачи (1.134) в смысле (1.118). Для неустойчивой задачи на различных $\alpha $-уровнях решения соответствуют различным критическим путям и возникает проблема объединения разнородных $\alpha $-уровневых решений ${{S}_{1}}\left( \alpha  \right)$. Очевидно, что задача будет устойчива, если критический путь не меняется при переходе с одного $\alpha $-уровня на другой. Поэтому в данной ситуации целесообразно исследовать задачу (1.134) на устойчивость и выяснить, при каких значениях параметра $\lambda $ критический путь одинаков на всех $\alpha $-уровнях. 

При решении задачи (1.134), будем использовать параметр \[\lambda _{s}^{*}=\frac{{{a}_{s}}}{{{a}_{s}}+{{b}_{s}}}\], оптимальный в смысле сохранения максимального количества характеристик нечёткого числа.

Согласно данному ранее определению (1.118), задачу (1.134) поиска критического пути на $\alpha$-уровне будем считать устойчивой по решению, если она устойчива и если $S_1$ не зависит от $\alpha$, т.е. на всех $\alpha$-уровнях критический путь не изменяется и проходит по одним и тем же рёбрам. Условие устойчивости выполняется автоматически, поскольку в сетевом графике всегда существует хотя бы один путь от истока к стоку, следовательно, независимо от величин весов рёбер, всегда существует путь максимальной длины [Кормен]. Что касается устойчивости по решению, то, поскольку результатом решения задачи о критическом пути является множество номеров критических операций, целесообразно выбрать в качестве метрики сходства решений мощность симметрической разности двух множеств. Очевидно, что если для ${{\alpha }_{1}},{{\alpha }_{2}}\in \left[ 0;1 \right];\ {{\alpha }_{1}}\ne {{\alpha }_{2}}$ ${{S}_{1}}\left( {{\alpha }_{1}} \right)={{S}_{1}}\left( {{\alpha }_{2}} \right)$, то мощность симметрической разности этих множеств ${{S}_{1}}\left( {{\alpha }_{1}} \right)\Delta {{S}_{2}}\left( {{\alpha }_{2}} \right)$ равна нулю, т.е. критические пути на разных $\alpha$-уровнях совпадают.

Таким образом, задачу поиска критического пути на $\alpha $-уровне будем считать устойчивой по решению, если она устойчива и если 
	$\forall {{\alpha }_{1}},{{\alpha }_{2}}\in \left[ 0;1 \right];\ {{\alpha }_{1}}\ne {{\alpha }_{2}}\ {{S}_{1}}\left( {{\alpha }_{1}} \right)\Delta {{S}_{2}}\left( {{\alpha }_{2}} \right)=\varnothing $ 	(1.135)
т.е. на всех $\alpha $-уровнях критический путь не изменяется и проходит по одним и тем же дугам. 
В п.~\ref{chapter2_3} было показано, что для получения функции принадлежности модифицированного нечёткого числа, достаточно знать его значения в двух точках – при $\alpha=0$ и $\alpha=1$. На этом основывается упрощённый способ вычислений

причём будем вести расчёты только на двух $\alpha$-уровнях – при $\alpha=0$ и $\alpha=1$.

В качестве невозмущённой задачи возьмём исходную задачу (1.134) при $\alpha =1$. Возмущения возникают при изменении $\alpha$ и $\lambda$, поскольку в этом случае меняется чёткое значение числа на каждом $\alpha$-срезе. На устойчивость задачи можно влиять, изменяя параметры $\lambda$ для каждого из чисел. Поскольку преобразование L является линейным, то для дальнейшего анализа задачи (1.134) на устойчивость достаточно решить её при $\alpha =1$, а затем, зафиксировав полученный критический путь с помощью изменения части ограничений на строгие равенства для соответствующих операций, проверить существование решения при $\alpha =0$ и степень его совпадения с решением при $\alpha =1$.

АЛГОРИТМ РЕШЕНИЯ ЗДЕСЬ!