\subsection{Постановка и способы решения чёткой задачи сетевого планирования}

Рассмотрим направленный ациклический граф (DAG, от directed acyclic graph, согласно терминологии~\cite{Kormen}) $G=(V,E)$, $\left| V \right|=n$, $\left| E \right|=m$, обладающий следующими свойствами:
\begin{itemize}
  \item существует ровно одна вершина $v_1\in V$, называемая истоком, из которой рёбра только исходят, т.е. $\forall i=2,n$ $\nexists \left( v_i, v_1 \right)$ ;
  \item существует ровно одна вершина $v_n\in V$, называемая стоком, в которую рёбра графа только входят, т.е.  $\forall i=\overline{1,n-1}$ $\nexists \left( v_n, v_i \right)$;
  \item для любой вершины графа $v_i\in V,\ i=\overline{1,n}$ существует путь $v_1\ldots v_n$, проходящий через неё;
  \item для любого ребра $e_j\in E,\ j=\overline{1,m}$ существует путь $v_1\ldots v_n$, содержащий это ребро.
\end{itemize}

В~задачах календарно-сетевого планирования и управления~(КСПУ) граф, удовлетворяющий перечисленным выше условиям, называется сетевым графиком, и часто применяется для определения продолжительности проекта, состоящего из набора зависящих друг от друга операций~\cite{Kosorukov_Mischenko, Eddous, Taha_Operation_Research, Balashov_IPU}. Обычно работам проекта $w_j$, длительностью $\tau_j$ каждая, сопоставлены дуги графа $e_j$, $j=\overline{1,m}$. Событиям проекта $z_i$ с~временами наступления $t_i$ сопоставления вершины графа $v_i$, $i=\overline{1,n}$. Событие $z_1$~--- начало работ по проекту, событие $z_n$~--- окончание проекта. События проекта (вершины графа) обычно нумеруются, а операции (рёбра) обозначаются словами или буквами латинского алфавита.

Нетрудно заметить, что общее время выполнения проекта $T$ будет равно длине максимального пути в графе, называемого также критическим. Соответственно, операции, которые принадлежат пути максимальной длины, также будут называться критическими. Изменение длительности любой из них приведёт к изменению общего времени выполнения проекта. Остальные операции являются некритическими и характеризуются т.н. резервом времени, т.е. максимальной задержкой по срокам выполнения, при которой общее время проекта не изменится.

Как указано в~\cite{Balashov_IPU}, для сетевого графика всегда существует <<правильная>> нумерация, т.\,е. такая, при которой из вершины с большим порядковым номером не идут дуги в вершину с меньшим порядковым номером. В дальнейшем будем считать, что события проекта перенумерованы так, что нумерация является «правильной».

Для каждого события проекта в~\cite{Eddous, Taha_Operation_Research, Balashov_IPU} определяются следующие понятия.
\begin{mydef}
  Величина $t_{i}^{-}$ называется наиболее ранним моментом наступления события $z_i$ и характеризует момент времени, раньше которого наступление $z_i$ невозможно.
\end{mydef}

\begin{mydef}
  Величина $t_{i}^{+}$ называется наиболее поздним моментом наступления события $z_i$ и характеризует максимальное время наступления $z_i$, при котором общая продолжительность проекта не меняется.
\end{mydef}

\begin{mydef}
  Полным резервом времени для события $z_i$ называется разность наиболее позднего и наиболее раннего моментов его наступления, т.\,е. величина
\begin{equation}
\label{eq:full-event-reserve}
  \Delta t_i=t^+_i-t^-_i.
\end{equation}
\end{mydef}

Для проектных работ вводится аналогичное определение.
\begin{mydef}
  Полным резервом времени работы $w_s$, которая начинается при наступлении события $z_{i_s}$ и заканчивается событием $z_{j_s}$, называется величина
\begin{equation}
\label{eq:full-work-reserve}
  \Delta \tau_s=t_{j_s}^{+}-t_{i_s}^{-}-\tau_s.
\end{equation}
\end{mydef}
	
События, у которых полный резерв времени равен нулю, являются потенциально критическими. Соответственно, работа $w_s$ будет считаться критической, если её начальное и конечное событие потенциально критические, а~её общий~резерв времени равен нулю, т.е. одновременно выполняются условия, описываемые формулами~\eqref{eq:full-event-reserve} и~\eqref{eq:full-work-reserve}:
\begin{equation}
\label{eq:critical-work-def}
  \left\{ \begin{aligned}
    & \Delta t_{i_s}=0; \\ 
    & \Delta t_{j_s}=0; \\ 
    & \Delta \tau_s=0.
  \end{aligned} \right.
\end{equation}

Задача поиска критического пути в~графе $G$ может быть решена различными способами, наиболее простым из которых является алгоритмический. Алгоритм поиска критического пути представляет собой алгоритм Дейкстры для~нахождения кратчайшего пути в графе с изменённой функцией релаксации~\cite{Kormen, Indians_CPM}, которая позволяет искать вместо самого короткого пути самый длинный. Каждая вершина графа снабжается двумя метками – временем наиболее раннего наступления события $t_{i}^{-}$ и временем наиболее позднего наступления события $t_{i}^{+}$. В начале работы алгоритма у всех вершин $t_{i}^{-}=0$, $t_{i}^{+}=\infty $. На каждом шаге для вершины $v_i;\ i=\overline{2,n}$ уточняется её $t_{i}^{-}$. Для этого вначале определяются вершины $\left\{ v_j \right\}$, непосредственно предшествующие $v_i$, т.\,е. такие, что существует дуга $e_{ji}$ между $v_j$ и $v_i$. После этого находится $t_{i}^{-}$ путём выбора максимума из сумм $t^{-}$ непосредственно предшествующих событий и длительностей операций, которые связывают каждую из $v_j$ с~$v_i$:
\begin{equation*}
  t_{i}^{-}=\underset{v_j}{\mathop{\max }}\,\left\{ t_{j}^{-}+\tau_{ji} \right\}.
\end{equation*}

Процесс повторяется до тех пор, пока у всех событий не будут рассчитаны наиболее ранние сроки их наступления. $t_{n}^{-}$ вершины-стока и будет общим временем выполнения проекта. Другими словами,
\begin{equation}
\label{eq:total-project-time}
  T=\underset{i=\overline{1,n}}{\mathop{\max }}\,\left\{ t_{i}^{-} \right\}.
\end{equation}

Псевдокод алгоритма приведён ниже.

Если необходимо, помимо общей длительности проекта, отыскать и критический путь, то по графу выполняется обратный проход, во время которого вычисляются наиболее поздние сроки наступления события. Для вершины-стока принимается, что $t_{n}^{+}=t_{n}^{-}$. Для остальных вершин $v_i;\ i=\overline{n-1,1}$ на каждом шаге вначале находятся непосредственно следующие за ними события $\left\{ v_k \right\}$, т.е. такие, что существует дуга $e_{ik}$ между $v_i$ и $v_k$, после чего ищется $t_{i}^{+}$ в виде
\begin{equation*}
  t_{i}^{+}=\underset{v_k}{\mathop{\min }}\,\left\{ t_{k}^{+}-{{\tau }_{ik}} \right\}.
\end{equation*}
Критические операции проекта определяются согласно~\eqref{eq:critical-work-def}.

На рис. изображён сетевой граф проекта после выполнения вышеописанного алгоритма. Для каждого события проекта указан его номер, наиболее ранний и наиболее поздний сроки его наступления. Пунктиром выделены так называемые фиктивные стрелки (операции) с нулевой длительностью, которые необходимы для того, чтобы показать зависимость между операциями, не нарушая свойств направленного ациклического графа. Критический путь выделен жирным.

// Дать определение устойчивой задачи КСПУ

К~недостаткам алгоритмического решения стоит отнести трудность исследования задачи на~устойчивость, поскольку результат алгоритмически, а не~аналитически, зависит от~входных данных. Для~исследования на~устойчивость больше подходит решение задачи методами линейного программирования. Классическая постановка этой задачи дана в~\cite{Kosorukov_Mischenko}. Требуется решить задачу
\begin{equation}
\label{eq:crisp-lp-cpm-task}
  T=t_n-t_1 \to \min
\end{equation}
при ограничениях на времена наступления событий
\begin{equation}
\label{eq:crisp-lp-cpm-restrictions}
  t_{j_s}-t_{i_s}\geqslant \tau_s;\ s=\overline{1,m},
\end{equation}
где $t_{i_s}$ и $t_{j_s}$ – времена наступления событий начала и окончания работы $w_s$ соответственно. Задача~\eqref{eq:crisp-lp-cpm-task} при ограничениях~\eqref{eq:crisp-lp-cpm-restrictions} является задачей линейного программирования.

В~результате решения данной задачи получается общее время выполнения проекта $T$, а также вектор времён $\mathbf{t}=\left\{ t_1, \ldots, t_n \right\}$, называемых календарным планом проекта, и~совокупность критических операций $\mathbf{S}_1$ с нулевым общим резервом времени: $\Delta \tau_{s_1}=0$, $\forall s_1 \in S_1$.

\subsection{Классификация методов решения задачи сетевого планирования с нечёткими параметрами}

Рассмотрим сформулированные выше задачи~\eqref{eq:total-project-time} и~\eqref{eq:crisp-lp-cpm-task} в~случае, когда для временных оценок длительностей операций используются нечёткие треугольные числа $\tilde \tau_i$. Решению задачи календарно-сетевого планирования и управления с нечёткими параметрами посвящено немало публикаций в отечественной и зарубежной литературе, однако в большинстве случаев методы её решения можно разбить на три группы:
\begin{enumerate}
  \item оперирующие нечёткими числами в неизменной форме с использованием принципа обобщения. К этим методам можно отнести, например, описанные в~\cite{Balashov_IPU, Chanas_Zielinski_FCPM};
  \item вводящие операции сложения, вычитания и сравнения для нечётких чисел. Подобные методы изложены в~\cite{Uskov_FCPM, Leondes};
  \item использующие дефаззификацию или получение чётких $\alpha$-уровневых значений. Такие методы, описанные в~\cite{Egyptians, Indians_FCPM}, обычно основываются на линейном программировании.
\end{enumerate}

Рассмотрим методы первой группы на примере решения задачи поиска критического пути в~\cite{Balashov_IPU}. Пусть заданы функции принадлежности для нечётких~оценок продолжительности операций $\mu_{\tilde \tau_s}\left( x \right)$. Если за $Q_i$ обозначить множество номеров вершин, непосредственно предшествующих вершине $v_i$, a за $R_i$~--- множество номеров вершин, непосредственно следующих за $v_i$, то, согласно принципу обобщения Заде, функция принадлежности нечёткого наиболее раннего срока наступления события $z_i$ имеет вид
\begin{equation}
\label{eq:fcpm-zadeh-earliest}
  \mu_{\tilde{t}_{i}^{-}}\left( x \right)=\underset{\left\{ \left( x_{ji} \right),j\in Q_i \left| \max \left( x_j+x_{ji} \right)=x \right. \right\}}{\mathop{\max }}\,\min \left[ \underset{j\in Q_i}{\mathop{\min }}\,\left( \mu_{\tilde{t}_{ji}^{-}}\left( x_{ji} \right) \right);{\mu_{\tilde{t}_{j}^{-}}}\left(x_j \right) \right].
\end{equation}
Функция принадлежности наиболее позднего срока наступления вычисляется следующим образом:
\begin{equation}
  \mu_{\tilde{t}_{i}^{+}}\left( x \right)=\underset{\left\{ \left( T,x_i \right)\left| T-x_i=x \right. \right\}}{\mathop{\max }}\,\underset{i=\overline{1,n}}{\mathop{\min }}\,\left[ \mu_{\tilde T}\left( T \right);\mu_{\tilde l_i}\left(x_i \right) \right],
\end{equation}
где $\tilde l_i$~--- нечёткая максимальная длина пути от вершины $v_i$ до стока $v_n$, определяемая через функцию принадлежности
\begin{equation}
  \mu_{\tilde l_i}\left( x \right)=\underset{\left\{ \left(x_{ij} \right),j\in R_i\left| \max \left( x_j+x_{ji} \right)=x \right. \right\}}{\mathop{\max }}\,\min \left[ \underset{j\in R_i}{\mathop{\min }}\,\left( \mu_{\tilde t_{ij}}\left( x_{ij} \right) \right);\mu_{\tilde l_j}\left(x_j \right) \right],
\end{equation}
a $\tilde T$ – общее время проекта, функция принадлежности которого рассчитывается как
\begin{equation}
  \mu_{\tilde T}\left( T \right)=\underset{\left\{ \left(x_i \right),i=\overline{1,n}\left| \min \left(x_j \right)=T \right. \right\}}{\mathop{\max }}\,\underset{j=\overline{1,n}}{\mathop{\min }}\,\left( \mu_{\tilde{t}_{j}^{-}}\left(x_j \right) \right).
\end{equation}
Функции принадлежности полных нечётких резервов времени для событий имеют вид
\begin{equation}
\label{eq:fcpm-event-reserves}
  \mu_{\Delta \tilde{t}_i}\left( x \right)=\underset{\left\{ \left( y_i, x_i \right)\left| y_i-x_i=x \right. \right\}}{\mathop{\max }}\,\min \left[ \mu_{\tilde{t}_{i}^{+}}\left(y_i \right); \mu_{\tilde{t}_{i}^{-}}\left(x_i \right) \right].
\end{equation}
	
В полученной модели, задаваемой формулами~\eqref{eq:fcpm-zadeh-earliest}--\eqref{eq:fcpm-event-reserves}, нельзя однозначно указать, какое из событий является критическим~---авторы~\cite{Balashov_IPU} указывают, что функции принадлежности полных резервов времени для каждого события~\eqref{eq:fcpm-event-reserves} при $x=0$ могут быть интерпретированы как степени принадлежности событий критическому пути. В~\cite{Balashov_IPU} также предлагается следующая классификация событий для~модели КСПУ с~интервальной неопределённостью~--- критические, полукритические и~некритические. Согласно результатам, полученным в~\cite{Chanas_Zielinski_Criticality}, схожая классификация может быть распространена и на нечёткий случай:
\begin{itemize}
  \item однозначно критической в~\cite{Chanas_Zielinski_Criticality} называется такая операция, которая при замене длительностей операций $\tilde \tau_i$ во всём проекте их любыми чёткими значениями $\tau_i\in supp\left( \tilde \tau_i \right)$ является критической в классическом понимании этого термина;
  \item потенциально некритической называется такая операция, которая при замене длительностей операций $\tilde \tau_i$ во всём проекте их любыми чёткими значениями $\tau_i\in supp\left( \tilde \tau_i \right)$, не является критической в классическом понимании этого термина.
\end{itemize}

Решение задачи, предложенное в~\cite{Balashov_IPU}, обладает несколькими существенными недостатками:
\begin{itemize}
  \item громоздкость вычислений, основанных на принципе обобщения Заде;
  \item как отмечено в~\cite{Chanas_Zielinski_Criticality}, до сих пор остаётся нерешённым в общем случае вопрос о степени критичности операции из-за трудности нахождения отдельных критических операций в сетях проектов, где не существует однозначно критического пути, т.\,е. такого, все операции которого являются однозначно критическими;
  \item сложность интерпретации результата лицом, принимающим решение~--- трудно выбрать из~нескольких критических путей <<наиболее критический>>, в~\cite{Chanas_Zielinski_Criticality} для этого используется довольно громоздкий алгоритм;
  \item не~решена проблема устойчивости решения~--- применение принципа обобщения может привести к~неоправданному расширению носителя результата, что~также сказывается на~полезности его практического применения.
\end{itemize}

//Раскрыть подробнее

Методы второй группы обычно получают проблемы в~наследство от используемых в~них алгебр и~способов сравнения нечётких чисел, а в методах третьей возникают трудности с~восстановлением функции принадлежности нечёткого результата по~значениям на~различных $\alpha$-уровнях.
