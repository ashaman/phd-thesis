\chapter*{Введение}
\addcontentsline{toc}{chapter}{Введение}	 % Добавляем его в оглавление

Почему используются треугольные числа - ввиду множества существующих линейных моделей с известными и хорошо изученными методами решения. Кратко – их просто решать!!!

\textbf{Цели и задачи исследования.} Целью диссертационной работы является построение и исследование моделей учёта нечёткой неопределённости, обеспечивающих требуемые свойства решения различных производственных задач.

Для достижения поставленной цели в работе решались следующие задачи.
\begin{enumerate}
  \item Анализ существующих методик нечётких вычислений с~точки зрения сохранения свойств решения задач.
  \item Разработка модели представления нечётких чисел, позволяющей максимально сохранять исходную экспертную информацию и обеспечить требуемые качественные свойства решений (устойчивость, сохранение математических соотношений и т.\,п.).
  \item Разработка методики эффективной численной реализации решения задач с нечёткими параметрами, основанной на подходящих алгебраических структурах и её тестирование на примере задачи сетевого планирования с нечёткими параметрами.
  \item Разработка и верификация программного обеспечения, реализущего предложенную модель представления нечётких параметров и методики численного решения задач с нечёткими параметрами.
\end{enumerate}

\textbf{Методы исследования.} В диссертационной работе использованы методы исследования операций, теории принятия решений, теории нечётких множеств, мягких вычислений, теории алгебраических структур, теории графов. При создании программного обеспечения использовались технологии модульного и объектно-ориентированного программирования.

\textbf{Научная новизна.} В диссертационной работе получены следующие результаты, характеризующиеся научной новизной:
\begin{itemize}
  \item предложена модификация метода моделирования экспертных числовых оценок, полученных в классе LR-чисел, отличающаяся наличием L-преобразования LR-числа в соответствующие LL/RR-числа;
  \item предложены эффективные вычислительные методы решения задач с нечёткими параметрами, отличающиеся использованием описанной в работе алгебраической структуры с групповыми свойствами (со свойствами, эквивалентными полю действительных чисел);
  \item разработан программный комплекс для решения задач с нечёткими параметрами, реализующий предложенные в работе вычислительные методы, модули которого используют стандартные вычислительные операции.
\end{itemize}

\textbf{Достоверность научных результатов.} Научные положения, теоретические выводы и практические рекомендации обоснованы корректным использованием выбранного математического аппарата и подтверждены результатами вычислительного эксперимента.

\textbf{Практическая значимость исследования} заключается в... Подходы к нечётким вычислениям, предложенные в диссертации, позволяют существенно упростить процедуру расчётов без значительных потерь экспертной информации, а также использовать существующее стандартное программное обеспечение для решения различных производственных задач.

\textbf{Реализация и внедрение результатов работы.}

\textbf{Апробация работы.} Основные результаты работы докладывались на ежегодных научных сессиях Воронежского государственного университа и следующих конференциях различного уровня: международная конференция <<Современные проблемы прикладной математики, теории управления и математического моделирования>> (Воронеж, 2012 г.); международная конференция <<Информатика: проблемы, методология, технологии>> (Воронеж, 2012--2013 гг.); международный научно-технический семинар <<Современные технологии в задачах управления, автоматики и обработки информации>> (Алушта, 2013--2014 гг.); научно--техническая конференция студентов и аспирантов <<Радиоэлектроника, электротехника и энергетика>> (Москва, 2014).

\textbf{Публикации.} По теме диссертационного исследования опубликовано 11 научных работ (\cite{PMTYMM}--\cite{Kanischeva}), в том числе 4~--- в изданиях, рекомендованных ВАК РФ. В~работах, выполненных в~соавторстве: в~\cite{Vorontsov_PI} предложено преобразование~L и~алгебра модифицированных нечётких чисел; в~\cite{Vorontsov_Compare} выполнен анализ существующих методов сравнения нечётких чисел и предложен метод сравнения для модифицированных LL/RR-чисел; а в~\cite{Vorontsov_VSTU}~--- предложено определение устойчивости задачи нечёткого линейного программирования и разработан алгоритм решения задачи календарно-сетевого планирования и управления с нечёткими параметрами, позволяющий получать устойчивое решение.