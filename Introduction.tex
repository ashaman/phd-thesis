\chapter*{Введение}
\addcontentsline{toc}{chapter}{Введение}	 % Добавляем его в оглавление
Немного о~моделях с~неопределённостью.

Модели с~вероятностной (стохастической) неопределённостью основываются на~случайных переменных, для~которых известны законы их~распределения. Ключевым фактором для~применения таких моделей является наличие подробных статистических данных, исходя из которых, можно восстановить вероятностные распределения случайных величин. Обычно статистические распределения справедливы на больших, очень больших и сверхбольших выборках данных. Если распределения величин неизвестны, или выборка данных мала, то использовать вероятностные модели не рекомендуется, и приходится прибегать к услугам экспертов, которые выражают своё мнение в виде качественных оценок, а принадлежность объектов задаётся с помощью лингвистических операторов («много», «мало», «около» и т.п.). Такой вид неопределённости называется нечёткой/лингвистической. Наконец, если информация об объекте моделирования или параметрах модели неполная, а её дополнение является либо невозможным, либо нецелесообразным, то речь идёт об информационной неопределённости.

Частным случаем моделей с лингвистической неопределённостью являются модели, использующие нечёткие числа в качестве параметров. Они удобны тем, что это хорошо проработанные и испытанные временем модели. Основная проблема – создание алгебраических систем для нечётких чисел либо численных методов решения таких моделей.

Почему используются треугольные числа - ввиду множества существующих линейных моделей с известными и хорошо изученными методами решения. Кратко – их просто решать!!!

Почему числа должны быть асимметричными – основной use case для нечётких параметров это определение рисков (!!!) С точки зрения оценки рисков, симметричное число не несёт в себе никакой информации, поскольку позитивный и негативный исходы <<равновероятны>> (по идее, такое число можно заменить симметричным относительно матожидания распределением). Риск предполагает только негативный исход, и тому эксперту, который будет формировать оценки, необходимо это учитывать. Таким образом, ответственность за результаты частично переносится на экспертов. Предлагаемая в данном исследовании методика решения будет работать и на симметричных числах, однако она будет эквивалентна в плане решения уже известным чётким задачам с чёткими параметрами (это обобщение/расширение обычной четкой арифметики).