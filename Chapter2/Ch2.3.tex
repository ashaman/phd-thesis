\subsection{Алгебра модифицированных нечетких чисел}
Для того, чтобы использовать модифицированные нечёткие числа в качестве параметров чётких задач, необходимо построить алгебраическую систему для множества всех нечётких модифицированных чисел $K$.

Будем строить чёткую алгебру $P=\left\langle K ;\ +,\,* \right\rangle$ на множестве модифицированных нечётких чисел $K=\left\{ \bar{x}\left( \alpha  \right) \right\};\ \alpha \in \left[ 0;1 \right]$ по аналогии с тем, как это делается в~\cite{Yakhyaeva}. Для удобства дальнейших вычислений преобразуем $\bar{x}\left( \alpha \right)$:
\begin{gather*}
  \bar{x}\left( \alpha  \right)=\lambda \left( m-a+a\alpha  \right)+\left( 1-\lambda  \right)\left( m+b-b\alpha  \right)=a\alpha \lambda +\lambda \left( m-a \right)+m+b-{}\\
  {}-b\alpha -\lambda \left( m+b \right)+b\alpha \lambda =\alpha \left( \lambda a+\lambda b-b \right)+m+b-\lambda \left( m+b-m+a \right)={}\\
  {}=a\left( \lambda \left( a+b \right)-b \right)+m+b-\lambda \left( a+b \right).
\end{gather*}

При построении алгебры будем использовать форму записи
\begin{equation}
\label{eq:modified-number-base}
  \bar{x}\left( \alpha  \right)=c+k\alpha,
\end{equation}
где
\begin{equation}
\label{eq:modified-number-from-abm}
  \begin{aligned}
    & \left[ \begin{aligned}
    & c=m+b-\lambda \left( a+b \right) \\ 
    & k=\lambda \left( a+b \right)-b \\ 
  \end{aligned} \right. \\ 
  & \lambda \in \left[ 0;1 \right];\ c,k\in \mathbb{R} \\ 
\end{aligned}
\end{equation}

\subsubsection*{Операция сложения и её свойства}

Введем на~множестве $K$ бинарную операцию сложения + следующим образом:
\begin{gather*}
  \bar{x}_1\left(\alpha \right)+\bar{x}_2\left(\alpha \right)=r_1\left( \alpha  \right)=c_1+c_2+\left(k_1+k_2 \right)\alpha, \notag \\ 
  r_1\left( \alpha  \right)\in K.
\end{gather*}
Докажем основные свойства операции сложения.

Коммутативность:
\begin{gather*}
\bar{x}_1\left( \alpha \right)+\bar{x}_2\left( \alpha \right)=c_1+c_2+\left(k_1+k_2\right)\alpha=c_2+c_1+\left(k_2+k_1 \right)\alpha =\bar{x}_2\left(\alpha \right)+\bar{x}_1\left(\alpha \right).
\end{gather*}

Ассоциативность:
\begin{gather*}
  \bar{x}_1\left(\alpha \right)+\left(\bar{x}_2\left(\alpha \right)+\bar{x}_3\left(\alpha \right) \right)=c_1+k_1\alpha +c_2+c_3+\left(k_2+k_3\right)\alpha={} \\ 
  {}=\left(c_1+c_2+\left(k_1+k_2\right)\alpha \right)+c_3+k_3\alpha =\left(\bar{x}_1\left(\alpha \right)+\bar{x}_2\left(\alpha \right) \right)+\bar{x}_3\left(\alpha \right).
\end{gather*}

\subsubsection*{Нулевой и обратный элементы}

Введём нейтральный (нулевой) элемент
\begin{equation}
\label{eq:fuzzy-kc-zero}
	\bar{0}=0+0\alpha \in K
\end{equation}
такой, что 
\begin{equation}
\label{eq:fuzzy-kc-zeroprop}
	\forall \bar{x}(\alpha )\in K:\ \bar{x}(\alpha )+\bar{0}=c+k\alpha +0+0\alpha =\bar{x}(\alpha )
\end{equation}

Также определим для каждого $\bar{x}\left( \alpha  \right)=c+k\alpha \in K$ единственный элемент $-\bar{x}\left(\alpha \right)\in K$, называемый противоположным, такой, что выполняется равенство
\begin{equation}
\label{eq:fuzzy-kc-inverse-minus}
	\bar{x}\left( \alpha  \right)+\left( -\bar{x}\left( \alpha  \right) \right)=\bar{0}
\end{equation}

Очевидно, что противоположный элемент можно определить следующим образом:
\begin{equation*}
  -\bar{x}\left( \alpha  \right)=-c-k\alpha
\end{equation*}
так что равенство \eqref{eq:fuzzy-kc-inverse-minus} будет справедливым:
\begin{equation*}
	\bar{x}\left( \alpha  \right)+\left( -\bar{x}\left( \alpha  \right) \right)=c+k\alpha -c-k\alpha =0+0\alpha =\bar{0}.
\end{equation*}

Операция вычитания нечётких чисел вводится как сложение числа $\bar{x}_1\left(\alpha \right)$ с числом, противоположным числу $\bar{x}_2\left( \alpha  \right)$:
\begin{gather*}
  \bar{x}_1\left(\alpha \right)-\bar{x}_2\left(\alpha \right)=\bar{x}_1\left(\alpha \right)+\left(-\bar{x}_2\left(\alpha \right) \right)=c_1-c_2+\left( k_1-k_2 \right)\alpha
\end{gather*}

\textbf{Пример.} Даны два нечётких числа $\tilde A=\left\langle 3;4;1 \right\rangle $ и $\tilde B=\left\langle 5;2;3 \right\rangle$. Выполнить операции $\tilde A^{*}+ \tilde B^{*}$, $\tilde A^{*}-\tilde B^{*}$.

Вначале запишем уравнения для левой и правой ветвей каждого из чисел и оптимальные в смысле сохранения нечёткой информации значения $\lambda $:
\begin{gather*}
\begin{matrix}
  \left[ \begin{aligned}
    & x_{\tilde A}^{L}\left( \alpha  \right)=-1+4\alpha;  \\ 
    & x_{\tilde A}^{R}\left( \alpha  \right)=4-\alpha;  \\ 
    & \lambda_{\tilde A}=\frac{4}{4+1}=\frac{4}{5};
  \end{aligned} \right.
  &
  \left[ \begin{aligned}
    & x_{\tilde B}^{L}\left( \alpha  \right)=3+2\alpha;  \\ 
    & x_{\tilde B}^{R}\left( \alpha  \right)=8-3\alpha;  \\ 
    & \lambda_{\tilde B}=\frac{2}{2+3}=\frac{2}{5}.
\end{aligned} \right.
\end{matrix}
\end{gather*}
Найдём модифицированные значения:
\begin{equation}
\label{eq:sample-modified-equations}
  \left[ \begin{aligned}
    & \bar{x}_{\tilde A}\left( \alpha  \right)=\frac{4}{5}\left( -1+4\alpha  \right)+\frac{1}{5}\left( 4-\alpha  \right)=\frac{-4+16\alpha +4-\alpha }{5}=3\alpha;  \\ 
    & \bar{x}_{\tilde B}\left( \alpha  \right)=\frac{2}{5}\left( 3+2\alpha  \right)+\frac{3}{5}\left( 8-3\alpha  \right)=\frac{6+4\alpha +24-9\alpha }{5}=6-\alpha.  \\ 
  \end{aligned} \right.
\end{equation}
Исходя из~\eqref{eq:sample-modified-equations}, значения искомых выражений равны:
\begin{equation*}
  \left[ \begin{aligned}
    & \tilde A^{*}+ \tilde B^{*}=3\alpha +6-\alpha =6+2\alpha; \\ 
    & \tilde A^{*}-\tilde B^{*}=3\alpha -6+\alpha =-6+4\alpha. \\ 
  \end{aligned} \right. 
\end{equation*}

\subsubsection{Операция умножения и её свойства}
Введём на множестве $K$ операцию умножения. Её можно было бы определить с помощью следующего выражения как сумму произведений компонент модифицированных нечётких чисел
\begin{gather}
  \bar{x}_1(\alpha )\cdot \bar{x}_2(\alpha )=r_{2}^{'}\left( \alpha  \right)=\left( c_1+k_1\alpha  \right)\left( c_2+k_2\alpha \right)={} \notag \\
  \label{eq:modified-invalid-multiplication}
  {}= c_1 c_2+c_1 k_2\alpha+c_2 k_1\alpha+k_1 k_2\alpha^2.
\end{gather}
Однако такое определение приводит к искажению треугольного вида результата нечётких операций, поскольку в~\eqref{eq:modified-invalid-multiplication} появляется слагаемое с~$\alpha^2$. А~это~означает, что~$r_{2}^{'}\left( \alpha  \right)\notin K$. Для того, чтобы результат операции умножения остался в~множестве~$K$, воспользуемся линейной интерполяцией~--- зависимость $r_2\left(\alpha \right)$ будет восстанавливаться в~виде линейной функции по значениям выражения~\eqref{eq:modified-invalid-multiplication} при~$\alpha =0$ и~$\alpha =1$. В~первом случае $r_{2}^{'}\left( 0 \right)=c_1c_2$, во втором $r_{2}^{'}\left( 1 \right)=\left(c_1+k_1 \right)\left( c_2+k_2 \right)$. Подставляя данные значения в~уравнение прямой~$r_2\left( \alpha \right)$, получаем:
\begin{equation*}
  \frac{\alpha-0}{1-0}=\frac{r_2\left( \alpha \right)-c_1 c_2}{\left(c_1+k_1 \right)\left( c_2+k_2 \right)-c_1 c_2},
\end{equation*}
откуда, упрощая знаменатель второй дроби, получаем:
\begin{gather*}
  \alpha =\frac{r_2\left( \alpha \right)-c_1 c_2}{c_1 k_2+c_2 k_1+k_1 k_2}; \\
  r_2\left( \alpha \right)=c_1 c_2+\left(c_1 k_2+ c_2 k_1 +k_1 k_2 \right)\alpha \in K.
\end{gather*}

Таким образом, операция умножения на $K$ вводится следующим образом:
\begin{equation}
\label{eq:modified-multiplication}
  r_2\left( \alpha \right)=c_1 c_2+\left(c_1 k_2+ c_2 k_1 +k_1 k_2 \right)\alpha;\ r_2\left( \alpha  \right)\in K.
\end{equation}
Умножение нечёткого числа на~скаляр $\beta \in \mathbb{R}$ является частным случаем операции~\eqref{eq:modified-multiplication}, поскольку скаляр представляется в виде нечёткого синглтона
\begin{equation*}
  \bar{\beta }=\beta +0\alpha \in K.
\end{equation*}

Докажем основные свойства операции умножения. Коммутативность доказывается элементарно:
\begin{gather*}
  \bar{x}_1(\alpha )\cdot \bar{x}_2(\alpha )=c_1 c_2+\left(c_1 k_2+c_2 k_1+k_1 k_2 \right)\alpha = {}\\ 
  {}=c_2 c_1+\left( c_2 k_1+c_1 k_2+k_2 k_1 \right)\alpha=\bar{x}_2(\alpha )\cdot \bar{x}_1(\alpha ).
\end{gather*}

Для доказательства свойства ассоциативности
\begin{equation*}
  \bar{x}_1\left( \alpha  \right)\cdot \left( \bar{x}_2\left( \alpha  \right)\bar{x}_3\left( \alpha \right) \right)=\left( \bar{x}_1\left( \alpha  \right)\bar{x}_2\left( \alpha \right) \right)\cdot \bar{x}_3\left( \alpha  \right)
\end{equation*}
вычислим по отдельности и сравним результаты правой и левой частей:
\begin{gather*}
\bar{x}_1\left( \alpha  \right)\cdot \left( \bar{x}_2\left( \alpha  \right)\bar{x}_3\left( \alpha  \right) \right)=\left( c_1+k_1\alpha  \right)\left( \left( c_2+k_2\alpha  \right)\left( c_3+k_3\alpha  \right) \right)={} \\ 
  {}=\left( c_1+k_1\alpha  \right)\left( c_2c_3+\left( k_2c_3+k_3c_2+k_2k_3 \right)\alpha  \right)={} \\ 
  {}=c_1c_2c_3+\left( c_1k_2c_3+c_1c_2k_3+c_1k_2k_3+k_1c_2c_3+k_1k_2c_3+k_1c_2k_3+k_1k_2k_3 \right)\alpha; \\
  \left( \bar{x}_1\left( \alpha  \right)\bar{x}_2\left( \alpha  \right) \right)\cdot \bar{x}_3\left( \alpha  \right)=\left( \left( c_1+k_1\alpha  \right)\left( c_2+k_2\alpha  \right) \right)\left( c_3+k_3\alpha  \right)={}\\ 
  {}=\left( c_1c_2+\left( k_1c_2+c_1k_2+k_1k_2 \right)\alpha  \right)\left( c_3+k_3\alpha  \right)={} \\ 
  {}=c_1c_2c_3+\left( k_1c_2c_3+c_1k_2c_3+k_1k_2c_3+c_1c_2k_3+k_1c_2k_3+c_1k_2k_3+k_1k_2k_3 \right). 
\end{gather*}
Путём сравнения результатов умножения можно убедиться, что свойство ассоциативности верно.

Для доказательства дистрибутивности умножения относительно сложения, т.\,е. справедливости равенства 
\begin{equation}
\label{eq:modified-multiplication-distrib}
  \bar{x}_1\left(\alpha \right)\cdot \bar{x}_2\left(\alpha \right)+\bar{x}_1\left(\alpha \right)\cdot \bar{x}_3\left(\alpha \right)=\bar{x}_1\left(\alpha \right)\cdot \left(\bar{x}_2(\alpha )+\bar{x}_3(\alpha ) \right)
\end{equation}
также выполним действия в левой и правой частях выражения~\eqref{eq:modified-multiplication-distrib} по~отдельности, а~затем сравним результаты: \allowbreak
\begin{gather*}
  \bar{x}_1\left( \alpha \right)\bar{x}_2\left( \alpha  \right)+\bar{x}_1\left( \alpha  \right)\bar{x}_3\left( \alpha  \right)=(c_1+k_1\alpha )\cdot (c_2+k_2\alpha )+(c_1+k_1\alpha )\cdot (c_3+k_3\alpha )={}\\ 
  {}=c_1c_2+(c_1k_2+c_2k_1+k_1k_2)\alpha +c_1c_3+(c_1k_3+c_3k_1+k_1k_3)\alpha; \\ 
  \bar{x}_1\left( \alpha  \right)\left( \bar{x}_2\left( \alpha  \right)+\bar{x}_3\left( \alpha  \right) \right)=(c_1+k_1\alpha )\cdot (c_2+k_2\alpha +c_3+k_3\alpha )= \\ 
  =\left(c_1+k_1\alpha \right)\cdot \left(c_2+c_3+\left(k_2+k_3\right)\alpha \right)={}\\ 
  {}=c_1c_2+c_1c_3+\left(c_1k_2+c_2k_1+k_1k_2\right)\alpha +\left(c_1k_3+c_3k_1+k_1k_3\right)\alpha.  
\end{gather*}
Сравнение результатов вычислений подтверждает верность равенства~\eqref{eq:modified-multiplication-distrib}, т.\,е.~дистрибутивность операции умножения относительно сложения.

Проиллюстрируем введённые операции примерами.

\textbf{Пример.} Даны нечёткие числа $\tilde{A}=\left\langle 4;1;2 \right\rangle $. $\tilde{B}=\left\langle 7;3;1 \right\rangle $ и $\tilde{C}=\left\langle 1;4;2 \right\rangle $. Выполнить операции $2\tilde{A}+\tilde{B}\tilde{C}$, $\tilde{C}\left( \tilde{A}-3\tilde{B} \right)$ с~использованием модифицированных чисел.

Вначале найдём оптимальные в смысле сохранения нечёткой информации значения $\lambda$:
\begin{equation*}
  \left[ \begin{aligned}
    & \lambda_{\tilde A}=\frac{1}{1+2}=\frac{1}{3}; \\ 
    & \lambda_{\tilde B}=\frac{3}{3+1}=\frac{3}{4}; \\ 
    & \lambda_{\tilde C}=\frac{4}{4+2}=\frac{2}{3}. \\ 
  \end{aligned} \right.
\end{equation*}
Найдём модифицированные нечёткие числа для каждого из исходных чисел согласно формулам~\eqref{eq:modified-number-base} и~\eqref{eq:modified-number-from-abm}:
\begin{equation*}
  \left[ \begin{aligned}
    & \bar{x}_{\tilde A}\left( \alpha  \right)=4+2-\frac{1}{3}\left( 1+2 \right)+\left( \frac{1}{3}\left( 1+2 \right)-2 \right)\alpha =5-\alpha;  \\ 
    & \bar{x}_{\tilde B}\left( \alpha  \right)=7+1-\frac{3}{4}\left( 3+1 \right)+\left( \frac{3}{4}\left( 3+1 \right)-1 \right)\alpha =5+2\alpha;  \\ 
    & \bar{x}_{\tilde C}\left( \alpha  \right)=1+2-\frac{2}{3}\left( 4+2 \right)+\left( \frac{2}{3}\left( 4+2 \right)-2 \right)\alpha =-1+2\alpha.
\end{aligned} \right. 
\end{equation*}

Значение первого выражения в модифицированном виде равно
\begin{gather*}
  2\tilde A^{*}+\tilde B^{*}\tilde C^{*}=2\left( 5-\alpha  \right)+\left( 5+2\alpha  \right)\left( -1+2\alpha \right)={} \\ 
  {}=10-2\alpha +\left( -5+\left( 10-2+4 \right)\alpha  \right)=5+10\alpha,
\end{gather*}
а второго
\begin{gather*}
  \tilde C^{*}\left(\tilde A^{*}-3\tilde B^{*} \right)=\left(-1+2\alpha  \right)\left( 5-\alpha -3\left( 5+2\alpha  \right) \right)=\left( -1+2\alpha  \right)\left( -10-7\alpha  \right)={} \\ 
  {}=10+\left( 7-20-14 \right)\alpha =10-27\alpha.
\end{gather*}

\subsubsection*{Единичный и обратный элементы}

Перейдём к рассмотрению нейтрального по умножению и обратного элементов. Введём единичный элемент
\begin{equation*}
  \bar{1}=1+0\alpha \in K
\end{equation*}
такой, что
\begin{equation*}
  \forall \bar{x}\left( \alpha  \right)\in K\quad \bar{x}\left( \alpha  \right)\cdot \bar{1}=\bar{x}\left( \alpha  \right).
\end{equation*}
Это~равенство легко подтверждается с~помощью формулы~\eqref{eq:modified-multiplication}:
\begin{equation*}
  \bar{1}\cdot \bar{x}(\alpha )=(1+0\alpha )(c+k\alpha )=c+k\alpha =\bar{x}(\alpha ).
\end{equation*}

Несколько сложнее вводится на множестве К обратный элемент $\bar{x}^{-1}\left( \alpha  \right)\in K$ такой, что
\begin{equation*}
  \bar{x}\left( \alpha  \right){{\bar{x}}^{-1}}\left( \alpha  \right)=\bar{1}.
\end{equation*}

Поскольку предполагается, что $\bar{x}^{-1}\left( \alpha  \right)\in K$, то~будем искать обратный элемент в~виде $\bar{x}^{-1}\left( \alpha \right)=c^{'}+k^{'}\alpha$. Имеем
\begin{equation}
\label{eq:modified-inverse-division-equation}
  \bar{x}\left( \alpha  \right)\bar{x}^{-1}\left( \alpha  \right)=\left( c+k\alpha  \right)\left(c^{'}+k^{'}\alpha \right)=cc^{'}+\left(ck^{'}+c^{'}k+kk^{'}\right)\alpha =1+0\alpha.
\end{equation}
Из~\eqref{eq:modified-inverse-division-equation} очевидно, что должны выполняться равенства
\begin{equation*}
  \left\{ \begin{aligned}
    & cc^{'}=1; \\ 
    & ck^{'}+c^{'}k+kk^{'}=0.
  \end{aligned} \right.
\end{equation*}

Из первого уравнения системы находим $\displaystyle c^{'}=\frac{1}{c};\ c\ne 0$. Подставляя найденное значение во второе уравнение, имеем:
\begin{equation*}
  ck^{'}+\frac{k}{c}+kk^{'}=0,
\end{equation*}
откуда
\begin{equation*}
  k^{'}=\frac{-k}{c\left(k+c\right)}.
\end{equation*}

Таким образом, обратный элемент вводится в~виде
\begin{equation}
\label{eq:modified-inverse-division}
  \bar{x}^{-1}(\alpha )=\frac{1}{c}-\frac{k}{c\left(c+k\right)}\alpha,\ c\ne 0.
\end{equation}

Очевидно, что для~существования обратного элемента число $\bar{x}\left( \alpha  \right)$ должно иметь ненулевую моду, поскольку, согласно~\eqref{eq:modified-number-from-abm}, $c+k=m\ne 0$.

Операция деления вводится как умножение числа $\bar{x}_1\left( \alpha  \right)$ на число, обратное числу $\bar{x}_2\left( \alpha  \right)$:
\begin{gather*}
  \frac{\bar{x}_1\left( \alpha \right)}{\bar{x}_2\left( \alpha  \right)}=\bar{x}_1\left( \alpha  \right)\bar{x}_{2}^{-1}\left( \alpha \right)=\left( c_1+k_1\alpha  \right)\left( \frac{1}{c_2}-\frac{k_2}{c_2\left( k_2+c_2 \right)}\alpha \right)={} \\ 
  {}=\frac{c_1}{c_2}+\left( -\frac{c_1k_2}{c_2\left( k_2+c_2 \right)}+\frac{k_1}{c_2}-\frac{k_1k_2}{c_2\left( k_2+c_2 \right)} \right)\alpha;\ c_2\ne 0;\ k_2+c_2\ne 0.
\end{gather*}

Таким образом, алгебра модифицированных нечётких чисел удовлетворяет всем аксиомам поля, а вводимые выше алгебраические операции над элементами множества $K$ модифицированных нечётких чисел позволяют рассматривать его как линейное пространство над полем P~\cite{Voevodin}.

\subsection{Изоморфизм алгебр}

Введённая выше алгебра на~множестве $K$ решает несколько проблем, обозначенных в главе~\ref{chapter1}~--- создание полевой структуры, уменьшение размытости результатов и т.д. Однако

Сказать про нерешённость построения отношения полного порядка - сослаться на статью в Вестнике.
Проблемы создания специального программного обеспечения решения задач на основе предложенной алгебры. Нерешенными остались проблемы построения полного порядка. Проведенное исследование показало, что подходящих методов сравнения нечетких чисел нет.
Предложен изоморфизм алгебры модифицированных нечетких чисел, который позволяет решать нечеткую задачу как две четких задачи на заданных альфа –уровнях.
Такой подход не только позволяет использовать все стандартные программные продукты для решения нечетких задач, но и решает проблему построения полного порядка.


Стоит отметить, что расчёты с использованием модифицированных нечётких чисел можно проводить несколько иным, более удобным с вычислительной точки зрения, методом. Поскольку все элементы множества $K$ имеют линейную структуру, то для восстановления конкретного модифицированного числа $\tilde{A}$ достаточно знать два значения~--- $\bar{x}_{\tilde A}\left( 0 \right)$ и~$\bar{x}_{\tilde A}\left( 1 \right)=m_{\tilde A}$. Их подстановка в уравнение прямой позволяет получить зависимость $\bar{x}_{\tilde A}\left( \alpha \right)$ или $\mu_{\tilde A}\left( x \right)$ в~явном виде:
\begin{gather}
  \frac{\alpha-0}{1-0}=\frac{\bar{x}_{\tilde A}\left( \alpha  \right)-\bar{x}_{\tilde A}\left( 0 \right)}{\bar{x}_{\tilde A}\left( 1 \right)-\bar{x}_{\tilde A}\left( 0 \right)}; \notag \\
  \label{eq:isomorphic-field}
  \bar{x}_{\tilde A}\left( \alpha \right)=\bar{x}_{\tilde A}\left( 0 \right)+\alpha \left(\bar{x}_{\tilde A}\left( 1 \right)-\bar{x}_{\tilde A}\left(0 \right) \right)=\alpha \bar{x}_{\tilde A}\left( 1 \right)+\left( 1-\alpha  \right) \bar{x}_{\tilde A}\left( 0 \right).
\end{gather}

Все вычисления с использованием данного способа ведутся только на~двух $\alpha $-уровнях над действительными числами без использования дополнительных параметров. Если обозначить за~$*$ произвольную арифметическую операцию, то для~чисел в~форме~\eqref{eq:isomorphic-field} её результат будет выглядеть следующим образом:
\begin{equation*}
  \bar{x}_{\tilde A}\left( \alpha \right)*\bar{x}_{\tilde B}\left(\alpha \right)=\alpha \left(\bar{x}_{\tilde A}\left( 1 \right)*\bar{x}_{\tilde B}\left(1 \right) \right)+\left(1-\alpha \right)\left(\bar{x}_{\tilde A}\left(0 \right)*\bar{x}_{\tilde B}\left(0 \right) \right).
\end{equation*}

По~сути, вводится другая алгебра на множестве $K$, оперирующая парой чётких значений и автоморфная введённой ранее алгебре модифицированных нечётких чисел. В самом деле, существует взаимно однозначная функция $f:K\to K$, которая позволяет сопоставить числу вида $c+k\alpha $ число вида $\alpha {{\bar{x}}_{\tilde A}}\left( 1 \right)+\left( 1-\alpha  \right){{\bar{x}}_{\tilde A}}\left( 0 \right)$. Воспользовавшись \eqref{eq:isomorphic-field}, \eqref{eq:modified-number-base} и приравняв свободные члены и коэффициенты при $\alpha$, получим следующее соответствие
\begin{equation*}
  \left[ \begin{aligned}
    & c=\bar{x}_{\tilde A}\left( 0 \right); \\ 
    & k=\bar{x}_{\tilde A}\left( 1 \right)-\bar{x}_{\tilde A}\left( 0 \right);
  \end{aligned} \right.
  \Leftrightarrow 
  \left[ \begin{aligned}
    & \bar{x}_{\tilde A}\left( 0 \right)=c; \\ 
    & \bar{x}_{\tilde A}\left( 1 \right)=c+k.
  \end{aligned} \right.
\end{equation*}

Стоит отметить, что~преобразование~$L$ и~вводимая алгебра модифицированных нечётких чисел снимают проблему сравнения двух нечётких чисел, поскольку все сравнения при решении задач происходят с действительными значениями нечёткого числа на выбранных $\alpha$-уровнях.

\textbf{Пример.} Даны нечёткие числа $\tilde{A}=\left\langle 4;2;3 \right\rangle $. $\tilde{B}=\left\langle -2;6;2 \right\rangle $ и $\tilde{C}=\left\langle 1;1;4 \right\rangle $. Вычислить значение $\displaystyle \tilde{D}=\frac{3\tilde{A}\tilde{C}+4\tilde{B}}{\tilde{C}\left( \tilde{A}-2\tilde{B} \right)}$ с использованием модифицированных чисел.

Воспользуемся упрощённой методикой вычислений. Оптимальные в смысле сохранения нечёткой информации значения $\lambda$ равны:
\begin{equation}
\label{eq:sample2-optimal-lambda}
  \left[ \begin{aligned}
    & \lambda_{\tilde A}=\frac{2}{2+3}=\frac{2}{5}; \\ 
    & \lambda_{\tilde B}=\frac{6}{6+2}=\frac{3}{4}; \\ 
    & \lambda_{\tilde C}=\frac{1}{1+4}=\frac{1}{5}.
  \end{aligned} \right.
\end{equation}
Используя формулы~\eqref{eq:modified-number-from-abm} c~учётом~\eqref{eq:sample2-optimal-lambda}, получим коэффициенты для модифицированных нечётких чисел:
\begin{gather*}
  c_{\tilde A}=4+3-\frac{2}{5}\left( 2+3 \right)=5;\quad k_{\tilde A}=\frac{2}{5}\left( 2+3 \right)-3=-1; \\ 
  c_{\tilde B}=-2+2-\frac{3}{4}\left( 6+2 \right)=-6;\quad k_{\tilde B}=\frac{3}{4}\left( 6+2 \right)-2=4; \\ 
  c_{\tilde C}=1+4-\frac{1}{5}\left( 1+4 \right)=4;\quad k_{\tilde C}=\frac{1}{5}\left( 1+4 \right)-4=-3.
\end{gather*}

Отсюда
\begin{equation}
\label{eq:sample2-modified-numbers}
  \left[ \begin{aligned}
    & \bar{x}_{\tilde A}\left( \alpha  \right)=5-\alpha;  \\ 
    & \bar{x}_{\tilde B}\left( \alpha  \right)=-6+4\alpha;  \\ 
    & \bar{x}_{\tilde C}\left( \alpha  \right)=4-3\alpha.
  \end{aligned} \right.
\end{equation}
При~$\alpha=1$ выражения~\eqref{eq:sample2-modified-numbers} принимают следующие значения
\begin{equation*}
  \bar{x}_{\tilde A}\left( 1 \right)=4;\ \bar{x}_{\tilde B}\left( 1 \right)=-2;\ \bar{x}_{\tilde C}\left( 1 \right)=1,
\end{equation*}
а при~$\alpha=0$
\begin{equation*}
  \bar{x}_{\tilde A}\left( 1 \right)=5;\ \bar{x}_{\tilde B}\left( 1 \right)=-6;\ \bar{x}_{\tilde C}\left( 1 \right)=4.
\end{equation*}

Подставляя в выражение $\displaystyle \tilde{D}=\frac{3\tilde{A}\tilde{C}+4\tilde{B}}{\tilde{C}\left( \tilde{A}-2\tilde{B} \right)}$ вместо $\tilde A$, $\tilde B$ и~$\tilde C$ соответствующие им чёткие значения модифицированных чисел при $\alpha=0$ и $\alpha=1$, получаем:
\begin{gather*}
  \bar{x}_{\tilde D}\left( 1 \right)={{\left. \frac{3{{\tilde A}^{*}}{{\tilde C}^{*}}+4{{\tilde B}^{*}}}{{{\tilde C}^{*}}\left( {{\tilde A}^{*}}-2{{\tilde B}^{*}} \right)} \right|}_{\alpha =1}}=\frac{3\cdot 4\cdot 1+4\cdot \left( -2 \right)}{1\cdot \left( 4-2\cdot \left( -2 \right) \right)}=\frac{12-8}{4+4}=\frac{1}{2}, \\
  {{\bar{x}}_{\tilde D}}\left( 0 \right)={{\left. \frac{3{{\tilde A}^{*}}{{\tilde C}^{*}}+4{{\tilde B}^{*}}}{{{\tilde C}^{*}}\left( {{\tilde A}^{*}}-2{{\tilde B}^{*}} \right)} \right|}_{\alpha =0}}=\frac{3\cdot 5\cdot 4+4\cdot \left( -6 \right)}{4\cdot \left( 5-2\cdot \left( -6 \right) \right)}=\frac{60-24}{4\cdot 17}=\frac{9}{17}.
\end{gather*}

Согласно формуле~\eqref{eq:isomorphic-field}, модифицированный результат будет равен
\begin{equation*}
  \bar{x}_{\tilde D}\left( \alpha \right)=\frac{9}{17}+\alpha \left(\frac{1}{2}-\frac{9}{17} \right)=\frac{9}{17}-\frac{1}{34}\alpha.
\end{equation*}
	
Рассмотрим более сложный пример. 

\textbf{Пример.} Решить уравнение $\tilde{A}x=\tilde{B}$, в котором $\tilde{A}=\left\langle 3;1;2 \right\rangle $, $\tilde{B}=\left\langle 4;4;1 \right\rangle $.

В терминах модифицированных нечётких чисел решение уравнения будет иметь вид
\begin{equation}
\label{eq:sample3-equation}
  x\left( \alpha  \right)=\frac{\bar{x}_{\tilde B}\left( \alpha  \right)}{\bar{x}_{\tilde A}\left(\alpha \right)}.
\end{equation}
Поскольку операция деления для преобразованных нечётких чисел вводится как умножение на обратное число, перепишем~\eqref{eq:sample3-equation} в~виде
\begin{equation}
\label{eq:sample3-equation-modified}
  x\left( \alpha  \right)=\bar x_{\tilde B}\left( \alpha  \right)\cdot \bar{x}_{\tilde A}^{-1}\left( \alpha  \right).
\end{equation}

Выберем значения $\lambda_{\tilde A}$ и $\lambda_{\tilde B}$, равные $\displaystyle \frac{a_{\tilde A}}{d_{\tilde A}}$ и~$\displaystyle \frac{a_{\tilde B}}{d_{\tilde B}}$, в~соответствие с критерием сохранения максимального количества нечёткой информации. Они равны
\begin{equation}
\label{eq:sample3-opt-lambda}
  {{\lambda }_{\tilde A}}=\frac{1}{3};\ {{\lambda }_{\tilde B}}=\frac{4}{5}.
\end{equation}

Воспользовавшись формулами~\eqref{eq:modified-number-base},~\eqref{eq:modified-number-from-abm} и~\eqref{eq:sample3-opt-lambda}, получаем
\begin{equation}
\label{eq:sample3-modified-functions}
  \left[ \begin{aligned}
    & \bar{x}_{\tilde A}\left( \alpha  \right)=4-\alpha;  \\ 
    & \bar{x}_{\tilde B}\left( \alpha  \right)=3\alpha+1.
  \end{aligned} \right.
\end{equation}

Найдём обратный элемент $\bar{x}_{\tilde A}^{-1}\left( \alpha  \right)$ согласно формуле \eqref{eq:modified-inverse-division}: 
\begin{equation}
\label{eq:sample3-inverse}
  \bar{x}_{\tilde A}^{-1}\left(\alpha \right)=\frac{1}{4}-\frac{1}{\left( -1 \right)\cdot \left( 4-1 \right)}\alpha =\frac{1}{4}+\frac{1}{12}\alpha.\end{equation}

Пользуясь значениями из~\eqref{eq:sample3-modified-functions},~\eqref{eq:sample3-inverse} и~подставляя~их в~\eqref{eq:sample3-equation-modified}, окончательно получаем:
\begin{equation}
\label{eq:sample3-solution}
  x\left( \alpha  \right)=\left( 1+3\alpha  \right)\left( \frac{1}{4}+\frac{1}{12}\alpha  \right)=\frac{1}{4}+\frac{13}{12}\alpha.
\end{equation}

Функция принадлежности модифицированного решения определяется как обратная к~\eqref{eq:sample3-solution}~--- $\displaystyle \mu_{x\left(\alpha \right)}\left(x \right)=\frac{12}{13}x-\frac{3}{13}$. Подстановка решения задачи в исходное уравнение $\tilde{A}x=\tilde{B}$ с~учётом формулы~\eqref{eq:modified-multiplication} приводит к верному равенству:
\begin{gather*}
  \left( 4-\alpha  \right)\left( \frac{1}{4}+\frac{13}{12}\alpha  \right)=3\alpha +1; \\
  \frac{52-13-3}{12}\alpha +1=3\alpha +1.  
\end{gather*}

\subsection{Алгебра двухкомпонентных нечётких чисел}

Описанные выше модель представления нечётких чисел и~методика нечётких вычислений применимы в~случаях использования треугольных асимметричных нечётких чисел, когда допустимы некоторые потери экспертной информации ради создания подходящей алгебры над множеством нечётких чисел. Однако существуют ситуации, когда в~задаче моделирования с чёткими отношениями и нечёткими параметрами потери экспертной информации недопустимы. В этом случае может применяться альтернативный подход к определению операций над нечёткими числами, описанный в~\cite{Kanischeva}. Её~ключевое отличие в~том, что не~используется преобразование для~$\alpha$-интервалов, а~операции над~числами выполняются как~операции над~левой и~правой ветвями функции принадлежности.

Для удобства дальнейших вычислений, нечёткое треугольное число записывается в~форме
\begin{equation}
\label{eq:two-component-number}
  \hat{X} = \left(x^L\left(\alpha \right); x^R\left(\alpha \right) \right),
\end{equation}
где $x^L\left(\alpha \right)$, $x^R\left(\alpha \right)$~--- функции, описываемые выражением~\eqref{eq:membership-alphacut-form}. Такое представление позволяет рассматривать вычислительные операции над нечеткими LR-числами как операции над их компонентами~---  функциями~$x^L\left(\alpha \right)$ и~$x^R\left(\alpha \right)$, а~также исследовать алгебраическую структуру множества таких функций. Символ <<$\hat{\ }$>> используется над обозначением числа для того, чтобы отличать двухкомпонентное представление нечёткого числа от классических вариантов.

Если преобразовать выражения для функций~\eqref{eq:membership-alphacut-form} и обозначить $c_1=m-a$, $c_2=m+b$, $k_1=a$, $k_2=b$, где параметры $c_1$ и~$c_2$ определяют границы носителя числа, то получится следующее эквивалентное представление  двухкомпонентного нечёткого числа:
\begin{equation}
\label{eq:two-component-number-equivalent}
  \hat{X} = \left(c_1+k_1\alpha; c_2-k_2\alpha \right) = \left(l\left(\alpha \right), r\left(\alpha \right) \right),ё c_1+k_1\alpha,\, c_2+k_2\alpha \in \mathbb{R},
\end{equation}
причём при $\alpha=1$ верно равенство $c_1+k_1=c_2-k_2=m$.

\begin{mydef}
  Нечёткое число, задаваемое выражением~\eqref{eq:two-component-number-equivalent}, называется двухкомпонентным нечётким числом.
\end{mydef}

Как уже было отмечено, oперации над двухкомпонентными числами числами вводятся как операции над левой и правой компонентами числа в~представлении~\eqref{eq:two-component-number-equivalent}. В~\cite{Kanischeva} отмечается, что эти компоненты можно рассматривать как два числа LL и RR-типа с~функциями, обратными функциям принадлежности, вида $x\left(\alpha \right)=c+k\alpha \in K$. Выше уже была введена алгебра типа поле для модифицированных чисел, которые структурно являются числами LL/RR-типа, поэтому рассмотрим подробнее взаимодействие между обеими компонентами при~выполнении операций над~нечёткими числами. Если обозначить за~$*$ одну из~четырёх арифметических операций, то~схема взаимодействия между компонентами выглядит следующим образом:
\begin{gather}
  \left( l_1\left(\alpha \right), r_1 \left(\alpha \right) \right) * \left(l_2\left(\alpha \right), r_2\left(\alpha \right) \right) = {}\notag \\
  \label{eq:two-component-generic-operation}
  {}=\left(l_1\left(\alpha \right)*l_2\left(\alpha \right), r_1\left(\alpha \right)*r_2\left(\alpha \right) \right) = \left(x_1\left(\alpha \right), x_2\left(\alpha \right) \right),
\end{gather}
где для результата справедливы следующие варианты: $\left(x_1, x_2 \right) \in \left\{ \left(l, r \right), \left(r, l \right),\allowbreak \left(l, l \right), \left(r, r \right) \right\}$, а тип функции ($l$ или $r$) определяется знаком параметра $k$ в~\eqref{eq:two-component-number-equivalent}.

При выполнении операций над двухкомпонентными нечёткими числами  согласно схеме~\eqref{eq:two-component-generic-operation}, справедливы следующие допущения~\cite{Kanischeva}, которые позволяют избавить операции от жёсткой привязки к понятиям границ числа и левого и правого коэффициентов нечёткости:
\begin{itemize}
  \item представление~\eqref{eq:two-component-number-equivalent} двухкомпонентного числа в первую очередь рассматривается как отображение экспертных оценок о~степени пессимизма и~оптимизма относительно достижения величины $m$. В этом случае нечеткие числа типа~$\left(l, r \right)$ и~$\left(r, l \right)$ будут интепретироваться как описания степени оптимизма/пессимизма относительно достижения $m$, а~числа типа $\left(l, l \right)$ и $\left(r, r \right)$ могут определяться как не совпадающее на численном уровне двойное суждение о степени оптимизма/пессимизма;
  \item в процессе обработки нечёткой информации допускается компенсация противоположных по смыслу оценок, что не противоречит здравому смыслу во многих задачах и позволяет сохранять тождественность чётких отношений, но не согласуется с постулатом теории вероятностей о возрастании дисперсии результата действий со случайными числами.
\end{itemize}

При невыполнении допущений, результат $\left(l, r \right)$ однозначно выбирался бы из~прямого произведения $\left(l_1, r_1 \right) \times \left(l_2, r_2 \right) = \left(l_1l_2, l_1r_2, r_1l_2, r_1r_2 \right)$ в зависимости от знаков коэффициентов функций $l$ и $r$, что полностью соответствует интервальным алгебрам нечётких чисел со всеми их недостатками, описанными в п.~\ref{chapter2_1}.

Рассмотрим два примера, иллюстрирующие вышеописанные утверждения.

\textbf{Пример.} Решить уравнение $\tilde A \tilde X + \tilde B = \tilde 0$, где $\tilde A = \left \langle 2; 1; 1 \right \rangle$, $\tilde B = \left \langle 5; 3; 2 \right \rangle$.

Решение будем искать в~виде $\tilde X = -\tilde B \tilde A^{-1}$. Найдём двухкомпонентные представления чисел $\tilde A$ и $\tilde B$:
\begin{gather*}
  \hat A = \left(2-1+\alpha; 2+1-\alpha \right) = \left(1+\alpha; 3-\alpha \right); \\
  \hat B = \left(5-3+3\alpha; 5+2-2\alpha \right) = \left(2+3\alpha; 7-2\alpha \right).
\end{gather*}
Обратное значение $\hat A^{-1}$ и противоположное $-\hat B$ равны
\begin{gather*}
  \hat A^{-1} = \left( \frac{1}{1} - \frac{1}{1\cdot \left( 1+1\right)}\alpha; \frac{1}{3}+\frac{1}{3\cdot\left(3-1\right)}\alpha \right) = \left(1-\frac{1}{2}\alpha; \frac{1}{3}+\frac{1}{6}\alpha \right); \\
  -\hat B = \left(-2-3\alpha; -7+2\alpha \right).
\end{gather*}
Отсюда
\begin{gather*}
\hat X = \left(-2-3\alpha; -7+2\alpha \right) \cdot \left(1-\frac{1}{2}\alpha; \frac{1}{3}+\frac{1}{6}\alpha \right) = {} \\{} = \left(-2+\left(1 - 3 + \frac{3}{2} \right)\alpha; -\frac{7}{3}+\left(-\frac{7}{6} + \frac{2}{3} + \frac{1}{3} \right)\alpha \right) = \left( -2 -\frac{1}{2}\alpha; -\frac{7}{3} -\frac{1}{6}\alpha \right).
\end{gather*}
Полученный результат является примером двойной численно несогласованной оценки.

\textbf{Пример.} Решить систему уравнений $Ax=B$ методом Крамера. Элементами матрицы $A$ и~вектора $B$ являются нечёткие треугольные числа
\begin{gather*}
  \tilde A_{11} = \left \langle 3; 2; 1 \right \rangle; \tilde A_{12} = \left \langle -2; 2; 1\right \rangle; \tilde A_{21} = \left \langle 3; 1; 1\right \rangle; \tilde A_{22} = \left \langle -1; 4; 1\right \rangle; \\
  \tilde B_1 = \left \langle 1; 1; 1 \right \rangle; \tilde B_2 = \left \langle 2; 1; 3\right \rangle.
\end{gather*}

Решение системы будем искать независимо для каждой из компонент чисел. Значения определителей системы равны
\begin{equation*}
  \left[ \begin{aligned}
    & \Delta = \tilde A_{11} \tilde A_{22}-\tilde A_{12} \tilde A_{21}; \\
    & \Delta_1 = \tilde B_1 \tilde A_{22}-\tilde B_2 \tilde A_{12}; \\
    & \Delta_2 = \tilde B_2 \tilde A_{11}-\tilde B_1 \tilde A_{21},
    \end{aligned} \right.
\end{equation*}
а числа в~двухкомпонентной форме выглядят следующим образом:
\begin{gather*}
  \hat A_{11} = \left(1+2\alpha; 4-\alpha \right); \hat A_{12} = \left(-4+2\alpha; -1-\alpha \right); \\
  \hat A_{21} = \left(2+\alpha; 4-\alpha \right); \hat A_{22} = \left(-5+4\alpha; -\alpha \right); \\
  \hat B_1 = \left(\alpha; 2-\alpha \right); \hat B_2 = \left(1+\alpha; 5-3\alpha \right).
\end{gather*}

Для левых компонент определители равны
\begin{equation*}
  \left[ \begin{aligned}
    & \hat \Delta^L = \left(1+2\alpha \right) \left(-5+4\alpha \right) - \left(-4+2\alpha \right) \left(2+\alpha \right) = -5+2\alpha+8-2\alpha = 3; \\
    & \hat \Delta_1^L = \left(0 + \alpha \right) \left( -5 + 4\alpha \right) - \left(1+\alpha \right) \left(-4+2\alpha \right) = 4-\alpha; \\
    & \hat \Delta_2^L = \left(1+\alpha \right) \left(1+2\alpha \right) - \left(0+\alpha \right) \left(2+\alpha \right) = 1+5\alpha-3\alpha =1+2\alpha;
  \end{aligned} \right.
\end{equation*}
откуда находим $\hat X_1^L$ и $\hat X_2^L$:
\begin{gather*}
  \hat X_1^L = \frac{\hat \Delta_1^L}{\hat \Delta^L} = \frac{4}{3} - \frac{1}{3}\alpha; \\
  \hat X_2^L = \frac{\hat \Delta_2^L}{\hat \Delta^L} = \frac{1}{3} + \frac{2}{3}\alpha.
\end{gather*}

Аналогично, находим определители для правых компонент
\begin{equation*}
  \left[ \begin{aligned}
    & \hat \Delta^R = \left(4-\alpha \right) \left(0-\alpha \right) - \left(-1-\alpha \right) \left(4-\alpha \right) = -3\alpha+4+2\alpha = 4-\alpha; \\
    & \hat \Delta_1^R = \left(2-\alpha \right) \left(0-\alpha \right) - \left(5-3\alpha \right) \left(-1-\alpha \right) = -\alpha+5-\alpha = 5-2\alpha; \\
    & \hat \Delta_2^R = \left(5-3\alpha \right) \left(4-\alpha \right) - \left(2-\alpha \right) \left(4-\alpha \right) = 20-14\alpha-8+5\alpha=12-9\alpha;
  \end{aligned} \right.
\end{equation*}
и~поэтому $\hat X_1^R$ и $\hat X_2^R$ равны
\begin{gather*}
  \hat X_1^R = \frac{\hat \Delta_1^R}{\hat \Delta^R} = \left(5-2\alpha \right) \left(\frac{1}{4}+\frac{1}{12}\alpha \right) = \frac{5}{4}-\frac{1}{4}\alpha; \\
  \hat X_2^R = \frac{\hat \Delta_2^R}{\hat \Delta^R} = \left(12-9\alpha \right) \left(\frac{1}{4}+\frac{1}{12}\alpha \right) = 3-2\alpha.
\end{gather*}

Окончательное решение записывается в~виде
\begin{equation*}
  \left[ \begin{aligned}
    \hat X_1 = \left(\frac{4}{3} - \frac{1}{3}\alpha; \frac{5}{4}-\frac{1}{4}\alpha \right); \\
    \hat X_2 = \left(\frac{1}{3} + \frac{2}{3}\alpha; 3-2\alpha \right).
  \end{aligned} \right.
\end{equation*}