\chapter{Методы моделирования и обработки нечетких числовых величин}
\label{chapter2}

В главе~\ref{chapter1} было отмечено, что частным случаем моделей с лингвистической неопределённостью являются модели, использующие чёткие отношения и нечёткие числа в качестве параметров. Удобство их применения состоит в том, что они достаточно хорошо проработаны и испытаны временем.

В связи с существованием множества линейных моделей с известными и хорошо изученными методами решения, в качестве их параметров проще использовать треугольные нечёткие числа, функция принадлежности которых является кусочно-линейной.

%Конечно, за удобство применения нечётких моделей приходится платить, поскольку предлагаемые теорией решения, основанные на нечёткой информации, несут на себе печать нечёткости. Они могут рассматриваться лишь как рекомендации по выбору для лица, принимающего решения, требуя от него выбора одного из предлагаемых вариантов. Тем не менее, даже этот факт можно рассматривать как достоинство теории~--- он показывает, как увеличение информированности ЛПР сказывается на достоверности и правильности принимаемых решений.

Как покажет последующий анализ существующих алгебр нечётких чисел, практически все они не обеспечивают, по крайней мере, одно из двух важных свойств решения модели~--- непротиворечивость чётким математическим отношениям и ограничение расширения неопределенности. Кроме того, только в двух работах (за авторством Ускова) ставится вопрос о создании такой алгебры нечётких чисел, которая позволяла бы использовать стандартное ПО для чётких задач и решать их без создания дополнительных программных пакетов/модулей

В связи с этим.... 

Почему числа должны быть асимметричными – основной use case для нечётких параметров это определение рисков (!!!) С точки зрения оценки рисков, симметричное число не несёт в себе никакой информации, поскольку позитивный и негативный исходы <<равновероятны>> (по идее, такое число можно заменить симметричным относительно матожидания распределением). Риск предполагает только негативный исход, и тому эксперту, который будет формировать оценки, необходимо это учитывать. Таким образом, ответственность за результаты частично переносится на экспертов. Предлагаемая в данном исследовании методика решения будет работать и на симметричных числах, однако она будет эквивалентна в плане решения уже известным чётким задачам с чёткими параметрами (это обобщение/расширение обычной четкой арифметики).

%Кроме того, существенным недостатком ин- тервальных вычислений является то, что интервальная арифметика абстрагирована от контекста операндов ￿ в результате выполнения бинарных интервальных операций с одним нечетким числом область определения результата операции может расшириться относительно действительного результата [7]

\section{Анализ существующих алгебр нечётких чисел}
\label{chapter2_1}
Модификация нечетких чисел с помощью L-преобразования, доказательства свойств L- преобразования и следствия.

Построение алгебры модифицированных чисел, введение нуля и единицы,  операций, доказательства групповых свойств. Иллюстрация выполнению четких отношений равенства при решении уравнений. Иллюстрация ограничения роста неопределенности результата. Упоминание о двухкомпонентных нечетких числах как возможной альтернативе, когда любые потери экспертной информации недопустимы.

\subsection{Модифицированные нечеткие числа}

1. Модифицированные нечеткие числа.
В статье [Матвеев Лебедев] для преодоления изложенных в параграфе 3 главы 1 недостатков нечётких арифметик, для решения нечётких задач был предложен следующий подход. Исходная задача $\tilde{Y}=\text{ }f\left( {\tilde{X}} \right)$ с нечёткими числовыми параметрами рассматривается как совокупность задач с интервальной неопределенностью
	\[\tilde{Y}=f\left( {\tilde{X}} \right)\to \bigcup\limits_{\alpha =0}^{1}{{{y}_{\alpha }}=f\left( {{X}_{\alpha }} \right)}\] 	(1.42)
с последующим переходом к полной определённости. Для этого на каждом $\alpha $-уровне внутри интервала ${{X}_{\alpha }}$ выбирается точка $\bar{x}\left( \alpha  \right)$. В [Лебедев Матвеев] для этого используется средняя точка $\alpha $-интервала
	\[\bar{x}\left( \alpha  \right)=\frac{{{x}^{L}}\left( \alpha  \right)+{{x}^{R}}\left( \alpha  \right)}{2}\] 	(1.43)
Для треугольных чисел $\bar{x}\left( \alpha  \right)$ является линейной функцией ввиду линейности .\[{{x}^{L}}\left( \alpha  \right)\] и \[{{x}^{R}}\left( \alpha  \right)\]. После решения N чётких $\alpha $-уровневых задач полученные результаты $y\left( \alpha  \right)$ аппроксимируются нечётким числом \[{{\tilde{Y}}^{*}}=\left\{ y(\alpha )\left| {{\mu }_{{\tilde{Y}}}}(y)=\alpha  \right. \right\}\], которое называется модифицированным решением задачи $\tilde{Y}=\text{ }f\left( {\tilde{X}} \right)$. В статье утверждается, что в реальных задачах модифицированного решения достаточно для, например, поддержки принятия решений. (в статье дан только частный случай, а также нет строгого математического доказательства. Это предложено в статье [Воронцов Матвеев ПИ8-2014]) Предлагаемый подход в своей основе имеет факторизацию, т.е. декомпозицию нечётких чисел по $\alpha $-уровням.
Если рассмотреть данный подход с точки зрения нечётких арифметик, то решение задачи с использованием факторизации нечётких чисел представляется как переход от использования арифметик «полноценных» нечётких чисел к арифметикам для чисел$LL/RR$-типа. Действительно, функция принадлежности такого числа является обратной к функции, которая определяет точки $\bar{x}\left( \alpha  \right)$:
	${{\mu }_{{{{\tilde{A}}}^{*}}}}\left( x \right)={{\left( \bar{x}\left( \alpha  \right) \right)}^{-1}}$ 	(1.44)
Определение. Число ${{\tilde{A}}^{*}}$, получаемое из числа $\tilde{A}$ с помощью преобразования (1.44), будем называть модифицированным нечётким числом. Модифицированное нечёткое число является числом $LL/RR$-типа, поскольку один из коэффициентов нечёткости равен нулю, а функция принадлежности имеет только левую или правую ветвь. В дальнейшем для модифицированных чисел, наряду с обозначением ${{\tilde{A}}^{*}}$, будем использовать обозначение $\bar{x}\left( \alpha  \right)$, которое указывает на механизм их построения как совокупности точек на выбранных $\alpha $-интервалах.
Пример. Пусть $\tilde{A}=\left\langle 2;2;4 \right\rangle $. Найдём модифицированное число ${{\tilde{A}}^{*}}$ в соответствие с (1.43).
Вначале запишем функцию принадлежности числа ${{\tilde{A}}^{*}}$:
	\[{{\mu }_{{\tilde{A}}}}\left( x \right)=\left\{ \begin{aligned}
  & \frac{x}{2};\ 0\le x\le 2 \\ 
 & \frac{6-x}{4};\ 2<x\le 6 \\ 
 & 0;\ x<0\ \ x>6 \\ 
\end{aligned} \right.\] 
Используя выражения из главы 1, получим
	\[\left[ \begin{aligned}
  & {{x}^{L}}\left( \alpha  \right)=m-a+a\alpha =2-2+2\alpha =2\alpha  \\ 
 & {{x}^{R}}\left( \alpha  \right)=m+b-b\alpha =2+4-4\alpha =6-4\alpha  \\ 
\end{aligned} \right.\] 
Тогда, согласно (1.43), получим:
	$\bar{x}\left( \alpha  \right)=\frac{{{x}^{L}}\left( \alpha  \right)+{{x}^{R}}\left( \alpha  \right)}{2}=\frac{2\alpha +6-4\alpha }{2}=3-\alpha $.
Т.к. прямая функция $\bar{x}\left( \alpha  \right)$ убывает на своей области определения $\alpha \in \left[ 0;1 \right]$, то и обратная ${{\mu }_{{{{\tilde{A}}}^{*}}}}\left( x \right)$ также будет убывать. Её областью определения будет являться область значений функции $\bar{x}\left( \alpha  \right)$, т.е. отрезок $\left[ 2;3 \right]$. В итоге${{\mu }_{{{{\tilde{A}}}^{*}}}}\left( x \right)$ состоит только из правой ветви:
	\[{{\mu }_{{{{\tilde{A}}}^{*}}}}\left( x \right)=\left\{ \begin{aligned}
  & 3-x;\ x\in \left[ 2;3 \right] \\ 
 & 0;\ x\notin \left[ 2;3 \right] \\ 
\end{aligned} \right.\] 
Оба числа – и исходное, и модифицированное – изображены на рис. ().
Очевидно, что на вид модифицированных чисел (и, соответственно, на итоговый результат решения задачи) влияние будут оказывать как характеристики самих нечётких чисел – мода, коэффициенты нечёткости, – так и принцип, согласно которому выбирается точка $\bar{x}\left( \alpha  \right)$. В статье [ПИ8-2014] предложено обобщение принципа, вводимого в [Матвеев Лебедев] – значение $\bar{x}\left( \alpha  \right)$ выбирается с помощью линейного параметрического преобразования $L$:
	\[\bar{x}\left( \alpha  \right)=L\left( {{X}_{\alpha }} \right)=\lambda {{x}^{L}}\left( \alpha  \right)+\left( 1-\lambda  \right){{x}^{R}}\left( \alpha  \right)\] 	(1.45)
Параметр преобразования $\lambda \in \left[ 0;1 \right]$ выбирается индивидуально для каждого числа согласно его характеристикам – величинам коэффициентов нечёткости и длине носителя. Нетрудно убедиться, что мода модифицированного числа ${{m}_{{{{\tilde{A}}}^{*}}}}$ равна
	\[{{m}_{{{{\tilde{A}}}^{*}}}}=\bar{x}\left( 1 \right)\],	(1.46)
а ненулевой коэффициент нечёткости и носитель равны по модулю
	\[{{d}_{{{{\tilde{A}}}^{*}}}}=\left| \bar{x}\left( 1 \right)-\bar{x}\left( 0 \right) \right|\].	(1.47)
Если известно уравнение (1.45), то, с учётом (1.46) и (1.47), модифицированное нечёткое число можно представить в виде тройки \[\left\langle \bar{x}\left( 1 \right);\left| \bar{x}\left( 1 \right)-\bar{x}\left( 0 \right) \right|;0 \right\rangle \] (число LL-типа) или \[\left\langle \bar{x}\left( 1 \right);0;\left| \bar{x}\left( 1 \right)-\bar{x}\left( 0 \right) \right| \right\rangle \] (число RR-типа). Тип модифицированного числа можно определить по коэффициенту при $\alpha $ в (1.45): если он больше нуля, то число LL-типа, если меньше – RR-типа.
Исходя из механизма построения модифицированных нечётких чисел, очевидно, что преобразование (1.45) сокращает информативность исходной нечёткой величины. Чтобы выяснить, насколько существенны потери нечёткой информации при различных значениях параметра $\lambda $, проведём исследование свойств преобразования L.
Для исследования свойств преобразования $L$ введём следующие характеристические показатели нечёткого числа, которые определяют его информативность с точки зрения принятия решений [5 ПИ8-2014], а также, по аналогии со [Cпесивцев], позволяют производить анализ и вычисления в форме, нечувствительной к знаку нечёткого числа:
•	длина носителя ${{d}_{{\tilde{A}}}}$ 
•	мода ${{m}_{{\tilde{A}}}}$ 
•	степень асимметрии $A{{S}_{{\tilde{A}}}}$
При использовании записи треугольного числа с помощью коэффициентов нечёткости, длина носителя определяется как их сумма:
	${{d}_{{\tilde{A}}}}=a+b$ 	(1.48)
Определение. Степенью асимметрии $A{{S}_{{\tilde{A}}}}$ будем называть характеристику треугольного нечёткого числа, определяемую как разность площадей прямоугольных треугольников, на которые исходное нечёткое число делится модой (рис!!!).
Площадь левого треугольника ${{S}_{1}}=\frac{1}{2}ah(\tilde{A})=\frac{a}{2}$, правого ${{S}_{2}}=\frac{1}{2}bh\left( {\tilde{A}} \right)=\frac{b}{2}$, отсюда
	\[A{{S}_{{\tilde{A}}}}={{S}_{2}}-{{S}_{1}}=\frac{b-a}{2}\in \left[ -\frac{a}{2};\frac{b}{2} \right]\] 	(1.49)
Если $\tilde{A}$ является числом $LL$($RR$)-типа, то $A{{S}_{{\tilde{A}}}}$, согласно (1.49), принимает значение $-\frac{a}{2}$ ($\frac{b}{2}$).
Очевидно, что запись треугольного нечёткого числа в виде тройки \[\left( {{m}_{{\tilde{A}}}},{{d}_{{\tilde{A}}}},A{{S}_{{\tilde{A}}}} \right)\] эквивалентна введённым ранее способам записи через коэффициенты нечёткости $\left( m;a;b \right)$ и точки пересечения с осью $Ox$ $\left( {{x}^{L}};m;{{x}^{R}} \right)$. При известных степени асимметрии $A{{S}_{{\tilde{A}}}}$ и длине носителя ${{d}_{{\tilde{A}}}}$, коэффициенты нечёткости определяются по формуле
	\[\left[ \begin{aligned}
  & a=\frac{{{d}_{{\tilde{A}}}}-2A{{S}_{{\tilde{A}}}}}{2} \\ 
 & b=\frac{{{d}_{{\tilde{A}}}}+2A{{S}_{{\tilde{A}}}}}{2} \\ 
\end{aligned} \right.\] 	(1.50)
Справедливость (1.50) можно проверить, подставив соответствующие значения для $A{{S}_{{\tilde{A}}}}$ и ${{d}_{{\tilde{A}}}}$ из формул (1.48) и (1.49).
Пример. Найдём эквивалентную форму записи для треугольного числа $\tilde{A}=\left\langle 5;3;2 \right\rangle $ в виде тройки «мода – носитель – степень асимметрии».
Согласно формуле (1.48) длина носителя равна
	\[{{d}_{{\tilde{A}}}}=a+b=3+2=5\]
Степень асимметрии $A{{S}_{{\tilde{A}}}}$ вычисляется по формуле (1.49)
	\[A{{S}_{{\tilde{A}}}}=\frac{2-3}{2}=-0,5\] 
Таким образом, число $\tilde{A}$ представляется в виде тройки $\left( 5;5;-0,5 \right)$.
2. Свойства преобразования L
Перейдём к непосредственному рассмотрению свойств преобразования $L$.
Свойство 1. Преобразование $L$ сохраняет моду нечёткого числа. Другими словами, \[\forall \lambda (\alpha ):\ {{m}_{{\tilde{A}}}}={{m}_{{{{\tilde{A}}}^{*}}}}\].
Доказательство. Перепишем с учётом равенств:
	\[\left[ \begin{matrix}
   \begin{aligned}
  & {{x}^{L}}(\alpha )=x_{A}^{L}+\alpha ({{m}_{A}}-x_{A}^{L}) \\ 
 & {{x}^{R}}(\alpha )=x_{A}^{R}+\alpha ({{m}_{A}}-x_{A}^{R}) \\ 
\end{aligned}  \\
\end{matrix} \right.\] 	(1.51)
При $\alpha =1$ имеем
	\[\left[ \begin{matrix}
   \begin{aligned}
  & {{x}^{L}}(1)=x_{{\tilde{A}}}^{L}+1\left( {{m}_{{\tilde{A}}}}-x_{{\tilde{A}}}^{L} \right)={{m}_{{\tilde{A}}}} \\ 
 & {{x}^{R}}(1)=x_{{\tilde{A}}}^{R}+1\left( {{m}_{{\tilde{A}}}}-x_{{\tilde{A}}}^{R} \right)={{m}_{{\tilde{A}}}} \\ 
\end{aligned}  \\
\end{matrix} \right.\],	(1.52)
поэтому при подстановке $\alpha =1$ в преобразование (1.45) получаем:
	\[\bar{x}\left( 1 \right)=\lambda {{x}^{L}}\left( 1 \right)+\left( 1-\lambda  \right){{x}^{R}}\left( 1 \right)=\lambda {{m}_{{\tilde{A}}}}+\left( 1-\lambda  \right){{m}_{{\tilde{A}}}}={{m}_{{\tilde{A}}}}\] 	(1.53)
Выражение (1.53) доказывает, что моды модифицированного и исходного чисел совпадают при любых значения параметров преобразования (1.45).
Свойство 2. При некоторых значениях параметра $\lambda $ преобразование $L$ сохраняет
А) знак степени асимметрии, т.е. \[\exists \lambda \in [0;1]:sign(A{{S}_{{\tilde{A}}}})=sign(A{{S}_{{{{\tilde{A}}}^{*}}}})\] 
Б) значение степени асимметрии, т.е. \[\exists \lambda \in [0;1]:\ A{{S}_{{\tilde{A}}}}=A{{S}_{{{{\tilde{A}}}^{*}}}}\] 
Доказательство. Вначале докажем утверждение А). Степень асимметрии исходного числа $\tilde{A}$ определяется выражением (1.49). Модифицированное число имеет только один ненулевой коэффициент нечёткости, который равен $\left| \bar{x}\left( 1 \right)-\bar{x}\left( 0 \right) \right|$. Кроме того, согласно свойству 1, мода числа при преобразовании сохраняется. Поэтому абсолютная величина степени асимметрии равна
	\[\left| A{{S}_{{{{\tilde{A}}}^{*}}}} \right|=\frac{\left| m-\bar{x}\left( 0 \right) \right|}{2}\] 	(1.54)
Поскольку
	\[\bar{x}\left( 0 \right)=\lambda {{x}^{L}}\left( 0 \right)+\left( 1-\lambda  \right){{x}^{R}}\left( 0 \right)=m+b-\lambda \left( a+b \right)\] 	(1.55)
то выражение (1.54) принимает вид
	\[\left| A{{S}_{{{{\tilde{A}}}^{*}}}} \right|=\frac{\left| b-\lambda \left( a+b \right) \right|}{2}\] 	(1.56)
Если исходное нечёткое число симметричное, т.е. $a=b$, то степень его асимметрии равна нулю. В этом случае равенство степеней асимметрии достигается при $\lambda =\frac{1}{2}$ -при подстановке данного значения в (1.56) имеем верное равенство:
	\[\left| A{{S}_{{{{\tilde{A}}}^{*}}}} \right|=\frac{\left| b-\lambda \left( a+b \right) \right|}{2}=\frac{1}{2}\left| b-\frac{b+b}{2} \right|=\frac{1}{2}\left| b-b \right|=0\].
Если $a>b$, то $A{{S}_{{\tilde{A}}}}<0$, и для выполнения первого пункта свойства необходимо, чтобы модифицированное число было числом$LL$-типа. Это достигается при 
	\[\bar{x}\left( 0 \right)<m\] 	(1.57)
Преобразовывая (1.57) с использованием (1.55), получаем, что \[b-\lambda \left( a+b \right)<0\], и в результате 
	\[\lambda \in \left( \frac{b}{a+b};1 \right]\] 	(1.58)
При $a<b$ $A{{S}_{{\tilde{A}}}}>0$, и модифицированное число должно быть числом$RR$-типа, т.е. 
	\[\bar{x}\left( 0 \right)>m\] 	(1.59)
По аналогии, подставляя в (1.59) значение $\bar{x}\left( 0 \right)$ из (1.55), получаем, что \[b-\lambda \left( a+b \right)>0\], и 
	\[\lambda \in \left[ 0;\frac{b}{a+b} \right)\] 	(1.60)
Таким образом, знак степени асимметрии сохраняется при выполнении следующих условий:
	\[\left[ \begin{aligned}
  & \lambda \in \left[ 0;\frac{b}{a+b} \right);\ a<b \\ 
 & \lambda =0,5;\ a=b \\ 
 & \lambda \in \left( \frac{b}{a+b};1 \right];\ a>b \\ 
\end{aligned} \right.\] 	(1.61)
Докажем теперь утверждение Б). Для этого покажем, что уравнение
	\[A{{S}_{{\tilde{A}}}}=A{{S}_{{{{\tilde{A}}}^{*}}}}\] 	(1.62)
имеет решения при $\lambda \in \left[ 0;1 \right]$. Пусть, для определённости, у исходного числа $a>b$, тогда $A{{S}_{{\tilde{A}}}}<0$, и поэтому должно быть справедливо неравенство$A{{S}_{{{{\tilde{A}}}^{*}}}}<0$. Пользуясь выражением (1.54), получаем
	\[A{{S}_{{{{\tilde{A}}}^{*}}}}=\frac{\bar{x}\left( 0 \right)-m}{2}=\frac{b-\lambda \left( a+b \right)}{2}<0\] 	(1.63)
Подставляя (1.49) и (1.63) в (1.62), приходим к уравнению
	\[\frac{b-a}{2}=\frac{b-\lambda \left( a+b \right)}{2}\] 	(1.64)
Его решением является значение
	\[\lambda =\frac{a}{a+b}=\frac{a}{d}\] 	(1.65)
Если же $a<b$, то $A{{S}_{{\tilde{A}}}}>0$, и выражение (1.63) должно быть положительным. В итоге имеем то же самое уравнение (1.64).
Таким образом, при \[\lambda =\frac{a}{a+b}\] значение степени асимметрии числа сохраняется.
Свойство 3. Модифицированное число всегда содержится внутри исходного числа. Другими словами, \[\forall \lambda \in \left[ 0;1 \right]:A_{\alpha }^{*}\subset {{A}_{\alpha }};{{d}_{{\tilde{A}}}}\ge {{d}_{{{{\tilde{A}}}^{*}}}}\], т.е. т.е. преобразование L уменьшает длину носителя нечёткого числа и оставляет \[\alpha \]-интервалы модифицированного числа в рамках \[\alpha \]-интервалов исходного числа.
Доказательство. Вначале докажем, что \[\forall \lambda \in \left[ 0;1 \right]\ {{d}_{{\tilde{A}}}}\ge {{d}_{{{{\tilde{A}}}^{*}}}}\]. Решим данное неравенство и покажем, что отрезок $\left[ 0;1 \right]$ содержится внутри решения. Очевидно, что 
	\[{{d}_{{{{\tilde{A}}}^{*}}}}=\left| \bar{x}\left( 1 \right)-\bar{x}\left( 0 \right) \right|=\left| {{m}_{{\tilde{A}}}}-\left( {{m}_{{\tilde{A}}}}-b+\lambda \left( a+b \right) \right) \right|=\left| b-\lambda \left( a+b \right) \right|\] 	(1.66)
и поэтому
	\[\left| b-\lambda (a+b) \right|\le a+b\] 	(1.67)
Раскрывая модуль, получаем систему неравенств
	\[\left\{ \begin{aligned}
  & b-\lambda \left( a+b \right)\ge -a-b \\ 
 & b-\lambda \left( a+b \right)\le a+b \\ 
\end{aligned} \right.\] 	(1.68)
Её решением является отрезок \[\left[ -\frac{a}{a+b};1+\frac{b}{a+b} \right]\]. Ввиду того, что $a,b\ge 0$, этот отрезок содержит в себе интервал $\left[ 0;1 \right]$.
Теперь покажем, что \[\forall \lambda \in \left[ 0;1 \right]:A_{\alpha }^{*}\subset {{A}_{\alpha }}\]. Очевидно, что $\alpha $-интервал модифицированного числа будет ограничен с одной стороны значением моды ${{m}_{{\tilde{A}}}}$, а с другой – значением $\bar{x}\left( \alpha  \right)$. В силу определения нечёткого числа
	\[\forall \alpha \in \left[ 0;1 \right]\ \ x_{{\tilde{A}}}^{L}(\alpha )\le {{m}_{{\tilde{A}}}}\le x_{{\tilde{A}}}^{R}(\alpha )\] 	(1.69)
Кроме того, из определения (1.45) преобразования L следует, что
	\[\forall \alpha \in \left[ 0;1 \right]\ \ x_{{\tilde{A}}}^{L}(\alpha )\le \bar{x}(\alpha )\le x_{{\tilde{A}}}^{R}(\alpha )\] 	(1.70)
Исходя из (1.69) и (1.70), \[\left[ x_{{\tilde{A}}}^{L}(\alpha );x_{{\tilde{A}}}^{R}(\alpha ) \right]\supset \left[ x_{{\tilde{A}}}^{M};\bar{x}(\alpha ) \right]\]  для LL-числа (соответственно, \[\left[ x_{{\tilde{A}}}^{L}(\alpha );x_{{\tilde{A}}}^{R}(\alpha ) \right]\supset \left[ \bar{x}(\alpha );x_{{\tilde{A}}}^{M} \right]\] для RR-числа).
Проиллюстрируем доказанные выше свойства примером. Пусть $\tilde{A}=\left\langle 4;5;1 \right\rangle $. Степень его асимметрии равна
	\[A{{S}_{{\tilde{A}}}}=\frac{1-5}{2}=-2\],
а длина носителя ${{d}_{{\tilde{A}}}}$
	\[{{d}_{{\tilde{A}}}}=5+1=6\].
Уравнения для левой и правой ветвей функции принадлежности принимают вид
	\[\left[ \begin{aligned}
  & {{x}^{L}}\left( \alpha  \right)=-1+5\alpha  \\ 
 & {{x}^{R}}\left( \alpha  \right)=5-\alpha  \\ 
\end{aligned} \right.\] 	(1.71)
Выполним преобразование $L$ с параметром $\lambda =\frac{1}{4}$ с учётом (1.71):
	\[\bar{x}\left( \alpha  \right)=\frac{1}{4}\left( 5\alpha -1 \right)+\frac{3}{4}\left( 5-\alpha  \right)=\frac{1}{4}\left( 5\alpha -1+15-3\alpha  \right)=\frac{1}{4}\left( 14+2\alpha  \right)=3,5+0,5\alpha \] 	(1.72)
Поскольку коэффициент при $\alpha $ в выражении (1.72) больше нуля, то число ${{\tilde{A}}^{*}}$ является числом LL-типа. Мода числа ${{\tilde{A}}^{*}}$, согласно свойству 1, равна 4, а левый коэффициент нечёткости равен длине носителя и равен
	\[a={{d}_{{{{\tilde{A}}}^{*}}}}=\left| \bar{x}\left( 1 \right)-\bar{x}\left( 0 \right) \right|=\left| 4-\left( 3,5-0,5\cdot 0 \right) \right|=0,5\].
Таким образом, число ${{\tilde{A}}^{*}}$ может быть записано в виде тройки $\left\langle 4;0,5;0 \right\rangle $. Степень его асимметрии, согласно свойству 2, сохраняет знак относительно $A{{S}_{{\tilde{A}}}}$ и равна
	\[A{{S}_{{{{\tilde{A}}}^{*}}}}=\frac{0-0,5}{2}=-0,25\].
Исходное и модифицированное числа изображены на рис. На нём наглядно иллюстрируется свойство 3 – модифицированное число ${{\tilde{A}}^{*}}$ целиком расположено внутри границ исходного числа.
Из доказанных выше свойств следует, что применение преобразования L к нечетким исходным данным в основном сохраняет их информативность при целенаправленном выборе параметра преобразования. Уменьшение длины носителя при определенных оговорках можно рассматривать как положительное явление [ПИ8-2014], поскольку при этом повышается общая устойчивость решения.
Следствия из свойств и рекомендации по выбору $\lambda $ 
1. Модифицированное нечёткое число, получаемое с помощью преобразования L с $\lambda =0,5$ из симметричного нечёткого числа $\left\langle m;a;a \right\rangle $, является нечётким синглтоном.
Действительно, при указанном значении $\lambda $, используя выражения, получаем:
	\[\bar{x}\left( \alpha  \right)=\frac{1}{2}\left( m-a+a\alpha  \right)+\frac{1}{2}\left( m+b-b\alpha  \right)=\frac{1}{2}\left( m-a+a\alpha +m+a-a\alpha  \right)=m\] 
Это следствие накладывает ограничения на возможность использования симметричных нечётких чисел в задачах, поскольку все нечёткие вычисления с использованием преобразования L в этом случае будут сведены к обычным алгебраическим операциям над модами чисел.
2. Применение преобразования L с параметром \[\lambda =\frac{a}{a+b}\] к числу LL/RR-типа не изменяет данного числа.
В самом деле, для LL-числа $b=0$, поэтому \[x_{{\tilde{A}}}^{R}\left( \alpha  \right)=m+b-b\alpha =m\], $\lambda =\frac{a}{a+0}=1$. Отсюда
	\[\bar{x}\left( \alpha  \right)=\lambda \left( m-a+a\alpha  \right)+\left( 1-\lambda  \right)m=m-a+a\alpha =x_{{\tilde{A}}}^{L}\left( \alpha  \right)\] 
Для RR-числа $a=0$, отсюда \[x_{{\tilde{A}}}^{L}\left( \alpha  \right)=m-a+a\alpha =m\] и $\lambda =\frac{0}{b+0}=0$. Ввиду этого
	\[\bar{x}\left( \alpha  \right)=\lambda m+\left( 1-\lambda  \right)\left( m+b-b\alpha  \right)=m+b-b\alpha =x_{{\tilde{A}}}^{R}\left( \alpha  \right)\] 
Отдельно выделим несколько наиболее интересных значений $\lambda $.
1. $\lambda =\frac{a}{a+b}=\frac{a}{{{d}_{{\tilde{A}}}}}$. Как уже было показано выше, при данном значении сохраняется значение степени асимметрии.
Проиллюстрируем это на следующем примере. Пусть $\tilde{A}=\left\langle 1;4;6 \right\rangle $. Значение $\lambda =\frac{4}{4+6}=\frac{2}{5}$, степень асимметрии $A{{S}_{{\tilde{A}}}}=\frac{6-4}{2}=1$, а уравнения правой и левой ветвей
	\[\left[ \begin{aligned}
  & {{x}^{L}}\left( \alpha  \right)=-3+4\alpha  \\ 
 & {{x}^{R}}\left( \alpha  \right)=7-6\alpha  \\ 
\end{aligned} \right.\].
Преобразование L даёт следующий результат:
	\[\bar{x}\left( \alpha  \right)=\frac{2}{5}\left( -3+4\alpha  \right)+\frac{3}{5}\left( 7-6\alpha  \right)=\frac{1}{5}\left( -6+8\alpha +21-18\alpha  \right)=3-2\alpha \].
Полученное в результате преобразования число ${{\tilde{A}}^{*}}$ является числом RR-типа, поэтому степень его асимметрии положительна и, согласно (1.54), равна
	\[A{{S}_{{{{\tilde{A}}}^{*}}}}=\frac{\left| 1-\left( 3-2\cdot 0 \right) \right|}{2}=1\].
2. \[\lambda =\frac{b}{a+b}=\frac{b}{{{d}_{{\tilde{A}}}}}\]. Преобразование L с таким значение параметра уничтожает нечёткую информацию, заложенную экспертом в число, поскольку в этом случае модифицированное число превращается в чёткое:
	\[\begin{matrix}
  \bar{x}\left( \alpha  \right)=\frac{b}{a+b}\left( m-a+a\alpha  \right)+\left( 1-\frac{b}{a+b} \right)\left( m+b-b\alpha  \right)=\frac{b\left( m-a+a\alpha  \right)+a\left( m+b-b\alpha  \right)}{a+b}= \\ 
  =\frac{bm-ab+ab\alpha +am+ab-ab\alpha }{a+b}=\frac{am+bm}{a+b}=m \\ 
\end{matrix}\] 
3. При \[\lambda =\frac{b}{{{d}_{{\tilde{A}}}}}-\frac{b-a}{3\left( b+a \right)}=\frac{2b+a}{3\left( b+a \right)}=\frac{2b+a}{3{{d}_{{\tilde{A}}}}}\] значение \[\bar{x}(\alpha )\] является проекцией центра тяжести треугольника, построенного на отрезке \[\left[ {{x}^{L}}(\alpha ),{{x}^{R}}(\alpha ) \right]\] как на основании, на ось Ox (см. рис.).
Данное значение получается следующим образом. Координата \[{{x}_{0}}\left( \alpha  \right)\] центра тяжести $\alpha $-сечения треугольного числа рассчитывается как
	\[{{x}_{0}}\left( \alpha  \right)=\frac{{{x}^{L}}\left( \alpha  \right)+m+{{x}^{R}}\left( \alpha  \right)}{3}\] 	(1.73)
Значения \[{{x}^{L}}\left( \alpha  \right)\] и ${{x}^{R}}\left( \alpha  \right)$ равны
	\[\left[ \begin{aligned}
  & {{x}^{L}}\left( \alpha  \right)=m-a+a\alpha  \\ 
 & {{x}^{R}}\left( \alpha  \right)=m+b-b\alpha  \\ 
\end{aligned} \right.\] 	(1.74)
Из (1.73) и (1.74) следует, что
	\[\begin{matrix}
  {{x}_{0}}\left( \alpha  \right)=\frac{{{x}^{L}}\left( \alpha  \right)+m+{{x}^{R}}\left( \alpha  \right)}{3}=\frac{m-a+a\alpha +m+m+b-b\alpha }{3}= \\ 
  =\frac{3m-\left( a-b \right)+\alpha \left( a-b \right)}{3}=m+\frac{\left( a-b \right)\left( \alpha -1 \right)}{3} \\ 
\end{matrix}\] 	(1.75)
Преобразование $L$ вычисляется по формуле
	\[\begin{matrix}
  \bar{x}\left( \alpha  \right)=\lambda {{x}^{L}}\left( \alpha  \right)+\left( 1-\lambda  \right){{x}^{R}}\left( \alpha  \right)=\lambda \left( m-a+a\alpha  \right)+\left( 1-\lambda  \right)\left( m+b-b\alpha  \right)= \\ 
  =\lambda \left( m-a+a\alpha -m-b+b\alpha  \right)+b\left( 1-\alpha  \right)+m=\lambda \left( a+b \right)\left( \alpha -1 \right)+b\left( 1-\alpha  \right)+m \\ 
\end{matrix}\] 	(1.76)
Приравнивая результаты (1.75) и (1.76), получаем:
	\[m+\frac{\left( a-b \right)\left( \alpha -1 \right)}{3}=m+b\left( 1-\alpha  \right)+\lambda \left( a+b \right)\left( \alpha -1 \right)\] 
	\[\left( a-b \right)\left( \alpha -1 \right)=-3b\left( \alpha -1 \right)+3\lambda \left( a+b \right)\left( \alpha -1 \right)\] 
При $\alpha =1$ равенство выполняется для всех $\lambda $. Для остальных $\alpha $, поделим обе части равенства на $\alpha -1$:
	\[a-b=-3b+3\lambda \left( a+b \right)\],
откуда получается искомое значение $\lambda $.



\section{Модифицированные нечёткие числа и параметрическое преобразование L}
\label{chapter2_2}
Проблемы создания специального программного обеспечения решения задач на основе предложенной алгебры. Нерешенными остались проблемы построения полного порядка. Проведенное исследование показало, что подходящих методов сравнения нечетких чисел нет.
Предложен изоморфизм алгебры модифицированных нечетких чисел, который позволяет решать нечеткую задачу как две четких задачи на заданных альфа –уровнях.
Такой подход не только позволяет использовать все стандартные программные продукты для решения нечетких задач, но и решает проблему построения полного порядка.


\subsection{Алгебра модифицированных нечетких чисел с примерами}
Для того, чтобы использовать модифицированные нечёткие числа в качестве параметров чётких задач, необходимо построить алгебраическую систему для множества всех нечётких модифицированных чисел $K$. 
Будем строить чёткую алгебру \[P=\left\langle ;\ +,\,* \right\rangle \] на множестве модифицированных нечётких чисел $K=\left\{ \bar{x}\left( \alpha  \right) \right\};\ \alpha \in \left[ 0;1 \right]$ по аналогии с тем, как это делается в (Яхъяева). Для удобства дальнейших вычислений преобразуем $\bar{x}\left( \alpha  \right)$:
	\[\begin{matrix}
  \bar{x}\left( \alpha  \right)=\lambda \left( m-a+a\alpha  \right)+\left( 1-\lambda  \right)\left( m+b-b\alpha  \right)=a\alpha \lambda +\lambda \left( m-a \right)+m+b-b\alpha -\lambda \left( m+b \right)+b\alpha \lambda = \\ 
  =\alpha \left( \lambda a+\lambda b-b \right)+m+b-\lambda \left( m+b-m+a \right)=a\left( \lambda \left( a+b \right)-b \right)+m+b-\lambda \left( a+b \right) \\ 
\end{matrix}\] 
При построении алгебры будем использовать форму записи
	\[\bar{x}\left( \alpha  \right)=c+k\alpha \] 	(1.77)
где
	\[\begin{aligned}
  & \left[ \begin{aligned}
  & c=m+b-\lambda \left( a+b \right) \\ 
 & k=\lambda \left( a+b \right)-b \\ 
\end{aligned} \right. \\ 
 & \lambda \in \left[ 0;1 \right];\ c,k\in \mathbb{R} \\ 
\end{aligned}\] 	(1.78)
Операция сложения
Введем на множестве $K$ бинарную операцию сложения + следующим образом:
	\[\begin{matrix}
  {{{\bar{x}}}_{1}}(\alpha )+{{{\bar{x}}}_{2}}(\alpha )={{r}_{1}}\left( \alpha  \right)={{c}_{1}}+{{c}_{2}}+\left( {{k}_{1}}+{{k}_{2}} \right)\alpha  \\ 
  {{r}_{1}}\left( \alpha  \right)\in K \\ 
\end{matrix}\] 	(1.79)
Докажем основные свойства операции сложения.
1. Коммутативность:
\[{{\bar{x}}_{1}}\left( \alpha  \right)+{{\bar{x}}_{2}}\left( \alpha  \right)={{c}_{1}}+{{c}_{2}}+({{k}_{1}}+{{k}_{2}})\alpha ={{c}_{2}}+{{c}_{1}}+({{k}_{2}}+{{k}_{1}})\alpha ={{\bar{x}}_{2}}\left( \alpha  \right)+{{\bar{x}}_{1}}\left( \alpha  \right)\] 
2. Ассоциативность
\[\begin{matrix}
  {{{\bar{x}}}_{1}}\left( \alpha  \right)+\left( {{{\bar{x}}}_{2}}\left( \alpha  \right)+{{{\bar{x}}}_{3}}\left( \alpha  \right) \right)={{c}_{1}}+{{k}_{1}}\alpha +({{c}_{2}}+{{c}_{3}}+({{k}_{2}}+{{k}_{3}})\alpha )= \\ 
  =({{c}_{1}}+{{c}_{2}}+({{k}_{1}}+{{k}_{2}})\alpha )+{{c}_{3}}+{{k}_{3}}\alpha =\left( {{{\bar{x}}}_{1}}\left( \alpha  \right)+{{{\bar{x}}}_{2}}\left( \alpha  \right) \right)+{{{\bar{x}}}_{3}}\left( \alpha  \right) \\ 
\end{matrix}\]
Введём нейтральный (нулевой) элемент
\begin{equation}
\label{eq:fuzzy-kc-zero}
	\bar{0}=0+0\alpha \in K
\end{equation}
такой, что 
\begin{equation}
\label{eq:fuzzy-kc-zeroprop}
	\forall \bar{x}(\alpha )\in K:\ \bar{x}(\alpha )+\bar{0}=c+k\alpha +0+0\alpha =\bar{x}(\alpha )
\end{equation}

Также определим для каждого $\bar{x}\left( \alpha  \right)=c+k\alpha \in K$ единственный элемент $-\bar{x}\left( \alpha  \right)\in K$, называемый противоположным, такой, что выполняется равенство
\begin{equation}
\label{eq:fuzzy-kc-inverse-minus}
	\bar{x}\left( \alpha  \right)+\left( -\bar{x}\left( \alpha  \right) \right)=\bar{0}
\end{equation}

Очевидно, что противоположный элемент можно определить следующим образом:
	\[-\bar{x}\left( \alpha  \right)=-c-k\alpha \], 	(1.83)
так что равенство \eqref{eq:fuzzy-kc-inverse-minus} будет справедливым:
\begin{equation*}
	\bar{x}\left( \alpha  \right)+\left( -\bar{x}\left( \alpha  \right) \right)=c+k\alpha -c-k\alpha =0+0\alpha =\bar{0}.
\end{equation*}

Операция вычитания нечётких чисел вводится как сложение числа $\bar{x}_1\left( \alpha  \right)$ с числом, противоположным числу $\bar{x}_2\left( \alpha  \right)$:
	\[{{\bar{x}}_{1}}\left( \alpha  \right)-{{\bar{x}}_{2}}\left( \alpha  \right)={{\bar{x}}_{1}}\left( \alpha  \right)+\left( -{{{\bar{x}}}_{2}}\left( \alpha  \right) \right)={{c}_{1}}-{{c}_{2}}+\left( {{k}_{1}}-{{k}_{2}} \right)\alpha \] 	(1.84)
Пример. Даны два нечётких числа $\tilde{A}=\left\langle 3;4;1 \right\rangle $ и $\tilde{B}=\left\langle 5;2;3 \right\rangle $. Выполнить операции ${{\tilde{A}}^{*}}+{{\tilde{B}}^{*}}$, ${{\tilde{A}}^{*}}-{{\tilde{B}}^{*}}$.
Вначале запишем уравнения для левой и правой ветвей каждого из чисел и оптимальные в смысле сохранения нечёткой информации значения $\lambda $:
	\[\begin{matrix}
   \left[ \begin{aligned}
  & x_{{\tilde{A}}}^{L}\left( \alpha  \right)=-1+4\alpha  \\ 
 & x_{{\tilde{A}}}^{R}\left( \alpha  \right)=4-\alpha  \\ 
 & {{\lambda }_{{\tilde{A}}}}=\frac{4}{4+1}=\frac{4}{5} \\ 
\end{aligned} \right. & \left[ \begin{aligned}
  & x_{{\tilde{B}}}^{L}\left( \alpha  \right)=3+2\alpha  \\ 
 & x_{{\tilde{B}}}^{R}\left( \alpha  \right)=8-3\alpha  \\ 
 & {{\lambda }_{{\tilde{B}}}}=\frac{2}{2+3}=\frac{2}{5} \\ 
\end{aligned} \right.  \\
\end{matrix}\] 
Найдём модифицированные значения:
	\[\left[ \begin{aligned}
  & {{{\bar{x}}}_{{\tilde{A}}}}\left( \alpha  \right)=\frac{4}{5}\left( -1+4\alpha  \right)+\frac{1}{5}\left( 4-\alpha  \right)=\frac{-4+16\alpha +4-\alpha }{5}=3\alpha  \\ 
 & {{{\bar{x}}}_{{\tilde{B}}}}\left( \alpha  \right)=\frac{2}{5}\left( 3+2\alpha  \right)+\frac{3}{5}\left( 8-3\alpha  \right)=\frac{6+4\alpha +24-9\alpha }{5}=6-\alpha  \\ 
\end{aligned} \right.\] 	(1.85)
Исходя из (1.85), значения искомых выражений равны:
	\[\left[ \begin{aligned}
  & {{{\tilde{A}}}^{*}}+{{{\tilde{B}}}^{*}}=3\alpha +6-\alpha =6+2\alpha  \\ 
 & {{{\tilde{A}}}^{*}}-{{{\tilde{B}}}^{*}}=3\alpha -6+\alpha =-6+4\alpha  \\ 
\end{aligned} \right.\] 
Операция умножения
Введём на множестве $K$ операцию умножения. Её можно было бы определить с помощью следующего выражения как сумму произведений компонент модифицированных нечётких чисел
	\[{{\bar{x}}_{1}}(\alpha )\cdot {{\bar{x}}_{2}}(\alpha )=r_{2}^{'}\left( \alpha  \right)=\left( {{c}_{1}}+{{k}_{1}}\alpha  \right)\left( {{c}_{2}}+{{k}_{2}}\alpha  \right)={{c}_{1}}{{c}_{2}}+{{c}_{1}}{{k}_{2}}\alpha +{{c}_{2}}{{k}_{1}}\alpha +{{k}_{1}}{{k}_{2}}{{\alpha }^{2}}\] 	(1.86)
Однако такое определение приводит к искажению треугольного вида результата нечётких операций, поскольку в (1.86) появляется слагаемое с ${{\alpha }^{2}}$. А это означает, что \[r_{2}^{'}\left( \alpha  \right)\notin K\].
Для того, чтобы результат операции умножения остался в множестве $K$, воспользуемся линейной интерполяцией – зависимость \[{{r}_{2}}(\alpha )\] будет восстанавливаться в виде линейной функции по значениям выражения (1.86) при $\alpha =0$ и $\alpha =1$. В первом случае $r_{2}^{'}\left( 0 \right)={{c}_{1}}{{c}_{2}}$, во втором \[r_{2}^{'}\left( 1 \right)=\left( {{c}_{1}}+{{k}_{1}} \right)\left( {{c}_{2}}+{{k}_{2}} \right)\]. Подставляя данные значения в уравнение прямой ${{r}_{2}}\left( \alpha  \right)$, получаем:
	\[\frac{\alpha -0}{1-0}=\frac{{{r}_{2}}\left( \alpha  \right)-{{c}_{1}}{{c}_{2}}}{\left( {{c}_{1}}+{{k}_{1}} \right)\left( {{c}_{2}}+{{k}_{2}} \right)-{{c}_{1}}{{c}_{2}}}\],	(1.87)
откуда, упрощая знаменатель второй дроби, получаем:
	\[\alpha =\frac{{{r}_{2}}\left( \alpha  \right)-{{c}_{1}}{{c}_{2}}}{{{c}_{1}}{{k}_{2}}+{{c}_{2}}{{k}_{1}}+{{k}_{1}}{{k}_{2}}}\] 	\[{{r}_{2}}\left( \alpha  \right)={{c}_{1}}{{c}_{2}}+\left( {{c}_{1}}{{k}_{2}}+{{c}_{2}}{{k}_{1}}+{{k}_{1}}{{k}_{2}} \right)\alpha \in K\] 	(1.88)
Таким образом, операция умножения на $K$ вводится следующим образом:
	\[{{\bar{x}}_{1}}(\alpha )\cdot {{\bar{x}}_{2}}(\alpha )={{r}_{2}}\left( \alpha  \right)={{c}_{1}}{{c}_{2}}+({{c}_{1}}{{k}_{2}}+{{c}_{2}}{{k}_{1}}+{{k}_{1}}{{k}_{2}})\alpha ;\ \ {{r}_{2}}\left( \alpha  \right)\in K\] 	(1.89)
Умножение нечёткого числа на скаляр $\beta $ является частным случаем операции (1.89), поскольку скаляр представляется в виде нечёткого синглтона
	\[\bar{\beta }=\beta +0\alpha \in K\].
Докажем основные свойства операции умножения.
Коммутативность доказывается элементарно:
	\[\begin{matrix}
  {{{\bar{x}}}_{1}}(\alpha )\cdot {{{\bar{x}}}_{2}}(\alpha )={{c}_{1}}{{c}_{2}}+({{c}_{1}}{{k}_{2}}+{{c}_{2}}{{k}_{1}}+{{k}_{1}}{{k}_{2}})\alpha = \\ 
  ={{c}_{2}}{{c}_{1}}+\left( {{c}_{2}}{{k}_{1}}+{{c}_{1}}{{k}_{2}}+{{k}_{2}}{{k}_{1}} \right)\alpha ={{{\bar{x}}}_{2}}(\alpha )\cdot {{{\bar{x}}}_{1}}(\alpha ) \\ 
\end{matrix}\] 
Ассоциативность. Для доказательства равенства
	\[{{\bar{x}}_{1}}\left( \alpha  \right)\cdot \left( {{{\bar{x}}}_{2}}\left( \alpha  \right){{{\bar{x}}}_{3}}\left( \alpha  \right) \right)=\left( {{{\bar{x}}}_{1}}\left( \alpha  \right){{{\bar{x}}}_{2}}\left( \alpha  \right) \right)\cdot {{\bar{x}}_{3}}\left( \alpha  \right)\] 	(1.90)
вычислим по отдельности и сравним результаты правой и левой частей:
$\begin{matrix}
  {{{\bar{x}}}_{1}}\left( \alpha  \right)\cdot \left( {{{\bar{x}}}_{2}}\left( \alpha  \right){{{\bar{x}}}_{3}}\left( \alpha  \right) \right)=\left( {{c}_{1}}+{{k}_{1}}\alpha  \right)\left( \left( {{c}_{2}}+{{k}_{2}}\alpha  \right)\left( {{c}_{3}}+{{k}_{3}}\alpha  \right) \right)= \\ 
  =\left( {{c}_{1}}+{{k}_{1}}\alpha  \right)\left( {{c}_{2}}{{c}_{3}}+\left( {{k}_{2}}{{c}_{3}}+{{k}_{3}}{{c}_{2}}+{{k}_{2}}{{k}_{3}} \right)\alpha  \right)= \\ 
  ={{c}_{1}}{{c}_{2}}{{c}_{3}}+\left( {{c}_{1}}{{k}_{2}}{{c}_{3}}+{{c}_{1}}{{c}_{2}}{{k}_{3}}+{{c}_{1}}{{k}_{2}}{{k}_{3}}+{{k}_{1}}{{c}_{2}}{{c}_{3}}+{{k}_{1}}{{k}_{2}}{{c}_{3}}+{{k}_{1}}{{c}_{2}}{{k}_{3}}+{{k}_{1}}{{k}_{2}}{{k}_{3}} \right)\alpha  \\ 
\end{matrix}$ 
$\begin{matrix}
  \left( {{{\bar{x}}}_{1}}\left( \alpha  \right){{{\bar{x}}}_{2}}\left( \alpha  \right) \right)\cdot {{{\bar{x}}}_{3}}\left( \alpha  \right)=\left( \left( {{c}_{1}}+{{k}_{1}}\alpha  \right)\left( {{c}_{2}}+{{k}_{2}}\alpha  \right) \right)\left( {{c}_{3}}+{{k}_{3}}\alpha  \right)= \\ 
  =\left( {{c}_{1}}{{c}_{2}}+\left( {{k}_{1}}{{c}_{2}}+{{c}_{1}}{{k}_{2}}+{{k}_{1}}{{k}_{2}} \right)\alpha  \right)\left( {{c}_{3}}+{{k}_{3}}\alpha  \right)= \\ 
  ={{c}_{1}}{{c}_{2}}{{c}_{3}}+\left( {{k}_{1}}{{c}_{2}}{{c}_{3}}+{{c}_{1}}{{k}_{2}}{{c}_{3}}+{{k}_{1}}{{k}_{2}}{{c}_{3}}+{{c}_{1}}{{c}_{2}}{{k}_{3}}+{{k}_{1}}{{c}_{2}}{{k}_{3}}+{{c}_{1}}{{k}_{2}}{{k}_{3}}+{{k}_{1}}{{k}_{2}}{{k}_{3}} \right) \\ 
\end{matrix}$ 
Путём сравнения результатов умножения можно убедиться, что свойство ассоциативности верно.
Для доказательства дистрибутивности умножения относительно сложения, т.е. справедливости равенства 
	\[{{\bar{x}}_{1}}(\alpha )\cdot {{\bar{x}}_{2}}(\alpha )+{{\bar{x}}_{1}}(\alpha )\cdot {{\bar{x}}_{3}}(\alpha )={{\bar{x}}_{1}}(\alpha )\cdot ({{\bar{x}}_{2}}(\alpha )+{{\bar{x}}_{3}}(\alpha ))\] 	(1.91)
также выполним действия в левой и правой частях выражения (1.91) по отдельности, а затем сравним результаты:
\[\begin{matrix}
  {{{\bar{x}}}_{1}}\left( \alpha  \right){{{\bar{x}}}_{2}}\left( \alpha  \right)+{{{\bar{x}}}_{1}}\left( \alpha  \right){{{\bar{x}}}_{3}}\left( \alpha  \right)=({{c}_{1}}+{{k}_{1}}\alpha )\cdot ({{c}_{2}}+{{k}_{2}}\alpha )+({{c}_{1}}+{{k}_{1}}\alpha )\cdot ({{c}_{3}}+{{k}_{3}}\alpha )= \\ 
  ={{c}_{1}}{{c}_{2}}+({{c}_{1}}{{k}_{2}}+{{c}_{2}}{{k}_{1}}+{{k}_{1}}{{k}_{2}})\alpha +{{c}_{1}}{{c}_{3}}+({{c}_{1}}{{k}_{3}}+{{c}_{3}}{{k}_{1}}+{{k}_{1}}{{k}_{3}})\alpha . \\ 
\end{matrix}\] 
\[\begin{matrix}
  {{{\bar{x}}}_{1}}\left( \alpha  \right)\left( {{{\bar{x}}}_{2}}\left( \alpha  \right)+{{{\bar{x}}}_{3}}\left( \alpha  \right) \right)=({{c}_{1}}+{{k}_{1}}\alpha )\cdot ({{c}_{2}}+{{k}_{2}}\alpha +{{c}_{3}}+{{k}_{3}}\alpha )= \\ 
  =({{c}_{1}}+{{k}_{1}}\alpha )\cdot ({{c}_{2}}+{{c}_{3}}+({{k}_{2}}+{{k}_{3}})\alpha )= \\ 
  ={{c}_{1}}{{c}_{2}}+{{c}_{1}}{{c}_{3}}+({{c}_{1}}{{k}_{2}}+{{c}_{2}}{{k}_{1}}+{{k}_{1}}{{k}_{2}})\alpha +({{c}_{1}}{{k}_{3}}+{{c}_{3}}{{k}_{1}}+{{k}_{1}}{{k}_{3}})\alpha . \\ 
\end{matrix}\] 
Сравнение результатов вычислений подтверждает верность равенства (1.91), т.е. дистрибутивность операции умножения относительно сложения.
Проиллюстрируем введённые операции примерами.
Пример. Даны нечёткие числа $\tilde{A}=\left\langle 4;1;2 \right\rangle $. $\tilde{B}=\left\langle 7;3;1 \right\rangle $ и $\tilde{C}=\left\langle 1;4;2 \right\rangle $. Выполнить операции $2\tilde{A}+\tilde{B}\tilde{C}$, $\tilde{C}\left( \tilde{A}-3\tilde{B} \right)$ с использованием модифицированных чисел.
Вначале найдём оптимальные в смысле сохранения нечёткой информации значения $\lambda$:
	\[\left[ \begin{aligned}
  & {{\lambda }_{{\tilde{A}}}}=\frac{1}{1+2}=\frac{1}{3} \\ 
 & {{\lambda }_{{\tilde{B}}}}=\frac{3}{3+1}=\frac{3}{4} \\ 
 & {{\lambda }_{{\tilde{C}}}}=\frac{4}{4+2}=\frac{2}{3} \\ 
\end{aligned} \right.\].
Найдём модифицированные нечёткие числа для каждого из исходных чисел согласно формулам (1.77) и (1.78):
	\[\left[ \begin{aligned}
  & {{{\bar{x}}}_{{\tilde{A}}}}\left( \alpha  \right)=4+2-\frac{1}{3}\left( 1+2 \right)+\left( \frac{1}{3}\left( 1+2 \right)-2 \right)\alpha =5-\alpha  \\ 
 & {{{\bar{x}}}_{{\tilde{B}}}}\left( \alpha  \right)=7+1-\frac{3}{4}\left( 3+1 \right)+\left( \frac{3}{4}\left( 3+1 \right)-1 \right)\alpha =5+2\alpha  \\ 
 & {{{\bar{x}}}_{{\tilde{C}}}}\left( \alpha  \right)=1+2-\frac{2}{3}\left( 4+2 \right)+\left( \frac{2}{3}\left( 4+2 \right)-2 \right)\alpha =-1+2\alpha  \\ 
\end{aligned} \right.\] 
Значение первого выражения в модифицированном виде равно
	$\begin{matrix}
  2{{{\tilde{A}}}^{*}}+{{{\tilde{B}}}^{*}}{{{\tilde{C}}}^{*}}=2\left( 5-\alpha  \right)+\left( 5+2\alpha  \right)\left( -1+2\alpha  \right)= \\ 
  =10-2\alpha +\left( -5+\left( 10-2+4 \right)\alpha  \right)=5+10\alpha  \\ 
\end{matrix}$
а второго
	\[\begin{matrix}
  {{{\tilde{C}}}^{*}}\left( {{{\tilde{A}}}^{*}}-3{{{\tilde{B}}}^{*}} \right)=\left( -1+2\alpha  \right)\left( 5-\alpha -3\left( 5+2\alpha  \right) \right)=\left( -1+2\alpha  \right)\left( -10-7\alpha  \right)= \\ 
  =10+\left( 7-20-14 \right)\alpha =10-27\alpha  \\ 
\end{matrix}\] 
Перейдём к рассмотрению нейтрального по умножению и обратного элементов. Введём единичный элемент
	\[\bar{1}=1+0\alpha \in K\] 	(1.92)
такой, что 
	\[\forall \bar{x}\left( \alpha  \right)\in K\quad \bar{x}\left( \alpha  \right)\cdot \bar{1}=\bar{x}\left( \alpha  \right)\] 	(1.93)
Равенство (1.93) легко подтверждается с помощью формулы (1.89):
	\[\bar{1}\cdot \bar{x}(\alpha )=(1+0\alpha )(c+k\alpha )=c+k\alpha =\bar{x}(\alpha )\] 
Несколько сложнее вводится на множестве К обратный элемент ${{\bar{x}}^{-1}}\left( \alpha  \right)\in K$ такой, что
	\[\bar{x}\left( \alpha  \right){{\bar{x}}^{-1}}\left( \alpha  \right)=\bar{1}\] 	(1.94)
Поскольку предполагается, что ${{\bar{x}}^{-1}}\left( \alpha  \right)\in K$, то будем искать обратный элемент в виде ${{\bar{x}}^{-1}}\left( \alpha  \right)={c}'+{k}'\alpha $. Имеем
	\[\bar{x}\left( \alpha  \right){{\bar{x}}^{-1}}\left( \alpha  \right)=\left( c+k\alpha  \right)\left( {c}'+{k}'\alpha  \right)=c{c}'+\left( c{k}'+{c}'k+k{k}' \right)\alpha =1+0\alpha \] 	(1.95)
Из (1.95) очевидно, что должны выполняться равенства
	\[\left\{ \begin{aligned}
  & c{c}'=1 \\ 
 & c{k}'+{c}'k+k{k}'=0 \\ 
\end{aligned} \right.\].
Из первого уравнения системы находим ${c}'=\frac{1}{c};\ c\ne 0$. Подставляя найденное значение во второе уравнение, имеем:
	\[c{k}'+\frac{k}{c}+k{k}'=0\],
откуда
	\[{k}'=\frac{-k}{c(k+c)}\] 
Таким образом, обратный элемент вводится в виде
	\[{{\bar{x}}^{-1}}(\alpha )=\frac{1}{c}-\frac{k}{c(c+k)}\alpha ,\quad c\ne 0\] 	(1.96)
Очевидно, что для существования обратного элемента число $\bar{x}\left( \alpha  \right)$ должно иметь ненулевую моду, поскольку, согласно (1.78), $c+k=m\ne 0$.
Операция деления вводится как умножение числа ${{\bar{x}}_{1}}\left( \alpha  \right)$ на число, обратное числу ${{\bar{x}}_{2}}\left( \alpha  \right)$:
	\[\begin{matrix}
  \frac{{{{\bar{x}}}_{1}}\left( \alpha  \right)}{{{{\bar{x}}}_{2}}\left( \alpha  \right)}={{{\bar{x}}}_{1}}\left( \alpha  \right)\bar{x}_{2}^{-1}\left( \alpha  \right)=\left( {{c}_{1}}+{{k}_{1}}\alpha  \right)\left( \frac{1}{{{c}_{2}}}-\frac{{{k}_{2}}}{{{c}_{2}}\left( {{k}_{2}}+{{c}_{2}} \right)}\alpha  \right)= \\ 
  =\frac{{{c}_{1}}}{{{c}_{2}}}+\left( -\frac{{{c}_{1}}{{k}_{2}}}{{{c}_{2}}\left( {{k}_{2}}+{{c}_{2}} \right)}+\frac{{{k}_{1}}}{{{c}_{2}}}-\frac{{{k}_{1}}{{k}_{2}}}{{{c}_{2}}\left( {{k}_{2}}+{{c}_{2}} \right)} \right)\alpha ;\ {{c}_{2}}\ne 0;\ {{k}_{2}}+{{c}_{2}}\ne 0 \\ 
\end{matrix}\] 	(1.97)
Таким образом, алгебра модифицированных нечётких чисел удовлетворяет всем аксиомам поля, а вводимые выше алгебраические операции над элементами множества $K$ модифицированных нечётких чисел позволяют рассматривать его как линейное пространство над полем P [Воеводин Линейная алгебра].
Стоит отметить, что расчёты с использованием модифицированных нечётких чисел можно проводить несколько иным, более удобным с вычислительной точки зрения, методом. Поскольку все элементы множества $K$ имеют линейную структуру, то для восстановления конкретного модифицированного числа $\tilde{A}$ достаточно знать два значения – ${{\bar{x}}_{{\tilde{A}}}}\left( 0 \right)$ и ${{\bar{x}}_{{\tilde{A}}}}\left( 1 \right)={{m}_{{\tilde{A}}}}$. Их подстановка в уравнение прямой позволяет получить зависимость ${{\bar{x}}_{{\tilde{A}}}}\left( \alpha  \right)$ или ${{\mu }_{{\tilde{A}}}}\left( x \right)$ в явном виде:
	\[\frac{\alpha -0}{1-0}=\frac{{{{\bar{x}}}_{{\tilde{A}}}}\left( \alpha  \right)-{{{\bar{x}}}_{{\tilde{A}}}}\left( 0 \right)}{{{{\bar{x}}}_{{\tilde{A}}}}\left( 1 \right)-{{{\bar{x}}}_{{\tilde{A}}}}\left( 0 \right)}\] 	(1.98)
	${{\bar{x}}_{{\tilde{A}}}}\left( \alpha  \right)={{\bar{x}}_{{\tilde{A}}}}\left( 0 \right)+\alpha \left( {{{\bar{x}}}_{{\tilde{A}}}}\left( 1 \right)-{{{\bar{x}}}_{{\tilde{A}}}}\left( 0 \right) \right)=\alpha {{\bar{x}}_{{\tilde{A}}}}\left( 1 \right)+\left( 1-\alpha  \right){{\bar{x}}_{{\tilde{A}}}}\left( 0 \right)$ 	(1.99)
Все вычисления с использованием данного способа ведутся только на двух $\alpha $-уровнях над действительными числами без использования дополнительных параметров. Если обозначить за $*$ произвольную арифметическую операцию, то для чисел в форме (1.99) её результат будет выглядеть следующим образом:
	\[{{\bar{x}}_{{\tilde{A}}}}\left( \alpha  \right)*{{\bar{x}}_{{\tilde{B}}}}\left( \alpha  \right)=\alpha \left( {{{\bar{x}}}_{{\tilde{A}}}}\left( 1 \right)*{{{\bar{x}}}_{{\tilde{B}}}}\left( 1 \right) \right)+\left( 1-\alpha  \right)\left( {{{\bar{x}}}_{{\tilde{A}}}}\left( 0 \right)*{{{\bar{x}}}_{{\tilde{B}}}}\left( 0 \right) \right)\] 	(1.100)
По сути, вводится другая алгебра на множестве $K$, оперирующая парой чётких значений и автоморфная введённой ранее алгебре модифицированных нечётких чисел. В самом деле, существует взаимно однозначная функция $f:K\to K$, которая позволяет сопоставить числу вида $c+k\alpha $ число вида $\alpha {{\bar{x}}_{{\tilde{A}}}}\left( 1 \right)+\left( 1-\alpha  \right){{\bar{x}}_{{\tilde{A}}}}\left( 0 \right)$. Воспользовавшись (1.99), (1.77) и приравняв свободные члены и коэффициенты при $\alpha$, получим следующее соответствие
	\[\left[ \begin{aligned}
  & c={{{\bar{x}}}_{{\tilde{A}}}}\left( 0 \right) \\ 
 & k={{{\bar{x}}}_{{\tilde{A}}}}\left( 1 \right)-{{{\bar{x}}}_{{\tilde{A}}}}\left( 0 \right) \\ 
\end{aligned} \right.\Leftrightarrow \left[ \begin{aligned}
  & {{{\bar{x}}}_{{\tilde{A}}}}\left( 0 \right)=c \\ 
 & {{{\bar{x}}}_{{\tilde{A}}}}\left( 1 \right)=c+k \\ 
\end{aligned} \right.\] 
Стоит отметить, что преобразование $L$ и вводимая алгебра модифицированных нечётких чисел снимают проблему сравнения двух нечётких чисел, поскольку все сравнения при решении задач происходят с действительными значениями нечёткого числа на выбранных $\alpha$-уровнях.
Пример. Даны нечёткие числа $\tilde{A}=\left\langle 4;2;3 \right\rangle $. $\tilde{B}=\left\langle -2;6;2 \right\rangle $ и $\tilde{C}=\left\langle 1;1;4 \right\rangle $. Вычислить значение $\tilde{D}=\frac{3\tilde{A}\tilde{C}+4\tilde{B}}{\tilde{C}\left( \tilde{A}-2\tilde{B} \right)}$ с использованием модифицированных чисел.
Воспользуемся упрощённой методикой вычислений. Оптимальные в смысле сохранения нечёткой информации значения $\lambda$ равны:
	\[\left[ \begin{aligned}
  & {{\lambda }_{{\tilde{A}}}}=\frac{2}{2+3}=\frac{2}{5} \\ 
 & {{\lambda }_{{\tilde{B}}}}=\frac{6}{6+2}=\frac{3}{4} \\ 
 & {{\lambda }_{{\tilde{C}}}}=\frac{1}{1+4}=\frac{1}{5} \\ 
\end{aligned} \right.\] 	(1.101)
Используя формулы (1.78) c учётом (1.101), получим коэффициенты для модифицированных нечётких чисел:
	\[\begin{aligned}
  & {{c}_{{\tilde{A}}}}=4+3-\frac{2}{5}\left( 2+3 \right)=5;\quad {{k}_{{\tilde{A}}}}=\frac{2}{5}\left( 2+3 \right)-3=-1 \\ 
 & {{c}_{{\tilde{B}}}}=-2+2-\frac{3}{4}\left( 6+2 \right)=-6;\quad {{k}_{{\tilde{B}}}}=\frac{3}{4}\left( 6+2 \right)-2=4 \\ 
 & {{c}_{{\tilde{C}}}}=1+4-\frac{1}{5}\left( 1+4 \right)=4;\quad {{k}_{{\tilde{C}}}}=\frac{1}{5}\left( 1+4 \right)-4=-3 \\ 
\end{aligned}\] 
Отсюда
	\[\left[ \begin{aligned}
  & {{{\bar{x}}}_{{\tilde{A}}}}\left( \alpha  \right)=5-\alpha  \\ 
 & {{{\bar{x}}}_{{\tilde{B}}}}\left( \alpha  \right)=-6+4\alpha  \\ 
 & {{{\bar{x}}}_{{\tilde{C}}}}\left( \alpha  \right)=4-3\alpha  \\ 
\end{aligned} \right.\] 	(1.102)
При $\alpha=1$ выражения (1.102) принимают следующие значения
	\[{{\bar{x}}_{{\tilde{A}}}}\left( 1 \right)=4;\ {{\bar{x}}_{{\tilde{B}}}}\left( 1 \right)=-2;\ {{\bar{x}}_{{\tilde{C}}}}\left( 1 \right)=1\] 	(1.103)
а при $\alpha=0$
	\[{{\bar{x}}_{{\tilde{A}}}}\left( 1 \right)=5;\ {{\bar{x}}_{{\tilde{B}}}}\left( 1 \right)=-6;\ {{\bar{x}}_{{\tilde{C}}}}\left( 1 \right)=4\] 	(1.104)
Подставляя в выражение $\tilde{D}=\frac{3\tilde{A}\tilde{C}+4\tilde{B}}{\tilde{C}\left( \tilde{A}-2\tilde{B} \right)}$ вместо $\tilde A$, $\tilde B$ и $\tilde C$ соответствующие им чёткие значения модифицированных чисел при $\alpha=0$ и $\alpha=1$, получаем:
	${{\bar{x}}_{{\tilde{D}}}}\left( 1 \right)={{\left. \frac{3{{{\tilde{A}}}^{*}}{{{\tilde{C}}}^{*}}+4{{{\tilde{B}}}^{*}}}{{{{\tilde{C}}}^{*}}\left( {{{\tilde{A}}}^{*}}-2{{{\tilde{B}}}^{*}} \right)} \right|}_{\alpha =1}}=\frac{3\cdot 4\cdot 1+4\cdot \left( -2 \right)}{1\cdot \left( 4-2\cdot \left( -2 \right) \right)}=\frac{12-8}{4+4}=\frac{1}{2}$,
	\[{{\bar{x}}_{{\tilde{D}}}}\left( 0 \right)={{\left. \frac{3{{{\tilde{A}}}^{*}}{{{\tilde{C}}}^{*}}+4{{{\tilde{B}}}^{*}}}{{{{\tilde{C}}}^{*}}\left( {{{\tilde{A}}}^{*}}-2{{{\tilde{B}}}^{*}} \right)} \right|}_{\alpha =0}}=\frac{3\cdot 5\cdot 4+4\cdot \left( -6 \right)}{4\cdot \left( 5-2\cdot \left( -6 \right) \right)}=\frac{60-24}{4\cdot 17}=\frac{9}{17}\].
Согласно формуле (1.99), модифицированный результат будет равен
	\[{{\bar{x}}_{{\tilde{D}}}}\left( \alpha  \right)=\frac{9}{17}+\alpha \left( \frac{1}{2}-\frac{9}{17} \right)=\frac{9}{17}-\frac{1}{34}\alpha \] 
Рассмотрим более сложный пример. Необходимо решить уравнение $\tilde{A}x=\tilde{B}$, в котором $\tilde{A}=\left\langle 3;1;2 \right\rangle $, $\tilde{B}=\left\langle 4;4;1 \right\rangle $. В терминах модифицированных нечётких чисел его решение будет иметь вид
	\[x\left( \alpha  \right)=\frac{{{{\bar{x}}}_{{\tilde{B}}}}\left( \alpha  \right)}{{{{\bar{x}}}_{{\tilde{A}}}}\left( \alpha  \right)}\] 	(1.105)
Поскольку операция деления для преобразованных нечётких чисел вводится как умножение на обратное число, перепишем (1.105) в виде
	\[x\left( \alpha  \right)=\bar x_{\tilde B}\left( \alpha  \right)\cdot \bar{x}_{{\tilde{A}}}^{-1}\left( \alpha  \right)\] 	(1.106)

Выберем значения $\lambda_{\tilde A}$ и $\lambda_{\tilde B}$, равные $\frac{a_{\tilde A}}{d_{\tilde A}}$ и $\frac{a_{\tilde B}}{d_{\tilde B}}$, в соответствие с критерием сохранения максимального количества нечёткой информации. Они равны
	\[{{\lambda }_{{\tilde{A}}}}=\frac{1}{3};\ {{\lambda }_{{\tilde{B}}}}=\frac{4}{5}\] 	(1.107)
Воспользовавшись формулами (1.77), (1.78) и (1.107), получаем
	\[\left[ \begin{aligned}
  & {{{\bar{x}}}_{{\tilde{A}}}}\left( \alpha  \right)=4-\alpha  \\ 
 & {{{\bar{x}}}_{{\tilde{B}}}}\left( \alpha  \right)=3\alpha +1 \\ 
\end{aligned} \right.\] 	(1.108)
Найдём обратный элемент $\bar{x}_{{\tilde{A}}}^{-1}\left( \alpha  \right)$ согласно формуле (1.96): 
	$\bar{x}_{{\tilde{A}}}^{-1}\left( \alpha  \right)=\frac{1}{4}-\frac{1}{\left( -1 \right)\cdot \left( 4-1 \right)}\alpha =\frac{1}{4}+\frac{1}{12}\alpha $ 	(1.109)
Пользуясь значениями из (1.108), (1.109) и подставляя их в (1.106), окончательно получаем
	\[x\left( \alpha  \right)=\left( 1+3\alpha  \right)\left( \frac{1}{4}+\frac{1}{12}\alpha  \right)=\frac{1}{4}+\frac{13}{12}\alpha \] 	(1.110)
Функция принадлежности модифицированного решения определяется как обратная к (1.110) – ${{\mu }_{x(\alpha )}}\left( x \right)=\frac{12}{13}x-\frac{3}{13}$. Подстановка решения задачи в исходное уравнение $\tilde{A}x=\tilde{B}$ с учётом формулы (1.89) приводит к верному равенству:
	\[\left( 4-\alpha  \right)\left( \frac{1}{4}+\frac{13}{12}\alpha  \right)=3\alpha +1\];
	\[\frac{52-13-3}{12}\alpha +1=3\alpha +1\].

Описанная выше модель применима в случаях использования асимметричных нечётких чисел. Однако существуют ситуации, когда  необходимо 

В статье \cite{Kanischeva} предложена алгебра для двухкомпонентных нечётких чисел. Её ключевое отличие в том, что не используется преобразование для $\alpha$-интервалов, а операции над числами выполняются как операции над


\section{Построение алгебры нечетких чисел, удовлетворяющей требованиям к решению задач} 
\label{chapter2_3}
Проблемы создания специального программного обеспечения решения задач на основе предложенной алгебры. Нерешенными остались проблемы построения полного порядка. Проведенное исследование показало, что подходящих методов сравнения нечетких чисел нет.
Предложен изоморфизм алгебры модифицированных нечетких чисел, который позволяет решать нечеткую задачу как две четких задачи на заданных альфа –уровнях.
Такой подход не только позволяет использовать все стандартные программные продукты для решения нечетких задач, но и решает проблему построения полного порядка.

\subsection{Алгебра модифицированных нечетких чисел с примерами}
Для того, чтобы использовать модифицированные нечёткие числа в качестве параметров чётких задач, необходимо построить алгебраическую систему для множества всех нечётких модифицированных чисел $K$. 
Будем строить чёткую алгебру \[P=\left\langle ;\ +,\,* \right\rangle \] на множестве модифицированных нечётких чисел $K=\left\{ \bar{x}\left( \alpha  \right) \right\};\ \alpha \in \left[ 0;1 \right]$ по аналогии с тем, как это делается в (Яхъяева). Для удобства дальнейших вычислений преобразуем $\bar{x}\left( \alpha  \right)$:
	\[\begin{matrix}
  \bar{x}\left( \alpha  \right)=\lambda \left( m-a+a\alpha  \right)+\left( 1-\lambda  \right)\left( m+b-b\alpha  \right)=a\alpha \lambda +\lambda \left( m-a \right)+m+b-b\alpha -\lambda \left( m+b \right)+b\alpha \lambda = \\ 
  =\alpha \left( \lambda a+\lambda b-b \right)+m+b-\lambda \left( m+b-m+a \right)=a\left( \lambda \left( a+b \right)-b \right)+m+b-\lambda \left( a+b \right) \\ 
\end{matrix}\] 
При построении алгебры будем использовать форму записи
	\[\bar{x}\left( \alpha  \right)=c+k\alpha \] 	(1.77)
где
	\[\begin{aligned}
  & \left[ \begin{aligned}
  & c=m+b-\lambda \left( a+b \right) \\ 
 & k=\lambda \left( a+b \right)-b \\ 
\end{aligned} \right. \\ 
 & \lambda \in \left[ 0;1 \right];\ c,k\in \mathbb{R} \\ 
\end{aligned}\] 	(1.78)
Операция сложения
Введем на множестве $K$ бинарную операцию сложения + следующим образом:
	\[\begin{matrix}
  {{{\bar{x}}}_{1}}(\alpha )+{{{\bar{x}}}_{2}}(\alpha )={{r}_{1}}\left( \alpha  \right)={{c}_{1}}+{{c}_{2}}+\left( {{k}_{1}}+{{k}_{2}} \right)\alpha  \\ 
  {{r}_{1}}\left( \alpha  \right)\in K \\ 
\end{matrix}\] 	(1.79)
Докажем основные свойства операции сложения.
1. Коммутативность:
\[{{\bar{x}}_{1}}\left( \alpha  \right)+{{\bar{x}}_{2}}\left( \alpha  \right)={{c}_{1}}+{{c}_{2}}+({{k}_{1}}+{{k}_{2}})\alpha ={{c}_{2}}+{{c}_{1}}+({{k}_{2}}+{{k}_{1}})\alpha ={{\bar{x}}_{2}}\left( \alpha  \right)+{{\bar{x}}_{1}}\left( \alpha  \right)\] 
2. Ассоциативность
\[\begin{matrix}
  {{{\bar{x}}}_{1}}\left( \alpha  \right)+\left( {{{\bar{x}}}_{2}}\left( \alpha  \right)+{{{\bar{x}}}_{3}}\left( \alpha  \right) \right)={{c}_{1}}+{{k}_{1}}\alpha +({{c}_{2}}+{{c}_{3}}+({{k}_{2}}+{{k}_{3}})\alpha )= \\ 
  =({{c}_{1}}+{{c}_{2}}+({{k}_{1}}+{{k}_{2}})\alpha )+{{c}_{3}}+{{k}_{3}}\alpha =\left( {{{\bar{x}}}_{1}}\left( \alpha  \right)+{{{\bar{x}}}_{2}}\left( \alpha  \right) \right)+{{{\bar{x}}}_{3}}\left( \alpha  \right) \\ 
\end{matrix}\]
Введём нейтральный (нулевой) элемент
\begin{equation}
\label{eq:fuzzy-kc-zero}
	\bar{0}=0+0\alpha \in K
\end{equation}
такой, что 
\begin{equation}
\label{eq:fuzzy-kc-zeroprop}
	\forall \bar{x}(\alpha )\in K:\ \bar{x}(\alpha )+\bar{0}=c+k\alpha +0+0\alpha =\bar{x}(\alpha )
\end{equation}

Также определим для каждого $\bar{x}\left( \alpha  \right)=c+k\alpha \in K$ единственный элемент $-\bar{x}\left( \alpha  \right)\in K$, называемый противоположным, такой, что выполняется равенство
\begin{equation}
\label{eq:fuzzy-kc-inverse-minus}
	\bar{x}\left( \alpha  \right)+\left( -\bar{x}\left( \alpha  \right) \right)=\bar{0}
\end{equation}

Очевидно, что противоположный элемент можно определить следующим образом:
	\[-\bar{x}\left( \alpha  \right)=-c-k\alpha \], 	(1.83)
так что равенство \eqref{eq:fuzzy-kc-inverse-minus} будет справедливым:
\begin{equation*}
	\bar{x}\left( \alpha  \right)+\left( -\bar{x}\left( \alpha  \right) \right)=c+k\alpha -c-k\alpha =0+0\alpha =\bar{0}.
\end{equation*}

Операция вычитания нечётких чисел вводится как сложение числа $\bar{x}_1\left( \alpha  \right)$ с числом, противоположным числу $\bar{x}_2\left( \alpha  \right)$:
	\[{{\bar{x}}_{1}}\left( \alpha  \right)-{{\bar{x}}_{2}}\left( \alpha  \right)={{\bar{x}}_{1}}\left( \alpha  \right)+\left( -{{{\bar{x}}}_{2}}\left( \alpha  \right) \right)={{c}_{1}}-{{c}_{2}}+\left( {{k}_{1}}-{{k}_{2}} \right)\alpha \] 	(1.84)
Пример. Даны два нечётких числа $\tilde{A}=\left\langle 3;4;1 \right\rangle $ и $\tilde{B}=\left\langle 5;2;3 \right\rangle $. Выполнить операции ${{\tilde{A}}^{*}}+{{\tilde{B}}^{*}}$, ${{\tilde{A}}^{*}}-{{\tilde{B}}^{*}}$.
Вначале запишем уравнения для левой и правой ветвей каждого из чисел и оптимальные в смысле сохранения нечёткой информации значения $\lambda $:
	\[\begin{matrix}
   \left[ \begin{aligned}
  & x_{{\tilde{A}}}^{L}\left( \alpha  \right)=-1+4\alpha  \\ 
 & x_{{\tilde{A}}}^{R}\left( \alpha  \right)=4-\alpha  \\ 
 & {{\lambda }_{{\tilde{A}}}}=\frac{4}{4+1}=\frac{4}{5} \\ 
\end{aligned} \right. & \left[ \begin{aligned}
  & x_{{\tilde{B}}}^{L}\left( \alpha  \right)=3+2\alpha  \\ 
 & x_{{\tilde{B}}}^{R}\left( \alpha  \right)=8-3\alpha  \\ 
 & {{\lambda }_{{\tilde{B}}}}=\frac{2}{2+3}=\frac{2}{5} \\ 
\end{aligned} \right.  \\
\end{matrix}\] 
Найдём модифицированные значения:
	\[\left[ \begin{aligned}
  & {{{\bar{x}}}_{{\tilde{A}}}}\left( \alpha  \right)=\frac{4}{5}\left( -1+4\alpha  \right)+\frac{1}{5}\left( 4-\alpha  \right)=\frac{-4+16\alpha +4-\alpha }{5}=3\alpha  \\ 
 & {{{\bar{x}}}_{{\tilde{B}}}}\left( \alpha  \right)=\frac{2}{5}\left( 3+2\alpha  \right)+\frac{3}{5}\left( 8-3\alpha  \right)=\frac{6+4\alpha +24-9\alpha }{5}=6-\alpha  \\ 
\end{aligned} \right.\] 	(1.85)
Исходя из (1.85), значения искомых выражений равны:
	\[\left[ \begin{aligned}
  & {{{\tilde{A}}}^{*}}+{{{\tilde{B}}}^{*}}=3\alpha +6-\alpha =6+2\alpha  \\ 
 & {{{\tilde{A}}}^{*}}-{{{\tilde{B}}}^{*}}=3\alpha -6+\alpha =-6+4\alpha  \\ 
\end{aligned} \right.\] 
Операция умножения
Введём на множестве $K$ операцию умножения. Её можно было бы определить с помощью следующего выражения как сумму произведений компонент модифицированных нечётких чисел
	\[{{\bar{x}}_{1}}(\alpha )\cdot {{\bar{x}}_{2}}(\alpha )=r_{2}^{'}\left( \alpha  \right)=\left( {{c}_{1}}+{{k}_{1}}\alpha  \right)\left( {{c}_{2}}+{{k}_{2}}\alpha  \right)={{c}_{1}}{{c}_{2}}+{{c}_{1}}{{k}_{2}}\alpha +{{c}_{2}}{{k}_{1}}\alpha +{{k}_{1}}{{k}_{2}}{{\alpha }^{2}}\] 	(1.86)
Однако такое определение приводит к искажению треугольного вида результата нечётких операций, поскольку в (1.86) появляется слагаемое с ${{\alpha }^{2}}$. А это означает, что \[r_{2}^{'}\left( \alpha  \right)\notin K\].
Для того, чтобы результат операции умножения остался в множестве $K$, воспользуемся линейной интерполяцией – зависимость \[{{r}_{2}}(\alpha )\] будет восстанавливаться в виде линейной функции по значениям выражения (1.86) при $\alpha =0$ и $\alpha =1$. В первом случае $r_{2}^{'}\left( 0 \right)={{c}_{1}}{{c}_{2}}$, во втором \[r_{2}^{'}\left( 1 \right)=\left( {{c}_{1}}+{{k}_{1}} \right)\left( {{c}_{2}}+{{k}_{2}} \right)\]. Подставляя данные значения в уравнение прямой ${{r}_{2}}\left( \alpha  \right)$, получаем:
	\[\frac{\alpha -0}{1-0}=\frac{{{r}_{2}}\left( \alpha  \right)-{{c}_{1}}{{c}_{2}}}{\left( {{c}_{1}}+{{k}_{1}} \right)\left( {{c}_{2}}+{{k}_{2}} \right)-{{c}_{1}}{{c}_{2}}}\],	(1.87)
откуда, упрощая знаменатель второй дроби, получаем:
	\[\alpha =\frac{{{r}_{2}}\left( \alpha  \right)-{{c}_{1}}{{c}_{2}}}{{{c}_{1}}{{k}_{2}}+{{c}_{2}}{{k}_{1}}+{{k}_{1}}{{k}_{2}}}\] 	\[{{r}_{2}}\left( \alpha  \right)={{c}_{1}}{{c}_{2}}+\left( {{c}_{1}}{{k}_{2}}+{{c}_{2}}{{k}_{1}}+{{k}_{1}}{{k}_{2}} \right)\alpha \in K\] 	(1.88)
Таким образом, операция умножения на $K$ вводится следующим образом:
	\[{{\bar{x}}_{1}}(\alpha )\cdot {{\bar{x}}_{2}}(\alpha )={{r}_{2}}\left( \alpha  \right)={{c}_{1}}{{c}_{2}}+({{c}_{1}}{{k}_{2}}+{{c}_{2}}{{k}_{1}}+{{k}_{1}}{{k}_{2}})\alpha ;\ \ {{r}_{2}}\left( \alpha  \right)\in K\] 	(1.89)
Умножение нечёткого числа на скаляр $\beta $ является частным случаем операции (1.89), поскольку скаляр представляется в виде нечёткого синглтона
	\[\bar{\beta }=\beta +0\alpha \in K\].
Докажем основные свойства операции умножения.
Коммутативность доказывается элементарно:
	\[\begin{matrix}
  {{{\bar{x}}}_{1}}(\alpha )\cdot {{{\bar{x}}}_{2}}(\alpha )={{c}_{1}}{{c}_{2}}+({{c}_{1}}{{k}_{2}}+{{c}_{2}}{{k}_{1}}+{{k}_{1}}{{k}_{2}})\alpha = \\ 
  ={{c}_{2}}{{c}_{1}}+\left( {{c}_{2}}{{k}_{1}}+{{c}_{1}}{{k}_{2}}+{{k}_{2}}{{k}_{1}} \right)\alpha ={{{\bar{x}}}_{2}}(\alpha )\cdot {{{\bar{x}}}_{1}}(\alpha ) \\ 
\end{matrix}\] 
Ассоциативность. Для доказательства равенства
	\[{{\bar{x}}_{1}}\left( \alpha  \right)\cdot \left( {{{\bar{x}}}_{2}}\left( \alpha  \right){{{\bar{x}}}_{3}}\left( \alpha  \right) \right)=\left( {{{\bar{x}}}_{1}}\left( \alpha  \right){{{\bar{x}}}_{2}}\left( \alpha  \right) \right)\cdot {{\bar{x}}_{3}}\left( \alpha  \right)\] 	(1.90)
вычислим по отдельности и сравним результаты правой и левой частей:
$\begin{matrix}
  {{{\bar{x}}}_{1}}\left( \alpha  \right)\cdot \left( {{{\bar{x}}}_{2}}\left( \alpha  \right){{{\bar{x}}}_{3}}\left( \alpha  \right) \right)=\left( {{c}_{1}}+{{k}_{1}}\alpha  \right)\left( \left( {{c}_{2}}+{{k}_{2}}\alpha  \right)\left( {{c}_{3}}+{{k}_{3}}\alpha  \right) \right)= \\ 
  =\left( {{c}_{1}}+{{k}_{1}}\alpha  \right)\left( {{c}_{2}}{{c}_{3}}+\left( {{k}_{2}}{{c}_{3}}+{{k}_{3}}{{c}_{2}}+{{k}_{2}}{{k}_{3}} \right)\alpha  \right)= \\ 
  ={{c}_{1}}{{c}_{2}}{{c}_{3}}+\left( {{c}_{1}}{{k}_{2}}{{c}_{3}}+{{c}_{1}}{{c}_{2}}{{k}_{3}}+{{c}_{1}}{{k}_{2}}{{k}_{3}}+{{k}_{1}}{{c}_{2}}{{c}_{3}}+{{k}_{1}}{{k}_{2}}{{c}_{3}}+{{k}_{1}}{{c}_{2}}{{k}_{3}}+{{k}_{1}}{{k}_{2}}{{k}_{3}} \right)\alpha  \\ 
\end{matrix}$ 
$\begin{matrix}
  \left( {{{\bar{x}}}_{1}}\left( \alpha  \right){{{\bar{x}}}_{2}}\left( \alpha  \right) \right)\cdot {{{\bar{x}}}_{3}}\left( \alpha  \right)=\left( \left( {{c}_{1}}+{{k}_{1}}\alpha  \right)\left( {{c}_{2}}+{{k}_{2}}\alpha  \right) \right)\left( {{c}_{3}}+{{k}_{3}}\alpha  \right)= \\ 
  =\left( {{c}_{1}}{{c}_{2}}+\left( {{k}_{1}}{{c}_{2}}+{{c}_{1}}{{k}_{2}}+{{k}_{1}}{{k}_{2}} \right)\alpha  \right)\left( {{c}_{3}}+{{k}_{3}}\alpha  \right)= \\ 
  ={{c}_{1}}{{c}_{2}}{{c}_{3}}+\left( {{k}_{1}}{{c}_{2}}{{c}_{3}}+{{c}_{1}}{{k}_{2}}{{c}_{3}}+{{k}_{1}}{{k}_{2}}{{c}_{3}}+{{c}_{1}}{{c}_{2}}{{k}_{3}}+{{k}_{1}}{{c}_{2}}{{k}_{3}}+{{c}_{1}}{{k}_{2}}{{k}_{3}}+{{k}_{1}}{{k}_{2}}{{k}_{3}} \right) \\ 
\end{matrix}$ 
Путём сравнения результатов умножения можно убедиться, что свойство ассоциативности верно.
Для доказательства дистрибутивности умножения относительно сложения, т.е. справедливости равенства 
	\[{{\bar{x}}_{1}}(\alpha )\cdot {{\bar{x}}_{2}}(\alpha )+{{\bar{x}}_{1}}(\alpha )\cdot {{\bar{x}}_{3}}(\alpha )={{\bar{x}}_{1}}(\alpha )\cdot ({{\bar{x}}_{2}}(\alpha )+{{\bar{x}}_{3}}(\alpha ))\] 	(1.91)
также выполним действия в левой и правой частях выражения (1.91) по отдельности, а затем сравним результаты:
\[\begin{matrix}
  {{{\bar{x}}}_{1}}\left( \alpha  \right){{{\bar{x}}}_{2}}\left( \alpha  \right)+{{{\bar{x}}}_{1}}\left( \alpha  \right){{{\bar{x}}}_{3}}\left( \alpha  \right)=({{c}_{1}}+{{k}_{1}}\alpha )\cdot ({{c}_{2}}+{{k}_{2}}\alpha )+({{c}_{1}}+{{k}_{1}}\alpha )\cdot ({{c}_{3}}+{{k}_{3}}\alpha )= \\ 
  ={{c}_{1}}{{c}_{2}}+({{c}_{1}}{{k}_{2}}+{{c}_{2}}{{k}_{1}}+{{k}_{1}}{{k}_{2}})\alpha +{{c}_{1}}{{c}_{3}}+({{c}_{1}}{{k}_{3}}+{{c}_{3}}{{k}_{1}}+{{k}_{1}}{{k}_{3}})\alpha . \\ 
\end{matrix}\] 
\[\begin{matrix}
  {{{\bar{x}}}_{1}}\left( \alpha  \right)\left( {{{\bar{x}}}_{2}}\left( \alpha  \right)+{{{\bar{x}}}_{3}}\left( \alpha  \right) \right)=({{c}_{1}}+{{k}_{1}}\alpha )\cdot ({{c}_{2}}+{{k}_{2}}\alpha +{{c}_{3}}+{{k}_{3}}\alpha )= \\ 
  =({{c}_{1}}+{{k}_{1}}\alpha )\cdot ({{c}_{2}}+{{c}_{3}}+({{k}_{2}}+{{k}_{3}})\alpha )= \\ 
  ={{c}_{1}}{{c}_{2}}+{{c}_{1}}{{c}_{3}}+({{c}_{1}}{{k}_{2}}+{{c}_{2}}{{k}_{1}}+{{k}_{1}}{{k}_{2}})\alpha +({{c}_{1}}{{k}_{3}}+{{c}_{3}}{{k}_{1}}+{{k}_{1}}{{k}_{3}})\alpha . \\ 
\end{matrix}\] 
Сравнение результатов вычислений подтверждает верность равенства (1.91), т.е. дистрибутивность операции умножения относительно сложения.
Проиллюстрируем введённые операции примерами.
Пример. Даны нечёткие числа $\tilde{A}=\left\langle 4;1;2 \right\rangle $. $\tilde{B}=\left\langle 7;3;1 \right\rangle $ и $\tilde{C}=\left\langle 1;4;2 \right\rangle $. Выполнить операции $2\tilde{A}+\tilde{B}\tilde{C}$, $\tilde{C}\left( \tilde{A}-3\tilde{B} \right)$ с использованием модифицированных чисел.
Вначале найдём оптимальные в смысле сохранения нечёткой информации значения $\lambda$:
	\[\left[ \begin{aligned}
  & {{\lambda }_{{\tilde{A}}}}=\frac{1}{1+2}=\frac{1}{3} \\ 
 & {{\lambda }_{{\tilde{B}}}}=\frac{3}{3+1}=\frac{3}{4} \\ 
 & {{\lambda }_{{\tilde{C}}}}=\frac{4}{4+2}=\frac{2}{3} \\ 
\end{aligned} \right.\].
Найдём модифицированные нечёткие числа для каждого из исходных чисел согласно формулам (1.77) и (1.78):
	\[\left[ \begin{aligned}
  & {{{\bar{x}}}_{{\tilde{A}}}}\left( \alpha  \right)=4+2-\frac{1}{3}\left( 1+2 \right)+\left( \frac{1}{3}\left( 1+2 \right)-2 \right)\alpha =5-\alpha  \\ 
 & {{{\bar{x}}}_{{\tilde{B}}}}\left( \alpha  \right)=7+1-\frac{3}{4}\left( 3+1 \right)+\left( \frac{3}{4}\left( 3+1 \right)-1 \right)\alpha =5+2\alpha  \\ 
 & {{{\bar{x}}}_{{\tilde{C}}}}\left( \alpha  \right)=1+2-\frac{2}{3}\left( 4+2 \right)+\left( \frac{2}{3}\left( 4+2 \right)-2 \right)\alpha =-1+2\alpha  \\ 
\end{aligned} \right.\] 
Значение первого выражения в модифицированном виде равно
	$\begin{matrix}
  2{{{\tilde{A}}}^{*}}+{{{\tilde{B}}}^{*}}{{{\tilde{C}}}^{*}}=2\left( 5-\alpha  \right)+\left( 5+2\alpha  \right)\left( -1+2\alpha  \right)= \\ 
  =10-2\alpha +\left( -5+\left( 10-2+4 \right)\alpha  \right)=5+10\alpha  \\ 
\end{matrix}$
а второго
	\[\begin{matrix}
  {{{\tilde{C}}}^{*}}\left( {{{\tilde{A}}}^{*}}-3{{{\tilde{B}}}^{*}} \right)=\left( -1+2\alpha  \right)\left( 5-\alpha -3\left( 5+2\alpha  \right) \right)=\left( -1+2\alpha  \right)\left( -10-7\alpha  \right)= \\ 
  =10+\left( 7-20-14 \right)\alpha =10-27\alpha  \\ 
\end{matrix}\] 
Перейдём к рассмотрению нейтрального по умножению и обратного элементов. Введём единичный элемент
	\[\bar{1}=1+0\alpha \in K\] 	(1.92)
такой, что 
	\[\forall \bar{x}\left( \alpha  \right)\in K\quad \bar{x}\left( \alpha  \right)\cdot \bar{1}=\bar{x}\left( \alpha  \right)\] 	(1.93)
Равенство (1.93) легко подтверждается с помощью формулы (1.89):
	\[\bar{1}\cdot \bar{x}(\alpha )=(1+0\alpha )(c+k\alpha )=c+k\alpha =\bar{x}(\alpha )\] 
Несколько сложнее вводится на множестве К обратный элемент ${{\bar{x}}^{-1}}\left( \alpha  \right)\in K$ такой, что
	\[\bar{x}\left( \alpha  \right){{\bar{x}}^{-1}}\left( \alpha  \right)=\bar{1}\] 	(1.94)
Поскольку предполагается, что ${{\bar{x}}^{-1}}\left( \alpha  \right)\in K$, то будем искать обратный элемент в виде ${{\bar{x}}^{-1}}\left( \alpha  \right)={c}'+{k}'\alpha $. Имеем
	\[\bar{x}\left( \alpha  \right){{\bar{x}}^{-1}}\left( \alpha  \right)=\left( c+k\alpha  \right)\left( {c}'+{k}'\alpha  \right)=c{c}'+\left( c{k}'+{c}'k+k{k}' \right)\alpha =1+0\alpha \] 	(1.95)
Из (1.95) очевидно, что должны выполняться равенства
	\[\left\{ \begin{aligned}
  & c{c}'=1 \\ 
 & c{k}'+{c}'k+k{k}'=0 \\ 
\end{aligned} \right.\].
Из первого уравнения системы находим ${c}'=\frac{1}{c};\ c\ne 0$. Подставляя найденное значение во второе уравнение, имеем:
	\[c{k}'+\frac{k}{c}+k{k}'=0\],
откуда
	\[{k}'=\frac{-k}{c(k+c)}\] 
Таким образом, обратный элемент вводится в виде
	\[{{\bar{x}}^{-1}}(\alpha )=\frac{1}{c}-\frac{k}{c(c+k)}\alpha ,\quad c\ne 0\] 	(1.96)
Очевидно, что для существования обратного элемента число $\bar{x}\left( \alpha  \right)$ должно иметь ненулевую моду, поскольку, согласно (1.78), $c+k=m\ne 0$.
Операция деления вводится как умножение числа ${{\bar{x}}_{1}}\left( \alpha  \right)$ на число, обратное числу ${{\bar{x}}_{2}}\left( \alpha  \right)$:
	\[\begin{matrix}
  \frac{{{{\bar{x}}}_{1}}\left( \alpha  \right)}{{{{\bar{x}}}_{2}}\left( \alpha  \right)}={{{\bar{x}}}_{1}}\left( \alpha  \right)\bar{x}_{2}^{-1}\left( \alpha  \right)=\left( {{c}_{1}}+{{k}_{1}}\alpha  \right)\left( \frac{1}{{{c}_{2}}}-\frac{{{k}_{2}}}{{{c}_{2}}\left( {{k}_{2}}+{{c}_{2}} \right)}\alpha  \right)= \\ 
  =\frac{{{c}_{1}}}{{{c}_{2}}}+\left( -\frac{{{c}_{1}}{{k}_{2}}}{{{c}_{2}}\left( {{k}_{2}}+{{c}_{2}} \right)}+\frac{{{k}_{1}}}{{{c}_{2}}}-\frac{{{k}_{1}}{{k}_{2}}}{{{c}_{2}}\left( {{k}_{2}}+{{c}_{2}} \right)} \right)\alpha ;\ {{c}_{2}}\ne 0;\ {{k}_{2}}+{{c}_{2}}\ne 0 \\ 
\end{matrix}\] 	(1.97)
Таким образом, алгебра модифицированных нечётких чисел удовлетворяет всем аксиомам поля, а вводимые выше алгебраические операции над элементами множества $K$ модифицированных нечётких чисел позволяют рассматривать его как линейное пространство над полем P [Воеводин Линейная алгебра].
Стоит отметить, что расчёты с использованием модифицированных нечётких чисел можно проводить несколько иным, более удобным с вычислительной точки зрения, методом. Поскольку все элементы множества $K$ имеют линейную структуру, то для восстановления конкретного модифицированного числа $\tilde{A}$ достаточно знать два значения – ${{\bar{x}}_{{\tilde{A}}}}\left( 0 \right)$ и ${{\bar{x}}_{{\tilde{A}}}}\left( 1 \right)={{m}_{{\tilde{A}}}}$. Их подстановка в уравнение прямой позволяет получить зависимость ${{\bar{x}}_{{\tilde{A}}}}\left( \alpha  \right)$ или ${{\mu }_{{\tilde{A}}}}\left( x \right)$ в явном виде:
	\[\frac{\alpha -0}{1-0}=\frac{{{{\bar{x}}}_{{\tilde{A}}}}\left( \alpha  \right)-{{{\bar{x}}}_{{\tilde{A}}}}\left( 0 \right)}{{{{\bar{x}}}_{{\tilde{A}}}}\left( 1 \right)-{{{\bar{x}}}_{{\tilde{A}}}}\left( 0 \right)}\] 	(1.98)
	${{\bar{x}}_{{\tilde{A}}}}\left( \alpha  \right)={{\bar{x}}_{{\tilde{A}}}}\left( 0 \right)+\alpha \left( {{{\bar{x}}}_{{\tilde{A}}}}\left( 1 \right)-{{{\bar{x}}}_{{\tilde{A}}}}\left( 0 \right) \right)=\alpha {{\bar{x}}_{{\tilde{A}}}}\left( 1 \right)+\left( 1-\alpha  \right){{\bar{x}}_{{\tilde{A}}}}\left( 0 \right)$ 	(1.99)
Все вычисления с использованием данного способа ведутся только на двух $\alpha $-уровнях над действительными числами без использования дополнительных параметров. Если обозначить за $*$ произвольную арифметическую операцию, то для чисел в форме (1.99) её результат будет выглядеть следующим образом:
	\[{{\bar{x}}_{{\tilde{A}}}}\left( \alpha  \right)*{{\bar{x}}_{{\tilde{B}}}}\left( \alpha  \right)=\alpha \left( {{{\bar{x}}}_{{\tilde{A}}}}\left( 1 \right)*{{{\bar{x}}}_{{\tilde{B}}}}\left( 1 \right) \right)+\left( 1-\alpha  \right)\left( {{{\bar{x}}}_{{\tilde{A}}}}\left( 0 \right)*{{{\bar{x}}}_{{\tilde{B}}}}\left( 0 \right) \right)\] 	(1.100)
По сути, вводится другая алгебра на множестве $K$, оперирующая парой чётких значений и автоморфная введённой ранее алгебре модифицированных нечётких чисел. В самом деле, существует взаимно однозначная функция $f:K\to K$, которая позволяет сопоставить числу вида $c+k\alpha $ число вида $\alpha {{\bar{x}}_{{\tilde{A}}}}\left( 1 \right)+\left( 1-\alpha  \right){{\bar{x}}_{{\tilde{A}}}}\left( 0 \right)$. Воспользовавшись (1.99), (1.77) и приравняв свободные члены и коэффициенты при $\alpha$, получим следующее соответствие
	\[\left[ \begin{aligned}
  & c={{{\bar{x}}}_{{\tilde{A}}}}\left( 0 \right) \\ 
 & k={{{\bar{x}}}_{{\tilde{A}}}}\left( 1 \right)-{{{\bar{x}}}_{{\tilde{A}}}}\left( 0 \right) \\ 
\end{aligned} \right.\Leftrightarrow \left[ \begin{aligned}
  & {{{\bar{x}}}_{{\tilde{A}}}}\left( 0 \right)=c \\ 
 & {{{\bar{x}}}_{{\tilde{A}}}}\left( 1 \right)=c+k \\ 
\end{aligned} \right.\] 
Стоит отметить, что преобразование $L$ и вводимая алгебра модифицированных нечётких чисел снимают проблему сравнения двух нечётких чисел, поскольку все сравнения при решении задач происходят с действительными значениями нечёткого числа на выбранных $\alpha$-уровнях.
Пример. Даны нечёткие числа $\tilde{A}=\left\langle 4;2;3 \right\rangle $. $\tilde{B}=\left\langle -2;6;2 \right\rangle $ и $\tilde{C}=\left\langle 1;1;4 \right\rangle $. Вычислить значение $\tilde{D}=\frac{3\tilde{A}\tilde{C}+4\tilde{B}}{\tilde{C}\left( \tilde{A}-2\tilde{B} \right)}$ с использованием модифицированных чисел.
Воспользуемся упрощённой методикой вычислений. Оптимальные в смысле сохранения нечёткой информации значения $\lambda$ равны:
	\[\left[ \begin{aligned}
  & {{\lambda }_{{\tilde{A}}}}=\frac{2}{2+3}=\frac{2}{5} \\ 
 & {{\lambda }_{{\tilde{B}}}}=\frac{6}{6+2}=\frac{3}{4} \\ 
 & {{\lambda }_{{\tilde{C}}}}=\frac{1}{1+4}=\frac{1}{5} \\ 
\end{aligned} \right.\] 	(1.101)
Используя формулы (1.78) c учётом (1.101), получим коэффициенты для модифицированных нечётких чисел:
	\[\begin{aligned}
  & {{c}_{{\tilde{A}}}}=4+3-\frac{2}{5}\left( 2+3 \right)=5;\quad {{k}_{{\tilde{A}}}}=\frac{2}{5}\left( 2+3 \right)-3=-1 \\ 
 & {{c}_{{\tilde{B}}}}=-2+2-\frac{3}{4}\left( 6+2 \right)=-6;\quad {{k}_{{\tilde{B}}}}=\frac{3}{4}\left( 6+2 \right)-2=4 \\ 
 & {{c}_{{\tilde{C}}}}=1+4-\frac{1}{5}\left( 1+4 \right)=4;\quad {{k}_{{\tilde{C}}}}=\frac{1}{5}\left( 1+4 \right)-4=-3 \\ 
\end{aligned}\] 
Отсюда
	\[\left[ \begin{aligned}
  & {{{\bar{x}}}_{{\tilde{A}}}}\left( \alpha  \right)=5-\alpha  \\ 
 & {{{\bar{x}}}_{{\tilde{B}}}}\left( \alpha  \right)=-6+4\alpha  \\ 
 & {{{\bar{x}}}_{{\tilde{C}}}}\left( \alpha  \right)=4-3\alpha  \\ 
\end{aligned} \right.\] 	(1.102)
При $\alpha=1$ выражения (1.102) принимают следующие значения
	\[{{\bar{x}}_{{\tilde{A}}}}\left( 1 \right)=4;\ {{\bar{x}}_{{\tilde{B}}}}\left( 1 \right)=-2;\ {{\bar{x}}_{{\tilde{C}}}}\left( 1 \right)=1\] 	(1.103)
а при $\alpha=0$
	\[{{\bar{x}}_{{\tilde{A}}}}\left( 1 \right)=5;\ {{\bar{x}}_{{\tilde{B}}}}\left( 1 \right)=-6;\ {{\bar{x}}_{{\tilde{C}}}}\left( 1 \right)=4\] 	(1.104)
Подставляя в выражение $\tilde{D}=\frac{3\tilde{A}\tilde{C}+4\tilde{B}}{\tilde{C}\left( \tilde{A}-2\tilde{B} \right)}$ вместо $\tilde A$, $\tilde B$ и $\tilde C$ соответствующие им чёткие значения модифицированных чисел при $\alpha=0$ и $\alpha=1$, получаем:
	${{\bar{x}}_{{\tilde{D}}}}\left( 1 \right)={{\left. \frac{3{{{\tilde{A}}}^{*}}{{{\tilde{C}}}^{*}}+4{{{\tilde{B}}}^{*}}}{{{{\tilde{C}}}^{*}}\left( {{{\tilde{A}}}^{*}}-2{{{\tilde{B}}}^{*}} \right)} \right|}_{\alpha =1}}=\frac{3\cdot 4\cdot 1+4\cdot \left( -2 \right)}{1\cdot \left( 4-2\cdot \left( -2 \right) \right)}=\frac{12-8}{4+4}=\frac{1}{2}$,
	\[{{\bar{x}}_{{\tilde{D}}}}\left( 0 \right)={{\left. \frac{3{{{\tilde{A}}}^{*}}{{{\tilde{C}}}^{*}}+4{{{\tilde{B}}}^{*}}}{{{{\tilde{C}}}^{*}}\left( {{{\tilde{A}}}^{*}}-2{{{\tilde{B}}}^{*}} \right)} \right|}_{\alpha =0}}=\frac{3\cdot 5\cdot 4+4\cdot \left( -6 \right)}{4\cdot \left( 5-2\cdot \left( -6 \right) \right)}=\frac{60-24}{4\cdot 17}=\frac{9}{17}\].
Согласно формуле (1.99), модифицированный результат будет равен
	\[{{\bar{x}}_{{\tilde{D}}}}\left( \alpha  \right)=\frac{9}{17}+\alpha \left( \frac{1}{2}-\frac{9}{17} \right)=\frac{9}{17}-\frac{1}{34}\alpha \] 
Рассмотрим более сложный пример. Необходимо решить уравнение $\tilde{A}x=\tilde{B}$, в котором $\tilde{A}=\left\langle 3;1;2 \right\rangle $, $\tilde{B}=\left\langle 4;4;1 \right\rangle $. В терминах модифицированных нечётких чисел его решение будет иметь вид
	\[x\left( \alpha  \right)=\frac{{{{\bar{x}}}_{{\tilde{B}}}}\left( \alpha  \right)}{{{{\bar{x}}}_{{\tilde{A}}}}\left( \alpha  \right)}\] 	(1.105)
Поскольку операция деления для преобразованных нечётких чисел вводится как умножение на обратное число, перепишем (1.105) в виде
	\[x\left( \alpha  \right)=\bar x_{\tilde B}\left( \alpha  \right)\cdot \bar{x}_{{\tilde{A}}}^{-1}\left( \alpha  \right)\] 	(1.106)

Выберем значения $\lambda_{\tilde A}$ и $\lambda_{\tilde B}$, равные $\frac{a_{\tilde A}}{d_{\tilde A}}$ и $\frac{a_{\tilde B}}{d_{\tilde B}}$, в соответствие с критерием сохранения максимального количества нечёткой информации. Они равны
	\[{{\lambda }_{{\tilde{A}}}}=\frac{1}{3};\ {{\lambda }_{{\tilde{B}}}}=\frac{4}{5}\] 	(1.107)
Воспользовавшись формулами (1.77), (1.78) и (1.107), получаем
	\[\left[ \begin{aligned}
  & {{{\bar{x}}}_{{\tilde{A}}}}\left( \alpha  \right)=4-\alpha  \\ 
 & {{{\bar{x}}}_{{\tilde{B}}}}\left( \alpha  \right)=3\alpha +1 \\ 
\end{aligned} \right.\] 	(1.108)
Найдём обратный элемент $\bar{x}_{{\tilde{A}}}^{-1}\left( \alpha  \right)$ согласно формуле (1.96): 
	$\bar{x}_{{\tilde{A}}}^{-1}\left( \alpha  \right)=\frac{1}{4}-\frac{1}{\left( -1 \right)\cdot \left( 4-1 \right)}\alpha =\frac{1}{4}+\frac{1}{12}\alpha $ 	(1.109)
Пользуясь значениями из (1.108), (1.109) и подставляя их в (1.106), окончательно получаем
	\[x\left( \alpha  \right)=\left( 1+3\alpha  \right)\left( \frac{1}{4}+\frac{1}{12}\alpha  \right)=\frac{1}{4}+\frac{13}{12}\alpha \] 	(1.110)
Функция принадлежности модифицированного решения определяется как обратная к (1.110) – ${{\mu }_{x(\alpha )}}\left( x \right)=\frac{12}{13}x-\frac{3}{13}$. Подстановка решения задачи в исходное уравнение $\tilde{A}x=\tilde{B}$ с учётом формулы (1.89) приводит к верному равенству:
	\[\left( 4-\alpha  \right)\left( \frac{1}{4}+\frac{13}{12}\alpha  \right)=3\alpha +1\];
	\[\frac{52-13-3}{12}\alpha +1=3\alpha +1\].

Описанная выше модель применима в случаях использования асимметричных нечётких чисел. Однако существуют ситуации, когда  необходимо 

В статье \cite{Kanischeva} предложена алгебра для двухкомпонентных нечётких чисел. Её ключевое отличие в том, что не используется преобразование для $\alpha$-интервалов, а операции над числами выполняются как операции над


% Проблема устойчивости нечётких решений на примере оптимальной задачи выбора с нечеткими параметрами
\section{Определение условий устойчивости численного решения для задачи линейного программирования с нечеткими параметрами} 
\label{chapter2_4}
Задача нечеткого математического программирования рассматривается в~\cite{Matveev_Fuzzy_LP, Matveev_PIIT_2011, Melkumova_Vestnik, Zak, Liu_Fuzzy_Programming} как удобное и~адекватное описание выбора в~условиях неполной определенности. Математический аппарат решения таких задач достаточно разнообразен~\cite{Melkumova_Vestnik, Liu_Fuzzy_Programming} и~соответствует вариантам трактовки понятия оптимальности в~различных условиях нечеткости~\cite{Matveev_Starodubtsev}. Для задач с~нечёткой целевой зависимостью и нечеткими отношениями при формулировке ограничений используется принцип Беллмана-Заде~\cite{Bellman_Zadeh, Ukhobotov_Chosen}, в~соответствие с~которым решением задачи является нечеткое множество~--- пересечение нечетких множеств целевой функции и ограничений, однако такие задачи не подпадают под второй тип нечёткости (нечёткие параметры при чётких отношениях) согласно классификации нечётких моделей из п.\ref{chapter1_2} и лежат за пределами данного исследования.

Следуя общей логике изложения, рассмотрим задачу нечёткого линейного программирования в~случае, когда целевая функция и ограничения содержат нечеткие параметры при чёткости отношений. Различные подходы к решению задач нечёткого линейного программирования при нечетких параметрах целевой функции и четких ограничениях ранее рассматривались в работах~\cite{Orlovskiy, Zak, Matveev_PIIT_2011, Matveev_Fuzzy_LP}. В~\cite{Vorontsov_PI} задача с нечёткими коэффициентами решается с~использованием преобразования~$L$ и~алгебры модифицированных нечётких чисел и~может приводить к~различным~$\alpha$-уровневым решениям. Рассмотрим следующий пример.

\textbf{Пример.} Решить задачу линейного программирования~\eqref{eq:stability-sample-task} при $d=10$, $\tilde A_1 = \left \langle 4; 4; 1 \right \rangle$, $\tilde A_2 = \left \langle 3; 1; 2\right \rangle$:
\begin{equation}
\label{eq:stability-sample-task}
  \left\{ \begin{aligned}
    & \tilde A_1 x_1 + \tilde A_2 x_2 \to \max; \\
    & x_1 + x_2 = d; \\
    & x_1, x_2 \geqslant 0.
  \end{aligned} \right.
\end{equation}

Применяя преобразование~$L$ к~параметрам $\tilde A_1$ и~$\tilde A_2$ c~оптимальными в~смысле минимума потерь экспертной информации коэффициентами $\displaystyle \lambda_{\tilde A_1}=\frac{4}{4+1}=\frac{4}{5}$ и $\displaystyle \lambda_{\tilde A_2}=\frac{1}{2+1}=\frac{1}{3}$, получим:
\begin{equation}
\label{eq:stability-sample-modified}
  \left\{ \begin{aligned}
    & \bar{x}_{\tilde A_1}\left(\alpha \right)=\frac{4}{5}\left(4-4+4\alpha \right)+\frac{1}{5}\left(4+1-\alpha \right)=3\alpha+1; \\
    & \bar{x}_{\tilde A_2}\left(\alpha \right)=\frac{1}{3}\left(3-1+\alpha \right)+\frac{2}{3}\left(3+2-2\alpha \right)=4-\alpha.
  \end{aligned} \right.
\end{equation}

С~учётом результатов преобразования~\eqref{eq:stability-sample-modified}, задача~\eqref{eq:stability-sample-task} будет выглядеть следующим образом:
\begin{equation}
\label{eq:stability-sample-task-modified}
  \left\{ \begin{aligned}
    & \left( 3\alpha+1\right)x_1+\left(4-\alpha \right)x_2 \to \max; \\
    & x_1 + x_2 = 10; \\
    & x_1, x_2 \geqslant 0.
  \end{aligned} \right.
\end{equation}

Решашь задачу~\eqref{eq:stability-sample-task-modified} будем на различных $\alpha$-уровнях, следуя методикам из~\cite{Matveev_Starodubtsev, Zak}. При её решении на~разных $\alpha$-уровнях получаются различные значения $x_1$ и~$x_2$. Например, при $\alpha=0$ задача принимает вид
\begin{equation*}
  \left\{ \begin{aligned}
    & x_1+4x_2 \to \max; \\
    & x_1 + x_2 = 10; \\
    & x_1, x_2 \geqslant 0,
  \end{aligned} \right.
\end{equation*}
а её решением является вектор $\left[0;10 \right]$. В~то~же~время, при~$\alpha=0$ задача~\eqref{eq:stability-sample-task-modified} сводится к~системе
\begin{equation*}
  \left\{ \begin{aligned}
    & 4x_1+3x_2 \to \max; \\
    & x_1 + x_2 = 10; \\
    & x_1, x_2 \geqslant 0,
  \end{aligned} \right.
\end{equation*}
решение которой~--- вектор $\left[10;0 \right]$.

С одной стороны, можно представить результат решения в~виде нечётких $LL/RR$-чисел согласно формуле~\eqref{eq:isomorphic-field}: $\bar{x}_{\tilde A_1}=10\alpha$, $\bar{x}_{\tilde A_2}=10-10\alpha$. С~другой стороны, если решить задачу~\eqref{eq:stability-sample-task-modified} при~$\alpha=0,5$, то~решение будет таким~же, как~и~при~$\alpha=1$, что наводит на мысль о том, что изменение решения происходит скачкообразно при~некотором значении $\alpha_0$, и поэтому $x_1$ и $x_2$ нечёткими быть не могут. Наконец, первое ограничение на~переменные в~исходной задаче~\eqref{eq:stability-sample-task} подразумевает, что значения искомых переменных $x_1$ и $x_2$ должны быть чёткими.

Если зафиксировать в~качестве <<эталонного>> решение задачи~\eqref{eq:stability-sample-task-modified} при~$\alpha=1$, то~очевидно, что при~изменении значений $\alpha$ при~фиксированных коэффициентах $\lambda_{A_i}$ преобразования~$L$, т.\,е. при~возмущении параметров исходной задачи, решение ведёт себя неустойчиво. Для~того, чтобы дать формальное определение устойчивости для~нечёткой задачи и~определить, возможно~ли влиять на~устойчивость решения с~помощью изменения параметров $\lambda_{A_i}$, вначале рассмотрим соответствующие формулировки для~классической задачи линейного программирования, приведённые в~\cite{Ashmanov}.

Пусть дана задача линейного программирования
\begin{equation}
\label{eq:crisp-lp-problem}
  \left\{ \begin{aligned}
    & \max \left\langle c,x \right\rangle \\ 
    & Ax\leqslant b
  \end{aligned} \right.
\end{equation}
имеющая решение $x^{*}$. Возмущённой задачей линейного программирования называется задача
\begin{equation}
\label{eq:crisp-lp-problem-unstable}
  \left\{ \begin{aligned}
    & \max \left\langle c\left(\delta \right),x \right\rangle; \\ 
    & A\left( \delta  \right)x\leqslant b\left(\delta \right),
  \end{aligned} \right.
\end{equation}
относительно параметров которой известно, что они в смысле определённой метрики близки параметрам исходной задачи~\eqref{eq:crisp-lp-problem}, т.\,е. для фиксированного $\delta>0$ выполняются неравенства
\begin{equation}
\label{eq:delta-metric}
  \left\{ \begin{aligned}
    & \left\| A\left( \delta  \right)-A \right\|<\delta; \\ 
    & \left\| b\left( \delta  \right)-b \right\|<\delta; \\ 
    & \left\| c\left( \delta  \right)-c \right\|<\delta.
  \end{aligned} \right.
\end{equation}

При этом для фиксированных матрицы~$A$ и~вектора~$b$ и~$\forall \delta >\text{0}$ выражения $A\left(\delta \right)$, $b\left(\delta \right)$ называют $\delta$-окрестностью матрицы $A$ и~вектора $b$ соответственно. В~качестве метрики близости в~\eqref{eq:delta-metric} может использоваться, например, евклидова метрика.

Задачу~\eqref{eq:crisp-lp-problem} будем называть устойчивой, если~$\exists \delta_0 > 0$: $\forall \delta \in \left[ 0; \delta_0 \right]$  задача~\eqref{eq:crisp-lp-problem-unstable} имеет решение. Если обозначить решение задачи~\eqref{eq:crisp-lp-problem-unstable} за~$x^{*} \left(\delta \right)$, то~задачу~\eqref{eq:crisp-lp-problem} будем называть устойчивой по~решению, если она устойчива и~$\forall \varepsilon>0\ \exists \delta > 0$ такое, что~при выполнении неравенств~\eqref{eq:delta-metric} для~$\forall x^{*} \left(\delta \right)\ \exists x^{*}$, удовлетворяющее условию $\left| x^{*}\left(\delta \right)- x^{*} \right| < \varepsilon$. Стоит отметить, что если задача~\eqref{eq:crisp-lp-problem} неустойчива в~смысле хотя~бы одного из~приведённых выше определений, то~она считается неустойчивой.

Проблема устойчивости по~решению в~задаче линейного программирования с~нечёткими коэффициентами рассматривалась в~работах~\cite{Matveev_Starodubtsev, PhD_Starodubtsev}, где для сведения нечёткой задачи к чёткой используется средневзвешенное значение для коэффициентов
\begin{equation*}
  \tilde A \to \frac{\sum \limits_k \alpha_k \left( x^L_{\tilde A} \left( \alpha_k \right) + x^R_{\tilde A} \left( \alpha_k \right) \right) }{2 \sum \limits_k \alpha_k }
\end{equation*}
Такое преобразование позволяет получать чёткое решение и~использовать классическое определение устойчивости~\eqref{eq:crisp-lp-problem-unstable}--\eqref{eq:delta-metric}.

В~\cite{Vorontsov_VSTU} отмечается, что для~задач линейного программирования с~нечёткими параметрами удобнее использовать определение устойчивости по~решению, поскольку оно включает в~себя более узкое определение общей устойчивости, т.\,е. принципиальное существование решения возмущённой задачи. Пусть дана задача с~нечёткими параметрами
\begin{equation*}
  \left\{ \begin{aligned}
    & f\left( \mathbf{x} \right)=\mathbf{Cx}\to \min;  \\ 
    & \mathbf{Ax}=\mathbf{B},
  \end{aligned} \right.
\end{equation*}
где $\mathbf{A}=\left\{ \tilde{A}_{ij} \right\}$~--- матрица, а~$\mathbf{B}=\left\{ \tilde{B}_i \right\}$, $\mathbf{C}=\left\{\tilde{C}_i \right\}$~--- векторы нечётких параметров. Применяя к~каждому из~элементов матрицы и~векторов преобразование~$L$, получаем модифицированную задачу
\begin{equation}
\label{eq:fuzzy-lp-unstable-problem}
  \left\{ \begin{aligned}
    & f\left( \mathbf{x} \right)={\mathbf{C}^{*}}\mathbf{x}\to \min;  \\ 
    & {\mathbf{A}^{*}}\mathbf{x}={\mathbf{B}}^{*},
  \end{aligned} \right.
\end{equation}
в~которой $\mathbf{A}^{*}=\left\{ \bar{x}_{\tilde{A}_{ij}}\left(\alpha \right) \right\}$, $\mathbf{B}^{*}=\left\{ \bar{x}_{\tilde{B}_i}\left(\alpha \right) \right\}$, $\mathbf{C}^{*}=\left\{ \bar{x}_{\tilde{C}_i}\left(\alpha \right) \right\}$.

Задачу~\eqref{eq:fuzzy-lp-unstable-problem} удобнее всего решать на~двух $\alpha$-уровнях и~восстанавливать решение согласно~\eqref{eq:isomorphic-field}. Ввиду свойства сохранения моды, решение модифицированной задачи~\eqref{eq:fuzzy-lp-unstable-problem} при $\alpha=1$ аналогично решению чёткой задачи с~коэффициентами, равными модам нечётких чисел. Если решать ту~же задачу при~$\alpha=0$ и~без дополнительных ограничений на~параметры $\lambda$ преобразования $L$, то~возникает ситуация, при~которой все~значения $\bar{x}_S\left(\alpha \right)$, где~$S$~--- один из индексов $\tilde A_{ij}$, $\tilde B_i$, $\tilde C_i$, принимают левое граничное значение. Это легко объясняется тем фактом, что при $\lambda_S=1$ максимальный вес в значении $\bar{x}_S\left(\alpha \right)$ имеет левая ветвь функции принадлежности, находящаяся ближе к нулю. Для решения данной проблемы введём дополнительные ограничения для параметров $\lambda_S$:
\begin{equation}
\label{eq:lambda-minimization-criterion}
  {\left( \lambda_{S}^{\star}-\lambda_S \right)}^2\to \min
\end{equation}
которые позволяют минимизировать отклонение параметров от оптимальных в~смысле сохранения нечёткой информации значений $\displaystyle \lambda_{S}^{\star}=\frac{a_S}{d_S}$ и, таким~образом, управлять устойчивостью решения. 

Критерии~\eqref{eq:lambda-minimization-criterion} и~целевая~функция задачи~\eqref{eq:fuzzy-lp-unstable-problem} противоречивы. Возникает задача векторной оптимизации, описанная в~\cite{MSU_Optimization}. Для~её решения воспользуемся аддитивной свёрткой критериев:
\begin{equation}
\label{eq:modified-target-function}
  f^{*}\left( \mathbf{x}, \mathbf{\lambda} \right)=\mathbf{C}^{*}\mathbf{x}+\gamma \sum\limits_{s}^{}{\left(\lambda_{S}^{*}-\lambda_S \right)}^{2} \to \min
\end{equation}
Семантика целевой функции~\eqref{eq:modified-target-function} такова: ищется решение $\mathbf{x}$ и вектор параметров преобразования $L$ $\mathbf{\lambda}_S$, которые позволяют удовлетворить исходный критерий оптимизации и~при~этом максимально сохранить нечёткую информацию, заложенную экспертами в параметры задачи. Безразмерный коэффициент~$\gamma$ позволяет привести значение свёртки к~одному~порядку со~значением исходной целевой функции.

Полученная пара векторов $\mathbf{x}\left( \alpha =1 \right)$ и $\mathbf{x}\left( \alpha =0 \right)$ позволяет восстановить модифицированные решения согласно~\eqref{eq:isomorphic-field}. Если рассмотреть устойчивость нечёткого решения в~смысле данного ранее определения для чёткой задачи, то~справедливо предположить, что нечёткая задача будет считаться устойчивой по решению, если при переходе с одного $\alpha$-уровня на другой не происходит значительного изменения решения, т.\,е.
\begin{equation}
\label{eq:fuzzy-solution-stability}
  \forall \varepsilon >0\ \exists \delta >0\ \forall \alpha_1, \alpha_2 \in \left[0; 1\right]\ \left| \alpha_1-\alpha_2 \right|<\delta \Rightarrow \left\| \mathbf{x}\left( \alpha_1 \right)-\mathbf{x}\left( \alpha_2  \right) \right\|<\varepsilon.
\end{equation}

Конкретное же условие устойчивости решения зависит от задачи и~может принимать различные формы.

\newpage
\section*{Выводы по второй главе:} 
\label{chapter2_5}
\begin{enumerate}
  \item Проведённый анализ существующих нечётких алгебр, принадлежащих к различным семействам, показал, что ни одна из них в полной мере не удовлетворяет требованиям к алгебре, выдвинутым в главе~\ref{chapter1}~--- ограничению роста неопределенности, сохранению чётких отношений в модельных уравнениях и возможности использования стандартных программных средств реализации численных методов решения.
  \item Для $\alpha$-интервалов треугольных чисел вводится линейное параметрического преобразования L, сопоставляющее обычному треугольному числу его модифицированное LL/RR-представление. Свойства преобразования L исследуются с точки зрения сохранения экспертной информации, заложенной в нечёткое число.
  \item На множестве $K$ модифицированных нечётких чисел строится алгебра, являющаяся полем. Вводится эквивалентная алгебре методика численного решения задач с нечёткими параметрами, позволяющая обойти проблему отсутствия линейного порядка на множестве $K$ и находить модифицированные решения задач на основе двухточечных вычислений.
  \item Кратко описывается алгебра двухкомпонентных нечётких чисел как альтернатива алгебре модифицированных нечётких чисел в тех случаях, когда потери экспертной информации недопустимы.
  \item Рассматривается задача линейного программирования с нечёткими коэффициентами, имеющая неустойчивое решение. Для общего случая нечёткой задачи формулируется условие устойчивости на основании классического определения для чёткого случая и решается проблема управления устойчивостью решения в виде задачи векторной оптимизации.
\end{enumerate}