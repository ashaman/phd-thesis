Модификация формулировки задачи устойчивости в терминах ограничений. Использование предложенного метода для решения задачи линейного программирования с иллюстрацией устойчивого и неустойчивого решения. 

Постановка задачи управления устойчивостью решения для модифицированных нечетких чисел в виде задачи векторной оптимизации. Обоснование построения линейной свертки критериев и алгоритм решения с использованием предложенного метода численной реализации.

Пример с задачей ЛП и переход к вопросам устойчивости решения

Рассмотрим следующую задачу линейного программирования, которая в \cite{Vorontsov_PI} решается с использованием преобразования $L$ и алгебры модифицированных нечётких чисел.


Рассмотрим проблему устойчивости, возникающую при решении нечётких задач, на примере задачи линейного программирования с нечёткими параметрами. Для того, чтобы ввести определения устойчивости для нечёткой задачи, вначале рассмотрим формулировки для чёткой задачи линейного программирования, приведённые в [Ашманов].
Пусть дана задача линейного программирования
	\[\left\{ \begin{aligned}
  & \max \left\langle c,x \right\rangle  \\ 
 & Ax\le b \\ 
\end{aligned} \right.\] 	(1.111)
имеющая решение ${{x}^{\star}}$. Возмущённой задачей линейного программирования называется задача
	\[\left\{ \begin{aligned}
  & \max \left\langle c\left( \delta  \right),x \right\rangle  \\ 
 & A\left( \delta  \right)x\le b\left( \delta  \right) \\ 
\end{aligned} \right.\] 	(1.112)
относительно параметров которой известно, что они в смысле определённой метрики близки параметрам исходной задачи (1.111), т.е. для фиксированного $\delta >0$ выполняются неравенства
	\[\left\{ \begin{aligned}
  & \left\| A\left( \delta  \right)-A \right\|<\delta  \\ 
 & \left\| b\left( \delta  \right)-b \right\|<\delta  \\ 
 & \left\| c\left( \delta  \right)-c \right\|<\delta  \\ 
\end{aligned} \right.\] 	(1.113)
При этом для фиксированных матрицы $A$ и вектора $b$ и $\forall \delta >\text{0}$ выражения $A\left( \delta  \right)$, $b\left( \delta  \right)$ называют $\delta$-окрестностью матрицы $A$ и вектора $b$ соответственно. В качестве метрики близости в (1.113) может использоваться, например, евклидова метрика.
%Задачу (1.111) будем называть устойчивой, если $\exists \delta_0 > 0$: $\forall \delta \in \left[ 0; \delta_0 \right]$  задача (1.112) имеет решение. Если обозначить решение задачи (1.112) за $x^{\star} \left( \delta  \right)$, то задачу (1.111) будем называть устойчивой по решению, если она устойчива и $\forall \varepsilon>0$ $\exists \delta > 0$ такое, что при выполнении неравенств (1.113) для $\forall x^{\star} \left( \delta  \right)$ $\exist x^{\star}$, удовлетворяющее условию $\left| x^{\star} \left( \delta  \right)- x^{\star} \right| < \varepsilon $. Стоит отметить, что если задача (1.111) неустойчива в смысле хотя бы одного из приведённых выше определений, то она считается неустойчивой.
Для задач линейного программирования с нечёткими параметрами удобнее использовать определение устойчивости по решению, поскольку оно включает себя более узкое определение общей устойчивости, т.е. принципиальное существование решения возмущённой задачи. Пусть дана задача с нечёткими параметрами
	\[\left\{ \begin{aligned}
  & f\left( \mathbf{x} \right)=\mathbf{Cx}\to \min  \\ 
 & \mathbf{Ax}=\mathbf{B} \\ 
\end{aligned} \right.\] 	(1.114)
где $\mathbf{A}=\left\{ {{{\tilde{A}}}_{ij}} \right\}$ – матрица, а $\mathbf{B}=\left\{ {{{\tilde{B}}}_{i}} \right\}$,$\mathbf{C}=\left\{ {{{\tilde{C}}}_{i}} \right\}$ – векторы нечётких параметров. Применяя к каждому из элементов матрицы и векторов преобразование $L$, получаем модифицированную задачу
	\[\left\{ \begin{aligned}
  & f\left( \mathbf{x} \right)={{\mathbf{C}}^{\star}}\mathbf{x}\to \min  \\ 
 & {{\mathbf{A}}^{\star}}\mathbf{x}={{\mathbf{B}}^{\star}} \\ 
\end{aligned} \right.\] 	(1.115)
в которой ${{\mathbf{A}}^{\star}}=\left\{ {{{\bar{x}}}_{{{{\tilde{A}}}_{ij}}}}\left( \alpha  \right) \right\}$, ${{\mathbf{B}}^{\star}}=\left\{ {{{\bar{x}}}_{{{{\tilde{B}}}_{i}}}}\left( \alpha  \right) \right\}$,${{\mathbf{C}}^{\star}}=\left\{ {{{\bar{x}}}_{{{{\tilde{C}}}_{i}}}}\left( \alpha  \right) \right\}$.
Задачу (1.115) удобнее всего решать на двух $\alpha$-уровнях и восстанавливать решение согласно (1.99). Ввиду свойства сохранения моды, решение модифицированной задачи (1.115) при $\alpha=1$ аналогично решению чёткой задачи с коэффициентами, равными модам нечётких чисел. Если решать ту же задачу при $\alpha =0$ и без дополнительных ограничений на параметры $\lambda$ преобразования $L$, то возникает ситуация, при которой все значения ${{\bar{x}}_{S}}\left( \alpha  \right)$, где $S$ – один из индексов $\tilde A_{ij}$, $\tilde B_i$, $\tilde C_i$, «сбиваются» в сторону минимума. Это легко объясняется тем фактом, что при ${{\lambda }_{S}}=1$ максимальный вес в значении ${{\bar{x}}_{S}}\left( \alpha  \right)$ имеет левая ветвь функции принадлежности. Для решения данной проблемы введём дополнительные ограничения для параметров ${{\lambda }_{S}}$:
	\[{{\left( \lambda _{S}^{\star}-{{\lambda }_{S}} \right)}^{2}}\to \min \] 	(1.116)
которые позволяют минимизировать отклонение параметров ${{\lambda }_{S}}$ от оптимального в смысле сохранения нечёткой информации значения \[\lambda _{S}^{\star}=\frac{{{a}_{S}}}{{{d}_{S}}}\]. Критерии (1.116) и целевая функция задачи (1.115) противоречивы. Возникает задача векторной оптимизации, описанная в [Методичка МГУ]. Для её решения воспользуемся аддитивной свёрткой критериев:
	\[{{f}^{\star}}\left( \mathbf{x} \right)={{\mathbf{C}}^{\star}}\mathbf{x}+\gamma \sum\limits_{s}^{{}}{{{\left( \lambda _{S}^{\star}-{{\lambda }_{S}} \right)}^{2}}\to \min }\] 	(1.117)
Семантика новой целевой функции такова: ищется решение $\mathbf{x}$ и вектор параметров преобразования $L$ ${{\mathbf{\lambda }}_{S}}$ , которые позволяют удовлетворить исходный критерий оптимизации и при этом максимально сохранить нечёткую информацию, заложенную экспертами в параметры задачи. Безразмерный коэффициент $\gamma$ позволяет привести значение свёртки к одному порядку со значением исходной целевой функции.
Полученная пара векторов $\mathbf{x}\left( \alpha =1 \right)$ и $\mathbf{x}\left( \alpha =0 \right)$ позволяет восстановить модифицированные решения согласно (1.99). Если рассмотреть устойчивость нечёткого решения в смысле данного ранее определения для чёткой задачи, то справедливо предположить, что нечёткая задача будет считаться устойчивой по решению, если при переходе с одного $\alpha$-уровня на другой не происходит значительного изменения решения относительно $\mathbf{x}\left( \alpha =1 \right)$, т.е.
	\[\forall \varepsilon >0\ \exists \delta >0:\forall \alpha :\left| \alpha -1 \right|<\delta \ \left\| \mathbf{x}\left( 1 \right)-\mathbf{x}\left( \alpha  \right) \right\|<\varepsilon \] 	(1.118)
Конкретное же условие устойчивости решения зависит от задачи и может принимать различные формы.