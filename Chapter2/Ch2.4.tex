Задача нечеткого математического программирования рассматривается в~\cite{Matveev_Fuzzy_LP, Matveev_PIIT_2011, Melkumova_Vestnik, Zak, Liu_Fuzzy_Programming} как удобное и~адекватное описание выбора в~условиях неполной определенности. Математический аппарат решения таких задач достаточно разнообразен~\cite{Melkumova_Vestnik, Liu_Fuzzy_Programming} и~соответствует вариантам трактовки понятия оптимальности в~различных условиях нечеткости~\cite{Matveev_Starodubtsev}. Для задач с~нечёткой целевой зависимостью и нечеткими отношениями при формулировке ограничений используется принцип Беллмана-Заде~\cite{Bellman_Zadeh, Ukhobotov_Chosen}, в~соответствие с~которым решением задачи является нечеткое множество~--- пересечение нечетких множеств целевой функции и ограничений, однако такие задачи не подпадают под второй тип нечёткости (нечёткие параметры при чётких отношениях) согласно классификации нечётких моделей из п.\ref{chapter1_2} и лежат за пределами данного исследования.

Следуя общей логике изложения, рассмотрим задачу нечёткого линейного программирования в~случае, когда целевая функция и ограничения содержат нечеткие параметры при чёткости отношений. Различные подходы к решению задач нечёткого линейного программирования при нечетких параметрах целевой функции и четких ограничениях ранее рассматривались в работах~\cite{Orlovskiy, Zak, Matveev_PIIT_2011, Matveev_Fuzzy_LP}. В~\cite{Vorontsov_PI} задача с нечёткими коэффициентами решается с~использованием преобразования~$L$ и~алгебры модифицированных нечётких чисел и~может приводить к~различным~$\alpha$-уровневым решениям. Рассмотрим следующий пример.

\textbf{Пример.} Решить задачу линейного программирования~\eqref{eq:stability-sample-task} при $d=10$, $\tilde A_1 = \left \langle 4; 4; 1 \right \rangle$, $\tilde A_2 = \left \langle 3; 1; 2\right \rangle$:
\begin{equation}
\label{eq:stability-sample-task}
  \left\{ \begin{aligned}
    & \tilde A_1 x_1 + \tilde A_2 x_2 \to \max; \\
    & x_1 + x_2 = d; \\
    & x_1, x_2 \geqslant 0.
  \end{aligned} \right.
\end{equation}

Применяя преобразование~$L$ к~параметрам $\tilde A_1$ и~$\tilde A_2$ c~оптимальными в~смысле минимума потерь экспертной информации коэффициентами $\displaystyle \lambda_{\tilde A_1}=\frac{4}{4+1}=\frac{4}{5}$ и $\displaystyle \lambda_{\tilde A_2}=\frac{1}{2+1}=\frac{1}{3}$, получим:
\begin{equation}
\label{eq:stability-sample-modified}
  \left\{ \begin{aligned}
    & \bar{x}_{\tilde A_1}\left(\alpha \right)=\frac{4}{5}\left(4-4+4\alpha \right)+\frac{1}{5}\left(4+1-\alpha \right)=3\alpha+1; \\
    & \bar{x}_{\tilde A_2}\left(\alpha \right)=\frac{1}{3}\left(3-1+\alpha \right)+\frac{2}{3}\left(3+2-2\alpha \right)=4-\alpha.
  \end{aligned} \right.
\end{equation}

С~учётом результатов преобразования~\eqref{eq:stability-sample-modified}, задача~\eqref{eq:stability-sample-task} будет выглядеть следующим образом:
\begin{equation}
\label{eq:stability-sample-task-modified}
  \left\{ \begin{aligned}
    & \left( 3\alpha+1\right)x_1+\left(4-\alpha \right)x_2 \to \max; \\
    & x_1 + x_2 = 10; \\
    & x_1, x_2 \geqslant 0.
  \end{aligned} \right.
\end{equation}

Решашь задачу~\eqref{eq:stability-sample-task-modified} будем на различных $\alpha$-уровнях, следуя методикам из~\cite{Matveev_Starodubtsev, Zak}. При её решении на~разных $\alpha$-уровнях получаются различные значения $x_1$ и~$x_2$. Например, при $\alpha=0$ задача принимает вид
\begin{equation*}
  \left\{ \begin{aligned}
    & x_1+4x_2 \to \max; \\
    & x_1 + x_2 = 10; \\
    & x_1, x_2 \geqslant 0,
  \end{aligned} \right.
\end{equation*}
а её решением является вектор $\left[0;10 \right]$. В~то~же~время, при~$\alpha=0$ задача~\eqref{eq:stability-sample-task-modified} сводится к~системе
\begin{equation*}
  \left\{ \begin{aligned}
    & 4x_1+3x_2 \to \max; \\
    & x_1 + x_2 = 10; \\
    & x_1, x_2 \geqslant 0,
  \end{aligned} \right.
\end{equation*}
решение которой~--- вектор $\left[10;0 \right]$.

С одной стороны, можно представить результат решения в~виде нечётких LL/RR-чисел согласно формуле~\eqref{eq:isomorphic-field}: $\bar{x}_{\tilde A_1}=10\alpha$, $\bar{x}_{\tilde A_2}=10-10\alpha$. С~другой стороны, если решить задачу~\eqref{eq:stability-sample-task-modified} при~$\alpha=0,5$, то~решение будет таким~же, как~и~при~$\alpha=1$, что наводит на мысль о том, что изменение решения происходит скачкообразно при~некотором значении $\alpha_0$, и поэтому $x_1$ и $x_2$ нечёткими быть не могут. Наконец, первое ограничение на~переменные в~исходной задаче~\eqref{eq:stability-sample-task} подразумевает, что значения искомых переменных $x_1$ и $x_2$ должны быть чёткими.

Если зафиксировать в~качестве <<эталонного>> решение задачи~\eqref{eq:stability-sample-task-modified} при~$\alpha=1$, то~очевидно, что при~изменении значений $\alpha$ при~фиксированных коэффициентах $\lambda_{A_i}$ преобразования~L, т.\,е. при~возмущении параметров исходной задачи, решение ведёт себя неустойчиво. Для~того, чтобы дать формальное определение устойчивости для~нечёткой задачи и~определить, возможно~ли влиять на~устойчивость решения с~помощью изменения параметров $\lambda_{A_i}$, вначале рассмотрим соответствующие формулировки для~классической задачи линейного программирования, приведённые в~\cite{Ashmanov}.

Пусть дана задача линейного программирования
\begin{equation}
\label{eq:crisp-lp-problem}
  \left\{ \begin{aligned}
    & \max \left\langle c,x \right\rangle \\ 
    & Ax\leqslant b
  \end{aligned} \right.
\end{equation}
имеющая решение $x^{*}$. Возмущённой задачей линейного программирования называется задача
\begin{equation}
\label{eq:crisp-lp-problem-unstable}
  \left\{ \begin{aligned}
    & \max \left\langle c\left(\delta \right),x \right\rangle; \\ 
    & A\left( \delta  \right)x\leqslant b\left(\delta \right),
  \end{aligned} \right.
\end{equation}
относительно параметров которой известно, что они в смысле определённой метрики близки параметрам исходной задачи~\eqref{eq:crisp-lp-problem}, т.\,е. для фиксированного $\delta>0$ выполняются неравенства
\begin{equation}
\label{eq:delta-metric}
  \left\{ \begin{aligned}
    & \left\| A\left( \delta  \right)-A \right\|<\delta; \\ 
    & \left\| b\left( \delta  \right)-b \right\|<\delta; \\ 
    & \left\| c\left( \delta  \right)-c \right\|<\delta.
  \end{aligned} \right.
\end{equation}

При этом для фиксированных матрицы~$A$ и~вектора~$b$ и~$\forall \delta >\text{0}$ выражения $A\left(\delta \right)$, $b\left(\delta \right)$ называют $\delta$-окрестностью матрицы $A$ и~вектора $b$ соответственно. В~качестве метрики близости в~\eqref{eq:delta-metric} может использоваться, например, евклидова метрика.

Задачу~\eqref{eq:crisp-lp-problem} будем называть устойчивой, если~$\exists \delta_0 > 0$: $\forall \delta \in \left[ 0; \delta_0 \right]$  задача~\eqref{eq:crisp-lp-problem-unstable} имеет решение. Если обозначить решение задачи~\eqref{eq:crisp-lp-problem-unstable} за~$x^{*} \left(\delta \right)$, то~задачу~\eqref{eq:crisp-lp-problem} будем называть устойчивой по~решению, если она устойчива и~$\forall \varepsilon>0\ \exists \delta > 0$ такое, что~при выполнении неравенств~\eqref{eq:delta-metric} для~$\forall x^{*} \left(\delta \right)\ \exists x^{*}$, удовлетворяющее условию $\left| x^{*}\left(\delta \right)- x^{*} \right| < \varepsilon$. Стоит отметить, что если задача~\eqref{eq:crisp-lp-problem} неустойчива в~смысле хотя~бы одного из~приведённых выше определений, то~она считается неустойчивой.

Проблема устойчивости по~решению в~задаче линейного программирования с~нечёткими коэффициентами рассматривалась в~работах~\cite{Matveev_Starodubtsev, PhD_Starodubtsev}, где для сведения нечёткой задачи к чёткой используется средневзвешенное значение для коэффициентов
\begin{equation*}
  \tilde A \to \frac{\sum \limits_k \alpha_k \left( x^L_{\tilde A} \left( \alpha_k \right) + x^R_{\tilde A} \left( \alpha_k \right) \right) }{2 \sum \limits_k \alpha_k }
\end{equation*}
Такое преобразование позволяет получать чёткое решение и~использовать классическое определение устойчивости~\eqref{eq:crisp-lp-problem-unstable}-\eqref{eq:delta-metric}.

В~\cite{Vorontsov_VSTU} отмечается, что для~задач линейного программирования с~нечёткими параметрами удобнее использовать определение устойчивости по~решению, поскольку оно включает в~себя более узкое определение общей устойчивости, т.\,е. принципиальное существование решения возмущённой задачи. Пусть дана задача с~нечёткими параметрами
\begin{equation*}
  \left\{ \begin{aligned}
    & f\left( \mathbf{x} \right)=\mathbf{Cx}\to \min;  \\ 
    & \mathbf{Ax}=\mathbf{B},
  \end{aligned} \right.
\end{equation*}
где $\mathbf{A}=\left\{ \tilde{A}_{ij} \right\}$~--- матрица, а~$\mathbf{B}=\left\{ \tilde{B}_i \right\}$, $\mathbf{C}=\left\{\tilde{C}_i \right\}$~--- векторы нечётких параметров. Применяя к~каждому из~элементов матрицы и~векторов преобразование~$L$, получаем модифицированную задачу
\begin{equation}
\label{eq:fuzzy-lp-unstable-problem}
  \left\{ \begin{aligned}
    & f\left( \mathbf{x} \right)={\mathbf{C}^{*}}\mathbf{x}\to \min;  \\ 
    & {\mathbf{A}^{*}}\mathbf{x}={\mathbf{B}}^{*},
  \end{aligned} \right.
\end{equation}
в~которой $\mathbf{A}^{*}=\left\{ \bar{x}_{\tilde{A}_{ij}}\left(\alpha \right) \right\}$, $\mathbf{B}^{*}=\left\{ \bar{x}_{\tilde{B}_i}\left(\alpha \right) \right\}$, $\mathbf{C}^{*}=\left\{ \bar{x}_{\tilde{C}_i}\left(\alpha \right) \right\}$.

Задачу~\eqref{eq:fuzzy-lp-unstable-problem} удобнее всего решать на~двух $\alpha$-уровнях и~восстанавливать решение согласно~\eqref{eq:isomorphic-field}. Ввиду свойства сохранения моды, решение модифицированной задачи~\eqref{eq:fuzzy-lp-unstable-problem} при $\alpha=1$ аналогично решению чёткой задачи с~коэффициентами, равными модам нечётких чисел. Если решать ту~же задачу при~$\alpha=0$ и~без дополнительных ограничений на~параметры $\lambda$ преобразования $L$, то~возникает ситуация, при~которой все~значения $\bar{x}_S\left(\alpha \right)$, где~$S$~--- один из индексов $\tilde A_{ij}$, $\tilde B_i$, $\tilde C_i$, <<сбиваются>> в~сторону минимума. Это легко объясняется тем фактом, что при $\lambda_S=1$ максимальный вес в значении $\bar{x}_S\left(\alpha \right)$ имеет левая ветвь функции принадлежности, находящаяся ближе к нулю. Для решения данной проблемы введём дополнительные ограничения для параметров $\lambda_S$:
\begin{equation}
\label{eq:lambda-minimization-criterion}
  {\left( \lambda_{S}^{\star}-\lambda_S \right)}^2\to \min
\end{equation}
которые позволяют минимизировать отклонение параметров от оптимальных в~смысле сохранения нечёткой информации значений $\displaystyle \lambda_{S}^{\star}=\frac{a_S}{d_S}$ и, таким~образом, управлять устойчивостью решения. 

Критерии~\eqref{eq:lambda-minimization-criterion} и~целевая~функция задачи~\eqref{eq:fuzzy-lp-unstable-problem} противоречивы. Возникает задача векторной оптимизации, описанная в~\cite{MSU_Optimization}. Для~её решения воспользуемся аддитивной свёрткой критериев:
\begin{equation}
\label{eq:modified-target-function}
  f^{*}\left( \mathbf{x} \right)=\mathbf{C}^{*}\mathbf{x}+\gamma \sum\limits_{s}^{}{\left(\lambda_{S}^{*}-\lambda_S \right)}^{2} \to \min
\end{equation}
Семантика целевой функции~\eqref{eq:modified-target-function} такова: ищется решение $\mathbf{x}$ и вектор параметров преобразования $L$ $\mathbf{\lambda}_S$, которые позволяют удовлетворить исходный критерий оптимизации и~при~этом максимально сохранить нечёткую информацию, заложенную экспертами в параметры задачи. Безразмерный коэффициент~$\gamma$ позволяет привести значение свёртки к~одному~порядку со~значением исходной целевой функции.

Полученная пара векторов $\mathbf{x}\left( \alpha =1 \right)$ и $\mathbf{x}\left( \alpha =0 \right)$ позволяет восстановить модифицированные решения согласно~\eqref{eq:isomorphic-field}. Если рассмотреть устойчивость нечёткого решения в~смысле данного ранее определения для чёткой задачи, то~справедливо предположить, что нечёткая задача будет считаться устойчивой по решению, если при переходе с одного $\alpha$-уровня на другой не происходит значительного изменения решения относительно $\mathbf{x}\left( \alpha =1 \right)$, т.\,е.
\begin{equation}
\label{eq:fuzzy-solution-stability}
  \forall \varepsilon >0\ \exists \delta >0:\forall \alpha \in \left[0; 1\right)\ \left| \alpha -1 \right|<\delta \Rightarrow \left\| \mathbf{x}\left( 1 \right)-\mathbf{x}\left( \alpha  \right) \right\|<\varepsilon.
\end{equation}

Конкретное же условие устойчивости решения зависит от задачи и~может принимать различные формы.