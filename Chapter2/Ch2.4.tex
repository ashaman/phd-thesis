Модификация формулировки задачи устойчивости в терминах ограничений. Использование предложенного метода для решения задачи линейного программирования с иллюстрацией устойчивого и неустойчивого решения. 

Постановка задачи управления устойчивостью решения для модифицированных нечетких чисел в виде задачи векторной оптимизации. Обоснование построения линейной свертки критериев и алгоритм решения с использованием предложенного метода численной реализации.

Пример с задачей ЛП и переход к вопросам устойчивости решения

Рассмотрим следующую задачу линейного программирования, которая в \cite{Vorontsov_PI} решается с~использованием преобразования~$L$ и алгебры модифицированных нечётких чисел.

Рассмотрим проблему устойчивости, возникающую при решении нечётких задач, на~примере задачи линейного программирования с~нечёткими параметрами. Для~того, чтобы ввести определения устойчивости для нечёткой задачи, вначале рассмотрим формулировки для чёткой задачи линейного программирования, приведённые в~\cite{Ashmanov}.

Пусть дана задача линейного программирования
\begin{equation}
\label{eq:crisp-lp-problem}
  \left\{ \begin{aligned}
    & \max \left\langle c,x \right\rangle \\ 
    & Ax\leqslant b
  \end{aligned} \right.
\end{equation}
имеющая решение $x^{*}$. Возмущённой задачей линейного программирования называется задача
\begin{equation}
\label{eq:crisp-lp-problem-unstable}
  \left\{ \begin{aligned}
    & \max \left\langle c\left(\delta \right),x \right\rangle \\ 
    & A\left( \delta  \right)x\leqslant b\left(\delta \right)
  \end{aligned} \right.
\end{equation}
относительно параметров которой известно, что они в смысле определённой метрики близки параметрам исходной задачи~\eqref{eq:crisp-lp-problem}, т.\,е. для фиксированного $\delta>0$ выполняются неравенства
\begin{equation}
\label{eq:delta-metric}
  \left\{ \begin{aligned}
    & \left\| A\left( \delta  \right)-A \right\|<\delta \\ 
    & \left\| b\left( \delta  \right)-b \right\|<\delta \\ 
    & \left\| c\left( \delta  \right)-c \right\|<\delta
  \end{aligned} \right.
\end{equation}

При этом для фиксированных матрицы~$A$ и~вектора~$b$ и~$\forall \delta >\text{0}$ выражения $A\left(\delta \right)$, $b\left(\delta \right)$ называют $\delta$-окрестностью матрицы $A$ и~вектора $b$ соответственно. В~качестве метрики близости в~\eqref{eq:delta-metric} может использоваться, например, евклидова метрика.

Задачу~\eqref{eq:crisp-lp-problem} будем называть устойчивой, если~$\exists \delta_0 > 0$: $\forall \delta \in \left[ 0; \delta_0 \right]$  задача~\eqref{eq:crisp-lp-problem-unstable} имеет решение. Если обозначить решение задачи~\eqref{eq:crisp-lp-problem-unstable} за~$x^{*} \left(\delta \right)$, то~задачу~\eqref{eq:crisp-lp-problem} будем называть устойчивой по~решению, если она устойчива и~$\forall \varepsilon>0\ \exists \delta > 0$ такое, что~при выполнении неравенств~\eqref{eq:delta-metric} для~$\forall x^{*} \left(\delta \right)\ \exists x^{*}$, удовлетворяющее условию $\left| x^{*}\left(\delta \right)- x^{*} \right| < \varepsilon$. Стоит отметить, что если задача~\eqref{eq:crisp-lp-problem} неустойчива в~смысле хотя бы одного из~приведённых выше определений, то~она считается неустойчивой.

Для задач линейного программирования с нечёткими параметрами удобнее использовать определение устойчивости по решению, поскольку оно включает себя более узкое определение общей устойчивости, т.\,е. принципиальное существование решения возмущённой задачи. Пусть дана задача с нечёткими параметрами
\begin{equation*}
  \left\{ \begin{aligned}
    & f\left( \mathbf{x} \right)=\mathbf{Cx}\to \min;  \\ 
    & \mathbf{Ax}=\mathbf{B},
  \end{aligned} \right.
\end{equation*}
где $\mathbf{A}=\left\{ \tilde{A}_{ij} \right\}$~--- матрица, а~$\mathbf{B}=\left\{ \tilde{B}_i \right\}$, $\mathbf{C}=\left\{\tilde{C}_i \right\}$~--- векторы нечётких параметров. Применяя к~каждому из~элементов матрицы и~векторов преобразование~$L$, получаем модифицированную задачу
\begin{equation}
\label{eq:fuzzy-lp-unstable-problem}
  \left\{ \begin{aligned}
    & f\left( \mathbf{x} \right)={\mathbf{C}^{\star}}\mathbf{x}\to \min  \\ 
    & {\mathbf{A}^{\star}}\mathbf{x}={\mathbf{B}}^{\star}
  \end{aligned} \right.
\end{equation}
в которой $\mathbf{A}^{*}=\left\{ \bar{x}_{\tilde{A}_{ij}}\left(\alpha \right) \right\}$, $\mathbf{B}^{*}=\left\{ \bar{x}_{\tilde{B}_i}\left(\alpha \right) \right\}$, $\mathbf{C}^{*}=\left\{ \bar{x}_{\tilde{C}_i}\left(\alpha \right) \right\}$.

Задачу~\eqref{eq:fuzzy-lp-unstable-problem} удобнее всего решать на двух $\alpha$-уровнях и восстанавливать решение согласно~\eqref{eq:isomorphic-field}. Ввиду свойства сохранения моды, решение модифицированной задачи~\eqref{eq:fuzzy-lp-unstable-problem} при $\alpha=1$ аналогично решению чёткой задачи с коэффициентами, равными модам нечётких чисел. Если решать ту же задачу при $\alpha=0$ и без дополнительных ограничений на параметры $\lambda$ преобразования $L$, то возникает ситуация, при которой все значения $\bar{x}_S\left(\alpha \right)$, где $S$ – один из индексов $\tilde A_{ij}$, $\tilde B_i$, $\tilde C_i$, «сбиваются» в сторону минимума. Это легко объясняется тем фактом, что при $\lambda_S=1$ максимальный вес в значении $\bar{x}_S\left(\alpha \right)$ имеет левая ветвь функции принадлежности. Для решения данной проблемы введём дополнительные ограничения для параметров $\lambda_S$:
\begin{equation}
\label{eq:lambda-minimization-criterion}
  {\left( \lambda_{S}^{\star}-\lambda_S \right)}^2\to \min
\end{equation}
которые позволяют минимизировать отклонение параметров от оптимальных в~смысле сохранения нечёткой информации значений $\displaystyle \lambda_{S}^{\star}=\frac{a_S}{d_S}$. 

Критерии~\eqref{eq:lambda-minimization-criterion} и~целевая~функция задачи~\eqref{eq:fuzzy-lp-unstable-problem} противоречивы. Возникает задача векторной оптимизации, описанная в~\cite{MSU_Optimization}. Для~её решения воспользуемся аддитивной свёрткой критериев:
\begin{equation}
\label{eq:modified-target-function}
  f^{*}\left( \mathbf{x} \right)=\mathbf{C}^{*}\mathbf{x}+\gamma \sum\limits_{s}^{}{\left(\lambda_{S}^{*}-\lambda_S \right)}^{2} \to \min
\end{equation}
Семантика целевой функции~\eqref{eq:modified-target-function} такова: ищется решение $\mathbf{x}$ и вектор параметров преобразования $L$ $\mathbf{\lambda}_S$, которые позволяют удовлетворить исходный критерий оптимизации и~при~этом максимально сохранить нечёткую информацию, заложенную экспертами в параметры задачи. Безразмерный коэффициент~$\gamma$ позволяет привести значение свёртки к~одному~порядку со~значением исходной целевой функции.

Полученная пара векторов $\mathbf{x}\left( \alpha =1 \right)$ и $\mathbf{x}\left( \alpha =0 \right)$ позволяет восстановить модифицированные решения согласно~\eqref{eq:isomorphic-field}. Если рассмотреть устойчивость нечёткого решения в~смысле данного ранее определения для чёткой задачи, то~справедливо предположить, что нечёткая задача будет считаться устойчивой по решению, если при переходе с одного $\alpha$-уровня на другой не происходит значительного изменения решения относительно $\mathbf{x}\left( \alpha =1 \right)$, т.\,е.
\begin{equation*}
  \forall \varepsilon >0\ \exists \delta >0:\forall \alpha \in \left[0; 1\right)\ \left| \alpha -1 \right|<\delta \Rightarrow \left\| \mathbf{x}\left( 1 \right)-\mathbf{x}\left( \alpha  \right) \right\|<\varepsilon.
\end{equation*}

Конкретное же условие устойчивости решения зависит от задачи и может принимать различные формы.