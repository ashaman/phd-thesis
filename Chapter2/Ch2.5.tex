\begin{enumerate}
  \item Проведённый анализ существующих нечётких алгебр, принадлежащих к различным семействам, показал, что ни одна из них в полной мере не удовлетворяет требованиям к алгебре, выдвинутым в главе~\ref{chapter1}~--- ограничению роста неопределенности, сохранению чётких отношений в модельных уравнениях и возможности использования стандартных программных средств реализации численных методов решения.
  \item Для $\alpha$-интервалов треугольных чисел вводится линейное параметрического преобразования $L$, сопоставляющее обычному треугольному числу его модифицированное $LL/RR$-представление. Свойства преобразования $L$ исследуются с точки зрения сохранения экспертной информации, заложенной в нечёткое число.
  \item Над множестве $K$ модифицированных нечётких чисел строится алгебра. Вводится эквивалентная алгебре методика численного решения задач с нечёткими параметрами, позволяющая обойти проблему отсутствия линейного порядка на множестве $K$ и находить модифицированные решения задач на основе двухточечных вычислений.
  \item Кратко описывается алгебра двухкомпонентных нечётких чисел как альтернатива алгебре модифицированных нечётких чисел в тех случаях, когда потери экспертной информации недопустимы.
  \item Рассматривается задача линейного программирования с нечёткими коэффициентами, имеющая неустойчивое решение. Для общего случая нечёткой задачи формулируется условие устойчивости на основании классического определения для чёткого случая и решается проблема управления устойчивостью решения в виде задачи векторной оптимизации.
\end{enumerate}