Проблемы создания специального программного обеспечения решения задач на основе предложенной алгебры. Нерешенными остались проблемы построения полного порядка. Проведенное исследование показало, что подходящих методов сравнения нечетких чисел нет.
Предложен изоморфизм алгебры модифицированных нечетких чисел, который позволяет решать нечеткую задачу как две четких задачи на заданных альфа –уровнях.
Такой подход не только позволяет использовать все стандартные программные продукты для решения нечетких задач, но и решает проблему построения полного порядка.


3. Алгебра модифицированных нечетких чисел с примерами
Для того, чтобы использовать модифицированные нечёткие числа в качестве параметров чётких задач, необходимо построить алгебраическую систему для множества всех нечётких модифицированных чисел $K$. 
Будем строить чёткую алгебру \[P=\left\langle ;\ +,\,* \right\rangle \] на множестве модифицированных нечётких чисел $K=\left\{ \bar{x}\left( \alpha  \right) \right\};\ \alpha \in \left[ 0;1 \right]$ по аналогии с тем, как это делается в (Яхъяева). Для удобства дальнейших вычислений преобразуем $\bar{x}\left( \alpha  \right)$:
	\[\begin{matrix}
  \bar{x}\left( \alpha  \right)=\lambda \left( m-a+a\alpha  \right)+\left( 1-\lambda  \right)\left( m+b-b\alpha  \right)=a\alpha \lambda +\lambda \left( m-a \right)+m+b-b\alpha -\lambda \left( m+b \right)+b\alpha \lambda = \\ 
  =\alpha \left( \lambda a+\lambda b-b \right)+m+b-\lambda \left( m+b-m+a \right)=a\left( \lambda \left( a+b \right)-b \right)+m+b-\lambda \left( a+b \right) \\ 
\end{matrix}\] 
При построении алгебры будем использовать форму записи
	\[\bar{x}\left( \alpha  \right)=c+k\alpha \] 	(1.77)
где
	\[\begin{aligned}
  & \left[ \begin{aligned}
  & c=m+b-\lambda \left( a+b \right) \\ 
 & k=\lambda \left( a+b \right)-b \\ 
\end{aligned} \right. \\ 
 & \lambda \in \left[ 0;1 \right];\ c,k\in \mathbb{R} \\ 
\end{aligned}\] 	(1.78)
Операция сложения
Введем на множестве $K$ бинарную операцию сложения + следующим образом:
	\[\begin{matrix}
  {{{\bar{x}}}_{1}}(\alpha )+{{{\bar{x}}}_{2}}(\alpha )={{r}_{1}}\left( \alpha  \right)={{c}_{1}}+{{c}_{2}}+\left( {{k}_{1}}+{{k}_{2}} \right)\alpha  \\ 
  {{r}_{1}}\left( \alpha  \right)\in K \\ 
\end{matrix}\] 	(1.79)
Докажем основные свойства операции сложения:
1. Коммутативность:
\[{{\bar{x}}_{1}}\left( \alpha  \right)+{{\bar{x}}_{2}}\left( \alpha  \right)={{c}_{1}}+{{c}_{2}}+({{k}_{1}}+{{k}_{2}})\alpha ={{c}_{2}}+{{c}_{1}}+({{k}_{2}}+{{k}_{1}})\alpha ={{\bar{x}}_{2}}\left( \alpha  \right)+{{\bar{x}}_{1}}\left( \alpha  \right)\] 
2. Ассоциативность
\[\begin{matrix}
  {{{\bar{x}}}_{1}}\left( \alpha  \right)+\left( {{{\bar{x}}}_{2}}\left( \alpha  \right)+{{{\bar{x}}}_{3}}\left( \alpha  \right) \right)={{c}_{1}}+{{k}_{1}}\alpha +({{c}_{2}}+{{c}_{3}}+({{k}_{2}}+{{k}_{3}})\alpha )= \\ 
  =({{c}_{1}}+{{c}_{2}}+({{k}_{1}}+{{k}_{2}})\alpha )+{{c}_{3}}+{{k}_{3}}\alpha =\left( {{{\bar{x}}}_{1}}\left( \alpha  \right)+{{{\bar{x}}}_{2}}\left( \alpha  \right) \right)+{{{\bar{x}}}_{3}}\left( \alpha  \right) \\ 
\end{matrix}\]
Введём нейтральный (нулевой) элемент
	$\bar{0}=0+0\alpha \in K$ 	(1.80)
такой, что 
	\[\forall \bar{x}(\alpha )\in K:\ \bar{x}(\alpha )+\bar{0}=c+k\alpha +0+0\alpha =\bar{x}(\alpha )\] 	(1.81)
Также определим для каждого $\bar{x}\left( \alpha  \right)=c+k\alpha \in K$ единственный элемент $-\bar{x}\left( \alpha  \right)\in K$, называемый противоположным, такой, что выполняется равенство
	\[\bar{x}\left( \alpha  \right)+\left( -\bar{x}\left( \alpha  \right) \right)=\bar{0}\]	(1.82)
Очевидно, что противоположный элемент можно определить следующим образом:
	\[-\bar{x}\left( \alpha  \right)=-c-k\alpha \], 	(1.83)
так что равенство (1.82) будет справедливым:
	\[\bar{x}\left( \alpha  \right)+\left( -\bar{x}\left( \alpha  \right) \right)=c+k\alpha -c-k\alpha =0+0\alpha =\bar{0}\].
Операция вычитания нечётких чисел вводится как сложение числа ${{\bar{x}}_{1}}\left( \alpha  \right)$ с числом, противоположным числу ${{\bar{x}}_{2}}\left( \alpha  \right)$:
	\[{{\bar{x}}_{1}}\left( \alpha  \right)-{{\bar{x}}_{2}}\left( \alpha  \right)={{\bar{x}}_{1}}\left( \alpha  \right)+\left( -{{{\bar{x}}}_{2}}\left( \alpha  \right) \right)={{c}_{1}}-{{c}_{2}}+\left( {{k}_{1}}-{{k}_{2}} \right)\alpha \] 	(1.84)
Пример. Даны два нечётких числа $\tilde{A}=\left\langle 3;4;1 \right\rangle $ и $\tilde{B}=\left\langle 5;2;3 \right\rangle $. Выполнить операции ${{\tilde{A}}^{*}}+{{\tilde{B}}^{*}}$, ${{\tilde{A}}^{*}}-{{\tilde{B}}^{*}}$.
Вначале запишем уравнения для левой и правой ветвей каждого из чисел и оптимальные в смысле сохранения нечёткой информации значения $\lambda $:
	\[\begin{matrix}
   \left[ \begin{aligned}
  & x_{{\tilde{A}}}^{L}\left( \alpha  \right)=-1+4\alpha  \\ 
 & x_{{\tilde{A}}}^{R}\left( \alpha  \right)=4-\alpha  \\ 
 & {{\lambda }_{{\tilde{A}}}}=\frac{4}{4+1}=\frac{4}{5} \\ 
\end{aligned} \right. & \left[ \begin{aligned}
  & x_{{\tilde{B}}}^{L}\left( \alpha  \right)=3+2\alpha  \\ 
 & x_{{\tilde{B}}}^{R}\left( \alpha  \right)=8-3\alpha  \\ 
 & {{\lambda }_{{\tilde{B}}}}=\frac{2}{2+3}=\frac{2}{5} \\ 
\end{aligned} \right.  \\
\end{matrix}\] 
Найдём модифицированные значения:
	\[\left[ \begin{aligned}
  & {{{\bar{x}}}_{{\tilde{A}}}}\left( \alpha  \right)=\frac{4}{5}\left( -1+4\alpha  \right)+\frac{1}{5}\left( 4-\alpha  \right)=\frac{-4+16\alpha +4-\alpha }{5}=3\alpha  \\ 
 & {{{\bar{x}}}_{{\tilde{B}}}}\left( \alpha  \right)=\frac{2}{5}\left( 3+2\alpha  \right)+\frac{3}{5}\left( 8-3\alpha  \right)=\frac{6+4\alpha +24-9\alpha }{5}=6-\alpha  \\ 
\end{aligned} \right.\] 	(1.85)
Исходя из (1.85), значения искомых выражений равны:
	\[\left[ \begin{aligned}
  & {{{\tilde{A}}}^{*}}+{{{\tilde{B}}}^{*}}=3\alpha +6-\alpha =6+2\alpha  \\ 
 & {{{\tilde{A}}}^{*}}-{{{\tilde{B}}}^{*}}=3\alpha -6+\alpha =-6+4\alpha  \\ 
\end{aligned} \right.\] 
Операция умножения
Введём на множестве $K$ операцию умножения. Её можно было бы определить с помощью следующего выражения как сумму произведений компонент модифицированных нечётких чисел
	\[{{\bar{x}}_{1}}(\alpha )\cdot {{\bar{x}}_{2}}(\alpha )=r_{2}^{'}\left( \alpha  \right)=\left( {{c}_{1}}+{{k}_{1}}\alpha  \right)\left( {{c}_{2}}+{{k}_{2}}\alpha  \right)={{c}_{1}}{{c}_{2}}+{{c}_{1}}{{k}_{2}}\alpha +{{c}_{2}}{{k}_{1}}\alpha +{{k}_{1}}{{k}_{2}}{{\alpha }^{2}}\] 	(1.86)
Однако такое определение приводит к искажению треугольного вида результата нечётких операций, поскольку в (1.86) появляется слагаемое с ${{\alpha }^{2}}$. А это означает, что \[r_{2}^{'}\left( \alpha  \right)\notin K\].
Для того, чтобы результат операции умножения остался в множестве $K$, воспользуемся линейной интерполяцией – зависимость \[{{r}_{2}}(\alpha )\] будет восстанавливаться в виде линейной функции по значениям выражения (1.86) при $\alpha =0$ и $\alpha =1$. В первом случае $r_{2}^{'}\left( 0 \right)={{c}_{1}}{{c}_{2}}$, во втором \[r_{2}^{'}\left( 1 \right)=\left( {{c}_{1}}+{{k}_{1}} \right)\left( {{c}_{2}}+{{k}_{2}} \right)\]. Подставляя данные значения в уравнение прямой ${{r}_{2}}\left( \alpha  \right)$, получаем:
	\[\frac{\alpha -0}{1-0}=\frac{{{r}_{2}}\left( \alpha  \right)-{{c}_{1}}{{c}_{2}}}{\left( {{c}_{1}}+{{k}_{1}} \right)\left( {{c}_{2}}+{{k}_{2}} \right)-{{c}_{1}}{{c}_{2}}}\],	(1.87)
откуда, упрощая знаменатель второй дроби, получаем:
	\[\alpha =\frac{{{r}_{2}}\left( \alpha  \right)-{{c}_{1}}{{c}_{2}}}{{{c}_{1}}{{k}_{2}}+{{c}_{2}}{{k}_{1}}+{{k}_{1}}{{k}_{2}}}\] 	\[{{r}_{2}}\left( \alpha  \right)={{c}_{1}}{{c}_{2}}+\left( {{c}_{1}}{{k}_{2}}+{{c}_{2}}{{k}_{1}}+{{k}_{1}}{{k}_{2}} \right)\alpha \in K\] 	(1.88)
Таким образом, операция умножения на $K$ вводится следующим образом:
	\[{{\bar{x}}_{1}}(\alpha )\cdot {{\bar{x}}_{2}}(\alpha )={{r}_{2}}\left( \alpha  \right)={{c}_{1}}{{c}_{2}}+({{c}_{1}}{{k}_{2}}+{{c}_{2}}{{k}_{1}}+{{k}_{1}}{{k}_{2}})\alpha ;\ \ {{r}_{2}}\left( \alpha  \right)\in K\] 	(1.89)
Умножение нечёткого числа на скаляр $\beta $ является частным случаем операции (1.89), поскольку скаляр представляется в виде нечёткого синглтона
	\[\bar{\beta }=\beta +0\alpha \in K\].
Докажем основные свойства операции умножения.
Коммутативность доказывается элементарно:
	\[\begin{matrix}
  {{{\bar{x}}}_{1}}(\alpha )\cdot {{{\bar{x}}}_{2}}(\alpha )={{c}_{1}}{{c}_{2}}+({{c}_{1}}{{k}_{2}}+{{c}_{2}}{{k}_{1}}+{{k}_{1}}{{k}_{2}})\alpha = \\ 
  ={{c}_{2}}{{c}_{1}}+\left( {{c}_{2}}{{k}_{1}}+{{c}_{1}}{{k}_{2}}+{{k}_{2}}{{k}_{1}} \right)\alpha ={{{\bar{x}}}_{2}}(\alpha )\cdot {{{\bar{x}}}_{1}}(\alpha ) \\ 
\end{matrix}\] 
Ассоциативность. Для доказательства равенства
	\[{{\bar{x}}_{1}}\left( \alpha  \right)\cdot \left( {{{\bar{x}}}_{2}}\left( \alpha  \right){{{\bar{x}}}_{3}}\left( \alpha  \right) \right)=\left( {{{\bar{x}}}_{1}}\left( \alpha  \right){{{\bar{x}}}_{2}}\left( \alpha  \right) \right)\cdot {{\bar{x}}_{3}}\left( \alpha  \right)\] 	(1.90)
вычислим по отдельности и сравним результаты правой и левой частей:
$\begin{matrix}
  {{{\bar{x}}}_{1}}\left( \alpha  \right)\cdot \left( {{{\bar{x}}}_{2}}\left( \alpha  \right){{{\bar{x}}}_{3}}\left( \alpha  \right) \right)=\left( {{c}_{1}}+{{k}_{1}}\alpha  \right)\left( \left( {{c}_{2}}+{{k}_{2}}\alpha  \right)\left( {{c}_{3}}+{{k}_{3}}\alpha  \right) \right)= \\ 
  =\left( {{c}_{1}}+{{k}_{1}}\alpha  \right)\left( {{c}_{2}}{{c}_{3}}+\left( {{k}_{2}}{{c}_{3}}+{{k}_{3}}{{c}_{2}}+{{k}_{2}}{{k}_{3}} \right)\alpha  \right)= \\ 
  ={{c}_{1}}{{c}_{2}}{{c}_{3}}+\left( {{c}_{1}}{{k}_{2}}{{c}_{3}}+{{c}_{1}}{{c}_{2}}{{k}_{3}}+{{c}_{1}}{{k}_{2}}{{k}_{3}}+{{k}_{1}}{{c}_{2}}{{c}_{3}}+{{k}_{1}}{{k}_{2}}{{c}_{3}}+{{k}_{1}}{{c}_{2}}{{k}_{3}}+{{k}_{1}}{{k}_{2}}{{k}_{3}} \right)\alpha  \\ 
\end{matrix}$ 
$\begin{matrix}
  \left( {{{\bar{x}}}_{1}}\left( \alpha  \right){{{\bar{x}}}_{2}}\left( \alpha  \right) \right)\cdot {{{\bar{x}}}_{3}}\left( \alpha  \right)=\left( \left( {{c}_{1}}+{{k}_{1}}\alpha  \right)\left( {{c}_{2}}+{{k}_{2}}\alpha  \right) \right)\left( {{c}_{3}}+{{k}_{3}}\alpha  \right)= \\ 
  =\left( {{c}_{1}}{{c}_{2}}+\left( {{k}_{1}}{{c}_{2}}+{{c}_{1}}{{k}_{2}}+{{k}_{1}}{{k}_{2}} \right)\alpha  \right)\left( {{c}_{3}}+{{k}_{3}}\alpha  \right)= \\ 
  ={{c}_{1}}{{c}_{2}}{{c}_{3}}+\left( {{k}_{1}}{{c}_{2}}{{c}_{3}}+{{c}_{1}}{{k}_{2}}{{c}_{3}}+{{k}_{1}}{{k}_{2}}{{c}_{3}}+{{c}_{1}}{{c}_{2}}{{k}_{3}}+{{k}_{1}}{{c}_{2}}{{k}_{3}}+{{c}_{1}}{{k}_{2}}{{k}_{3}}+{{k}_{1}}{{k}_{2}}{{k}_{3}} \right) \\ 
\end{matrix}$ 
Путём сравнения результатов умножения можно убедиться, что свойство ассоциативности верно.
Для доказательства дистрибутивности умножения относительно сложения, т.е. справедливости равенства 
	\[{{\bar{x}}_{1}}(\alpha )\cdot {{\bar{x}}_{2}}(\alpha )+{{\bar{x}}_{1}}(\alpha )\cdot {{\bar{x}}_{3}}(\alpha )={{\bar{x}}_{1}}(\alpha )\cdot ({{\bar{x}}_{2}}(\alpha )+{{\bar{x}}_{3}}(\alpha ))\] 	(1.91)
также выполним действия в левой и правой частях выражения (1.91) по отдельности, а затем сравним результаты:
\[\begin{matrix}
  {{{\bar{x}}}_{1}}\left( \alpha  \right){{{\bar{x}}}_{2}}\left( \alpha  \right)+{{{\bar{x}}}_{1}}\left( \alpha  \right){{{\bar{x}}}_{3}}\left( \alpha  \right)=({{c}_{1}}+{{k}_{1}}\alpha )\cdot ({{c}_{2}}+{{k}_{2}}\alpha )+({{c}_{1}}+{{k}_{1}}\alpha )\cdot ({{c}_{3}}+{{k}_{3}}\alpha )= \\ 
  ={{c}_{1}}{{c}_{2}}+({{c}_{1}}{{k}_{2}}+{{c}_{2}}{{k}_{1}}+{{k}_{1}}{{k}_{2}})\alpha +{{c}_{1}}{{c}_{3}}+({{c}_{1}}{{k}_{3}}+{{c}_{3}}{{k}_{1}}+{{k}_{1}}{{k}_{3}})\alpha . \\ 
\end{matrix}\] 
\[\begin{matrix}
  {{{\bar{x}}}_{1}}\left( \alpha  \right)\left( {{{\bar{x}}}_{2}}\left( \alpha  \right)+{{{\bar{x}}}_{3}}\left( \alpha  \right) \right)=({{c}_{1}}+{{k}_{1}}\alpha )\cdot ({{c}_{2}}+{{k}_{2}}\alpha +{{c}_{3}}+{{k}_{3}}\alpha )= \\ 
  =({{c}_{1}}+{{k}_{1}}\alpha )\cdot ({{c}_{2}}+{{c}_{3}}+({{k}_{2}}+{{k}_{3}})\alpha )= \\ 
  ={{c}_{1}}{{c}_{2}}+{{c}_{1}}{{c}_{3}}+({{c}_{1}}{{k}_{2}}+{{c}_{2}}{{k}_{1}}+{{k}_{1}}{{k}_{2}})\alpha +({{c}_{1}}{{k}_{3}}+{{c}_{3}}{{k}_{1}}+{{k}_{1}}{{k}_{3}})\alpha . \\ 
\end{matrix}\] 
Сравнение результатов вычислений подтверждает верность равенства (1.91), т.е. дистрибутивность операции умножения относительно сложения.
Проиллюстрируем введённые операции примерами.
Пример. Даны нечёткие числа $\tilde{A}=\left\langle 4;1;2 \right\rangle $. $\tilde{B}=\left\langle 7;3;1 \right\rangle $ и $\tilde{C}=\left\langle 1;4;2 \right\rangle $. Выполнить операции $2\tilde{A}+\tilde{B}\tilde{C}$, $\tilde{C}\left( \tilde{A}-3\tilde{B} \right)$ с использованием модифицированных чисел.
Вначале найдём оптимальные в смысле сохранения нечёткой информации значения $\lambda$:
	\[\left[ \begin{aligned}
  & {{\lambda }_{{\tilde{A}}}}=\frac{1}{1+2}=\frac{1}{3} \\ 
 & {{\lambda }_{{\tilde{B}}}}=\frac{3}{3+1}=\frac{3}{4} \\ 
 & {{\lambda }_{{\tilde{C}}}}=\frac{4}{4+2}=\frac{2}{3} \\ 
\end{aligned} \right.\].
Найдём модифицированные нечёткие числа для каждого из исходных чисел согласно формулам (1.77) и (1.78):
	\[\left[ \begin{aligned}
  & {{{\bar{x}}}_{{\tilde{A}}}}\left( \alpha  \right)=4+2-\frac{1}{3}\left( 1+2 \right)+\left( \frac{1}{3}\left( 1+2 \right)-2 \right)\alpha =5-\alpha  \\ 
 & {{{\bar{x}}}_{{\tilde{B}}}}\left( \alpha  \right)=7+1-\frac{3}{4}\left( 3+1 \right)+\left( \frac{3}{4}\left( 3+1 \right)-1 \right)\alpha =5+2\alpha  \\ 
 & {{{\bar{x}}}_{{\tilde{C}}}}\left( \alpha  \right)=1+2-\frac{2}{3}\left( 4+2 \right)+\left( \frac{2}{3}\left( 4+2 \right)-2 \right)\alpha =-1+2\alpha  \\ 
\end{aligned} \right.\] 
Значение первого выражения в модифицированном виде равно
	$\begin{matrix}
  2{{{\tilde{A}}}^{*}}+{{{\tilde{B}}}^{*}}{{{\tilde{C}}}^{*}}=2\left( 5-\alpha  \right)+\left( 5+2\alpha  \right)\left( -1+2\alpha  \right)= \\ 
  =10-2\alpha +\left( -5+\left( 10-2+4 \right)\alpha  \right)=5+10\alpha  \\ 
\end{matrix}$
а второго
	\[\begin{matrix}
  {{{\tilde{C}}}^{*}}\left( {{{\tilde{A}}}^{*}}-3{{{\tilde{B}}}^{*}} \right)=\left( -1+2\alpha  \right)\left( 5-\alpha -3\left( 5+2\alpha  \right) \right)=\left( -1+2\alpha  \right)\left( -10-7\alpha  \right)= \\ 
  =10+\left( 7-20-14 \right)\alpha =10-27\alpha  \\ 
\end{matrix}\] 
Перейдём к рассмотрению нейтрального по умножению и обратного элементов. Введём единичный элемент
	\[\bar{1}=1+0\alpha \in K\] 	(1.92)
такой, что 
	\[\forall \bar{x}\left( \alpha  \right)\in K\quad \bar{x}\left( \alpha  \right)\cdot \bar{1}=\bar{x}\left( \alpha  \right)\] 	(1.93)
Равенство (1.93) легко подтверждается с помощью формулы (1.89):
	\[\bar{1}\cdot \bar{x}(\alpha )=(1+0\alpha )(c+k\alpha )=c+k\alpha =\bar{x}(\alpha )\] 
Несколько сложнее вводится на множестве К обратный элемент ${{\bar{x}}^{-1}}\left( \alpha  \right)\in K$ такой, что
	\[\bar{x}\left( \alpha  \right){{\bar{x}}^{-1}}\left( \alpha  \right)=\bar{1}\] 	(1.94)
Поскольку предполагается, что ${{\bar{x}}^{-1}}\left( \alpha  \right)\in K$, то будем искать обратный элемент в виде ${{\bar{x}}^{-1}}\left( \alpha  \right)={c}'+{k}'\alpha $. Имеем
	\[\bar{x}\left( \alpha  \right){{\bar{x}}^{-1}}\left( \alpha  \right)=\left( c+k\alpha  \right)\left( {c}'+{k}'\alpha  \right)=c{c}'+\left( c{k}'+{c}'k+k{k}' \right)\alpha =1+0\alpha \] 	(1.95)
Из (1.95) очевидно, что должны выполняться равенства
	\[\left\{ \begin{aligned}
  & c{c}'=1 \\ 
 & c{k}'+{c}'k+k{k}'=0 \\ 
\end{aligned} \right.\].
Из первого уравнения системы находим ${c}'=\frac{1}{c};\ c\ne 0$. Подставляя найденное значение во второе уравнение, имеем:
	\[c{k}'+\frac{k}{c}+k{k}'=0\],
откуда
	\[{k}'=\frac{-k}{c(k+c)}\] 
Таким образом, обратный элемент вводится в виде
	\[{{\bar{x}}^{-1}}(\alpha )=\frac{1}{c}-\frac{k}{c(c+k)}\alpha ,\quad c\ne 0\] 	(1.96)
Очевидно, что для существования обратного элемента число $\bar{x}\left( \alpha  \right)$ должно иметь ненулевую моду, поскольку, согласно (1.78), $c+k=m\ne 0$.
Операция деления вводится как умножение числа ${{\bar{x}}_{1}}\left( \alpha  \right)$ на число, обратное числу ${{\bar{x}}_{2}}\left( \alpha  \right)$:
	\[\begin{matrix}
  \frac{{{{\bar{x}}}_{1}}\left( \alpha  \right)}{{{{\bar{x}}}_{2}}\left( \alpha  \right)}={{{\bar{x}}}_{1}}\left( \alpha  \right)\bar{x}_{2}^{-1}\left( \alpha  \right)=\left( {{c}_{1}}+{{k}_{1}}\alpha  \right)\left( \frac{1}{{{c}_{2}}}-\frac{{{k}_{2}}}{{{c}_{2}}\left( {{k}_{2}}+{{c}_{2}} \right)}\alpha  \right)= \\ 
  =\frac{{{c}_{1}}}{{{c}_{2}}}+\left( -\frac{{{c}_{1}}{{k}_{2}}}{{{c}_{2}}\left( {{k}_{2}}+{{c}_{2}} \right)}+\frac{{{k}_{1}}}{{{c}_{2}}}-\frac{{{k}_{1}}{{k}_{2}}}{{{c}_{2}}\left( {{k}_{2}}+{{c}_{2}} \right)} \right)\alpha ;\ {{c}_{2}}\ne 0;\ {{k}_{2}}+{{c}_{2}}\ne 0 \\ 
\end{matrix}\] 	(1.97)
Таким образом, алгебра модифицированных нечётких чисел удовлетворяет всем аксиомам поля, а вводимые выше алгебраические операции над элементами множества $K$ модифицированных нечётких чисел позволяют рассматривать его как линейное пространство над полем P [Воеводин Линейная алгебра].
Стоит отметить, что расчёты с использованием модифицированных нечётких чисел можно проводить несколько иным, более удобным с вычислительной точки зрения, методом. Поскольку все элементы множества $K$ имеют линейную структуру, то для восстановления конкретного модифицированного числа $\tilde{A}$ достаточно знать два значения – ${{\bar{x}}_{{\tilde{A}}}}\left( 0 \right)$ и ${{\bar{x}}_{{\tilde{A}}}}\left( 1 \right)={{m}_{{\tilde{A}}}}$. Их подстановка в уравнение прямой позволяет получить зависимость ${{\bar{x}}_{{\tilde{A}}}}\left( \alpha  \right)$ или ${{\mu }_{{\tilde{A}}}}\left( x \right)$ в явном виде:
	\[\frac{\alpha -0}{1-0}=\frac{{{{\bar{x}}}_{{\tilde{A}}}}\left( \alpha  \right)-{{{\bar{x}}}_{{\tilde{A}}}}\left( 0 \right)}{{{{\bar{x}}}_{{\tilde{A}}}}\left( 1 \right)-{{{\bar{x}}}_{{\tilde{A}}}}\left( 0 \right)}\] 	(1.98)
	${{\bar{x}}_{{\tilde{A}}}}\left( \alpha  \right)={{\bar{x}}_{{\tilde{A}}}}\left( 0 \right)+\alpha \left( {{{\bar{x}}}_{{\tilde{A}}}}\left( 1 \right)-{{{\bar{x}}}_{{\tilde{A}}}}\left( 0 \right) \right)=\alpha {{\bar{x}}_{{\tilde{A}}}}\left( 1 \right)+\left( 1-\alpha  \right){{\bar{x}}_{{\tilde{A}}}}\left( 0 \right)$ 	(1.99)
Все вычисления с использованием данного способа ведутся только на двух $\alpha $-уровнях над действительными числами без использования дополнительных параметров. Если обозначить за $*$ произвольную арифметическую операцию, то для чисел в форме (1.99) её результат будет выглядеть следующим образом:
	\[{{\bar{x}}_{{\tilde{A}}}}\left( \alpha  \right)*{{\bar{x}}_{{\tilde{B}}}}\left( \alpha  \right)=\alpha \left( {{{\bar{x}}}_{{\tilde{A}}}}\left( 1 \right)*{{{\bar{x}}}_{{\tilde{B}}}}\left( 1 \right) \right)+\left( 1-\alpha  \right)\left( {{{\bar{x}}}_{{\tilde{A}}}}\left( 0 \right)*{{{\bar{x}}}_{{\tilde{B}}}}\left( 0 \right) \right)\] 	(1.100)
По сути, вводится другая алгебра на множестве $K$, оперирующая парой чётких значений и автоморфная введённой ранее алгебре модифицированных нечётких чисел. В самом деле, существует взаимно однозначная функция $f:K\to K$, которая позволяет сопоставить числу вида $c+k\alpha $ число вида $\alpha {{\bar{x}}_{{\tilde{A}}}}\left( 1 \right)+\left( 1-\alpha  \right){{\bar{x}}_{{\tilde{A}}}}\left( 0 \right)$. Воспользовавшись (1.99), (1.77) и приравняв свободные члены и коэффициенты при $\alpha$, получим следующее соответствие
	\[\left[ \begin{aligned}
  & c={{{\bar{x}}}_{{\tilde{A}}}}\left( 0 \right) \\ 
 & k={{{\bar{x}}}_{{\tilde{A}}}}\left( 1 \right)-{{{\bar{x}}}_{{\tilde{A}}}}\left( 0 \right) \\ 
\end{aligned} \right.\Leftrightarrow \left[ \begin{aligned}
  & {{{\bar{x}}}_{{\tilde{A}}}}\left( 0 \right)=c \\ 
 & {{{\bar{x}}}_{{\tilde{A}}}}\left( 1 \right)=c+k \\ 
\end{aligned} \right.\] 
Стоит отметить, что преобразование $L$ и вводимая алгебра модифицированных нечётких чисел снимают проблему сравнения двух нечётких чисел, поскольку все сравнения при решении задач происходят с действительными значениями нечёткого числа на выбранных $\alpha$-уровнях.
Пример. Даны нечёткие числа $\tilde{A}=\left\langle 4;2;3 \right\rangle $. $\tilde{B}=\left\langle -2;6;2 \right\rangle $ и $\tilde{C}=\left\langle 1;1;4 \right\rangle $. Вычислить значение $\tilde{D}=\frac{3\tilde{A}\tilde{C}+4\tilde{B}}{\tilde{C}\left( \tilde{A}-2\tilde{B} \right)}$ с использованием модифицированных чисел.
Воспользуемся упрощённой методикой вычислений. Оптимальные в смысле сохранения нечёткой информации значения $\lambda$ равны:
	\[\left[ \begin{aligned}
  & {{\lambda }_{{\tilde{A}}}}=\frac{2}{2+3}=\frac{2}{5} \\ 
 & {{\lambda }_{{\tilde{B}}}}=\frac{6}{6+2}=\frac{3}{4} \\ 
 & {{\lambda }_{{\tilde{C}}}}=\frac{1}{1+4}=\frac{1}{5} \\ 
\end{aligned} \right.\] 	(1.101)
Используя формулы (1.78) c учётом (1.101), получим коэффициенты для модифицированных нечётких чисел:
	\[\begin{aligned}
  & {{c}_{{\tilde{A}}}}=4+3-\frac{2}{5}\left( 2+3 \right)=5;\quad {{k}_{{\tilde{A}}}}=\frac{2}{5}\left( 2+3 \right)-3=-1 \\ 
 & {{c}_{{\tilde{B}}}}=-2+2-\frac{3}{4}\left( 6+2 \right)=-6;\quad {{k}_{{\tilde{B}}}}=\frac{3}{4}\left( 6+2 \right)-2=4 \\ 
 & {{c}_{{\tilde{C}}}}=1+4-\frac{1}{5}\left( 1+4 \right)=4;\quad {{k}_{{\tilde{C}}}}=\frac{1}{5}\left( 1+4 \right)-4=-3 \\ 
\end{aligned}\] 
Отсюда
	\[\left[ \begin{aligned}
  & {{{\bar{x}}}_{{\tilde{A}}}}\left( \alpha  \right)=5-\alpha  \\ 
 & {{{\bar{x}}}_{{\tilde{B}}}}\left( \alpha  \right)=-6+4\alpha  \\ 
 & {{{\bar{x}}}_{{\tilde{C}}}}\left( \alpha  \right)=4-3\alpha  \\ 
\end{aligned} \right.\] 	(1.102)
При $\alpha=1$ выражения (1.102) принимают следующие значения
	\[{{\bar{x}}_{{\tilde{A}}}}\left( 1 \right)=4;\ {{\bar{x}}_{{\tilde{B}}}}\left( 1 \right)=-2;\ {{\bar{x}}_{{\tilde{C}}}}\left( 1 \right)=1\] 	(1.103)
а при $\alpha=0$
	\[{{\bar{x}}_{{\tilde{A}}}}\left( 1 \right)=5;\ {{\bar{x}}_{{\tilde{B}}}}\left( 1 \right)=-6;\ {{\bar{x}}_{{\tilde{C}}}}\left( 1 \right)=4\] 	(1.104)
Подставляя в выражение $\tilde{D}=\frac{3\tilde{A}\tilde{C}+4\tilde{B}}{\tilde{C}\left( \tilde{A}-2\tilde{B} \right)}$ вместо $\tilde A$, $\tilde B$ и $\tilde C$ соответствующие им чёткие значения модифицированных чисел при $\alpha=0$ и $\alpha=1$, получаем:
	${{\bar{x}}_{{\tilde{D}}}}\left( 1 \right)={{\left. \frac{3{{{\tilde{A}}}^{*}}{{{\tilde{C}}}^{*}}+4{{{\tilde{B}}}^{*}}}{{{{\tilde{C}}}^{*}}\left( {{{\tilde{A}}}^{*}}-2{{{\tilde{B}}}^{*}} \right)} \right|}_{\alpha =1}}=\frac{3\cdot 4\cdot 1+4\cdot \left( -2 \right)}{1\cdot \left( 4-2\cdot \left( -2 \right) \right)}=\frac{12-8}{4+4}=\frac{1}{2}$,
	\[{{\bar{x}}_{{\tilde{D}}}}\left( 0 \right)={{\left. \frac{3{{{\tilde{A}}}^{*}}{{{\tilde{C}}}^{*}}+4{{{\tilde{B}}}^{*}}}{{{{\tilde{C}}}^{*}}\left( {{{\tilde{A}}}^{*}}-2{{{\tilde{B}}}^{*}} \right)} \right|}_{\alpha =0}}=\frac{3\cdot 5\cdot 4+4\cdot \left( -6 \right)}{4\cdot \left( 5-2\cdot \left( -6 \right) \right)}=\frac{60-24}{4\cdot 17}=\frac{9}{17}\].
Согласно формуле (1.99), модифицированный результат будет равен
	\[{{\bar{x}}_{{\tilde{D}}}}\left( \alpha  \right)=\frac{9}{17}+\alpha \left( \frac{1}{2}-\frac{9}{17} \right)=\frac{9}{17}-\frac{1}{34}\alpha \] 
Рассмотрим более сложный пример. Необходимо решить уравнение $\tilde{A}x=\tilde{B}$, в котором $\tilde{A}=\left\langle 3;1;2 \right\rangle $, $\tilde{B}=\left\langle 4;4;1 \right\rangle $. В терминах модифицированных нечётких чисел его решение будет иметь вид
	\[x\left( \alpha  \right)=\frac{{{{\bar{x}}}_{{\tilde{B}}}}\left( \alpha  \right)}{{{{\bar{x}}}_{{\tilde{A}}}}\left( \alpha  \right)}\] 	(1.105)
Поскольку операция деления для преобразованных нечётких чисел вводится как умножение на обратное число, перепишем (1.105) в виде
	\[x\left( \alpha  \right)=\bar x_{\tilde B}\left( \alpha  \right)\cdot \bar{x}_{{\tilde{A}}}^{-1}\left( \alpha  \right)\] 	(1.106)

Выберем значения $\lambda_{\tilde A}$ и $\lambda_{\tilde B}$, равные $\frac{a_{\tilde A}}{d_{\tilde A}}$ и $\frac{a_{\tilde B}}{d_{\tilde B}}$, в соответствие с критерием сохранения максимального количества нечёткой информации. Они равны
	\[{{\lambda }_{{\tilde{A}}}}=\frac{1}{3};\ {{\lambda }_{{\tilde{B}}}}=\frac{4}{5}\] 	(1.107)
Воспользовавшись формулами (1.77), (1.78) и (1.107), получаем
	\[\left[ \begin{aligned}
  & {{{\bar{x}}}_{{\tilde{A}}}}\left( \alpha  \right)=4-\alpha  \\ 
 & {{{\bar{x}}}_{{\tilde{B}}}}\left( \alpha  \right)=3\alpha +1 \\ 
\end{aligned} \right.\] 	(1.108)
Найдём обратный элемент $\bar{x}_{{\tilde{A}}}^{-1}\left( \alpha  \right)$ согласно формуле (1.96): 
	$\bar{x}_{{\tilde{A}}}^{-1}\left( \alpha  \right)=\frac{1}{4}-\frac{1}{\left( -1 \right)\cdot \left( 4-1 \right)}\alpha =\frac{1}{4}+\frac{1}{12}\alpha $ 	(1.109)
Пользуясь значениями из (1.108), (1.109) и подставляя их в (1.106), окончательно получаем
	\[x\left( \alpha  \right)=\left( 1+3\alpha  \right)\left( \frac{1}{4}+\frac{1}{12}\alpha  \right)=\frac{1}{4}+\frac{13}{12}\alpha \] 	(1.110)
Функция принадлежности модифицированного решения определяется как обратная к (1.110) – ${{\mu }_{x(\alpha )}}\left( x \right)=\frac{12}{13}x-\frac{3}{13}$. Подстановка решения задачи в исходное уравнение $\tilde{A}x=\tilde{B}$ с учётом формулы (1.89) приводит к верному равенству:
	\[\left( 4-\alpha  \right)\left( \frac{1}{4}+\frac{13}{12}\alpha  \right)=3\alpha +1\];
	\[\frac{52-13-3}{12}\alpha +1=3\alpha +1\].
Пример с задачей ЛП и переход к вопросам устойчивости решения
