Модификация нечетких чисел с помощью L-преобразования, доказательства свойств L- преобразования и следствия.

Построение алгебры модифицированных чисел, введение нуля и единицы,  операций, доказательства групповых свойств. Иллюстрация выполнению четких отношений равенства при решении уравнений. Иллюстрация ограничения роста неопределенности результата. Упоминание о двухкомпонентных нечетких числах как возможной альтернативе, когда любые потери экспертной информации недопустимы.

\subsection{Модифицированные нечеткие числа}

В~статье~\cite{Lebedev}, для преодоления изложенных в предыдущем параграфе недостатков нечётких алгебр, для решения нечётких задач был предложен следующий подход. Исходная задача $\tilde{Y}=\text{ }f\left( {\tilde{X}} \right)$ с нечёткими числовыми параметрами рассматривается как совокупность задач с интервальной неопределенностью
\begin{equation}
\label{eq:alpha-equivalence}
	\tilde{Y}=f\left( {\tilde{X}} \right)\to \bigcup\limits_{\alpha =0}^{1}{{{y}_{\alpha }}=f\left( {{X}_{\alpha }} \right)}
\end{equation}
с последующим переходом к полной определённости. Для этого на каждом $\alpha $-уровне внутри интервала $X_\alpha$ выбирается точка $\bar{x}\left( \alpha  \right)$. В [Лебедев Матвеев] для этого используется средняя точка $\alpha $-интервала
	\[\bar{x}\left( \alpha  \right)=\frac{{{x}^{L}}\left( \alpha  \right)+{{x}^{R}}\left( \alpha  \right)}{2}\] 	(1.43)
Для треугольных чисел $\bar{x}\left( \alpha  \right)$ является линейной функцией ввиду линейности .\[{{x}^{L}}\left( \alpha  \right)\] и \[{{x}^{R}}\left( \alpha  \right)\]. После решения N чётких $\alpha $-уровневых задач полученные результаты $y\left( \alpha  \right)$ аппроксимируются нечётким числом \[{{\tilde{Y}}^{*}}=\left\{ y(\alpha )\left| {{\mu }_{{\tilde{Y}}}}(y)=\alpha  \right. \right\}\], которое называется модифицированным решением задачи $\tilde{Y}=\text{ }f\left( {\tilde{X}} \right)$. В статье утверждается, что в реальных задачах модифицированного решения достаточно для, например, поддержки принятия решений. (в статье дан только частный случай, а также нет строгого математического доказательства. Это предложено в статье [Воронцов Матвеев ПИ8-2014]) Предлагаемый подход в своей основе имеет факторизацию, т.е. декомпозицию нечётких чисел по $\alpha $-уровням.
Если рассмотреть данный подход с точки зрения нечётких арифметик, то решение задачи с использованием факторизации нечётких чисел представляется как переход от использования арифметик «полноценных» нечётких чисел к арифметикам для чисел$LL/RR$-типа. Действительно, функция принадлежности такого числа является обратной к функции, которая определяет точки $\bar{x}\left( \alpha  \right)$:
	${{\mu }_{{{{\tilde{A}}}^{*}}}}\left( x \right)={{\left( \bar{x}\left( \alpha  \right) \right)}^{-1}}$ 	(1.44)
Определение. Число ${{\tilde{A}}^{*}}$, получаемое из числа $\tilde{A}$ с помощью преобразования (1.44), будем называть модифицированным нечётким числом. Модифицированное нечёткое число является числом $LL/RR$-типа, поскольку один из коэффициентов нечёткости равен нулю, а функция принадлежности имеет только левую или правую ветвь. В дальнейшем для модифицированных чисел, наряду с обозначением ${{\tilde{A}}^{*}}$, будем использовать обозначение $\bar{x}\left( \alpha  \right)$, которое указывает на механизм их построения как совокупности точек на выбранных $\alpha $-интервалах.
Пример. Пусть $\tilde{A}=\left\langle 2;2;4 \right\rangle $. Найдём модифицированное число ${{\tilde{A}}^{*}}$ в соответствие с (1.43).
Вначале запишем функцию принадлежности числа ${{\tilde{A}}^{*}}$:
	\[{{\mu }_{{\tilde{A}}}}\left( x \right)=\left\{ \begin{aligned}
  & \frac{x}{2};\ 0\le x\le 2 \\ 
 & \frac{6-x}{4};\ 2<x\le 6 \\ 
 & 0;\ x<0\ \ x>6 \\ 
\end{aligned} \right.\] 
Используя выражения из главы 1, получим
	\[\left[ \begin{aligned}
  & {{x}^{L}}\left( \alpha  \right)=m-a+a\alpha =2-2+2\alpha =2\alpha  \\ 
 & {{x}^{R}}\left( \alpha  \right)=m+b-b\alpha =2+4-4\alpha =6-4\alpha  \\ 
\end{aligned} \right.\] 
Тогда, согласно (1.43), получим:
	$\bar{x}\left( \alpha  \right)=\frac{{{x}^{L}}\left( \alpha  \right)+{{x}^{R}}\left( \alpha  \right)}{2}=\frac{2\alpha +6-4\alpha }{2}=3-\alpha $.
Т.к. прямая функция $\bar{x}\left( \alpha  \right)$ убывает на своей области определения $\alpha \in \left[ 0;1 \right]$, то и обратная ${{\mu }_{{{{\tilde{A}}}^{*}}}}\left( x \right)$ также будет убывать. Её областью определения будет являться область значений функции $\bar{x}\left( \alpha  \right)$, т.е. отрезок $\left[ 2;3 \right]$. В итоге${{\mu }_{{{{\tilde{A}}}^{*}}}}\left( x \right)$ состоит только из правой ветви:
	\[{{\mu }_{{{{\tilde{A}}}^{*}}}}\left( x \right)=\left\{ \begin{aligned}
  & 3-x;\ x\in \left[ 2;3 \right] \\ 
 & 0;\ x\notin \left[ 2;3 \right] \\ 
\end{aligned} \right.\] 
Оба числа – и исходное, и модифицированное – изображены на рис. ().
Очевидно, что на вид модифицированных чисел (и, соответственно, на итоговый результат решения задачи) влияние будут оказывать как характеристики самих нечётких чисел – мода, коэффициенты нечёткости, – так и принцип, согласно которому выбирается точка $\bar{x}\left( \alpha  \right)$. В статье [ПИ8-2014] предложено обобщение принципа, вводимого в [Матвеев Лебедев] – значение $\bar{x}\left( \alpha  \right)$ выбирается с помощью линейного параметрического преобразования $L$:
	\[\bar{x}\left( \alpha  \right)=L\left( {{X}_{\alpha }} \right)=\lambda {{x}^{L}}\left( \alpha  \right)+\left( 1-\lambda  \right){{x}^{R}}\left( \alpha  \right)\] 	(1.45)
Параметр преобразования $\lambda \in \left[ 0;1 \right]$ выбирается индивидуально для каждого числа согласно его характеристикам – величинам коэффициентов нечёткости и длине носителя. Нетрудно убедиться, что мода модифицированного числа ${{m}_{{{{\tilde{A}}}^{*}}}}$ равна
	\[{{m}_{{{{\tilde{A}}}^{*}}}}=\bar{x}\left( 1 \right)\],	(1.46)
а ненулевой коэффициент нечёткости и носитель равны по модулю
	\[{{d}_{{{{\tilde{A}}}^{*}}}}=\left| \bar{x}\left( 1 \right)-\bar{x}\left( 0 \right) \right|\].	(1.47)
Если известно уравнение (1.45), то, с учётом (1.46) и (1.47), модифицированное нечёткое число можно представить в виде тройки \[\left\langle \bar{x}\left( 1 \right);\left| \bar{x}\left( 1 \right)-\bar{x}\left( 0 \right) \right|;0 \right\rangle \] (число LL-типа) или \[\left\langle \bar{x}\left( 1 \right);0;\left| \bar{x}\left( 1 \right)-\bar{x}\left( 0 \right) \right| \right\rangle \] (число RR-типа). Тип модифицированного числа можно определить по коэффициенту при $\alpha $ в (1.45): если он больше нуля, то число LL-типа, если меньше – RR-типа.
Исходя из механизма построения модифицированных нечётких чисел, очевидно, что преобразование (1.45) сокращает информативность исходной нечёткой величины. Чтобы выяснить, насколько существенны потери нечёткой информации при различных значениях параметра $\lambda $, проведём исследование свойств преобразования L.
Для исследования свойств преобразования $L$ введём следующие характеристические показатели нечёткого числа, которые определяют его информативность с точки зрения принятия решений [5 ПИ8-2014], а также, по аналогии со [Cпесивцев], позволяют производить анализ и вычисления в форме, нечувствительной к знаку нечёткого числа:
•	длина носителя ${{d}_{{\tilde{A}}}}$ 
•	мода ${{m}_{{\tilde{A}}}}$ 
•	степень асимметрии $A{{S}_{{\tilde{A}}}}$
При использовании записи треугольного числа с помощью коэффициентов нечёткости, длина носителя определяется как их сумма:
	${{d}_{{\tilde{A}}}}=a+b$ 	(1.48)
Определение. Степенью асимметрии $A{{S}_{{\tilde{A}}}}$ будем называть характеристику треугольного нечёткого числа, определяемую как разность площадей прямоугольных треугольников, на которые исходное нечёткое число делится модой (рис!!!).
Площадь левого треугольника ${{S}_{1}}=\frac{1}{2}ah(\tilde{A})=\frac{a}{2}$, правого ${{S}_{2}}=\frac{1}{2}bh\left( {\tilde{A}} \right)=\frac{b}{2}$, отсюда
	\[A{{S}_{{\tilde{A}}}}={{S}_{2}}-{{S}_{1}}=\frac{b-a}{2}\in \left[ -\frac{a}{2};\frac{b}{2} \right]\] 	(1.49)
Если $\tilde{A}$ является числом $LL$($RR$)-типа, то $A{{S}_{{\tilde{A}}}}$, согласно (1.49), принимает значение $-\frac{a}{2}$ ($\frac{b}{2}$).
Очевидно, что запись треугольного нечёткого числа в виде тройки \[\left( {{m}_{{\tilde{A}}}},{{d}_{{\tilde{A}}}},A{{S}_{{\tilde{A}}}} \right)\] эквивалентна введённым ранее способам записи через коэффициенты нечёткости $\left( m;a;b \right)$ и точки пересечения с осью $Ox$ $\left( {{x}^{L}};m;{{x}^{R}} \right)$. При известных степени асимметрии $A{{S}_{{\tilde{A}}}}$ и длине носителя ${{d}_{{\tilde{A}}}}$, коэффициенты нечёткости определяются по формуле
	\[\left[ \begin{aligned}
  & a=\frac{{{d}_{{\tilde{A}}}}-2A{{S}_{{\tilde{A}}}}}{2} \\ 
 & b=\frac{{{d}_{{\tilde{A}}}}+2A{{S}_{{\tilde{A}}}}}{2} \\ 
\end{aligned} \right.\] 	(1.50)
Справедливость (1.50) можно проверить, подставив соответствующие значения для $A{{S}_{{\tilde{A}}}}$ и ${{d}_{{\tilde{A}}}}$ из формул (1.48) и (1.49).
Пример. Найдём эквивалентную форму записи для треугольного числа $\tilde{A}=\left\langle 5;3;2 \right\rangle $ в виде тройки «мода – носитель – степень асимметрии».
Согласно формуле (1.48) длина носителя равна
	\[{{d}_{{\tilde{A}}}}=a+b=3+2=5\]
Степень асимметрии $A{{S}_{{\tilde{A}}}}$ вычисляется по формуле (1.49)
	\[A{{S}_{{\tilde{A}}}}=\frac{2-3}{2}=-0,5\] 
Таким образом, число $\tilde{A}$ представляется в виде тройки $\left( 5;5;-0,5 \right)$.
2. Свойства преобразования L
Перейдём к непосредственному рассмотрению свойств преобразования $L$.
Свойство 1. Преобразование $L$ сохраняет моду нечёткого числа. Другими словами, \[\forall \lambda (\alpha ):\ {{m}_{{\tilde{A}}}}={{m}_{{{{\tilde{A}}}^{*}}}}\].
Доказательство. Перепишем с учётом равенств:
	\[\left[ \begin{matrix}
   \begin{aligned}
  & {{x}^{L}}(\alpha )=x_{A}^{L}+\alpha ({{m}_{A}}-x_{A}^{L}) \\ 
 & {{x}^{R}}(\alpha )=x_{A}^{R}+\alpha ({{m}_{A}}-x_{A}^{R}) \\ 
\end{aligned}  \\
\end{matrix} \right.\] 	(1.51)
При $\alpha =1$ имеем
	\[\left[ \begin{matrix}
   \begin{aligned}
  & {{x}^{L}}(1)=x_{{\tilde{A}}}^{L}+1\left( {{m}_{{\tilde{A}}}}-x_{{\tilde{A}}}^{L} \right)={{m}_{{\tilde{A}}}} \\ 
 & {{x}^{R}}(1)=x_{{\tilde{A}}}^{R}+1\left( {{m}_{{\tilde{A}}}}-x_{{\tilde{A}}}^{R} \right)={{m}_{{\tilde{A}}}} \\ 
\end{aligned}  \\
\end{matrix} \right.\],	(1.52)
поэтому при подстановке $\alpha =1$ в преобразование (1.45) получаем:
	\[\bar{x}\left( 1 \right)=\lambda {{x}^{L}}\left( 1 \right)+\left( 1-\lambda  \right){{x}^{R}}\left( 1 \right)=\lambda {{m}_{{\tilde{A}}}}+\left( 1-\lambda  \right){{m}_{{\tilde{A}}}}={{m}_{{\tilde{A}}}}\] 	(1.53)
Выражение (1.53) доказывает, что моды модифицированного и исходного чисел совпадают при любых значения параметров преобразования (1.45).
Свойство 2. При некоторых значениях параметра $\lambda $ преобразование $L$ сохраняет
А) знак степени асимметрии, т.е. \[\exists \lambda \in [0;1]:sign(A{{S}_{{\tilde{A}}}})=sign(A{{S}_{{{{\tilde{A}}}^{*}}}})\] 
Б) значение степени асимметрии, т.е. \[\exists \lambda \in [0;1]:\ A{{S}_{{\tilde{A}}}}=A{{S}_{{{{\tilde{A}}}^{*}}}}\] 
Доказательство. Вначале докажем утверждение А). Степень асимметрии исходного числа $\tilde{A}$ определяется выражением (1.49). Модифицированное число имеет только один ненулевой коэффициент нечёткости, который равен $\left| \bar{x}\left( 1 \right)-\bar{x}\left( 0 \right) \right|$. Кроме того, согласно свойству 1, мода числа при преобразовании сохраняется. Поэтому абсолютная величина степени асимметрии равна
	\[\left| A{{S}_{{{{\tilde{A}}}^{*}}}} \right|=\frac{\left| m-\bar{x}\left( 0 \right) \right|}{2}\] 	(1.54)
Поскольку
	\[\bar{x}\left( 0 \right)=\lambda {{x}^{L}}\left( 0 \right)+\left( 1-\lambda  \right){{x}^{R}}\left( 0 \right)=m+b-\lambda \left( a+b \right)\] 	(1.55)
то выражение (1.54) принимает вид
	\[\left| A{{S}_{{{{\tilde{A}}}^{*}}}} \right|=\frac{\left| b-\lambda \left( a+b \right) \right|}{2}\] 	(1.56)
Если исходное нечёткое число симметричное, т.е. $a=b$, то степень его асимметрии равна нулю. В этом случае равенство степеней асимметрии достигается при $\lambda =\frac{1}{2}$ -при подстановке данного значения в (1.56) имеем верное равенство:
	\[\left| A{{S}_{{{{\tilde{A}}}^{*}}}} \right|=\frac{\left| b-\lambda \left( a+b \right) \right|}{2}=\frac{1}{2}\left| b-\frac{b+b}{2} \right|=\frac{1}{2}\left| b-b \right|=0\].
Если $a>b$, то $A{{S}_{{\tilde{A}}}}<0$, и для выполнения первого пункта свойства необходимо, чтобы модифицированное число было числом$LL$-типа. Это достигается при 
	\[\bar{x}\left( 0 \right)<m\] 	(1.57)
Преобразовывая (1.57) с использованием (1.55), получаем, что \[b-\lambda \left( a+b \right)<0\], и в результате 
	\[\lambda \in \left( \frac{b}{a+b};1 \right]\] 	(1.58)
При $a<b$ $A{{S}_{{\tilde{A}}}}>0$, и модифицированное число должно быть числом$RR$-типа, т.е. 
	\[\bar{x}\left( 0 \right)>m\] 	(1.59)
По аналогии, подставляя в (1.59) значение $\bar{x}\left( 0 \right)$ из (1.55), получаем, что \[b-\lambda \left( a+b \right)>0\], и 
	\[\lambda \in \left[ 0;\frac{b}{a+b} \right)\] 	(1.60)
Таким образом, знак степени асимметрии сохраняется при выполнении следующих условий:
	\[\left[ \begin{aligned}
  & \lambda \in \left[ 0;\frac{b}{a+b} \right);\ a<b \\ 
 & \lambda =0,5;\ a=b \\ 
 & \lambda \in \left( \frac{b}{a+b};1 \right];\ a>b \\ 
\end{aligned} \right.\] 	(1.61)
Докажем теперь утверждение Б). Для этого покажем, что уравнение
	\[A{{S}_{{\tilde{A}}}}=A{{S}_{{{{\tilde{A}}}^{*}}}}\] 	(1.62)
имеет решения при $\lambda \in \left[ 0;1 \right]$. Пусть, для определённости, у исходного числа $a>b$, тогда $A{{S}_{{\tilde{A}}}}<0$, и поэтому должно быть справедливо неравенство$A{{S}_{{{{\tilde{A}}}^{*}}}}<0$. Пользуясь выражением (1.54), получаем
	\[A{{S}_{{{{\tilde{A}}}^{*}}}}=\frac{\bar{x}\left( 0 \right)-m}{2}=\frac{b-\lambda \left( a+b \right)}{2}<0\] 	(1.63)
Подставляя (1.49) и (1.63) в (1.62), приходим к уравнению
	\[\frac{b-a}{2}=\frac{b-\lambda \left( a+b \right)}{2}\] 	(1.64)
Его решением является значение
	\[\lambda =\frac{a}{a+b}=\frac{a}{d}\] 	(1.65)
Если же $a<b$, то $A{{S}_{{\tilde{A}}}}>0$, и выражение (1.63) должно быть положительным. В итоге имеем то же самое уравнение (1.64).
Таким образом, при \[\lambda =\frac{a}{a+b}\] значение степени асимметрии числа сохраняется.
Свойство 3. Модифицированное число всегда содержится внутри исходного числа. Другими словами, \[\forall \lambda \in \left[ 0;1 \right]:A_{\alpha }^{*}\subset {{A}_{\alpha }};{{d}_{{\tilde{A}}}}\ge {{d}_{{{{\tilde{A}}}^{*}}}}\], т.е. т.е. преобразование L уменьшает длину носителя нечёткого числа и оставляет \[\alpha \]-интервалы модифицированного числа в рамках \[\alpha \]-интервалов исходного числа.
Доказательство. Вначале докажем, что \[\forall \lambda \in \left[ 0;1 \right]\ {{d}_{{\tilde{A}}}}\ge {{d}_{{{{\tilde{A}}}^{*}}}}\]. Решим данное неравенство и покажем, что отрезок $\left[ 0;1 \right]$ содержится внутри решения. Очевидно, что 
	\[{{d}_{{{{\tilde{A}}}^{*}}}}=\left| \bar{x}\left( 1 \right)-\bar{x}\left( 0 \right) \right|=\left| {{m}_{{\tilde{A}}}}-\left( {{m}_{{\tilde{A}}}}-b+\lambda \left( a+b \right) \right) \right|=\left| b-\lambda \left( a+b \right) \right|\] 	(1.66)
и поэтому
	\[\left| b-\lambda (a+b) \right|\le a+b\] 	(1.67)
Раскрывая модуль, получаем систему неравенств
	\[\left\{ \begin{aligned}
  & b-\lambda \left( a+b \right)\ge -a-b \\ 
 & b-\lambda \left( a+b \right)\le a+b \\ 
\end{aligned} \right.\] 	(1.68)
Её решением является отрезок \[\left[ -\frac{a}{a+b};1+\frac{b}{a+b} \right]\]. Ввиду того, что $a,b\ge 0$, этот отрезок содержит в себе интервал $\left[ 0;1 \right]$.
Теперь покажем, что \[\forall \lambda \in \left[ 0;1 \right]:A_{\alpha }^{*}\subset {{A}_{\alpha }}\]. Очевидно, что $\alpha $-интервал модифицированного числа будет ограничен с одной стороны значением моды ${{m}_{{\tilde{A}}}}$, а с другой – значением $\bar{x}\left( \alpha  \right)$. В силу определения нечёткого числа
	\[\forall \alpha \in \left[ 0;1 \right]\ \ x_{{\tilde{A}}}^{L}(\alpha )\le {{m}_{{\tilde{A}}}}\le x_{{\tilde{A}}}^{R}(\alpha )\] 	(1.69)
Кроме того, из определения (1.45) преобразования L следует, что
	\[\forall \alpha \in \left[ 0;1 \right]\ \ x_{{\tilde{A}}}^{L}(\alpha )\le \bar{x}(\alpha )\le x_{{\tilde{A}}}^{R}(\alpha )\] 	(1.70)
Исходя из (1.69) и (1.70), \[\left[ x_{{\tilde{A}}}^{L}(\alpha );x_{{\tilde{A}}}^{R}(\alpha ) \right]\supset \left[ x_{{\tilde{A}}}^{M};\bar{x}(\alpha ) \right]\]  для LL-числа (соответственно, \[\left[ x_{{\tilde{A}}}^{L}(\alpha );x_{{\tilde{A}}}^{R}(\alpha ) \right]\supset \left[ \bar{x}(\alpha );x_{{\tilde{A}}}^{M} \right]\] для RR-числа).
Проиллюстрируем доказанные выше свойства примером. Пусть $\tilde{A}=\left\langle 4;5;1 \right\rangle $. Степень его асимметрии равна
	\[A{{S}_{{\tilde{A}}}}=\frac{1-5}{2}=-2\],
а длина носителя ${{d}_{{\tilde{A}}}}$
	\[{{d}_{{\tilde{A}}}}=5+1=6\].
Уравнения для левой и правой ветвей функции принадлежности принимают вид
	\[\left[ \begin{aligned}
  & {{x}^{L}}\left( \alpha  \right)=-1+5\alpha  \\ 
 & {{x}^{R}}\left( \alpha  \right)=5-\alpha  \\ 
\end{aligned} \right.\] 	(1.71)
Выполним преобразование $L$ с параметром $\lambda =\frac{1}{4}$ с учётом (1.71):
	\[\bar{x}\left( \alpha  \right)=\frac{1}{4}\left( 5\alpha -1 \right)+\frac{3}{4}\left( 5-\alpha  \right)=\frac{1}{4}\left( 5\alpha -1+15-3\alpha  \right)=\frac{1}{4}\left( 14+2\alpha  \right)=3,5+0,5\alpha \] 	(1.72)
Поскольку коэффициент при $\alpha $ в выражении (1.72) больше нуля, то число ${{\tilde{A}}^{*}}$ является числом LL-типа. Мода числа ${{\tilde{A}}^{*}}$, согласно свойству 1, равна 4, а левый коэффициент нечёткости равен длине носителя и равен
	\[a={{d}_{{{{\tilde{A}}}^{*}}}}=\left| \bar{x}\left( 1 \right)-\bar{x}\left( 0 \right) \right|=\left| 4-\left( 3,5-0,5\cdot 0 \right) \right|=0,5\].
Таким образом, число ${{\tilde{A}}^{*}}$ может быть записано в виде тройки $\left\langle 4;0,5;0 \right\rangle $. Степень его асимметрии, согласно свойству 2, сохраняет знак относительно $A{{S}_{{\tilde{A}}}}$ и равна
	\[A{{S}_{{{{\tilde{A}}}^{*}}}}=\frac{0-0,5}{2}=-0,25\].
Исходное и модифицированное числа изображены на рис. На нём наглядно иллюстрируется свойство 3 – модифицированное число ${{\tilde{A}}^{*}}$ целиком расположено внутри границ исходного числа.
Из доказанных выше свойств следует, что применение преобразования L к нечетким исходным данным в основном сохраняет их информативность при целенаправленном выборе параметра преобразования. Уменьшение длины носителя при определенных оговорках можно рассматривать как положительное явление, поскольку при этом повышается общая устойчивость решения~\cite{Vorontsov_PI}.
Следствия из свойств и рекомендации по выбору $\lambda $ 
1. Модифицированное нечёткое число, получаемое с помощью преобразования L с $\lambda =0,5$ из симметричного нечёткого числа $\left\langle m;a;a \right\rangle $, является нечётким синглтоном.
Действительно, при указанном значении $\lambda $, используя выражения, получаем:
	\[\bar{x}\left( \alpha  \right)=\frac{1}{2}\left( m-a+a\alpha  \right)+\frac{1}{2}\left( m+b-b\alpha  \right)=\frac{1}{2}\left( m-a+a\alpha +m+a-a\alpha  \right)=m\] 
Это следствие накладывает ограничения на возможность использования симметричных нечётких чисел в задачах, поскольку все нечёткие вычисления с использованием преобразования L в этом случае будут сведены к обычным алгебраическим операциям над модами чисел.
2. Применение преобразования L с параметром \[\lambda =\frac{a}{a+b}\] к числу LL/RR-типа не изменяет данного числа.
В самом деле, для LL-числа $b=0$, поэтому \[x_{{\tilde{A}}}^{R}\left( \alpha  \right)=m+b-b\alpha =m\], $\lambda =\frac{a}{a+0}=1$. Отсюда
	\[\bar{x}\left( \alpha  \right)=\lambda \left( m-a+a\alpha  \right)+\left( 1-\lambda  \right)m=m-a+a\alpha =x_{{\tilde{A}}}^{L}\left( \alpha  \right)\] 
Для RR-числа $a=0$, отсюда \[x_{{\tilde{A}}}^{L}\left( \alpha  \right)=m-a+a\alpha =m\] и $\lambda =\frac{0}{b+0}=0$. Ввиду этого
	\[\bar{x}\left( \alpha  \right)=\lambda m+\left( 1-\lambda  \right)\left( m+b-b\alpha  \right)=m+b-b\alpha =x_{{\tilde{A}}}^{R}\left( \alpha  \right)\] 
Отдельно выделим несколько наиболее интересных значений $\lambda $.
1. $\lambda =\frac{a}{a+b}=\frac{a}{{{d}_{{\tilde{A}}}}}$. Как уже было показано выше, при данном значении сохраняется значение степени асимметрии.
Проиллюстрируем это на следующем примере. Пусть $\tilde{A}=\left\langle 1;4;6 \right\rangle $. Значение $\lambda =\frac{4}{4+6}=\frac{2}{5}$, степень асимметрии $A{{S}_{{\tilde{A}}}}=\frac{6-4}{2}=1$, а уравнения правой и левой ветвей
	\[\left[ \begin{aligned}
  & {{x}^{L}}\left( \alpha  \right)=-3+4\alpha  \\ 
 & {{x}^{R}}\left( \alpha  \right)=7-6\alpha  \\ 
\end{aligned} \right.\].
Преобразование L даёт следующий результат:
	\[\bar{x}\left( \alpha  \right)=\frac{2}{5}\left( -3+4\alpha  \right)+\frac{3}{5}\left( 7-6\alpha  \right)=\frac{1}{5}\left( -6+8\alpha +21-18\alpha  \right)=3-2\alpha \].
Полученное в результате преобразования число ${{\tilde{A}}^{*}}$ является числом RR-типа, поэтому степень его асимметрии положительна и, согласно (1.54), равна
	\[A{{S}_{{{{\tilde{A}}}^{*}}}}=\frac{\left| 1-\left( 3-2\cdot 0 \right) \right|}{2}=1\].
2. \[\lambda =\frac{b}{a+b}=\frac{b}{{{d}_{{\tilde{A}}}}}\]. Преобразование L с таким значение параметра уничтожает нечёткую информацию, заложенную экспертом в число, поскольку в этом случае модифицированное число превращается в чёткое:
	\[\begin{matrix}
  \bar{x}\left( \alpha  \right)=\frac{b}{a+b}\left( m-a+a\alpha  \right)+\left( 1-\frac{b}{a+b} \right)\left( m+b-b\alpha  \right)=\frac{b\left( m-a+a\alpha  \right)+a\left( m+b-b\alpha  \right)}{a+b}= \\ 
  =\frac{bm-ab+ab\alpha +am+ab-ab\alpha }{a+b}=\frac{am+bm}{a+b}=m \\ 
\end{matrix}\] 
3. При \[\lambda =\frac{b}{{{d}_{{\tilde{A}}}}}-\frac{b-a}{3\left( b+a \right)}=\frac{2b+a}{3\left( b+a \right)}=\frac{2b+a}{3{{d}_{{\tilde{A}}}}}\] значение \[\bar{x}(\alpha )\] является проекцией центра тяжести треугольника, построенного на отрезке \[\left[ {{x}^{L}}(\alpha ),{{x}^{R}}(\alpha ) \right]\] как на основании, на ось Ox (см. рис.).
Данное значение получается следующим образом. Координата \[{{x}_{0}}\left( \alpha  \right)\] центра тяжести $\alpha $-сечения треугольного числа рассчитывается как
	\[{{x}_{0}}\left( \alpha  \right)=\frac{{{x}^{L}}\left( \alpha  \right)+m+{{x}^{R}}\left( \alpha  \right)}{3}\] 	(1.73)
Значения \[{{x}^{L}}\left( \alpha  \right)\] и ${{x}^{R}}\left( \alpha  \right)$ равны
	\[\left[ \begin{aligned}
  & {{x}^{L}}\left( \alpha  \right)=m-a+a\alpha  \\ 
 & {{x}^{R}}\left( \alpha  \right)=m+b-b\alpha  \\ 
\end{aligned} \right.\] 	(1.74)
Из (1.73) и (1.74) следует, что
	\[\begin{matrix}
  {{x}_{0}}\left( \alpha  \right)=\frac{{{x}^{L}}\left( \alpha  \right)+m+{{x}^{R}}\left( \alpha  \right)}{3}=\frac{m-a+a\alpha +m+m+b-b\alpha }{3}= \\ 
  =\frac{3m-\left( a-b \right)+\alpha \left( a-b \right)}{3}=m+\frac{\left( a-b \right)\left( \alpha -1 \right)}{3} \\ 
\end{matrix}\] 	(1.75)
Преобразование $L$ вычисляется по формуле
	\[\begin{matrix}
  \bar{x}\left( \alpha  \right)=\lambda {{x}^{L}}\left( \alpha  \right)+\left( 1-\lambda  \right){{x}^{R}}\left( \alpha  \right)=\lambda \left( m-a+a\alpha  \right)+\left( 1-\lambda  \right)\left( m+b-b\alpha  \right)= \\ 
  =\lambda \left( m-a+a\alpha -m-b+b\alpha  \right)+b\left( 1-\alpha  \right)+m=\lambda \left( a+b \right)\left( \alpha -1 \right)+b\left( 1-\alpha  \right)+m \\ 
\end{matrix}\] 	(1.76)
Приравнивая результаты (1.75) и (1.76), получаем:
	\[m+\frac{\left( a-b \right)\left( \alpha -1 \right)}{3}=m+b\left( 1-\alpha  \right)+\lambda \left( a+b \right)\left( \alpha -1 \right)\] 
	\[\left( a-b \right)\left( \alpha -1 \right)=-3b\left( \alpha -1 \right)+3\lambda \left( a+b \right)\left( \alpha -1 \right)\] 
При $\alpha =1$ равенство выполняется для всех $\lambda $. Для остальных $\alpha $, поделим обе части равенства на $\alpha -1$:
	\[a-b=-3b+3\lambda \left( a+b \right)\],
откуда получается искомое значение $\lambda $.

Почему числа должны быть асимметричными – основной use case для нечётких параметров это определение рисков (!!!) С точки зрения оценки рисков, симметричное число не несёт в себе никакой информации, поскольку позитивный и негативный исходы <<равновероятны>> (по идее, такое число можно заменить симметричным относительно матожидания распределением). Риск предполагает только негативный исход, и тому эксперту, который будет формировать оценки, необходимо это учитывать. Таким образом, ответственность за результаты частично переносится на экспертов. Предлагаемая в данном исследовании методика решения будет работать и на симметричных числах, однако она будет эквивалентна в плане решения уже известным чётким задачам с чёткими параметрами (это обобщение/расширение обычной четкой арифметики).

